Dla porządku -- podajmy pełną treść pytania, bo była zbyt długa by ją dać do nazwy sekcji.

\begin{question}[Proces Poissona]
	Definicja i potrzebne własności aby wykazać co następuje. Niech \((N(t),t_0)\) będzie procesem Poissona o parametrze \(\lambda\). Wykazać, że jeśli w przedziale czasowym \((0,t]\) zaszło dokładnie jedno zdarzenie, czyli \(N(t) = 1\) to czas zajścia tego zdarzenia \(X_1\) ma rozkład jednostajny na przedziale \((0,t]\) (w szczególności nie zależy od \(\lambda\)).
\end{question}