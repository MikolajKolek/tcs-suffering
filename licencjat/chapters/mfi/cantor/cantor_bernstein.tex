\begin{theorem}[Cantor-Bernstein]
	Dla dowolnych zbiorów \( A, B \) zachodzi implikacja:
	\[
		(A \leqnum B \land B \leqnum A) \implies A \eqnum B
	\]
\end{theorem}
\begin{proof}
	Z założeń mamy dane dwie iniekcje -- \( f: A \rightarrow B \) oraz \( g: B \rightarrow A \).

	Korzystając z lematu Banacha (\ref{mfi:banach_lemma}) możemy podzielić zbiory \(A\) i \(B\):
	\[
		A = A_1 \cup A_2, A_1 \cap A_2 = \varnothing
	\]
	\[
		B = B_1 \cup B_2, B_1 \cap B_2 = \varnothing
	\]
	w taki sposób, że
	\[
		\fIm{f}(A_1) = B_1, \fIm{g}(B_2) = A_2
	\]

	Ostatnia własność mówi nam, że zawężenia \( f_{\mid A_1} : A_1 \rightarrow B_1 \) i  \( g_{\mid B_2} : B_2 \rightarrow A_2 \) są suriekcjami.
	Ponieważ jednak z założeń \( f, g \) są iniekcjami, to ich zawężenia są eleganckimi bijekcjami.

	Dzięki nim możemy zdefiniować \( h: A \rightarrow B \)
	\[
		h(x) = \begin{cases}
			f_{\mid A_1}(x)        & \text{ gdy } x \in A_1 \\
			(g_{\mid B_2})^{-1}(x) & \text{ gdy } x \in A_2
		\end{cases}
	\]
	które jest bijekcją, o czym świadczy odwrotność
	\[
		h^{-1}(x) = \begin{cases}
			(f_{\mid A_1})^{-1}(x) & \text{ gdy } x \in B_1 \\
			g_{\mid B_2}(x)        & \text{ gdy } x \in B_2
		\end{cases}
	\]

	Mamy zatem bijekcję między \( A \) i \( B \), a zatem \( A \eqnum B \)


\end{proof}
