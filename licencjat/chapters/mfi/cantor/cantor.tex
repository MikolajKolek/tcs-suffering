\begin{theorem}[Cantor]
	Dla dowolnego zbioru \( A \)
	\[
		A \ltnum \powerset(A)
	\]
\end{theorem}
\begin{proof}
	Oczywiście \( A \leqnum \powerset(A) \) bo mamy iniekcję \( f: A \ni a \mapsto \set{a} \in \powerset(A) \)

	Załóżmy teraz nie wprost, że istnieje bijekcja \( f: A \function \powerset(A) \) i zdefiniujmy
	\[
		R = \set{ x \in A \mid x \notin f(x) }
	\]
	\( R \) zawiera jedynie elementy z \( A \), więc niewątpliwie \( R \in \powerset(A) \), a ponieważ \( f \) jest bijekcją to \( \exists x_0 : f(x_0) = R \).

	Mamy teraz dwie możliwości:
	\begin{itemize}
		\item Jeśli \( x_0 \in R = f(x_0) \) to \( x_0 \notin f(x_0) = R \)
		\item Jeśli \( x_0 \notin R \) to \( x_0 \in f(x_0) = R \)
	\end{itemize}
	W obu przypadkach mamy sprzeczność, co dowodzi, że nie może istnieć bijekcja między \( A \) i \( \mathcal{P}(A) \)
\end{proof}