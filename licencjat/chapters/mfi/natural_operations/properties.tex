\subsection{Własności operacji}

Zdefiniowaliśmy operacje o których wiemy (a posteriori) że mają pewne własności np. przemienność i łączność. Zwykle te własności przyjmuje się jako dane, tu zobaczymy że faktycznie wynikają one z konstrukcji liczb.

Większość z nich idzie bardzo podobnie (indukcyjnie) dlatego przedstawimy jedynie parę szkiców.

\begin{lemma}
	\( n + 0 = n \)
\end{lemma}
\begin{proof}
	Może i wygląda to głupio, ale o dodawaniu wiemy z definicji tylko że \( 0 + n = n \), a jeszcze nie mamy przemienności.

	Robimy indukcję po \( n \)
	\begin{itemize}
		\item Baza: \( n = 0 \), mamy \( n + 0 = 0 + 0 = 0 = n \) z definicji
		\item Krok indukcyjny: \( n' + 0 = (n + 0)' = n' \)
	\end{itemize}
\end{proof}

\begin{lemma}
	\( a' + b = a + b' \)
\end{lemma}
\begin{proof}
	Robimy indukcję po \( a \)
	\begin{itemize}
		\item Baza: \( 0' + b = (0 + b)' = b' = 0 + b' \)
		\item Krok indukcyjny: \( a'' + b \stackrel{\text{def.} +}{=} (a' + b)' \stackrel{\text{zał. ind}}{=} (a + b')' \stackrel{\text{def.} +}{=} a' + b' \)
	\end{itemize}
\end{proof}


\begin{theorem}[Przemienność dodawania]
	\( a + b = b + a \)
\end{theorem}
\begin{proof}
	Robimy indukcję po \( a \)
	\begin{itemize}
		\item Baza: \( 0 + b = b = b + 0 \)
		\item Krok indukcyjny: \( a' + b = a + b' = b' + a = b + a' \)
	\end{itemize}
\end{proof}

\begin{theorem}[Prawo skróceń]
	\( n + p = k + p \implies n = k \)
\end{theorem}
\begin{proof}
	Indukcja po \( p \)
	\begin{itemize}
		\item Baza: \( n + 0 = k + 0 \implies n = k \) tak po prostu
		\item Krok indukcyjny: \( n + p' = k + p' \implies n' + p = k' + p \implies n' = k' \implies n = k \)
	\end{itemize}
\end{proof}