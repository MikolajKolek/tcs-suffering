\begin{theorem}
	Dla dowolnych zbiorów \(A, B, C\) zachodzi
	\begin{equation*}
		\pars{A^B}^C \eqnum A^{B \times C}
	\end{equation*}
\end{theorem}
\begin{proof}
	Z~definicji --- konstruujemy bijekcję
	\begin{equation*}
		\alpha\colon \pars{A^B}^C \function A^{B \times C}
	\end{equation*}
	Funkcja \(\alpha\)~przyjmuje funkcję \(C \function A^B\) i~zwraca funkcję \(B \times C \function A\). Zdefiniujmy \(\alpha\)~następująco:
	\begin{equation*}
		\begin{split}
			\alpha\pars{f}
			 & \coloneqq \pars{\text{taka funkcja \(\gamma\), że \(\gamma\pars{b, c} = \pars{f\pars{c}}\pars{b}\)}} \\
			 & \coloneqq \set{\pars{\pars{b, c}, \pars{f\pars{c}}\pars{b}} : \pars{b, c} \in B \times C}
		\end{split}
	\end{equation*}
	Nadużywając trochę notacji, możemy to bardziej intuicyjnie zapisać jako
	\begin{equation*}
		\pars{\alpha\pars{f}}\pars{b, c} \coloneqq \pars{f\pars{c}}\pars{b}
	\end{equation*}
	Musimy pokazać, że \(\alpha\)~bijekcją.
	\begin{description}
		\item[Injektywność.] Weźmy różne \(f_1, f_2 \in \pars{A^B}^C\) i~pokażmy, że \(\alpha\pars{f_1} \neq \alpha\pars{f_2}\).

		      Co to znaczy, że \(f_1, f_2\) są różne? To są funkcje \(C \function A^B\), więc gdy są różne, to na jakimś argumencie \(c_0 \in C\) przyjmują różne wartości:
		      \begin{equation*}
			      f_1\pars{c_0} \neq f_2\pars{c_0}
		      \end{equation*}
		      Ale to też są funkcje, tym razem \(B \function A\). Czyli skoro są różne, to na pewnym argumencie \(b_0 \in B\) przyjmują różne wartości:
		      \begin{equation*}
			      \pars{f_1\pars{c_0}}\pars{b_0} \neq \pars{f_2\pars{c_0}}\pars{b_0}
		      \end{equation*}
		      Możemy to zapisać za pomocą \(\alpha\)~jako
		      \begin{equation*}
			      \pars{\alpha\pars{f_1}}\pars{b_0, c_0} \neq \pars{\alpha\pars{f_2}}\pars{b_0, c_0}
		      \end{equation*}
		      Oznacza to, że funkcje \(\alpha\pars{f_1}, \alpha\pars{f_2}\colon B \times C \function \alpha\) przyjmują różne wartości na argumencie \(\pars{b_0, c_0} \in B \times C\). A~zatem są to różne funkcje \(\alpha\pars{f_1} \neq \alpha\pars{f_2}\), co właśnie chcieliśmy udowodnić.
		\item[Surjektywność.] Weźmy \(g \in A^{B \times C}\). Musimy pokazać, że istnieje \(f \in \pars{A^B}^C\) takie, że
		      \begin{equation*}
			      \alpha\pars{f} = g
		      \end{equation*}
		      Niewątpliwie, w~dziedzinie funkcji \(\alpha\)~znajduje się w~szczególności \(f_0\colon C \function A^B\)~zdefiniowane następująco
		      \begin{equation*}
			      \begin{split}
				      f_0\pars{c}
				       & \coloneqq \pars{\text{taka funkcja \(\gamma\), że \(\gamma\pars{b} = g\pars{b, c}\)}} \\
				       & \coloneqq \set{\pars{b, g\pars{b, c}} : b \in B}
			      \end{split}
		      \end{equation*}
		      co można, ponownie z~drobnym nadużyciem notacji można zapisać jako
		      \begin{equation*}
			      \pars{f_0\pars{c}}\pars{b} \coloneqq g\pars{b, c}
		      \end{equation*}
		      Zobaczmy, że dla dowolnej pary \(\pars{b_0, c_0} \in B \times C\) zachodzi
		      \begin{align*}
			      \pars{\alpha\pars{f_0}}\pars{b_0, c_0}
			       & = \pars{f_0\pars{c_0}}\pars{b_0} &  & \text{z~definicji \(\alpha\)} \\
			       & = g\pars{b_0, c_0}               &  & \text{z~definicji \(f_0\)}
		      \end{align*}
		      Zatem istotnie \(\alpha\pars{f_0} = g\) jako funkcje.
	\end{description}
\end{proof}