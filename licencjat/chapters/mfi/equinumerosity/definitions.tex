\begin{definition}[Równoliczność i~podobne]
	Mówimy, że zbiory \(A\) i~\(B\) są \textbf{równoliczne}, gdy istnieje bijekcja \(f\colon A \function B\). Oznaczamy to jako \(A \eqnum B\).

	Gdy istnieje injekcja \(f\colon A \function B\), to oznaczamy \(A \leqnum B\).

	Gdy istnieje injekcja i~nieprawda, że \(A \eqnum B\), to oznaczamy \(A \ltnum B\).
\end{definition}
Zauważmy, że relacja \({\eqnum}\) jest relacją równoważności, ponieważ jest
\begin{itemize}
	\item zwrotna --- każdy zbiór jest w~bijekcji z~samym sobą (identyczność)
	\item symetryczna --- bijekcję da się odwrócić
	\item przechodnia --- złożenie bijekcji jest bijekcją.
\end{itemize}
