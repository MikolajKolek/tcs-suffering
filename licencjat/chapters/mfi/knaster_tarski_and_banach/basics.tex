Aby zrozumieć, o~czym w~ogóle mówią twierdzenia z~tego pytania, musimy zdefiniować kilka pojęć.
\begin{definition}[Obraz i~przeciwobraz]
	Niech \(f\colon X \function Y\) oraz \(A \subseteq X\) i~\(B \subseteq Y\).

	\textbf{Obrazem} zbioru \(A\)~względem \(f\)~nazywamy
	\begin{equation*}
		\fIm{f}\pars{A} \coloneqq \set{y \in Y : \exists_{x \in A} f\pars{x} = y}
	\end{equation*}
	czyli po prostu zbiór wartości, które \(f\)~przyjmuje na argumentach pochodzących z~\(A\).

	\textbf{Przeciwobrazem} zbioru \(B\)~względem \(f\)~nazywamy
	\begin{equation*}
		\fInvIm{f}\pars{B} \coloneqq \set{x \in X : \exists_{y \in B} f\pars{x} = y}
	\end{equation*}
	czyli argumenty, które \(f\)~przenosi na wartości pochodzące z~\(B\).

	Możemy traktować je jak funkcje
	\begin{align*}
		\fIm{f}    & \colon \powerset\pars{X} \function \powerset\pars{Y} \\
		\fInvIm{f} & \colon \powerset\pars{Y} \function \powerset\pars{X}
	\end{align*}

	Jeśli chcemy być bardzo fancy i~niezrozumiali, to przeciwobraz singletonu nazywamy \textbf{włóknem} (lub \textbf{poziomicą} lub \textbf{warstwicą}, jak podaje \href{https://pl.wikipedia.org/wiki/Przeciwobraz}{Wikipedia}).
\end{definition}

\begin{definition}[Monotoniczność]
	Mówimy, że \(f\colon \powerset\pars{X} \function \powerset\pars{Y}\) jest \textbf{monotoniczna ze względu na inkluzję, gdy}
	\begin{equation*}
		x \subseteq y \implies f\pars{x} \subseteq f\pars{y}
	\end{equation*}
	,,Większy'' argument przechodzi na ,,większą'' wartość.
\end{definition}
Zauważmy, że funkcje obrazu i~przeciwobrazu są monotoniczne.

\begin{definition}[Punkt stały]
	Jeśli dla funkcji \(f\colon \powerset\pars{X} \function \powerset\pars{X}\) i~zbioru \(x_0 \subseteq X\) zachodzi
	\begin{equation*}
		f\pars{x_0} = x_0
	\end{equation*}
	to \(x_0\) nazywamy \textbf{punktem stałym} funkcji \(f\).

	Jest \textbf{najmniejszym} punktem stałym, gdy
	\begin{equation*}
		f\pars{y} = y \implies x_0 \subseteq y
	\end{equation*}

	Jest \textbf{największym} punktem stałym, gdy
	\begin{equation*}
		f\pars{y} = y \implies y \subseteq x_0
	\end{equation*}
\end{definition}
