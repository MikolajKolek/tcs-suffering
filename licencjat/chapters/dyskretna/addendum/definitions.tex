\line(1,0){330} \\

\section*{Wykaz definicji}

\vspace{2pt}

\(R^{(k)}(l_1,l_2,l_3,\dots,l_s;s)\) -- ogólna liczba Ramseya; patrz sekcja \ref{ramsey}

\textit{Kolorowanie wierzchołkowe grafu} -- funkcja \(c: V \rightarrow [n]\) przypisująca każdemu wierzchołkowi grafu \(G = (V, E)\) jakąś liczbę naturalną w taki sposób, że dla każdej pary \(v_1, v_2 \in V\) mamy \(  c(v_1) = c(v_2) \implies (v_1, v_2) \not\in E\); o liczbie \(n\) mówimy wówczas, że jest liczbą kolorów użytą przez kolorowanie. Po ludzku: kolorujemy wierzchołki grafu tak, by wierzchołki o tym samym kolorze nie były połączone.

\textit{Kolorowanie krawędziowe grafu} -- funkcja \(c: E \rightarrow [n]\) przypisująca każdej krawędzi grafu \(G = (V,E\) jakąs liczbę naturalną w ten sposób, że dla każdej pary \(e_1 = (v_a, v_b), e_2 = (v_c, v_d) \in E\) mamy \(c(e_1) = c(e_2) \implies \set{v_a,v_b} \cap \set{v_c, v_d} = \varnothing \); o liczbie \(n\) mówimy wówczas, że jest liczbą kolorów użytą przez kolorowanie. Po ludzku: kolorujemy krawędzie grafu tak, by krawędzie o tym samym kolorze nie wychodziły z jednego wierzchołka.

\( \chi(G) \) -- liczba chromatyczna grafu \(G\); najmniejsza liczba kolorów potrzebna do wykonania poprawnego kolorowania wierzchołkowego

\(\omega(G)\) -- liczba klikowa grafu \(G\); rozmiar największej kliki w grafie \(G\)

\(\mathrm{col}(G)\) -- liczba kolorująca grafu \(G\); patrz sekcja \ref{colouring_number}

\(\delta(G)\) -- stopień najmniejszego wierzchołka w grafie \(G\)

\(\Delta(G)\) -- stopień największego wierzchołka w grafie \(G\)

\(\mathrm{N}(A)\) -- zbiór wszystkich sąsiadów danego podzbioru \( A \subseteq V\) w jakimś grafie \(G\)

\textit{Przepływ całkowitoliczbowy}, etc. -- patrz sekcja \ref{flows}

\vspace{2pt}

\line(1,0){330}
