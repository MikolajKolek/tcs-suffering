\subsection{Definicja}
\begin{definition}
	Grupą nazywamy tuplę \((G, \cdot)\), gdzie:

	\begin{enumerate}
		\item \(G\) to zbiór elementów grupy;
		\item \(\cdot\) to funkcja z \(G \times G\) w \(G\), spełniająca następujące warunki:
		      \begin{itemize}
			      \item \textbf{Łączność} -- dla każdych trzech elementów \(a, b, c \in G\) jest spełnione: \(a\cdot(b\cdot c) = (a\cdot b)\cdot c\)
			      \item \textbf{Element neutralny} -- istnieje element \(e \in G\), taki, że dla każdego elementu \(a \in G\) prawdą jest \(e\cdot a = a\cdot e = a\)
			      \item \textbf{Element odwrotny} -- dla każdego elementu \(a\) istnieje element \(a^{-1}\), taki, że \(a \cdot a^{-1} = a^{-1} \cdot a = e\)
		      \end{itemize}
	\end{enumerate}

\end{definition}

\begin{definition}
	Mówimy, że grupa jest \textbf{abelowa}, jeśli dla dowolnych dwóch elementów \(x, y \in G\) jest tak, że:

	\[
		x \cdot y = y \cdot x
	\]
\end{definition}

\subsection{Przykłady}
\begin{lemma}
	Liczby całkowite ze składaniem \((\integer, +)\) są grupą.
\end{lemma}
\begin{proof}
	Łączność jest oczywista. Element neutralny to \(0\), a żeby otrzymać element odwrotny dla \(a\), wystarczy wziąć \(-a\).
\end{proof}

\begin{lemma}
	Dla każdego \(n \in \natural_{>0}\) wszystkie permutacji z operacją kompozycji \((S_n, \circ)\) są grupą.
\end{lemma}
\begin{proof}
	Element neutralny to permutacja w której dla każdego \(i\) na miejscu \(i\) stoi liczba \(i\). Dla pokazania łączności wystarczy przy składaniu permutacji rozpatrzyć każdy element osobno.

	Żeby uzyskać element odwrotny z danej permutacji \(\sigma\), wystarczy zrobić taką permutacje \[
		\sigma^{-1}(i) = j\text{, gdzie } \sigma(j = i)
	\],
	skoro każda permutacja jest bijekcją, to to się uda.
\end{proof}

\subsection{Zastosowania}

Definicja grupy obejmuje wiele znanych struktur matematycznych, więc dowodząc jakieś twierdzenie dla grup, tak naprawdę dostajemy dużo wyników. Na przykład grupy są często wykorzystane w teorii liczb.

Grupy są również użyteczne do definiowania innych obiektów matematycznych, np. ciał.