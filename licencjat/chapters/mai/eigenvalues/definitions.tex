\begin{definition}[Wektor własny macierzy]
	Niech \(\field = (\fieldset, +, \cdot)\) będzie ciałem, a \(M \in \fieldset^{n \times n}\) będzie macierzą nad tym ciałem (gdzie \(n \in \natural\)).

	Wówczas wektor \(v \in \fieldset^n\) nazwiemy \textbf{wektorem własnym} macierzy \(M\), jeżeli istnieje takie \(\lambda \in \fieldset\) że:

	\[
		M \cdot v = \lambda \cdot v
	\]

	W szczególności, wektor zerowy jest wektorem własnym dowolnej macierzy.
\end{definition}
\begin{definition}[Wartość własna macierzy]
	Niech \(\field = (\fieldset, +, \cdot)\) będzie ciałem, a \(M \in \fieldset^{n \times n}\) będzie macierzą nad tym ciałem (gdzie \(n \in \natural\)).

	Wówczas skalar \( \lambda \in \fieldset\) nazwiemy \textbf{wartością własną} macierzy \(M\) jeśli istnieje taki (niezerowy\footnote{Inaczej ta definicja byłaby niepoważna}) wektor \(v \in \fieldset^n\) że:

	\[
		M \cdot v = \lambda \cdot v
	\]
\end{definition}
