\begin{theorem}
	\(\complement L_{HALT} \not \in \re\)
\end{theorem}

Powyższe twierdzenie wynika wprost z poniższego lematu:

\begin{lemma}
	Jeśli \( L \in RE \setminus R \), to \( \complement{L} \not \in RE\)
\end{lemma}
\begin{proof}
	Załóżmy, że \( L \in RE \setminus R \) i \( \complement{L} \in RE\).
	Mamy więc maszynę \( M \) rozpoznającą \( L \) oraz \( N \) rozpoznającą \( \complement{L} \)

	Konstruujemy DMT \( M' \) która:
	\begin{enumerate}
		\item Wczytuje wejście \( w \)
		\item Aż do akceptacji:
		      \begin{enumerate}
			      \item wykonaj jeden krok symulacji \( M \) na \( w \)
			      \item wykonaj jeden krok symulacji \( N \) na \( w \)
		      \end{enumerate}
		\item Jeśli \( M \) zaakceptowało to wypisz ,,TAK''
		\item Jeśli \( N \) zaakceptowało to wypisz ,,NIE''
	\end{enumerate}

	Oczywiście \( w \in L \lor w \in \complement{L} \) więc albo \( M \) zaakceptuje \( w \) albo zrobi to \( N \) -- któraś w końcu musi.
	Stanie się to po skończonej liczbie kroków niezależnie od \( w \) zatem \( L(M') = M \) oraz \( M' \) ma własność stopu.

	Trochę przypał bo w takim razie \( L \in (RE \setminus R) \cap R \) czyli nie istnieje.

\end{proof}