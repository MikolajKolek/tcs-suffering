Niech \( M = (Q, \Sigma, \Gamma, \delta, q_0, \blank, F) \) będzie (deterministyczną) maszyną którą chcemy symulować. Pokażemy jak zakodować \( M \) nad alfabetem \( \Sigma_U = \set{0, 1} \).

\begin{itemize}
	\item Niech \( Q = \set{q_1, \dots, q_n} \)

	      Kodujemy \( q_i \) jako ciąg \( i \) jedynek.

	\item \( \blank \) będziemy kodować jako pojedynczą jedynkę.


	\item Podobnie niech \( \Sigma = \set{a_1, \dots, a_n} \)

	      Kodujemy \( a_i \) jako ciąg \( i + 1 \) jedynek.

	\item Dla \( \Gamma = \set{z_1, \dots z_n} \)

	      Kodujemy \( z_i \) jako ciąg \( i + \card{\Sigma + 1} \) jedynek -- dla odróżnienia od \( \Sigma \).

	\item \( \delta \) jest funkcją, czyli zbiorem par \( ((q_1, z_1), (q_2, z_2, \pm 1) \)

	      Parę \( ((q_i, z_j), (q_k, z_l), 1) \) zakodujemy jako

	      \[
		      \underbrace{1 \dots 1}_i
		      0
		      \underbrace{1 \dots 1}_j
		      0
		      \underbrace{1 \dots 1}_k
		      0
		      \underbrace{1 \dots 1}_l
		      0
		      1
	      \]

	      Natomiast parę \( ((q_i, z_j), (q_k, z_l), -1) \) zakodujemy jako

	      \[
		      \underbrace{1 \dots 1}_i
		      0
		      \underbrace{1 \dots 1}_j
		      0
		      \underbrace{1 \dots 1}_k
		      0
		      \underbrace{1 \dots 1}_l
		      0
		      11
	      \]

	      Całą deltę kodujemy kodując wszystkie jej pary (w dowolnej kolejności) i oddzielając je dwoma zerami.

	\item \( F = \set{q_{i_1}, \dots q_{i_n}} \) kodujemy jako
	      \[
		      \underbrace{1 \dots 1}_{i_1} 0 \dots 0 \underbrace{1 \dots 1}_{i_n}
	      \]
\end{itemize}

Całą maszynę \( M \) zapisujemy wpisując \( \delta, q_0, F \) oddzielając je ciągiem \( 000 \).

Słowo \( w = a_{i_1} \dots a_{i_n} \) kodujemy jako
\[
	\underbrace{1 \dots 1}_{i_1} 0 \dots 0 \underbrace{1 \dots 1}_{i_n}
\]

Teraz musimy jeszcze powiedzieć jak kodujemy konfigurację maszyny \( M \) a robimy to dość prosto --
Konfigurację \( X_1 \dots X_k q_i X_{k+1} \dots X_n \) kodujemy jako
\[
	0\underbrace{1 \dots 1}_{\text{kod } X_1} 0 \underbrace{1 \dots 1}_{\text{kod } X_k} 00 \underbrace{1 \dots 1}_{\text{kod } q_i} 00 \underbrace{1 \dots 1}_{\text{kod } X_{k + 1}} 0 \underbrace{1 \dots 1}_{\text{kod } X_n} 0
\]

Możemy założyć, że kodowanie \( M \) jest zapisane na taśmie wejściowej po której nigdy nie piszemy, zaś konfigurację \( M \) będziemy trzymać na osobnej taśmie.

Krok symulacji przebiega teraz następująco:
\begin{enumerate}
	\item Zlokalizuj głowicę \( M \) (szukamy \( 00 \))
	\item Znajdź w kodowaniu \( \delta \) wpis odpowiadający przejściu na \( q_i X_k \)
	\item Wykonaj zadane przejście -- możemy wpisać nową konfigurację na osobną, tymczasową taśmę i przepisać ją z powrotem na taśmę na której operujemy.

\end{enumerate}
