\begin{theorem}
	Języki regularne są zamknięte na przecięcie.
\end{theorem}
\begin{proof}
	Mając DFA \(A_1 = (Q_1, \Sigma, \delta_1, s_1, F_1)\) i DFA \(A_2 = (Q_2, \Sigma, \delta_2, s_2, F_2)\) możemy skonstruować DFA \(B\), które będzie rozpoznawać język \(L(A_1) \cap L(A_2)\), takie że:

	\[
		B = (Q', \Sigma, \delta', q', F')
	\]

	gdzie:

	\begin{itemize}
		\item \( Q' = Q_1 \times Q_2 \)
		\item \( \delta'(q_1, q_2) = (\delta_1(q_1), \delta_2(q_2)) \)
		\item \( s' = (s_1, s_2) \)
		\item \( F' = \set{(q_1, q_2) : q_1 \in F_1 \land q_2 \in F_2} \)

	\end{itemize}

	Intuicyjnie: DFA \(B\) symuluje chodzenie po \(A_1\) i \(A_2\) akceptując wtedy i tylko wtedy, gdy w obu DFA dane słowo znalazłoby się w stanie akceptującym. Formalniej:

	\[
		w \in L(A_1) \cap L(A_2) \iff w \in L(A_1) \land w \in L(A_2) \iff
	\]
	\[
		\tilde \delta_1(s_1, w) \in F_1 \land \tilde \delta_2(s_2, w) \in F_2 \iff \delta'((s_1, s_2), w) \in F' \iff w \in L(B)
	\]

\end{proof}