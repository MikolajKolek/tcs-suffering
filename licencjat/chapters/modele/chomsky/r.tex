\textbf{Uwaga.} Przynajmniej według Wikipedii, języki rekurencyjne nie wchodzą do hierarchii Chomsky'ego. Języki rekurencyjnie przeliczalne już za to tak. Proszę na to uważać, żeby ktoś się nie wykopał na egzaminie.

\begin{definition}
	Mówimy, że MT \( M \) ma \textbf{własność stopu} jeśli po skończonej liczbie kroków trafia do jednego z wyróżnionych stanów \( q_{acc}, q_{rej} \) z których jedyne przejścia prowadzą z powrotem do nich samych.
\end{definition}

Powyższa definicja działa zarówno w przypadku deterministycznym (w którym mamy jedną ścieżkę obliczeń) jak i niedeterministycznym (wtedy chcemy żeby każda ścieżka obliczeń miała taką własność).
Niestety formalnie Maszyna Turinga się nie zatrzymuje, ale przyjmujemy że jak już wpadniemy w stan \( q_{acc} \) to akceptujemy słowo i się zatrzymujemy a gdy wpadniemy w stan \( q_{rej} \) to odrzucamy słowo i się zatrzymujemy.

\begin{definition}
	Mówimy że język \( L \) jest \textbf{rekurencyjny} albo \textbf{rozstrzygalny} jeśli istnieje Maszyna Turinga z własnością stopu \( M \) taka że \( L(M) = L \).

	Zbiór języków rekurencyjnych oznaczamy przez \( \r \).
\end{definition}