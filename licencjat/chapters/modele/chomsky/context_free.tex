\begin{definition}
	\textbf{Gramatyka bezkontekstowa} (Context-Free Grammar) to czwórka \( G = (N, \Sigma, P, S) \) gdzie
	\begin{itemize}
		\item \( N \) to skończony zbiór zmiennych (nieterminale)
		\item \( \Sigma \) - alfabet (terminale)
		\item \( P \) - produkcje \( P \subseteq N \times (N \cup \Sigma)^* \)
		\item \( S \in N \) - symbol startowy
	\end{itemize}
\end{definition}

\begin{definition}
	\textbf{Forma zdaniowa} to dowolne słowo nad \( (N \cup \Sigma)^* \)
\end{definition}

Intuicyjnie -- gramatyka bezkontekstowa definiuje zasady na podstawie których możemy tworzyć nowe słowa nad alfabetem \( \Sigma \).
Nieterminale \( N \) są takimi stanami pośrednimi, które możemy podmieniać używając produkcji z \( P \).
Każda produkcja jest postaci
\[
	\text{nieterminal} \rightarrow \text{forma zdaniowa}
\]


\begin{definition}
	Dla gramatyki \( G \) definiujemy relację \( \rightarrow_G \) na formach zdaniowych.
	Mówimy, że
	\[
		\alpha \rightarrow_G \beta
	\]
	wtedy i tylko wtedy gdy:
	\[
		\exists_{\alpha_1, \alpha_2, \gamma \in (N \cup \Sigma)^*} : \exists_{A \in N} : (A, \gamma) \in P \land
		\alpha = \alpha_1  A \alpha_2 \land \beta = \alpha_1 \gamma \alpha_2
	\]
\end{definition}

Bardziej po ludzku -- formę zdaniową \( \beta \) możemy uzyskać (w jednym kroku) z formy zdaniowej \( \alpha \) o ile w  \( \alpha \) jest jakieś wystąpienie nieterminala \( A \), które możemy zgodnie z produkcjami podmienić na formę zdaniową \( \gamma \) aby uzyskać \( \beta \).

\begin{definition}
	\( \rightarrow_G^* \) to zwrotne i przechodnie domknięcie \( \rightarrow_G \)
\end{definition}

Z \( \alpha \) możemy uzyskać \( \beta \) w dowolnej liczbie kroków, jeśli istnieje ciąg przekształceń \( \alpha \rightarrow_G \alpha_1 \rightarrow_G \dots \rightarrow_G \gamma \)


\begin{definition}
	\textbf{Język  generowany} przez gramatykę G to
	\[
		L(G) = \set{w \in \Sigma^* \mid S \rightarrow_G^* w}
	\]
	podobnie dla \( A \in N \)
	\[
		L(G, A) = \set{w \in \Sigma^* \mid A \rightarrow_G^* w}
	\]
\end{definition}

\begin{definition}
	Język jest bezkontekstowy (Context-Free Language) jeśli jest generowany przez jakąś gramatykę bezkontekstową.
\end{definition}