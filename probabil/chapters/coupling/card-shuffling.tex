\begin{example}
    Powrócimy na moment do poprzedniego przykładu z tasowaniem kart. Ustalmy \(\pi\) jako rozkład stacjonarny oraz \(x\) jako pewną ustaloną permutację kart.
    Będziemy tasować, aż obecny stan ma rozkład \(D\) taki, że \(\left\|D-\pi\right\|_{TV} > \epsilon\) dla pewnego \(\epsilon > 0\).
    \begin{align*}
        \epsilon > \left\|D-\pi\right\|_{TV} = \max_{A \subseteq S}\left|D(A) - \pi(A)\right|
        \ge \left|D(x) - \frac{1}{n!}\right| \\
        \frac{1}{n!} - \epsilon < D(x) < \frac{1}{n!} + \epsilon
    \end{align*}
\end{example}

\begin{example}
    Dla kontrastu, gdy będziemy oszukiwać przy tasowaniu zostawiając asa pik na wierzchu talii.
    Ustalmy \(B\) jako zbiór wszystkich permutacji z asem pik na wierzchu.
    \begin{align*}
        \epsilon > \left\|D-\pi\right\|_{TV} &= \max_{A \subseteq S}\left|D(A) - \pi(A)\right|\\
        \ge \left|D(B) - \pi(B)\right| &= \left|1 - \frac{1}{52}\right| = \frac{51}{52}
    \end{align*}
\end{example}