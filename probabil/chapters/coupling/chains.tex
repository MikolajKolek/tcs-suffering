Będziemy rozważać łańcuch Markowa \(\left( X_t \right) _{t \in \N}\) (skończony, nieprzywiedlny, nieokresowy) o macierzy przejścia \(P\), zbiorze stanów \(S\) i rozkładzie stacjonarnym \(\left( \pi_x \right) _{x \in S}\). Przez \(P^{t}\left( x, \cdot \right) \) oznaczamy rozkład \(X_t\) przy założeniu \(X_0 = x \).

\begin{definition}
	Definiujemy \[ \Delta_x\left( t  \right) = \left\|P^{t}\left( x, \cdot  \right) - \pi\right\|_{TV}, \]
	\[ \tau_x\left( \varepsilon  \right) = \min \left\{ t : \Delta_x\left( t  \right) \le \varepsilon  \right\} . \]
	Mamy też maksima tych wartości:
	\[ \Delta\left( t  \right) = \max_{x \in S} \Delta_x\left( t  \right) , \]
	\[ \tau_{\mathrm{mix}}\left( \varepsilon  \right) = \max_{x \in S} \tau_x\left( \varepsilon  \right) . \]
	Ostatnią z tych wartości nazywamy czasem mieszania łańcucha Markowa. Będziemy też (bez większego powodu) oznaczać \(\tau_{\mathrm{mix}} = \tau_{\mathrm{mix}}\left( \frac{1}{4} \right) \).
\end{definition}

\begin{definition}
	Sprzęganiem łańcuchów Markowa \(X,Y\) o macierzy przejścia \(P\) i zbiorze stanów \(S\) jest dowolny łańcuch Markowa \(\left( Z_t = \left( X_t, Y_t \right)  \right) _{t \in \N}\) na przestrzeni stanów \(S \times S\) taki, że
	\[ P\left( X_{t+1} = x' \mid Z_t = \left( x,y \right)  \right) = P\left( x,x' \right)  \]
	\[ P\left( Y_{t+1} = y' \mid Z_t = \left( x,y \right)  \right) = P\left( y, y' \right)  \]
	dla każdego \(t\ge 0, x,y,x',y' \in S\).

	Sprzęgane łańcuchy to dwie równoległe kopie jednego procesu. Nie zawsze mają one te same stany, ale też nie zawsze są niezależne. Nie ustalamy nic o stanach początkowych. Będą nas interesować takie sprzęgania, które sprowadzają obie kopie do tego samego stanu i potem je tak utrzymują.
\end{definition}

\begin{lemma}
	Niech \(\left( \left( X_t,Y_t \right)  \right) _{t \in \N}\) będzie sprzęganiem łańcuchów (skończonych, nieprzywiedlnych, nieokresowych) z macierzą przejścia \(P\) i zbiorem stanów \(S\). Niech \(T \in \N\) i \(\varepsilon > 0\) będą takie, że dla każdego \(x,y \in S\) zachodzi
	\[ P\left( X_T \neq Y_T \mid X_0 = x, Y_0 = y  \right) \le \varepsilon. \]
	Wtedy czas mieszania łańcucha z macierzą \(P\) jest ograniczony:
	\[ \forall_{x \in S} \   \Delta_x\left( T \right) \le \varepsilon \]
	\[ \tau_{\mathrm{mix}}\left( \varepsilon  \right) \le T. \]
\end{lemma}
\begin{proof}
	Zauważmy, że sprzęganie spełnia założenia niezależnie od tego, w jaki sposób ustalimy \(X_0\) i \(Y_0\). Ustalmy dowolne \(x \in S\). Niech \(X_0 = x \) i niech \(Y_0\) będzie wybrany losowo z rozkładu stacjonarnego \(\pi\). Wtedy \(Y_t\) ma rozkład \(\pi\) dla każdego \(t\).

	Niech \(A \subseteq S\). Mamy
	\begin{align*}
		P\left( X_T \in A  \right) & \ge P\left( Y_T \in A \cap X_T = Y_T \right) = 1- P\left( Y_T \notin A \cup X_T \neq Y_T \right)                                                      \\
		                           & \ge 1-P\left( Y_T \notin A  \right) - P\left( X_T \neq Y_T \right) \ge P\left( Y_T \in A  \right) - \varepsilon = \pi\left( A  \right) - \varepsilon.
	\end{align*}
	Analogicznie \(P\left( X_T \in A^{c} \right) \ge \pi\left( A^{c} \right) - \varepsilon \), czyli \(P\left( X_T \in A  \right) \le \pi\left( A  \right) + \varepsilon \).

	Mamy zatem
	\[ \forall_{x \in S} \ \Delta_x\left( T \right) =  \max_{A \subseteq S} \left|P^{T}\left( x,A \right) - \pi\left( A  \right) \right|\le \varepsilon,\]
	a z tego wynika
	\[ \tau_{\mathrm{mix}}\left( \varepsilon  \right) \le T. \]
\end{proof}

\begin{lemma}[O monotoniczności]
	Niech \(P\) będzie macierzą przejścia skończonego, nieprzywiedlnego i nieokresowego łańcucha Markowa ze zbiorem stanów \(S\) i rozkładem stacjonarnym \(\pi\). Dla każdych \(t\ge 0, x \in S\) zachodzi
	\[ \Delta_x\left( t+1 \right) \le \Delta_x\left( t  \right) . \]
\end{lemma}
\begin{proof}
	Ustalmy \(t\ge 0\) i \(x \in S\). Niech \(\left( X_t,Y_t \right) \) będzie sprzęganiem rozkładów \(P^{t}\left( x,\cdot  \right) \) i \(\pi\) spełniającym \(P\left( X_t \neq Y_t \right) = \Delta_x\left( t  \right) \) (przedtem pokazaliśmy, że istnieje sprzęganie, dla którego ta równość zachodzi). Definiujemy \(\left( X_{t+1},Y_{t+1} \right) \) w następujący sposób: jeśli \(X_t = Y_t\), wykonujemy krok łańcucha zgodnie z macierzą \(P\) (na obu współrzędnych taki sam), a w przeciwnym wypadku wykonujemy dwa niezależne kroki. Zauważmy, że \(Y_{t+1}\) dalej ma rozkład \(\pi\). Mamy
	\[ \Delta_x\left( t  \right) = P\left( X_t \neq Y_t \right) \ge P\left( X_{t+1}\neq Y_{t+1} \right) \ge \left\|P^{t+1}\left( x,\cdot  \right) - \pi \right\|_{TV} = \Delta_x\left( t+1 \right) . \]
\end{proof}

\begin{theorem}[O geometrycznej zbieżności]
	Niech \(P\) będzie macierzą przejścia skończonego, nieprzywiedlnego i nieokresowego łańcucha Markowa ze zbiorem stanów \(S\) i rozkładem stacjonarnym \(\pi\). Wtedy istnieją \(\alpha \in (0,1)\) i \(C > 0\) takie, że
	\[ \forall_{n \in \N} \ \Delta\left( n  \right) \le C \alpha^{n}.  \]
\end{theorem}
\begin{proof}
	Ustalmy \(r\ge 1\) takie, że dla każdych \(x,y \in S\) jest \(P^{r}\left( x,y \right) > 0\) (dla konkretnych dwóch istnieje, bo łańcuch jest nieokresowy i nieprzywiedlny, a ze skończoności można wziąć maksimum).

	Niech \(m_y = \min_{x \in S} P^{r}\left( x,y \right)\) dla \(y \in S\). Jest to najmniejsze z prawdopodobieństw, z jakimi da się przejść do \(y\) krokiem macierzy \(P^{r}\). Niech \(m = \sum_{y \in S}^{} m_y \le 1\) (ta suma ogranicza z dołu dowolny wiersz, a wiersz sumuje się do \(1\)).

	Niech \(\left( \left( X_t,Y_t \right)  \right) _{t \in  \N}\) będzie sprzęganiem łańcuchów o macierzy przejścia \(P^{r}\) zadanym w następujący sposób: mając zadane \(X_{t} = x, Y_{t} = y \) (początkowe wartości wybieramy dowolnie) oznaczamy \(\mu = P^{r}\left( x, \cdot  \right) , \nu = P^{r}\left( y,\cdot  \right) \) i \(B = \left\{ x \in S : \mu\left( x  \right) \ge \nu\left( x \right)  \right\} \). Niech \(p_1 = \mu\left( B \right) - \nu\left( B  \right) \), \(p_2 = \nu\left( B^{c} \right) - \mu\left( B^{c} \right) \) i \(p_3 = 1-p_1 = 1-p_2\).

	Rzucamy monetą z prawdopodobieństwem orła \(p_3\). Jeśli wypadnie orzeł, to ustalamy \(X_{t+1} = Y_{t+1} = s \), gdzie \(s\) wybieramy z \(S\) z rozkładem \(\left( \frac{1}{p_3} \min\left( u \left( s  \right) , \nu\left( s  \right)  \right) : s \in S \right) \). Jeśli wypadnie reszka ustalamy \(X_{t+1} = x \) i \(Y_{t+1}=y\), gdzie \(x\) wybieramy losowo z \(S \) z rozkładem \(\left( \frac{1}{p_1} \max\left( \mu\left( x  \right) -\nu\left( x  \right) , 0 \right) : x \in S \right) \), a \(y\) z rozkładem \(\left( \frac{1}{p_2} \max\left( \nu\left( x  \right) -\mu\left( x  \right) , 0 \right) : x \in S \right) \).

	W ten sposób skonstruowaliśmy sprzęganie, dla którego zachodzi
	\[ \forall_{t \ge 0, y \in S} \ P\left( X_{t+1}= Y_{t+1}= y  \right) \ge p_3 \cdot \frac{1}{p_3} \min\left( \mu\left( y  \right) , \nu\left( y  \right)  \right) \ge m_y.  \]
	Mamy więc \[\forall_{t\ge 1} \ P\left( X_t = Y_t \right) = \sum_{y \in S}^{} P\left( X_t = Y_t = y  \right) \ge \sum_{y \in S}^{} m_y = m, \]
	a z tego wynika \(P\left( X_t \neq Y_t \right) \le \left( 1-m \right) ^{t}.\) Teraz mając zadane \(n = rt + j\) dla \(j \in \left\{ 0,\ldots,r-1 \right\} \) możemy zapisać
	\[ \Delta_x\left( n  \right) \le \Delta_x\left( rt \right) = \left\|P^{rt}\left( x,\cdot  \right) -\pi\right\|_{TV} \le P\left( X_t \neq Y_t \right) \le \left( 1-m \right) ^{t} = \alpha ^{rt} \le C\alpha ^{n}, \]
	gdzie położyliśmy \(\alpha = \left( 1-m \right) ^{\frac{1}{r}}\) i \(C = \alpha ^{-r}\).
\end{proof}


