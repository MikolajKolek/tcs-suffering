\begin{definition}
	\textbf{Funkcją ekstrakcji} nazywamy taką funkcję \( \Ext \), która zwraca bity w oparciu o wartości zwracane przez zmienną \( X \) w taki sposób, że dla dowolnego ciągu binarnego \( y \) zachodzi
	\[
		P(\abs{\Ext(X)} = k) \neq 0 \implies P(\Ext(X) = y \mid \abs{y} = k) = 2^{-k}
	\]
\end{definition}
Innymi słowy -- jeśli możemy dostać \textit{jakiś} ciąg \( k \)-bitowy to możemy dostać \textit{każdy} ciąg \(k\)-bitowy, wszystkie z równym prawdopodobieństwem.
Dalece nie jest oczywiste, że taka funkcja ekstrakcji w ogóle istnieje (być może nasza zmienna zachowuje się w brzydki sposób).

Czasem jednak jest dobrze.
\begin{theorem}[Twierdzenie 10.4 P\&C]
	Niech \( X \) ma rozkład jednostajny na zbiorze \( \set{0, \dots, m - 1} \).
	Wtedy istnieje funkcja ekstrakcji \( \Ext \), która średnio zwraca co najmniej
	\( \floor{H(X)} - 1  = \floor{\lg m} - 1\) bitów.
\end{theorem}
\begin{proof}
	Naszą funkcję konstruujemy rekurencyjnie. Niech \( \Ext_m \) będzie funkcją ekstrakcji dla ustalonego \( m \).

	Jeśli \( m = 1 \) to zwracamy pusty ciąg \( \Ext_1(0) = \varnothing \).
	Rozważmy zatem \( m \geq 2 \).

	Niech \( \alpha = \floor{\lg m} \). Definiujemy
	\[
		\Ext_m(x) = \begin{cases}
			x                                 & \text{ dla } x < 2^\alpha    \\
			\Ext_{m - 2^\alpha}(x - 2^\alpha) & \text{ dla } x \geq 2^\alpha
		\end{cases}
	\]
	Intuicyjnie działa to tak, że wybieramy największą potęgę dwójki, która mieści się w \( m \)
	i dla wszystkich liczb mniejszych od niej zwracamy po prostu ich reprezentację bitową. (
	wraz z zerami wiodącymi, aby miała ona długość \( \alpha \))
	Pozostałe wartości wypełniamy rekurencyjnie krótszymi ciągami.

	Pozostaje policzyć oczekiwaną liczbę zwróconych bitów -- nazwijmy ją \( \expected{X} \).
	Ponadto niech \( \beta = \floor{\lg \pars{m - 2^\alpha}} \)
	\begin{align*}
		\expected{Y}
		 & \geq \frac{2^\alpha}{m} \cdot \alpha
		+ \frac{m - 2^\alpha}{m} \cdot \pars{\floor{\lg\pars{m - 2^\alpha}} - 1} \\
		 & =\frac{m + 2^\alpha - m}{m} \cdot \alpha
		+ \frac{m - 2^\alpha}{m} \cdot \pars{\beta - 1}                          \\
		 & = \alpha + \frac{m - 2^\alpha}{m} \cdot \pars{-\alpha + \beta - 1}    \\
	\end{align*}
	Mamy ponadto oszacowanie
	\[
		\frac{m - 2^\alpha}{m} \leq \frac{2^{\beta + 1} - 1}{2^\alpha + 2^{\beta + 1} - 1}
	\]
	Zatem
	\begin{align*}
		\expected{Y}
		 & \geq \alpha - \frac{2^{\beta + 1} - 1}{2^\alpha + 2^{\beta + 1} - 1} \cdot (\alpha - \beta + 1)                          \\
		 & \geq \alpha - \frac{2^{\beta + 1} - 1} {(2^{\beta + 1} - 1)\cdot(2^{\alpha - \beta - 1} + 1)} \cdot (\alpha - \beta + 1) \\
		 & \geq \alpha - \frac{\alpha - \beta + 1}{2^{\alpha - \beta - 1} + 1}                                                      \\
		 & \geq \alpha - 1
	\end{align*}
\end{proof}

\begin{theorem}[Twierdzenie 10.5 P\&C]
	Niech \( p \in \pars{\frac{1}{2}, 1} \) będzie prawdopodobieństwem sukcesu pojedynczej próby oraz \( \delta > 0 \)

	Wtedy dla odpowiednio dużego \( n \) oraz ciągu \( n \) niezależnych prób:
	\begin{enumerate}
		\item Istnieje funkcja ekstrakcji, która średnio zwraca co najmniej \((1 - \delta) \cdot n \cdot H(p) \) niezależnych bitów.
		\item Średnia liczba bitów zwracana przez dowolną funkcję ekstrakcji to co najwyżej \( n \cdot H(p) \)
	\end{enumerate}
\end{theorem}
\begin{proof} \( \)
	\begin{enumerate}
		\item Zaczynamy od pierwszego podpunktu

		      Niech \( j \) będzie liczbą sukcesów. Mamy \( \binom{n}{j} \) możliwych ciągów, każdy wypada z takim samym prawdopodobieństwem.

		      Na podstawie poprzedniego twierdzenia możemy zdefiniować na takich ciągach funkcję ekstrakcji.

		      Niech \( Z \) będzie liczbą sukcesów które wypadły, a \( B \) liczbą bitów, którą zwracamy w opisany sposób. Wtedy
		      \[
			      \expected{B} = \sum_{k=0}^n P(Z = k) \cdot \expected{B \mid Z = k}
		      \]
		      Z poprzedniego twierdzenia mamy
		      \[
			      \expected{B \mid Z = k} \geq \floor{\lg \binom{n}{k}} - 1
		      \]

		      Będziemy chcieli skorzystać z oszacowania współczynników dwumianowych przez entropię.

		      Dobieramy zatem jakiegoś małego \( 0 < \varepsilon < \min\pars{p-\frac{1}{2}, 1-p} \)
		      i będziemy się zajmować \( k \in \brackets{n(p-\varepsilon), n(p+\varepsilon)} \)

		      Teraz mówimy, że
		      \[
			      \binom{n}{k} \geq \binom{n}{\floor{n(p+\varepsilon)}} \geq \frac{2^{nH(p + \varepsilon)}}{n+1}
		      \]

		      Teraz szacujemy
		      \begin{align*}
			      \expected{B}
			       & \geq \sum_{k=\floor{n(p-\varepsilon)}}^{\ceil{n(p+\varepsilon)}} P(Z = k) \cdot \expected{B \mid Z = k}                  \\
			       & \geq  \sum_{k=\floor{n(p-\varepsilon)}}^{\ceil{n(p+\varepsilon)}} P(Z = k) \cdot \pars{\floor{\lg \binom{n}{k}} - 1}     \\
			       & \geq \sum_{k=\floor{n(p-\varepsilon)}}^{\ceil{n(p+\varepsilon)}} P(Z = k) \cdot \pars{nH(p+\varepsilon) - \lg (n+1) - 2} \\
			       & \geq \pars{nH(p+\varepsilon) - \lg (n+1) - 2} \cdot P\pars{\abs{Z - np} \leq \varepsilon n}
		      \end{align*}
		      Drugi czynnik możemy ograniczyć nierównością Chernoffa na niezależne próby Poissona. Mamy:
		      \[
			      P\pars{\abs{Z - np} > \varepsilon n} \leq 2\cdot \exp\pars{\frac{-n\varepsilon^2}{3p}}
		      \]
		      Zatem
		      \begin{align*}
			      \expected{B}
			       & \geq \pars{nH(p+\varepsilon) - \lg (n+1) - 2}
			      \cdot \pars{1 -\exp\pars{\frac{-n\varepsilon^2}{3p}}} \\
			       & \geq (1 - \delta)n H(p)
		      \end{align*}

		\item Teraz drugi podpunkt, na szczęście prostszy.

		      Jeśli mamy jakąś funkcję ekstrakcji \( \Ext \) i wrzucimy do niej jakiś ciąg \( x \) taki, że
		      \( P(X = x) = q \) to \( \Ext \) może zwrócić co najwyżej \( \lg \frac{1}{q} \) bitów.

		      Dzieje się tak, ponieważ jeśli wyplujemy jakieś \( k \) bitów z prawdopodobieństwem \( q \) to każde \( k \) bitów musimy zwracać z takim prawdopodobieństwem.

		      Wychodzi nam z tego, że \( 2^{\abs{\Ext(x)}} \cdot q \leq 1 \).

		      Zatem
		      \begin{align*}
			      \expected{B}
			       & = \sum P(X = x) \cdot \abs{\Ext(x)}             \\
			       & \leq \sum P(X = x) \cdot \lg \frac{1}{P(X = x)} \\
			       & = H(X)                                          \\
			       & = n \cdot H(p)
		      \end{align*}
	\end{enumerate}
\end{proof}