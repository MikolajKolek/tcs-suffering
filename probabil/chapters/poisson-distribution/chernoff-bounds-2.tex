\begin{theorem}
	Niech \(X\) będzie zmienną o rozkładzie Poissona z parametrem \(\mu\). Wtedy:
	\begin{enumerate}
		\item jeśli \(\delta > 0\), to \(\prob \pars{X \ge \pars{1+\delta}\mu} \le \pars{\frac{e^{\delta}}{ \pars{1+\delta}^{1+\delta}}} ^{\mu}\)   \\
		\item jeśli \(1 > \delta > 0\), to \(\prob \pars{X \ge \pars{1+\delta}\mu} \le e^{\frac{-\mu\delta^2}{3}}\)                                 \\
		\item jeśli \(R \ge 6\mu\), to \(\prob \pars{X \ge R} \le 2^{-R}\)
	\end{enumerate}
\end{theorem}
Punkt \(1\) pokazaliśmy już w \ref{poisson-chernoff-bounds}. Udowodnimy jeszcze punkty \(2\) i \(3\).

\begin{proof}
    Chcemy wpierw pokazać, że \(\frac{e^\delta}{\pars{1+\delta}^{1+\delta}} \le e^{\frac{-\delta^2}{3}}\) dla \(\delta \in \pars{0, 1}\).
    Weźmiemy obustronnie logarytm i po odrobinie analizy wyjdzie:

    \begin{align*}
        f(\delta) &= \delta - (1+\delta)\ln{1+\delta} + \frac{\delta^2}{3} \le 0                         \\
        \frac{d}{d\delta}f(\delta) &= 1 - \ln{1+\delta} = \frac{1+\delta}{1+\delta} + \frac{2\delta}{3}  \\
        \frac{d^2}{d\delta^2}f(\delta) &= \frac{2}{3}-\frac{1}{1+\delta}                                 \\
    \end{align*}
    Zauważmy, że druga pochodna jest równa 0 tylko dla \(\delta = \frac{1}{2}\): dla mniejszych wartości jest ujemna, a dla większych dodatnia.
    Pierwsza pochodna jest w zerze równa 0 i w jedynce ujemna - wraz z poprzednim implikuje to, że dla \(\delta \in \pars{0, 1}\) mamy ujemną pochodną.
    Dodając do tego jeszcze \(f(0) = 0\) dostajemy zadaną na początku nierówność.
\end{proof}

\begin{proof}
    Do trzeciego punktu położymy \(R = (1+\delta)\mu\). I dalej:

    \begin{align*}
        \prob \pars{X \ge R} &= \prob \pars{X \ge (1+\delta)\mu}            \\
        &\le \pars{\frac{e^{\delta}}{\pars{1+\delta}^{1+\delta}}}^{\mu}     \\
        &\le \pars{\frac{e}{\pars{1+\delta}}}^{(1+\delta)\mu}               \\
        &\le \pars{\frac{e}{6}}^R \le \pars{\frac{1}{2}}^R = 2^{-R}
    \end{align*}
\end{proof}