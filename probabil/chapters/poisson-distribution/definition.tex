\begin{definition}
	Mówimy, że zmienna losowa \( X \) ma rozkład
	\textbf{Poissona} z parametrem \( \lambda \) jeśli
	\[
		P(X = n) = e^{-\lambda} \cdot \frac{\lambda^n}{n!}
	\]
\end{definition}

Aby upewnić się, że jest to poprawny rozkład policzmy \( \sum_{n = 0}^\infty P(X = n) \)

\begin{align*}
	\sum_{n = 0}^\infty P(X = n)
	 & = \sum_{n = 0}^\infty e^{-\lambda} \cdot \frac{\lambda^n}{n!} \\
	 & = e^{-\lambda} \sum_{n=0}^\infty \frac{\lambda^n}{n!}         \\
	 & = e^{-\lambda} \cdot  e^\lambda = 1
\end{align*}

\begin{theorem}
	Niech \( X \) ma rozkład Poissona z parametrem \( \lambda \). Wtedy
	\[
		\expected{X} = \lambda
	\]
\end{theorem}
\begin{proof}
	\begin{align*}
		\expected{X} & = \sum_{n = 0}^\infty n \cdot P(X = n)                                \\
		             & = \sum_{n = 0}^\infty n \cdot e^{-\lambda} \cdot \frac{\lambda^n}{n!} \\
		             & = e^{-\lambda} \sum_{n=1}^\infty \frac{\lambda^n}{(n-1)!}             \\
		             & = \lambda e^{-\lambda} \sum_{n=1}^\infty \frac{\lambda^{n-1}}{n!}     \\
		             & = \lambda e^{-\lambda} \sum_{n=0}^\infty \frac{\lambda^n}{n!}         \\
		             & = \lambda
	\end{align*}
\end{proof}

\begin{theorem}[Lemat 5.3 P\&C]
	Jeśli zmienna \( X \) ma rozkład Poissona z parametrem \( \lambda \) to
	\[
		M_X(t) = \exp\pars{\lambda\pars{e^t -  1}}
	\]
\end{theorem}
\begin{proof}
	\begin{align*}
		M_X(t)
		 & = \expected{e^{tX}}                                                      \\
		 & = \sum_{n=0}^\infty e^{tn} \cdot e^{-\lambda} \cdot \frac{\lambda^n}{n!} \\
		 & = e^{-\lambda} \sum_{n=0}^\infty \frac{\pars{\lambda e^t}^n}{n!}         \\
		 & = \exp(-\lambda) \cdot \exp\pars{\lambda e^t}                            \\
		 & = \exp\pars{\lambda(e^t - 1)}
	\end{align*}
	W przedostatnim przejściu korzystamy z faktu, że \( \sum_{0}^\infty \frac{x^n}{n!} = \exp(x) \)
\end{proof}

\begin{theorem}[Lemat 5.2 P\&C]
	Jeśli \( X \) ma rozkład Poissona z parametrem \( \lambda_X \) a \( Y \) rozkład Poissona z parametrem \( \lambda_Y \), a ponadto obie zmienne są niezależne to \( X + Y \) ma rozkład Poissona z parametrem \( \lambda_X + \lambda_Y\)
\end{theorem}
\begin{proof}
	Są dwie ścieżki aby to pokazać -- jedna polega na przeliczeniu explicite \( P(X + Y = n) \) -- zostawiamy to jako ćwiczenie dla Czytelnika.

	Pokażemy bardziej elegancki dowód korzystający z funkcji tworzących momentów.

	Ponieważ \( X \) i \( Y \) są niezależne to
	\begin{align*}
		M_{X + Y}(t)
		 & = M_X(t) \cdot M_Y(t)                                               \\
		 & = \exp\pars{\lambda_X(e^t - 1)} \cdot \exp\pars{\lambda_Y(e^t - 1)} \\
		 & = \exp\pars{(\lambda_X + \lambda_Y)(e^t - 1)}
	\end{align*}

	Skoro rozkład zmiennej \( X + Y \) tworzony jest przez funkcję, która wygląda jak rozkład Poissona, to musi być ona rozkładem Poissona z parametrem \( \lambda_X + \lambda_Y \)
\end{proof}
\begin{theorem}
	\label{poisson-variable-variance}
	Niech \( X \) ma rozkład Poissona z parametrem \( \lambda \). Wtedy
	\[
		\variance{X} = \lambda
	\]
\end{theorem}
\begin{proof}
	Liczymy drugą pochodną \( M_X(t) = \exp\pars{\lambda \pars{e^t - 1}} \) i wychodzi.
\end{proof}