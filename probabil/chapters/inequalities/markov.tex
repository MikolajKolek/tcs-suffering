%<*probabil-2025-10-10-markov>
\subsection{Definicja}
\begin{theorem}
	Jeśli \( X \) jest zmienną losową, która przyjmuje nieujemne wartości to
	\[
		\prob(X \geq a) \leq \frac{\expected{X}}{a}
	\]
\end{theorem}
\begin{proof}
	Niech \( I \) będzie indykatorem
	\[
		I = \begin{cases}
			1 & \text{ gdy } X \geq a \\
			0 & \text{ wpp. }
		\end{cases}
	\]
	Skoro \( X \geq 0 \) to \( I \leq \frac{X}{a} \).
	Zatem
	\[
		\prob(X \geq a) = P(I = 1) = \expected{I} \leq \frac{\expected{X}}{a}
	\]
\end{proof}

\subsection{Rzuty monetą}
Rzucamy \(n\) razy monetą - zliczamy liczbę orłów.

Jak \textit{bardzo} możemy ograniczyć prawdopodobieństwo otrzymania \textit{dużej} liczby orłów?
(Jest to troche nieformalne, nie wiemy co dokładnie znaczy \textit{bardzo} ani \textit{duża liczba orłów}, ale jakieś przykładowe ograniczenie możemy podać)

Niech \(X_i = \begin{cases}
	1, \text{ orzeł za i-tym razem} \\
	0 \text{ wpp}
\end{cases}\)

\[
	\ev{X} = \ev{\sum_i X_i} = \sum_i \ev{X_i} = \frac{n}{2}
\]

Chcemy ograniczyć prawdopodobieństwo tego, że w więcej niż \(\frac{3}{4}\) rzutów otrzymamy orła.

Z Markowa:
\[
	\prob(X \geq \frac{3}{4} \cdot n) \leq \frac{\frac{n}{2}}{\frac{3}{4} \cdot n} = \frac{2}{3}
\]

Intuicyjnie ta wartość powinna być dużo mniejsza (i coraz mniejsza dla większych \(n\)) ale Markow daje formalne ograniczenie (chociaż jak widać dość słabe).
%</probabil-2025-10-10-markov>

\subsection{Kolekcjoner kuponów}
Spróbujmy użyć nierówności Markowa do oszacowania jakoś czasu zebrania wszystkich kuponów.
Niech \( X \) będzie czasem zebrania wszystkich kuponów.
W sekcji \ref{coupon-collectors-problem} pokazaliśmy, że \( \expected{X} = nH_n = \Theta(n \ln n) \)

Możemy zatem skorzystać z nierówności Markowa aby otrzymać
\[
	P(X \geq 2n H_n) \leq \frac{\expected{X}}{2n H_n} = \frac{1}{2}
\]

Nie jest to jakieś szczególnie satysfakcjonujące oszacowanie -- prawdopodobieństwo, że musimy czekać dwa
razy dłużej niż tego oczekujemy może wynosić aż \( \frac{1}{2} \) :(
