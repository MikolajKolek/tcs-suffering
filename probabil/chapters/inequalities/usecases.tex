Pokażmy przykład zastosowania nierówności na zmiennych losowych poprzez wypełnienie tabelki pokazującej rezultaty jakie dają nam poszczególne nierówności.

Dla wprowadzenia, gdy mówimy o problemie kolekcjonera kuponów odwołujemy się do \ref{coupon-collectors-problem}. Za to gdy mówimy o rzutach monetą, to \(X_i\) to zmienna losowa dla której 
\[
	\prob(X_i = 0) = \prob(X_i = 1) = \frac{1}{2}
\]
oraz
\[
	X = \sum_{i=1}^n X_i, \ev{X} = \frac{n}{2}
\]

\begin{tabular}{p{6cm} p{5.5cm} p{5.5cm}}
\toprule
 & Rzuty monetą \newline Ograniczenie \(\prob(X \geq \frac{3}{4}n)\)  & Kolekcjoner kuponów \newline Ograniczenie \(\prob(X \geq 2 n H_n)\)\\
\midrule
Nierówność Markowa (\ref{markov-inequality}) & \(\frac{2}{3}\) (\ref{markov-coin-tosses}) & \(\frac{1}{2}\) (\ref{markov-coupon-collector}) \\

Nierówność Czebyszewa (\ref{chebyshev-inequality}) & \(\frac{4}{n}\) (\ref{chebyshev-coin-tosses}) & \(O(\frac{1}{\ln^2 n})\) (\ref{chebyshev-coupon-collector})\\

Nierówność Chernoffa (\ref{chernoff-inequality}) & \(e^{-\frac{n}{20}}\) (\ref{chernoff-coin-tosses})& \(e^{-n}\) \\

Union bound (\ref{union-bound}) & \(\) & \(\frac{1}{n}\) (\ref{union-bound-coupon-collector}) \\
\bottomrule
\end{tabular}

W następnych sekcjach pokażdemy odpowiednie definicje tych nierówności oraz dowody wartości podanych w tabelce.

\begin{example} Union bound dla kolekcjonera kuponów\\
	\label{union-bound-coupon-collector}
	Niech \(A_i\) to zdarzenie polegające na nieuzyskaniu \(i\)-tego kuponu po \(n \ln(n) + cn\) krokach
	\[
		\prob\pars{A_i} = \pars{1 - \frac{1}{n}}^{n \ln(n) + cn} = \pars{1 + \frac{1}{-n}}^{-n(-\ln(n) -c)} \leq e^{-(\ln(n) + c)} = \frac{1}{e^c n}
	\]
	Z \ref{union-bound} wiemy, że
	\[
		\prob\pars{\bigcup_{i=1}^n A_i} = \sum_{i=1}^n \prob\pars{A_i} = \frac{1}{e^c}
	\]
	Niech \(c = \ln(n)\), a więc \(A_i\) to zdarzenie polegające na nieuzyskaniu \(i\)-tego kuponu po \(2n \ln(n)\) krokach, a \(2n \ln(n) \leq 2n H_n\), a więc
	\[
		\prob(X \geq 2n H_n) \leq \prob\pars{X \geq 2n \ln(n)} = \prob\pars{\bigcup_{i=1}^n A_i} \leq \frac{1}{e^{\ln(n)}} = \frac{1}{n}
	\]
\end{example}