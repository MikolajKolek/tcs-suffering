Nierówność Markowa nie daje nam zbyt dobrych ograniczeń, ale jeśli jedyne co wiemy o zmiennej \( X \) to jej wartość oczekiwana, to to jest i tak dobry wynik.
\subsection{Definicja}
Spodziewamy się, że jeśli wiemy coś więcej o rozkładzie \( X \) to możemy lepiej szacować pewne prawdopodobieństwa.

Faktycznie, tak jest -- w nierówności Czebyszewa przyjmujemy że znamy wariancję.

\begin{theorem}[Twierdzenie 3.6 P\&C]
	Dla dowolnego \( a > 0 \)
	\[
		P(\abs{X - \expected{X}} \geq a) \leq \frac{\variance{X}}{a^2}
	\]
\end{theorem}
\begin{proof}
	Korzystamy z nierówności Markowa
	\[
		P(\abs{X - \expected{X}} \geq a) = P(\pars{X - \expected{X}}^2 \geq a^2)
		\leq \frac{\expected{\pars{X - \expected{X}}^2}}{a^2} = \frac{\variance{X}}{a^2}
	\]
\end{proof}

\subsection{Kolekcjoner kuponów}
Niech \( X_1, \dots, X_n \) opisują czasy czekania na \( i \)-ty kupon oraz \( X = \sum X_i \) -- łączny czas czekania.

Aby w ogóle móc liczyć coś nierównością Czebyszewa potrzebujemy obliczyć \( \variance{X} \).

Skorzystamy tutaj z bardzo wygodnego twierdzenia \ref{variance-of-sum-of-independent-variables}
a następnie z aby dostać
\begin{align*}
	\variance{X}
	 & = \sum_{i=1}^n \variance{X_i}          \\
	 & = \sum_{i=1}^n \frac{1 - p_i}{p_i^2}   \\
	 & \leq \sum_{i=1}^n \frac{1}{p_i^2}      \\
	 & =\sum_{i=1}^n \pars{\frac{n}{n-i+1}}^2 \\
	 & =n^2 \cdot \sum_{i=1}^n \frac{1}{i^2}  \\
	 & \leq n^2\frac{\pi^2}{6}
\end{align*}

Teraz wkładamy to do nierówności Czebyszewa:
\[
	P(\abs{X - nH_n} \geq nH_n) \leq \frac{\variance{X}}{n^2H_n^2} = \frac{\pi^2}{6H_n^2} = O\pars{\frac{1}{\ln^2 n}}
\]
