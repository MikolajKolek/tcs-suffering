Powyższa definicja nie jest jedyną możliwą definicją procesu Poissona. Okazuje się, że możemy skorzystać też z nieco wygodniejszej definicji bez warunków 3. i 4., ale za to korzystającej z rozkładu Poissona.

Pokażemy teraz dwa lematy, które dadzą nam równoważność między dwoma definicjami.

\begin{theorem}[Twierdzenie 8.7 P\&C]
	Niech \( \set{N(t) \mid t \geq 0} \) będzie procesem Poissona z parametrem \( \lambda \). Wtedy dla dowolnego \( t \geq 0 \) oraz \( n \in \natural \)
	\[
		\prob_n(t) \coloneq P(N(t) = n) = e^{-\lambda t} \frac{\pars{\lambda t}^n}{n!}
	\]
\end{theorem}

\begin{proof}
	Zaczynamy od policzenia \( \prob_0(t) \); dowód będzie indukcyjny.
	
	Zauważmy, że z niezależności przyrostów mamy
	\begin{align*}
		\prob_0(t+h) &= \prob(N(t+h) = 0)\\
		&= \prob(N(t) = 0 \land N(t + h) - N(t) = 0)\\
		&= \prob(N(t) = 0)\prob(N(t + h) - N(t) = 0)\\
		&= \prob(N(t) = 0) \prob(N(h) = 0)\\
		&= \prob_0(t) \prob_0(h)
	\end{align*}
	
	Robimy więc pierwszą rzecz, która nam przychodzi do głowy tj. liczymy pochodną \( P_0(t) \), a co.
	\begin{align*}
		P_0'(t)
		 & = \lim_{h \rightarrow 0} \frac{P_0(t + h) - P_0(t)}{h}                                                             \\
		 & = \lim_{h \rightarrow 0} P_0(t) \cdot \frac{P_0(h) - 1}{h}                                                         \\
		 & = \lim_{h \rightarrow 0} P_0(t) \cdot \frac{(1 - P(N(h) = 1) - P(N(h)  > 1)) - 1}{h}                                 \\
		 & = \lim_{h \rightarrow 0} \pars{ P_0(t) \cdot \pars{ \frac{- P(N(h) = 1)}{h} - \frac{P(N(h) > 1)}{h}}}              \\
		 & = P_0(t) \cdot \pars{-\lim_{h \rightarrow 0} \frac{P(N(h) = 1)}{h} - \lim_{h \rightarrow 0} \frac{P(N(h) > 1)}{h}} \\
		 & = P_0(t) \cdot \pars{-\lambda - 0}                                                                                 \\
		 & = -\lambda P_0(t)
	\end{align*}
	Wyniki poszczególnych limesów biorą się z własności 4 i 5 procesu Poissona.

	Mamy zatem równanie różniczkowe
	\[
		P_0'(t) = -\lambda P_0(t)
	\]
	\[
		\frac{P_0'(t)}{P_0(t)} = -\lambda
	\]

	Całkujemy po \( t \) i dostajemy
	\[
		\ln P_0(t) = -\lambda t + C
	\]
	\[
		P_0(t) = e^{-\lambda t + C}
	\]
	Ponieważ \( P_0(0) = 1 \) to \( C = 0\), czyli \( P_0(t) = e^{-\lambda t} \). Tym samym bazę indukcji mamy udowodnioną.

	Podobnie zabawny motyw dzieje się gdy obliczamy kolejne \( P_n(t) \). Na początek zaobserwujmy jednak jedną rzecz.

	\begin{fact}
		\[ P_{n}(t+h) = \sum_{k=0}^{n} P_{n-k}(t) \cdot P_k(h) \]
	\end{fact}
	\begin{proof}
		Jeśli wiemy, że w czasie \(t + h\) zaistniało \(n\) zdarzeń, to wiemy, że jakieś \(k\) (być może \(0\)) zdarzeń musiało zaistnieć w czasie \(h\), a więc \(n - k\) zdarzeń zaistniało w czasie \(t\). Aby policzyć prawdopodobieństwo takiej sytuacji wystarczy wymnożyć 2 takie prawdopodobieństwa (bo niezależność) a z racji tego że kolejne składniki sumy opisują zdarzenia które są rozłączne to zsumowanie jest legalne.
	\end{proof}

	Korzystając z wyżej wymienionego faktu, mamy:
	\begin{align*}
		P_n(t + h)
		 & = \sum_{k=0}^n P_{n-k}(t) \cdot P_k(h)                                                      \\
		 & = P_n(t) \cdot P_0(h) + P_{n-1}(t) \cdot P_1(h) + \sum_{k=2}^n P_{n-k}(t) \cdot P(N(h) = k)
	\end{align*}
	Zrobiliśmy tu bardzo sprytną rzecz -- mianowicie rozbiliśmy sumę na trzy części tak, aby przy liczeniu pochodnych wszystko nam się ładnie zwinęło.
	\begin{align*}
		P_n'(t)
		 & = \lim_{h \rightarrow 0} \frac{P_n(t + h) - P_n(t)}{h}                                                                                                                                                         \\
		 & =  \lim_{h \rightarrow 0} \pars{ \frac{
				P_n(t) \cdot P_0(h) + P_{n-1}(t) \cdot P_1(h) + \sum_{k=2}^n \pars{P_{n-k}(t) \cdot P(N(h) = k)} - P_n(t)}{h}
		}                                                                                                                                                                                                                 \\
		 & =  \lim_{h \rightarrow 0} \pars{ \frac{
				P_n(t) \cdot (P_0(h) - 1) + P_{n-1}(t) \cdot P(N(h)=1) + \sum_{k=2}^n (P_{n-k}(t) \cdot P(N(h) = k))}{h}
		}                                                                                                                                                                                                                 \\
		 & =  \lim_{h \rightarrow 0} \pars{
			\frac{P_n(t) \cdot (P_0(h) - 1)}{h}
			+ \frac{P_{n-1}(t)\cdot P(N(h) = 1)}{h}
		+ \sum_{k=2}^n P_{n-k}(t) \cdot \frac{P(N(h) = k)}{h}}                                                                                                                                                            \\
		 & = P_n(t) \lim_{h \rightarrow 0} \pars{\frac{P_0(h) - 1}{h}} + P_{n-1}(t) \lim_{h \rightarrow 0} \pars{\frac{P(N(h)=1)}{h}} + \sum_{k=2}^n P_{n-k}(t) \cdot \lim_{h \rightarrow 0} \pars{\frac{P(N(h) = k)}{h}} \\
		 & = P_n(t) \lim_{h \rightarrow 0} \pars{\frac{(1 - \mathrm{P}(N(h) = 1) - \mathrm{P}(N(h) > 1)) - 1}{h}} + P_{n-1}(t) \cdot \lambda + \sum_{k=2}^n P_{n-k}(t) \cdot 0                                              \\
		 & = P_n(t) \pars{-\lim_{h \rightarrow 0}\frac{\mathrm{P}(N(h)=1)}{h} - \lim_{h \rightarrow 0} \frac{\mathrm{P}(N(h)>1)}{h}} + \lambda P_{n-1}(t)                                                                 \\
		 & = -\lambda P_n(t) + \lambda P_{n-1}(t)
	\end{align*}
	Znowu dostajemy równanie różniczkowe
	\begin{align*}
		P_n'(t)                                                & = -\lambda P_n(t) + \lambda P_{n-1}(t) \\
		P_n'(t) + \lambda P_n(t)                               & = \lambda P_{n-1}(t)                   \\
		e^{\lambda t}\pars{P_n'(t) + \lambda P_n(t)}           & = \lambda e^{\lambda t} P_{n-1}(t)     \\
		e^{\lambda t} P'_{n}(t) + e^{\lambda t} \lambda P_n(t) & = \lambda e^{\lambda t} P_{n-1}(t)     \\
		\frac{d}{dt}\pars{e^{\lambda t}\cdot P_n(t)}           & = \lambda e^{\lambda t} P_{n-1}(t)
	\end{align*}

	I z założenia indukcyjnego:
	\[
		\lambda e^{\lambda t} P_{n-1}(t) = \lambda e^{\lambda t} \cdot e^{-\lambda t} \cdot \frac{(\lambda t)^{n-1}}{(n-1)!} = \frac{\lambda^n \cdot t^{n-1}}{(n-1)!}
	\]

	Całkujemy obustronnie:

	\[
		\int \frac{d}{dt}\pars{e^{\lambda t} P_n(t)} \; dt = e^{\lambda t} P_n(t) + C_1
	\]

	\[
		\int \frac{\lambda^n \cdot t^{n-1}}{(n-1)!} \; dt = \frac{\lambda^n}{(n-1)!} \cdot \int t^{n-1} \; dt  =
		\frac{\lambda^n}{(n-1)!} \cdot \frac{t^n}{n} + C_2 = \frac{\lambda^n t^n}{n!} + C_2
	\]

	Definiujemy \(C = C_2 - C_1\) by musieć mniej myśleć o stałych:

	\begin{align*}
		e^{\lambda t} P_n(t) + C_1 & = \frac{\lambda^n t^n}{n!} + C_2                             \\
		e^{\lambda t} P_n(t)       & = \frac{\lambda^n t^n}{n!} + C_2 - C_1                       \\
		e^{\lambda t} P_n(t)       & = \frac{\lambda^n t^n}{n!} + C                               \\
		P_n(t)                     & = e^{-\lambda t} \frac{\lambda^n t^n}{n!} + C e^{-\lambda t}
	\end{align*}

	Wiemy, że \( P_n(0) = 0 \), zatem \(C = 0\). W takim razie:

	\[
		P_n(t) = e^{-\lambda t} \frac{\lambda^n t^n}{n!} = e^{-\lambda t} \frac{(\lambda t)^n}{n!}
	\]

\end{proof}

\begin{theorem}[Twierdzenie 8.8 P\&C]
	Niech \(\set{N(t) \mid t \geq 0}\) będzie procesem stochastycznym liczącym takim, że
	\begin{enumerate}
		\item \(N(0) = 0\)
		\item Proces ma niezależne przyrosty
		\item \(\forall_{t,s}\prob\pars{N(s+t) - N(t) = n} = e^{-\lambda s}\frac{(\lambda s)^n}{n!}\)
	\end{enumerate}
	Wtedy jest to proces Poissona z parametrem \(\lambda\).
\end{theorem}
\begin{proof}
	Pokazujemy co następuje z definicji \ref{poisson-process-definition}
	\begin{enumerate}
		\item \( N(0) = 0 \) z założeń
		\item Niezależność i stacjonarność przyrostów również bezpośrednio z założeń
		\item
			\[
				\lim_{t \to 0}\frac{\prob\pars{N(t) = 1}}{t} = \lim_{t \to 0}\frac{e^{-\lambda s}\frac{\lambda t}{1!}}{t} =
					\lim_{t \to 0}\lambda e^{-\lambda t} = \lambda
			\]
			
		\item
			\begin{align*}
				\lim_{t \to 0}\frac{\prob\pars{N(t) \geq 2}}{t} &= \lim_{t \to 0}\frac{1 - \prob\pars{N(t) = 0} - \prob\pars{N(t) = 1}}{t}\\
				&= \lim_{t \to 0}\frac{1-e^{-\lambda t}-e^{-\lambda t}\frac{\lambda t}{1!}}{t}\\
				&= \lim_{t \to 0}\frac{1-e^{-\lambda t}}{t} - \lim_{t \to 0} \lambda e^{-\lambda t}\\
				&= \lim_{t \to 0}\frac{\lambda e^{-\lambda t}}{1} - \lambda = \lambda - \lambda = 0
			\end{align*}
	\end{enumerate}
\end{proof}