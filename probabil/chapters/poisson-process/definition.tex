\begin{definition}
	\textbf{Stochastycznym procesem liczącym} nazywamy \hyperref[stochastic-process-definition]{proces stochastyczny} 
	\[\set{N(t) \mid t \geq 0}\] 
	
	spełniający
	\begin{enumerate}
		\item \(N(t) \in \natural_0\)
		\item \(\forall_{s < t} N(s) \leq N(t)\)
	\end{enumerate}
\end{definition}

Intuicyjnie: \(N(t)\) mówi ile \textit{jakichś} zdarzeń zaszło od momentu rozpoczęcia procesu do chwili \(t\), a dla \(s \leq t\) liczba zdarzeń które zaszły w przedziale czasu \((s, t]\) to \(N(t) - N(s)\).

\begin{definition}
	\label{poisson-process-definition}
	\textbf{Procesem Poissona} z parametrem \( \lambda \) nazywamy stochastyczny proces liczący \( \set{N(t) \mid t \in \real, t \geq 0} \) taki, że:

	\begin{enumerate}
		\item \( N(0) = 0 \)
		
		\item Proces ma stacjonarne i niezależne przyrosty, tzn.
		\begin{enumerate}[label=\arabic{enumi}\alph*.]
			\item Stacjonarność: \(\forall_{s, t \geq 0}\) zmienne \(N(s)\) oraz \(N(s+t) - N(t)\) mają taki sam rozkład
			\item Niezależność: \(\forall_{t_1 < t_2 \leq t_3 < t_4}\) zmienne \( N(t_2) - N(t_1) \) oraz \( N(t_4) - N(t_3) \) są niezależne
		\end{enumerate}
		
		\item Prawdopodobieństwo jednego zdarzenia w małym przedziale długości \( t \) zbiega do \( \lambda  \) \\
		      \[ \lim_{t \rightarrow 0} \frac{\prob(N(t) = 1)}{t} = \lambda \]

		\item Prawdopodobieństwo więcej niż jednego zdarzenia w małym przedziale zbiega do zera \\
		      \[ \lim_{t \rightarrow 0} \frac{\prob(N(t) > 1)}{t} = 0 \]
	\end{enumerate}
\end{definition}

