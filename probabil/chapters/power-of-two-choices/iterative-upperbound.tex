Przejdźmy wreszcie do pierwszego dowodu ograniczenia górnego twierdzenia \ref{power-of-two-choices-theorem}. Tak naprawdę będziemy dowodzić że 
\[
	\prob\pars{\max(X_1, \dots, X_n) \geq \log_2 \ln n + \mathcal{O}(1)} < \frac{1}{n^{\alpha - 1}}
\]
ale jak się temu przyjrzymy to możemy zauważyć, że z odpowiednim \(\alpha\) da nam to co chcemy.

Niech
\[
	\beta_4 = \frac{n}{4}
\]
\[
	\beta_{i+1} = 2\frac{\beta_i^2}{n} \text{ dla } 4 \leq i \leq i^*
\]
\(i^*\) zostanie zdefiniowane później.

\begin{lemma}
	\[
		\beta_{i+4} = \frac{n}{2^{2^i + 1}}
	\]
	a więc
	\[
		\beta_{i+4} \leq \frac{n}{2^{2^i}}
	\]
\end{lemma}
\begin{proof}
	Przeprowadźmy dowód indukcyjny. Dla \(i = 0\)
	\[
		\beta_4 = \frac{n}{4} = \frac{n}{2^{1 + 1}} = \frac{n}{2^{2^0 + 1}} = \beta_{0 + 4}
	\]
	Jeśli lemat zachodzi dla \(i\) to
	\[
		\beta_{i+5} = 2\frac{\beta_{i+4}^2}{n} 
		= 2\frac{\pars{\frac{n}{2^{2^i + 1}}}^2}{n} 
		= 2\frac{n}{2^{2(2^i + 1)}}
		= 2\frac{n}{2^{2^{i + 1} + 2}}
		= \frac{n}{2^{2^{i + 1} + 1}}
		= \beta_{(i + 1) + 4}
	\]
	a więc zachodzi też dla \(i + 1\)
\end{proof}

Zdefiniujmy teraz \(\mathcal{E}_i\) jako zdarzenie zachodzące jeśli \(\nu_i(n) \leq \beta_i\). Widzimy, że
\[
	\prob(\mathcal{E}_4) = \prob\pars{\nu_4(n) \leq \frac{n}{4}} = 1
\]
bo oczywiście \(\geq \frac{n}{4}\) kul może mieć co najwyżej \(\frac{n}{4}\) urn.

Chcemy teraz wykazać, że jeśli \(\mathcal{E}_i\) zaszło, to prawie na pewno \(\mathcal{E}_{i + 1}\) też zaszło. Dla \(t \in [n]\) definiujemy binarną zmienną losową \(Y_t\)
\[
	Y_t = \left\{ \begin{array}{lr} 1 & \text{ dla } h(t) \geq i + 1 \land \nu_i(t - 1) \leq \beta_i \\ 0 & \text{wpp.} \end{array} \right.
\]
Zauważmy, że dla \((K_1, \dots, K_{t - 1})\) będącego zmiennymi losowymi reprezentującymi urny do których trafiły kolejne kule oraz \((\omega_1, \dots, \omega_{t - 1})\) takiego, że \(\forall_{i \in [t - 1]} \, \omega_i \in [n]\) zachodzi
\[
	\prob(Y_t = 1 \mid (K_1, \dots, K_{t - 1}) = (\omega_1, \dots, \omega_{t - 1})) \leq \pars{\frac{\beta_i}{n}}^2
\]
Jest tak, ponieważ aby zaszło \(h(t) \geq i + 1\) \(t\)-ta kula musi obiema symulacjami trafić w urnę o przynajmniej \(i\) kulach, ale wiemy z \(\nu_i(t - 1) \leq \beta_i \) że takich urn jest co najwyżej \(\beta_i\) z \(n\) możliwych.

Jeśli \(\mathcal{E}_i\) zaszło, to \(\mu_{i+1}(n) = \sum_{t \in [n]} Y_t\), ponieważ z \(\mathcal{E}_i\) wiemy, że \(\nu_i(t - 1) \leq \beta_i\) jest prawdziwe dla każdego \(t\), a więc \(Y_t = 1 \iff h(t) \geq i + 1\), a więc suma \(Y_t\) zlicza liczbę kul o wysokości \(\geq i + 1\), co jest dokładnie definicją \(\mu_{i+1}(n)\).

Niech \(p_i \coloneq \pars{\frac{\beta_i}{n}}^2\). Z lematu \ref{binary-random-variable-binomial-upperbound} dostajemy
\[
	\prob\pars{\sum_{t \in [n]} Y_t > k} \leq \prob(\Bin(n, p_i) > k)
\]

Teraz przy użyciu tego pokażemy, że \( \prob\pars{\neg\mathcal{E}_{i+1}} \) jest małe, jeżeli \( \mathcal{E}_i \) zaszło.
Najpierw policzymy prawdopodobieństwo warunkowe:
\begin{align*}
	\prob\pars{\neg\mathcal{E}_{i+1} \mid \mathcal{E}_i} &= \prob\pars{\nu_{i+1}\pars{n} > \beta_{i+1} \mid \mathcal{E}_i} \\
	&\leq \prob\pars{\mu_{i+1}\pars{n} > \beta_{i+1} \mid \mathcal{E}_i} \\
	&= \prob\pars{\sum_{t \in [n]} Y_t > 2np_i \mid \mathcal{E}_i} \\
\end{align*}
Tutaj w ostatniej równości robimy dwie rzeczy: korzystamy z faku, że jeżeli \( \mathcal{E}_i \) zaszło, to \(\mu_{i+1} = \sum_{t \in [n]} Y_t \), oraz 
z definicji \( \beta_{i+1} \) i \( p_i \) podstawiamy \( \beta_{i+1} = 2np_i \)
\begin{align*}
	\prob\pars{\sum_{t \in [n]} Y_t > 2np_i \mid \mathcal{E}_i} &\leq \frac{\prob\pars{\sum_{t \in [n]} Y_t > 2np_i}}{\prob\pars{\mathcal{E}_i}} \\
	&\leq \frac{\prob\pars{\Bin(n,p_i) > 2np_i}}{\prob\pars{\mathcal{E}_i}} \\
\end{align*}
W drugiej nierówności korzystamy z faktu z lematu \ref{binary-random-variable-binomial-upperbound}. Przekształacając dalej:
\begin{align*}
	\frac{\prob\pars{\Bin(n,p_i) > 2np_i}}{\prob\pars{\mathcal{E}_i}} &\leq \frac{1}{e^{\frac{np_i}{3}}\prob\pars{\mathcal{E}_i}} \\
	&\leq \frac{1}{n^{\alpha}\prob\pars{\mathcal{E}_i}}
\end{align*}
Na koniec korzystamy z nierówności Czernowa (lemat \ref{chernow-for-power-of-two}), oraz zakładamy, że \( np_i \geq 3\alpha\ln(n) \).
Ostatecznie otrzymujemy:
\[
	\prob\pars{\neg\mathcal{E}_{i+1} \mid \mathcal{E}_i} \leq \frac{1}{n^{\alpha}\prob\pars{\mathcal{E}_i}}
\]

Teraz możemy ograniczyć \( \prob\pars{\neg\mathcal{E}_{i+1}} \):
\begin{align*}
	\prob\pars{\neg\mathcal{E}_{i+1}} &= \prob\pars{\neg\mathcal{E}_{i+1} \mid \mathcal{E}_i}\prob\pars{\mathcal{E}_i} + \prob\pars{\neg\mathcal{E}_{i+1} \mid \neg\mathcal{E}_i}\prob\pars{\neg\mathcal{E}_i} \\
	&\leq \prob\pars{\neg\mathcal{E}_{i+1} \mid \mathcal{E}_i}\prob\pars{\mathcal{E}_i} + \prob\pars{\neg\mathcal{E}_i} \\
	&\leq \frac{1}{n^{\alpha}} + \prob\pars{\neg\mathcal{E}_i}
\end{align*}
Gdzie na końcu ponownie korzystamy z założenia, że \( np_i \geq 3\alpha\ln(n) \).
Pokazujemy przez indukcję:
\[
	\prob\pars{\neg\mathcal{E}_{i+1}} \leq \frac{1}{n^{\alpha}} + \prob\pars{\neg\mathcal{E}_i} \leq \frac{1}{n^{\alpha}} + \frac{i}{n^{\alpha}} = \frac{i + 1}{n^{\alpha}}
\]
Ponieważ wiemy, że \(\prob\pars{\mathcal{E}_4} = 1 \), zatem \( \prob\pars{\neg\mathcal{E}_4} = 0 \leq \frac{4}{n^\alpha} \)

Teraz wracamy do \( i^* \) i definiujemy je jako minimalne \( i \) takie, że \( np_i < 3\alpha\ln n \).
Możemy zauważyć, że to co pokazaliśmy z indukcji jest prawdziwe jedynie dla \( i < i^* \), lecz działa jeszcze dla \(\prob\pars{\neg\mathcal{E}_{i^*}}\) bo pracowaliśmy na \(\mathcal{E}_{i + 1}\), a więc \( \prob\pars{\neg\mathcal{E}_{i^*}} \leq \frac{i^*}{n^{\alpha}} \)

Chcemy jeszcze znaleźć \( i^* \) więc rozważamy następujący ciąg nierówności:
\[
	np_i < 3\alpha\ln n
\]
Podstawiamy \( p_i = \pars{\frac{\beta_i}{n}}^2 \)
\[
	n\pars{\frac{\beta_i}{n}}^2 < 3\alpha\ln n
\]
Tworzymy mocniejszą nierówność, korzystając z faktu, że \( \beta_i \leq \frac{n}{2^{2^{i-4}}} \)
\[
	n\pars{\frac{\frac{n}{2^{2^{i-4}}}}{n}}^2 < 3\alpha\ln n
\]
\[
	\frac{n}{3\alpha\ln n} < 2^{2^{i-4} \cdot 2} = 2^{2^{i-3}}
\]
Logarytmujemy stronami i podstawiamy \( c = \frac{1}{\ln2} \):
\[
	c \cdot \ln n - c \cdot \ln(3\alpha) - \log_2 \ln n < 2^{i-3}
\]
Ponownie tworzymy mocniejszą nierówność przez zwiększenie lewej strony:
\[
	c \cdot \ln n < 2^{i-3}
\]
Znowu logarytmujemy stronami:
\[
	\ln c + \log_2 \ln n < i-3
\]
Ostatecznie otrzymujemy
\[
	\log_2 \ln n + \mathcal{O}(1) < i \implies i^* \leq \log_2 \ln n + \mathcal{O}(1)
\]
Znając \( i^* \) możemy dalej ograniczać prawdopodobieństwo, że \( \nu_i(n) \) będzie duże dla \( i \geq i^* \)
\begin{align*}
	\prob\pars{\nu_{i^*+1}(n) > 9\alpha\ln n \mid \mathcal{E}_{i^*}} &\leq \prob\pars{\mu_{i^*+1}(n) > 9\alpha\ln n \mid \mathcal{E}_{i^*}} \\
	&= \prob\pars{\sum_{t \in [n]} Y_t > 9\alpha\ln n \mid \mathcal{E}_{i^*}} \\
	&\leq \frac{\prob\pars{\Bin(n, p_i) > 9\alpha\ln n}}{\prob\pars{\mathcal{E}_{i^*}}} \\
	&\leq \frac{\prob\pars{\Bin(n, \frac{3\alpha\ln n}{n}) > 9\alpha\ln n}}{\prob\pars{\mathcal{E}_{i^*}}} \\
	&\leq \frac{1}{n^{\alpha}\prob\pars{\mathcal{E}_{i^*}}}
\end{align*}
Pierwsze przekształcenia robimy tak samo jak wyżej i korzystając z \( np_i < 3\alpha\ln n \). Ostatnia nierówność wynika z Czernowa:
\[
	\prob\pars{X > (1 + 2) \cdot 3\alpha\ln n} \leq \pars{\frac{e^2}{3^3}}^{3\alpha\ln n} < \pars{e^{-\frac{1}{3}}}^{3\alpha\ln n} = \frac{1}{n^{\alpha}}
\]
Z tego otrzymujemy:
\begin{align*}
	\prob\pars{\nu_{i^*+1}(n) > 9\alpha\ln n} &\leq \prob\pars{\neg\mathcal{E}_{i^*}} \prob\pars{\nu_{i^*+1}(n) > 9\alpha\ln n \mid \neg\mathcal{E}_{i^*}} + \prob\pars{\mathcal{E}_{i^*}} \prob\pars{\nu_{i^*+1}(n) > 9\alpha\ln n \mid \mathcal{E}_{i^*}}\\
	&\leq \prob\pars{\neg\mathcal{E}_{i^*}} + \frac{1}{n^{\alpha}}\\
	&\leq \frac{i^* + 1}{n^{\alpha}}
\end{align*}
Zostało nam ostatnie warunkowe ograniczenie:
\begin{align*}
	\prob\pars{\mu_{i^* + 2}(n) \geq \beta \mid \nu_{i^* + 1}(n) \leq 9\alpha\ln n} &\leq \frac{\prob\pars{\Bin(n, \pars{\frac{9\alpha\ln n}{n}}^2) \geq \beta}}{\prob\pars{\nu_{i^* + 1}(n) \leq 9\alpha\ln n}} \\
	&\leq \frac{n^{\beta}\pars{\frac{9\alpha\ln n}{n}}^{2\beta}}{\prob\pars{\nu_{i^* + 1}(n) \leq 9\alpha\ln n}} \\
	&\leq \frac{\pars{9\alpha\ln n}^{2 \beta}}{n^{\beta}\prob\pars{\nu_{i^* + 1}(n) \leq 9\alpha\ln n}}
\end{align*}
Gdzie w pierwszej nierówności jako \( p \) dla w rozkładzie dwumianowym podstawiamy prawdopodobieństwo, że dwukrotnie trafimy do wystarczająco wypełnionych urn. Druga nierówność to union bound.

Teraz mamy wszystko, żeby pokazać górne ograniczenie.
\begin{align*}
	\prob\pars{\nu_{i^* + 2 + 2\alpha}(n) \geq 1} &\leq \prob\pars{\mu_{i^* + 2 + 2\alpha}(n) \geq 1} \\
	&\leq \prob\pars{\mu_{i^* + 2}(n) \geq 2\alpha} \\
	&\leq \prob\pars{\mu_{i^* + 2}(n) \geq 2\alpha \mid \nu_{i^* + 1}(n) \leq 9\alpha\ln n}\prob\pars{\nu_{i^* + 1}(n) \leq 9\alpha\ln n} \\
	&\quad + \prob\pars{\nu_{i^* + 1}(n) > 9\alpha\ln n} \\
	&\leq \frac{\pars{9\alpha\ln n}^{4 \alpha}}{n^{2\alpha}} + \frac{i^* + 1}{n^{\alpha}} \\
	&< \frac{1}{n^{\alpha - 1}}
\end{align*}
Gdzie ostatnia nierówność zachodzi dla odpowiednio dużego \( n \).
