Dowód ograniczenia górnego przez drzewa świadczące wzorowany \href{https://dl.acm.org/doi/10.1145/792538.792546}{artykułem}.

\subsection*{Definicje drzew świadczących}

Jak poprzednio, rzucając kulę losujemy jednostajnie \(d \geq 2\) urn i wrzucamy kulę do najmniejszej.
Będziemy dalej nazywać \(i\)-tą wylosowaną urnę dla kuli \(b\) \(i\)-tą lokacją kuli \(b\).

\begin{definition}
    \textbf{Drzewo świadczące} rzędu \(L\) to pełne, ukorzenione drzewo \(d\)-arne wysokości \(L\).
    Każdy wierzchołek w drzewie reprezentuje pewną kulę, niekoniecznie unikalną - kula może być reprezentowana przez wiele wierzchołków.
    Dodatkowo, dla każdego wierzchołka \(v\) niebędącego liściem, kule odpowiadające dzieciom \(v\) muszą już znajdować się w urnach w momencie rzucania kuli \(v\).
\end{definition}

Ogólna idea dowodu będzie następująca; rozważmy wszystkie drzewa świadczące.
Powiemy za moment, co oznacza że jakieś konkretne drzewo świadczące jest \textit{aktywne}, a następnie pokażemy, że jeśli w którejś z urn jest dużo kul, to któreś z drzew świadczących jest \textit{aktywne}.
Tym samym prawdopodobieństwo, że któraś urna ma dużo kul jest ograniczone przez prawdopodobieństwo, że jest jakieś \textit{aktywne} drzewo świadczące.

\begin{definition}
    Dodajmy sobie dodatkową intepretację dla struktury dowolnego drzewa świadczącego.

    \begin{itemize}
        \item \textbf{Krawędziozdarzenie} (\textit{"edge event"}) - dla krawędzi \(e = \pars{u, v}\), gdzie \(v\) jest \(i\)-tym dzieckiem \(u\),
        krawędź \(e\) określa zdarzenie, że \(i\)-ta lokacja kuli \(u\) jest taka sama, co któraś z lokacji kuli \(v\).
        \item \textbf{Liściozdarzenie} (\textit{"leaf event"}) - liść drzewa świadczącego \(v\) określa zdarzenie, że każda z \(d\) lokacji kuli \(v\) wskazuje na urnę, gdzie znajdują się już co najmniej 3 inne kule, które nie są zareprezentowane wierzchołkami drzewa.
    \end{itemize}
\end{definition}

\begin{definition}
    Krawędź lub liść drzewa świadczącego są aktywne, jeśli podczas rzucania kul do urn zaszły odpowiadające im krawędzio- lub liściozdarzenia.
    Drzewo świadczące jest aktywne, jeśli wszystkie jego krawędzie i liście są aktywne.
\end{definition}

\subsection*{Konstrukcja drzew świadczących}

Z początku będziemy zakładać, że zdarzenia opisywane przez drzewa świadczące są niezależne od siebie.
W szczególności, rozważamy tylko drzewa, gdzie wierzchołki reprezentują parami różne kule - ta prostsza wersja posłuży później do pełnego dowodu.

\begin{lemma}
    Jeśli któraś z urn posiada więcej niż \(L+3\) kule, to istnieje aktywne drzewo świadczące rzędu \(L\).
\end{lemma}
\begin{proof}
    Niech urna \(x\) ma co najmniej \(L+4\) kule. Skonstruujemy aktywne drzewo następująco.

    Niech korzeń reprezentuje ostatnio wrzuconą kulę z \(x\). Zauważmy, że każda z \(d\) lokacji tej kuli wskazuje na urnę, gdzie były już \(L+3\) kule.
    Przypisujemy dzieciom korzenia kule, które znajdowały się najwyżej w tych urnach w momencie wrzucania kuli korzenia. Dalej postępujemy tak samo z dziećmi aż całe drzewo zostanie skonstruowane.

    Można zauważyć, że kula korzenia została wrzucona do swojej urny po co najmniej \(L+3\) innych kulach. Podobnie w momencie wrzucania kul każdego z dzieci korzenia, w ich urnach były już co najmniej \(L+2\) kule itd.

    Zauważmy też, że każda kula reprezentowana przez liść miała już co najmniej 3 inne kule w urnie w momencie jej wrzucania.

    Otrzymaliśmy drzewo świadczące o wszystkich krawędziach i liściach aktywnych, więc jest to drzewo aktywne.
\end{proof}

Jeśli udałoby się ograniczyć z góry prawdopodobieństwo na istnienie aktywnego drzewa świadczącego rzędu \(L\), to byłoby to także ograniczenie na prawdopodobieństwo, że istnieje urna o co najmniej \(L+4\) kulach. 

Jeśli w drzewie świadczącym jest \(m\) wierzchołków, to możemy przypisać im kule na \(n^m\) sposobów.
Prawdopodobieństwo, że krawędź \(\pars{u,v}\) jest aktywna jest co najwyżej \(\frac{d}{n}\), ponieważ prawdopodobieństwo, że \(i\)-ta lokacja \(u\) trafi w wybraną z lokacji \(v\) jest co najwyżej \(\frac{1}{n}\).
Z niezależności prawdopodobieństwo, że wszystkie krawędzie są aktywne jest co najwyżej \(\pars{\frac{d}{n}}^{m-1}\).
Prawdopodobieństwo aktywacji konkretnego liścia jest co najwyżej \(\frac{1}{3^d}\), ponieważ każda z lokacji kuli liścia musi trafić w urnę o co najmniej \(3\) innych kulach - a takich urn jest co najwyżej \(\frac{n}{3}\).
Tym samym gdy oznaczymy przez \(q\) liczbę liści w drzewie, to prawdopodobieństwo aktywacji wszystkich liści jest co najwyżej \(3^{-dq}\).

Łącząc to wszystko ze sobą i nakładając union bounda dostajemy ograniczenie na istnienie aktywnego drzewa świadczącego:

\[
    n^m \cdot \pars{\frac{d}{n}}^{m-1} \cdot 3^{-dq}
\]

Zachodzą także nastepujące ograniczenia: \(m \leq 2q\) oraz \(2d^2 \leq 3^d\). Gdy podstawimy je wraz z \(q = d^L\) dostaniemy ograniczenie:

\begin{align*}
    n^m \cdot \pars{\frac{d}{n}}^{m-1} \cdot 3^{-dq} &= n \cdot d^{m-1} \cdot 3^{-dq} \leq n \cdot d^{2q} \cdot 3^{-dq}\\
    &\leq n \cdot 2^{-q} \cdot 3^{dq} \cdot 3^{-dq} = n \cdot 2^{-d^L}
\end{align*}

A gdy weźmiemy \(L \geq \log_d \log_2 n + \log_d(1 + \alpha)\) dla \(\alpha > 0\), ograniczymy prawdopodobieństwo istnienia odpowiedniego aktywnego drzewa świadczącego przez \(n^{-\alpha}\).

\subsection*{Konstrukcja pełnych drzew świadczących}

W poprzedniej części dowodu rozważaliśmy tylko prawdopodobieństwo aktywacji drzew świadczących o parami różnych kulach.
W rzeczywistości jednak ta sama kula może wystąpić w drzewie wielokrotnie - i wtedy zdarzenia konstruujące drzewo nie są już niezależne.
W tym celu możemy poodcinać niektóre wierzchołki, które nam się nie będą podobać, aby przywrócić niezależność.
Odcinanie wierzchołków zwiększa jednak prawdopodobieństwo aktywacji takiego drzewa, więc zaczniemy od większej struktury.

\begin{definition}
    \textbf{Pełne drzewo świadczące} rzędu \(L\) dla pewnej stałej \(\kappa \in \mathbb{N}_2\) ma następującą konstrukcję:
    Korzeń drzewa ma \(\kappa\) dzieci, każde z których ma dokładnie jedno dziecko - tym samym korzeń ma także \(\kappa\) wnuków.
    Każdy z wnuków jest korzeniem standardowego drzewa świadczącego rzędu \(L\).
    W dodatku, w pełnym drzewie świadczącym kule przypisane do dzieci korzenia muszą być parami różne, a korzeń jako jedyny nie ma przypisanej kuli.
\end{definition}

\begin{lemma}
    Jeśli któraś z urn posiada więcej niż \(L + 3 + \kappa\) kul, to istnieje aktywne pełne drzewo świadczące.
\end{lemma}
\begin{proof}
    Niech urna \(x\) posiada co najmniej \(L + 4 + \kappa\) kul. Wybieramy z niej \(\kappa\) ostatnio wrzuconych kul i przyporządkowujemy je dzieciom korzenia.
    Następnie przyporządkujemy kule wnukom. Rozważmy kulę \(b\) przypisaną do \(v\) - dziecka korzenia. Przynajmniej jedna z lokacji \(b\) wskazuje na urnę \(x\) - jeśli jest ich więcej, należy wybrać jedną z nich.
    Niech \(i\) to indeks tej lokacji. Weźmy następną lokację, czyli \(i+1 \mod d\). W momencie wrzucania \(b\), ta lokacja wskazuje na urnę o co najmniej \(L+3\) kulach. 
    Ostatnio wrzuconą do tej urny kulę przed wrzuceniem \(b\) przypisujemy do dziecka \(v\), czyli wnuka korzenia.
\end{proof}

\subsection*{Przycinanie pełnych drzew świadczących}

\begin{definition}
    Zdefiniujemy \textbf{przycięte pełne drzewo świadczące} konstrukcyjnie. Zaczynając od pełnego drzewa świadczącego, będziemy odcinać jego krawędzie w odpowiedni sposób.
    Począwszy od korzenia, przeglądamy drzewo w kolejności BFS.
    Za każdym razem gdy przeglądamy wierzchołek \(v\), który reprezentuje kulę, która była już widziana wcześniej, przecinamy krawędź łączącą \(v\) z jego rodzicem.
    Odcinamy w ten sposób \(v\) wraz z całym jego poddrzewem. Zauważmy, że procedura ta nie odetnie nigdy dzieci korzenia, gdyż reprezentują kule parami różne.
    Przecięte w ten sposób krawędzie będziemy dalej nazywać \textit{odciętymi} (\textit{"cutoff edge"}). Kontynuujemy odcinanie do momentu, aż przeglądniemy całe drzewo albo odetniemy \(\kappa\) krawędzi.
    Do przyciętego drzewa świadczącego trafiają jednak tylko przeglądnięte wierzchołki i krawędzie, wraz z krawędziami odciętymi, jako świadectwo miejsc odcięcia.
\end{definition}

Wyróżniamy dwa przypadki.

\subsubsection*{Przypadek 1 - mniej niż \(\kappa\) odciętych}

Jeśli odcięliśmy mniej niż \(\kappa\) krawędzi, to znaczy, że jeden z wnuków korzenia wraz z poddrzewem przetrwał przycinanie.
To oznacza, że mamy aktywne drzewo świadczące o parami różnych kulach, a prawdopodobieństwo na to już ograniczyliśmy przez \(n^{-\alpha}\).

\subsubsection*{Przypadek 2 - \(\kappa\) odciętych}

Pozostaje jedynie ograniczyć prawdopodobieństwo wystąpienia aktywnego przyciętego drzewa świadczącego o \(\kappa\) odciętych krawędziach.
Ograniczymy to prawdopodobieństwo przy założeniu, że liczba kul reprezentowanych przez pełne drzewo świadczące jest co najwyżej \(M = 2\kappa(\alpha+1)\log_2n\).
Jest co najwyżej \(M^\kappa\) sposobów na przycięcie pełnego drzewa świadczącego - zamianę pełnego drzewa świadczącego w przycięte.
Niech \(m\) będzie liczbą kul reprezentowanych przez przycięte drzewo, a \(q\) to liczba liści. Liczba sposobów na dopasowanie kul do wierzchołków to co najwyżej \(n^m\).
Przyjmijmy, że korzeń przyciętego drzewa (który nie ma przypisanej kuli) zamiast tego ma przypisaną urnę, z której wyjęliśmy kule dla jego dzieci.
Jest \(n\) sposobów na wybranie tej urny, i mając tę urnę prawdopodobieństwo aktywacji wszystkich krawędzi to co najwyżej \(\pars{\frac{d}{n}}^m\), a prawdopodobieństwo, że wszystkie liście są aktywne jest ograniczone przez \(3^{-dq}\).

Okazuje się, że dostaliśmy niemal to samo ograniczenie. Mimo to, \(q\) i \(m\) mogą być mniejsze niż poprzednio.
Możemy jednak wciąż wyciągnąć lepsze ograniczenie patrząc na odcięte krawędzie.

Każda odcięta krawędź jest świadkiem, że jakaś kula \(b\) reprezentowana przez nieprzycięty wierzchołek \(u\) dzieli lokację z jakąś kulą \(b'\) innego wierzchołka \(u'\) - jest to powód, dlaczego ta krawędź została odcięta.
Wierzchołek \(u'\) został przeglądnięty przed \(u\), więc z pewnością jest częścią przyciętego drzewa świadczącego. Przycięta krawędź opisuje, która lokacja kuli \(b\) trafiła w lokację dzieloną z kulą \(b'\).
Liczba możliwości na wybranie \(u'\) i tym samym \(b'\) jest ograniczona z góry przez \(m \leq M\). Prawdopodobieństwo, że wybrana lokacja \(b\) trafi w tę samą urnę co któraś z lokacji \(b'\) jest co najwyżej \(\frac{d}{n}\).
Tym samym prawdopodobieństwo posiadania \(\kappa\) odciętych krawędzi ograniczamy przez \(\pars{M\frac{d}{n}}^\kappa\).

Możemy teraz zauważyć, że skoro pozbyliśmy się zbędnych kul, a wszystkie liściozdarzenia, krawędziozdarzenia i odcięciozdarzenia są niezależne, to pozostaje przejść do ostatecznego ograniczenia, co następuje:

\begin{align*}
    M^k \cdot n^{m+1} \cdot \pars{\frac{d}{n}}^m \cdot 3^{-dq} \cdot \pars{\frac{Md}{n}}^\kappa = n \cdot d^m \cdot 3^{-dq} \cdot \pars{\frac{M^2 d}{n}}^\kappa\\
\end{align*}

Stosujemy teraz podobne ograniczenia jak poprzednio: \(m \leq 2q\), \(d^2 \leq 3^d\) oraz \(M \leq 2\kappa(\alpha+1)\log_2n\).

\begin{align*}
    n \cdot d^m \cdot 3^{-dq} \cdot \pars{\frac{M^2 d}{n}}^\kappa &\leq n \cdot \pars{\frac{M^2 d}{n}}\\
    &\leq n\pars{\frac{d\pars{2\kappa(\alpha+1)\log_2n}}{n}}^\kappa = n^{-\kappa+1+o(1)} 
\end{align*}

To zamyka ograniczenie drugiego przypadku, a tym samym cały dowód.