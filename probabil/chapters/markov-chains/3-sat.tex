2-sata rozwiązaliśmy wolniej, ale 3-sat jest już problemem NP-zupełnym, więc jest większe pole do usprawnień. Z drugiej strony, raczej nie spodziewamy się lepszego wyniku niż wykładniczego, nawet w randomizowanym przypadku -- świat byłby wtedy zbyt idealny.

Pokażemy dwa algorytmy -- pierwszy będzie działał tak jak 2-sat, drugi będzie z lekkim twistem.

\subsection{Naiwny algorytm}
\label{3-sat-naive-algorithm}
Niech \( \lambda \in \natural \) będzie parametrem algorytmu (stałą czasu działania).
\begin{enumerate}
	\item Wylosuj dowolne wartościowanie zmniennych \( x_1, \dots, x_n \)
	\item Powtarzaj maksymalnie \( \lambda \) razy lub do znalezienia rozwiązania
	      \begin{enumerate}
		      \item Wylosuj niespełnioną klazulę
		      \item Wylosuj literał z tej klauzuli i odwróć wartość zmiennej tego literału
	      \end{enumerate}
	\item Jeśli mamy wartościowanie to je zwracamy
	\item W przeciwnym razie orzekamy, że formuła jest niespełnialna
\end{enumerate}

\subsection{Czas działania naiwnego algorytmu}
\begin{theorem}
	Jeśli dana jest spełnialna formuła to algorytm \ref{3-sat-naive-algorithm} wykonuje w oczekiwaniu \( O(2^n) \) kroków zanim ją znajdzie.
\end{theorem}
\begin{proof}

	Sytuacja ma się dość podobnie jak w 2-sacie -- mamy ciąg \( X_0, X_1, \dots \) zmiennych
	liczących zgodność z ustalonym wartościowaniem \( S \).
	Tylko oszacowania są nieco gorsze bo spośród trzech literałów, dwa mogą się zgadzać, a tylko jeden nie.
	\[
		P(X_{i+1} = j + 1 \mid X_i = j) \geq \frac{1}{3}
	\]
	\[
		P(X_{i+1} = j - 1 \mid X_i = j) \leq \frac{2}{3}
	\]

	Zatem pesymistyczny przypadek modelujemy ciągiem \( Y_0, Y_1, \dots \)
	\[
		Y_0 = X_0
	\]
	\[
		P(Y_{i+1} = 1 \mid Y_i = 0) = 1
	\]
	\[
		P(Y_{i+1} = j + 1 \mid Y_i = j) = \frac{1}{3}
	\]
	\[
		P(Y_{i+1} = j - 1 \mid Y_i = j) = \frac{2}{3}
	\]

	Ponownie, niech \( h_i \) oznacza oczekiwaną liczbę kroków zaczynając w stanie \( i \).
	Dostajemy układ równań
	\[
		\begin{cases}
			h_0 = h_1 + 1                                           \\
			h_i = 1 + \frac{2}{3} h_{i - 1} + \frac{1}{3} h_{i + 1} \\
			h_n = 0
		\end{cases}
	\]
	Przekształcamy środek do
	\[
		h_{i + 1} = 3h_i - 2h_{i - 1} - 3
	\]

	a następnie rozwiązujemy indukcyjnie
	\begin{align*}
		h_1 & = h_0 - 1                                                     \\
		h_2 & = 3h_1 - 2h_0 - 3 = h_1 - 5  = h_1 - (8 - 3)                  \\
		h_3 & = 3h_2 - 2h_1 - 3 = h_2 - 2 \cdot(8 - 3) - 3 = h_2 - (16 - 3)
	\end{align*}
	Widzimy, że wyjdzie
	\begin{align*}
		h_{i+1} & = h_i - (2^{i+2} - 3)   \\
		h_i     & = h_{i+1} + 2^{i+1} - 3
	\end{align*}

	W takim razie mamy
	\begin{align*}
		h_n     & = 0                                                                         \\
		h_{n-1} & = h_n + 2^{n+1} - 3 = 2^{n+1} - 3                                           \\
		h_{n-2} & = h_{n-1} + 2^n - 3 = 2^{n+1} + 2^n - 2\cdot 3                              \\
		h_i     & = 2^{n+1} + 2^n + \dots + 2^{i+2} - 3(n - i) = 2^{n+2} - 2^{i+2} - 3(n - i)
	\end{align*}

	Trochę do bani -- równie dobrze można sprawdzić wszystkie wartościowania w \( O(n^2 \cdot 2^n) \)
\end{proof}

Naiwny algorytm ma pewną wadę -- będzie się kręcił blisko wartościowań, które mają mało wspólnych zmiennych, z prostej przyczyny, że łatwiej popsuć niż poprawić.

Zauważmy jednak, że nic nie stoi na przeszkodzie aby zresetować algorytm tj. wylosować nowe wartościowanie i zacząć szukać od niego.
Losowe wartościowanie ma średnio połowę zmiennych ustawionych poprawnie, więc z dużą szansą jest bliżej rozwiązania niż wartościowanie naiwnego algorytmu po długim czasie.

\subsection{Sprytny algorytm}
\label{3-sat-better-algorithm}

Niech \( \lambda \in \natural \) będzie parametrem algorytmu (stałą czasu działania).
\begin{enumerate}
	\item Powtarzaj maksymalnie \( \lambda \) razy lub do znalezienia rozwiązania
	      \begin{enumerate}
		      \item Wylosuj dowolne wartościowanie zmniennych \( x_1, \dots, x_n \)
		      \item Powtarzaj maksymalnie \( 3n \) razy lub do znalezienia rozwiązania
		            \begin{enumerate}
			            \item Wylosuj niespełnioną klazulę
			            \item Wylosuj literał z tej klauzuli i odwróć wartość zmiennej tego literału
		            \end{enumerate}
	      \end{enumerate}
	\item Jeśli mamy wartościowanie to je zwracamy
	\item W przeciwnym razie orzekamy, że formuła jest niespełnialna
\end{enumerate}

\subsection{Czas działania sprytnego algorytmu}
Zaczynamy od lematu, który przyjmujemy na wiarę (jego dowód to przykra analiza)
\begin{lemma}[Wzór Stirlinga]
	\[
		\forall n > 0: n! = \sqrt{2\pi n}\cdot\pars{\frac{n}{e}}^n \cdot (1 \pm o(1))
	\]
	W szczególności
	\[
		\forall n > 0: \sqrt{2\pi n}\cdot\pars{\frac{n}{e}}^n \leq n! \leq 2\sqrt{2\pi n}\cdot\pars{\frac{n}{e}}^n
	\]
\end{lemma}
Używamy tego lematu aby pokazać
\begin{align*}
	\binom{3i}{i}
	 & = \frac{(3i)!}{i!(2i)!}                                                               \\
	 & \geq \frac{\sqrt{2\pi 3i}}{\pars{2\sqrt{2 \pi i}} \cdot \pars{2\sqrt{2 \pi 2i}}}
	\cdot \pars{\frac{3i}{e}}^{3i} \cdot \pars{\frac{e}{i}}^i \cdot \pars{\frac{e}{2i}}^{2i} \\
	 & = \frac{\sqrt{3}}{8\sqrt{\pi}\sqrt{i}} \cdot \pars{\frac{27}{4}}^i                    \\
	 & = \frac{c}{\sqrt{i}}\cdot \pars{\frac{27}{4}}^i
\end{align*}
Gdzie \( c \) jest stałą odsyfiającą wyrażenie.

\begin{theorem}
	Algorytm \ref{3-sat-better-algorithm} wykonuje w oczekiwaniu \( O\pars{n^{3/2} \cdot \pars{\frac{4}{3}}^n} \) kroków zanim znajdzie poprawne wartościowanie.
\end{theorem}
\begin{proof}
	Chcemy oszacować od dołu prawdopodobieństwo \( q \), że po co najwyżej 3n krokach, z losowego wartościowania znajdziemy poprawne.

	Niech \( q_i \) oznacza prawdopodobieństwo, że startując z wartościowania, któremu brakuje dokładnie \( i \) zmiennych to po 3n krokach dojdziemy do poprawnego wartościowania.

	\( q_i\) szacujemy od dołu pesymistycznym przypadkiem, który szacujemy przez możliwe rozłożenia kroków ,,w górę'' i ,,w dół''
	\begin{align*}
		q_i
		 & \geq \max_{k = 0, \dots, i} \binom{i + 2k}{k} \cdot \pars{\frac{2}{3}}^k\cdot\pars{\frac{1}{3}}^{i + k}   \\
		 & \geq \binom{3i}{i} \cdot \pars{\frac{2}{3}}^i\cdot\pars{\frac{1}{3}}^{2i}                                 \\
		 & \geq \frac{c}{\sqrt{i}}\cdot \pars{\frac{27}{4}}^i \cdot \pars{\frac{2}{3}}^i\cdot\pars{\frac{1}{3}}^{2i} \\
		 & = \frac{c}{\sqrt{i}}\cdot2^{-i}
	\end{align*}

	Niech \( \mathcal{E}_i \) oznacza zdarzenie, w którym wylosowane wartościowanie jest \textbf{różne} od zadanego \( S \) na dokładnie \( i \) pozycjach.


	Mamy zatem
	\begin{align*}
		q
		 & = \sum_{i=0}^n P(\mathcal{E}_i) \cdot q_i                                                                      \\
		 & = 2^{-n} + \sum_{i=1}^n \binom{n}{i}\pars{\frac{1}{2}}^n \cdot q_i                                             \\
		 & \geq \sum_{i=1}^n \binom{n}{i}\pars{\frac{1}{2}}^n \cdot \frac{c}{\sqrt{i}}\cdot2^{-i}                         \\
		 & = \sum_{i=1}^n \binom{n}{i}\pars{\frac{1}{2}}^{n+i} \cdot \frac{c}{\sqrt{i}}                                   \\
		 & \geq \frac{c}{\sqrt{n}} \cdot \pars{\frac{1}{2}}^n \sum_{i=1}^n \binom{n}{i}\pars{\frac{1}{2}}^i \cdot 1^{n-i} \\
		 & = \frac{c}{\sqrt{n}} \cdot \pars{\frac{1}{2}}^n \cdot \pars{\frac{3}{2}}^n                                     \\
		 & = \frac{c}{\sqrt{n}} \cdot \pars{\frac{3}{4}}^n                                                                \\
	\end{align*}

	Najgorsze za nami.

	\( q \) to jest prawdopodobieństwo, że w konkretnym bloku (dla konkretnego wartościowania startowego) nam się udało.
	Bloki są niezależne, więc całość ma rozkład geometryczny z parametrem \( q \),
	zatem w oczekiwaniu musimy przejść przez \( \frac{1}{q} = O\pars{\sqrt{n} \cdot \pars{\frac{4}{3}}^n}\) bloków, a każdy blok zajmuje czas \( O(n) \) co daje tezę.
\end{proof}