Zanim zaczniemy, zaprezentujemy dwa proste lematy potrzebne w oszacowaniu
\begin{lemma}
	Dla dowolnych \( n \geq M \)
	\[
		\binom{n}{M} \pars{\frac{1}{n}}^M \leq \frac{1}{M!}
	\]
\end{lemma}
\begin{proof}
	\[
		\binom{n}{M} \pars{\frac{1}{n}}^M = \frac{n!}{M!\cdot(n-M)!n^M} =
		\frac{1}{M!}\cdot\frac{(n-M+1)\cdot \dots \cdot n}{n^M} \leq \frac{1}{M!}
	\]
\end{proof}

\begin{lemma}
	Dla dowolnego \( n \)
	\[
		\frac{1}{n!} \leq \pars{\frac{e}{n}}^n
	\]
\end{lemma}
\begin{proof}
	Korzystamy z rozwinięcia \( e^k \) w szereg Taylora:
	\[
		e^k = \sum_{i=0}^\infty \frac{k^i}{i!} > \frac{k^k}{k!}
	\]
	Przekształcając otrzymujemy
	\[
		\frac{e^k}{k^k} > \frac{1}{k!}
	\]
	co daje nierówność z tezy.
\end{proof}

Rozważmy bardzo prosty model - wrzucamy sobie \( n \) kul do \( n \) urn niezależnie i jednostajnie.
Oczywiście średnio w jednej urnie spodziewamy się zobaczyć jedną kulę, ale ile spodziewamy się zobaczyć kul w najbardziej zapełnionej urnie?
Na to pytanie odpowiemy twierdzeniem.
\begin{theorem}[Lemat 5.1 P\&C]
	\label{balls-and-bins-max-load-upper-bound}
	Jeśli wrzucamy \( n \) kul do \( n \) urn to prawdopodobieństwo, że najcięższa urna zawiera
	\textbf{co najmniej} \( M = \frac{3 \ln n}{\ln \ln n} \) kul wynosi co najwyżej \( \frac{1}{n} \) dla odpowiednio dużych \( n \).
\end{theorem}
\begin{proof}
	Nie ma co się zrażać mnogością logarytmów; sam w sobie dowód jest względnie prosty -- stosujemy dwa razy \textit{union-bound} a ograniczenie z tezy po prostu pałujemy naszymi lematami a egzaminie raczej nie będziecie potrzebowali obliczeń.


	Prawdopodobieństwo, że ustalony podzbiór \( M \) kul wyląduje w ustalonej urnie wynosi
	\( \pars{\frac{1}{n}}^M \)
	Różnym podzbiorów jest \( \binom{n}{M} \), zatem z union bounda dostajemy ograniczenie na prawdopodobieństwo, że w ustalonej urnie jest co najmniej \( M \) kul wynosi
	\[
		\binom{n}{M} \cdot \pars{\frac{1}{n}}^M
	\]
	Korzystamy teraz z obu lematów i ograniczamy prawdopodobieństwo na co najmniej \( M \) kul w urnie przez co najwyżej
	\[
		n\pars{\frac{e}{M}}^M
	\]

	Teraz wstawiamy magiczne \( M \) z tezy i dostajemy:
	\[
		n\pars{\frac{e}{M}}^M \leq n\pars{\frac{e \ln \ln n}{3 \ln n}}^{(3 \ln n)/(\ln \ln n)}
	\]
	Zauważamy, że \( e \leq 3 \)
	\[
		\leq n\pars{\frac{\ln \ln n}{\ln n}}^{(3 \ln n)/(\ln \ln n)}
	\]
	Aby pokazać postulowaną w tezie nierówność bierzemy obustronnie logarytm
	\begin{align*}
		n\pars{\frac{\ln \ln n}{\ln n}}^{(3 \ln n)/(\ln \ln n)}                      & \leq \frac{1}{n} \\
		\ln n + \pars{(\ln \ln \ln n) - (\ln \ln n)}\pars{\frac{3 \ln n}{\ln \ln n}} & \leq -\ln n
	\end{align*}
	Wymnażamy i przenosimy na jedną stronę
	\[
		-\ln n + \frac{3 (\ln n)(\ln \ln \ln n)}{\ln \ln n} \leq 0
	\]
	Sprowadzamy do wspólnego mianownika
	\[
		\frac{(\ln n) \cdot \pars{3(\ln \ln \ln n) - (\ln \ln n)}}{\ln \ln n} \leq 0
	\]
	Ponieważ \( \ln n \) i \(\ln \ln n\) są od pewnego momentu dodatnie
	to nierówność sprowadza się do pokazania, że
	\[
		\ln \ln n \geq 3 \ln \ln \ln n
	\]
	co już jest trywialne.

\end{proof}