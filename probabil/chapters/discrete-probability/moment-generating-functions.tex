\begin{definition}
	\textbf{Funkcję tworzącą momenty} zmiennej losowej \( X \) definiujemy jako
	\[
		M_X(t) = \expected{e^{tX}}
	\]
\end{definition}

\begin{theorem}[Twierdzenie 4.1  P\&C]
	Jeśli \( M_X(t) \) tworzy momenty zmiennej \( X \) to
	\[
		\expected{X^n} = M_X^{(n)}(0)
	\]
\end{theorem}
\begin{proof}
	Zakładamy tutaj, że możemy zamieniać kolejnością operatory różniczkowania i wartości oczekiwanej.
	To założenie działa jeśli tworząca istnieje blisko zera i okazuje się, że zachodzi dla rozkładów, którymi się będziemy zajmować.

	\[
		M_X^{(n)}(t) = \expected{e^{tX}}^{(n)}
		= \expected{X^n e^{tX}}
	\]
	\[
		M_X^{(n)}(0) = \expected{X^n}
	\]
\end{proof}

\begin{theorem}[Twierdzenie 4.3 P\&C]
	Dla niezależnych zmiennych losowych \( X \) oraz \( Y \) zachodzi

	\[
		M_{X + Y}(t) = M_X(t) \cdot M_Y(t)
	\]
\end{theorem}
\begin{proof}
	\[
		M_{X + Y}(t) = \expected{e^{t(X + Y)}} = \expected{e^{tX}\cdot e^{tY}} = \expected{e^{tX}}\expected{e^{tX}} = M_X(t) \cdot M_Y(t)
	\]
\end{proof}
