%<*probabil-egzamin-16-gamma-1>
\begin{definition}
    Funkcją gamma nazywamy funkcję:
    \[
        \Gamma(a) = \int_0^{\infty} x^a e^{-x} \frac{\diff x}{x} = \int_0^{\infty} x^{a-1} e^{-x} \diff x
    \]
    dla \( a > 0 \).
    Powyższe dwie notacje są równoważne, my będziemy korzystać z tej pierwszej.
\end{definition}

Parę faktów o funkcji gamma:
\begin{fact}
    \( \Gamma(1) = 1 \)
\end{fact}
\begin{proof}
    \[
        \Gamma(1) = \int_0^{\infty} x e^{-x} \frac{\diff x}{x} = \int_0^{\infty} e^{-x} \diff x = \left[-e^{-x} \right]_0^{\infty} = 1
    \]
\end{proof}

\begin{fact}
    \( \forall_{a > 0} \Gamma(a+1) = a\Gamma(a) \)
\end{fact}
\begin{proof}
    \[
        \Gamma(a+1) = \int_0^{\infty} x^a e^{-x} \diff x = \left[ -x^a e^{-x}  \right]_0^{\infty} + a\int_0^{\infty} x^{a-1} e^{-x} \diff x = 0 + a\Gamma(a) = a\Gamma(a)
    \]
\end{proof}

\begin{fact}
    \( \forall_{n \geq 1} \Gamma(n) = (n-1)! \) \\
    Wynika to bezpośrednio z poprzedniego faktu.
\end{fact}

\begin{definition} 
    Mówimy, że ciągła zmienna losowa ma \textbf{Rozkład Gamma} z parametrem \( (a, 1) \), jeżeli jej funkcja gęstości jest równa:
    \[
        f(x) = \frac{1}{\Gamma(a)} x^a e^{-x} \frac{1}{x}
    \]
    Wtedy:
    \[
        \int_0^{\infty} \frac{1}{\Gamma(a)}x^a e^{-x} \frac{\diff x}{x} = \frac{1}{\Gamma(a)} \int_0^{\infty} x^a e^{-x} \frac{\diff x}{x} = \frac{1}{\Gamma(a)} \cdot \Gamma(a) = 1
    \]
    Zatem jest to poprawny rozkład.
\end{definition}

\begin{definition}
	Dla \(X \sim \Gam(a, 1)\) oraz \(\lambda > 0\) definiujemy \(Y \sim \Gam(a, \lambda)\) jako
	\[
		Y = \frac{X}{\lambda}
	\]
\end{definition}

\begin{theorem}
	Funkcja gęstości \(Y \sim \Gam(a, \lambda)\) jest równa
	\[
        f(x) = \frac{1}{\Gamma(a)} (\lambda x)^a e^{-\lambda x} \frac{1}{x}
    \]
\end{theorem}
\begin{proof}
	Niech \( X \sim \Gam(a, 1), Y = \frac{X}{\lambda} \). Liczymy gęstość
    \[
        f_Y(y) = f_X(x) \cdot \abs{\frac{\diff x}{\diff y}} = \frac{1}{\Gamma(a)}(\lambda y)^a e^{-\lambda y} \frac{1}{\lambda y} \lambda = \frac{1}{\Gamma(a)}(\lambda y)^a e^{-\lambda y} \frac{1}{y}
    \]
\end{proof}
%</probabil-egzamin-16-gamma-1>

\begin{fact}
    Dla \( X \sim \Gam(a, 1)\) zachodzi
    \[ 
        \expected{X} = a 
    \] 
    \[ 
        \expected{X^2} = a(a+1) 
    \]
\end{fact}
\begin{proof}
    \[
        \expected{X} = \int_0^{\infty} x \frac{1}{\Gamma(a)} x^a e^{-x} \frac{\diff x}{x} = \frac{1}{\Gamma(a)} \int_0^{\infty} x^{a+1} e^{-x} \frac{\diff x}{x} = \frac{\Gamma(a+1)}{\Gamma{a}} = \frac{a\Gamma(a)}{\Gamma(a)} = a
    \]
    \[
        \expected{X^2} = \int_0^{\infty} x^2 \frac{1}{\Gamma(a)} x^a e^{-x} \frac{\diff x}{x} = \frac{1}{\Gamma(a)} \int_0^{\infty} x^{a+2} e^{-x} \frac{\diff x}{x} = \frac{\Gamma(a+2)}{\Gamma(a)} = a(a+1)
    \]
\end{proof}

\begin{fact}
    Dla \( Y \sim \Gam(a, \lambda)\) (czyli \(Y = \frac{X}{\lambda} \) dla \( X \sim \Gam(a, 1)\)) zachodzi
    \[
        \expected{Y} = \expected{\frac{X}{\lambda}} = \frac{1}{\lambda}\expected{X} = \frac{a}{\lambda}
    \]
\end{fact}

%<*probabil-egzamin-16-gamma-2>
\begin{fact}
	\[
		\Gam(1, 1) \equiv \Exp(1)
	\]
    \[
		\Gam(1, \lambda) \equiv \Exp(\lambda)	
	\]
\end{fact}

\begin{theorem}
    Niech \(X_1, \ldots, X_n\) - niezależne zmienne losowe o rozkładzie wykładniczym z parametrem \( \lambda \). Wtedy:\\
    \( X_1 + \ldots + X_n \sim \) Gamma(\(n, \lambda\)).
\end{theorem}
\begin{proof}
    Skorzystamy z twierdzenia \ref{generating-function-equality}. Policzymy funkcję tworzącą dla sumy \(X_i\) oraz dla \(Y \sim \) Gamma(\(n, \lambda\)).
    \[
        M_{X_i}(t) = \expected{e^{tX_i}} = \int_0^{\infty} e^{tx} \lambda e^{-\lambda x} \diff x = \frac{\lambda}{\lambda - t}
    \]
    \[
        X = \sum_{i \in [n]}X_i \implies M_X(t) = \prod_{i \in [n]} M_{X_i}(t) = \pars{\frac{\lambda}{\lambda - t}}^n
    \]
    \begin{align*}
        M_Y(t) &= \expected{e^{tY}} = \int_0^{\infty} e^{tY} \frac{1}{\Gamma(n)} (\lambda y)^n \frac{\diff y}{y} \\
        &= \frac{\lambda^n}{(\lambda - t)^n}\int_0^{\infty} \frac{1}{\Gamma(n)} e^{-(\lambda - t)y}((\lambda - t)y)^n \frac{\diff y}{y} \\
        &= \pars{\frac{\lambda}{\lambda - t}}^n \frac{\Gamma(n)}{\Gamma(n)} = \pars{\frac{\lambda}{\lambda - t}}^n
    \end{align*}
    Obie funkcje tworzące są równe, więc zmienne \( X \) i \( Y \) mają ten sam rozkład.
\end{proof}
%</probabil-egzamin-16-gamma-2>
