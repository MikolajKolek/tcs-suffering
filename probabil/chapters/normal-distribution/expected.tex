%<*probabil-egzamin-22-wartosc-oczekiwana-wariancja>
%<*probabil-2025-12-19-wartosc-oczekiwana>
Najpierw wyliczmy wartość oczekiwaną standardowego rozkładu normalnego.
\begin{theorem}
	Wartość oczekiwana standardowego rozkładu normalnego wynosi 0, wariancja wynosi 1.
\end{theorem}

\begin{proof}
	Wartość oczekiwana wynosi 0, ponieważ standardowy rozkład normalny jest symetryczny wobec prostej \( OY\)
	Wariancja:
	\[
		\variance{Z} = \expected{Z^2} - \expected{Z}^2 = \expected{Z^2} = \\
	\]
	ponieważ \( \expected{Z} = 0 \)
	\[
		= \frac{1}{\sqrt{2\pi}}\int_{-\infty}^{z}t^2e^{-t^2/2} dt =
	\]
	\[
		= \frac{1}{\sqrt{2\pi}}\int_{-\infty}^{z}(t)(te^{-t^2/2}) dt =
	\]
	całkowanie przez części
	\[
		-\frac{1}{\sqrt{2\pi}}te^{-t^2/2}|_{-\infty}^{\infty} + \frac{1}{\sqrt{2\pi}}\int_{-\infty}^{\infty}e^{-t^2/2} dt = 1
	\]
	Ponieważ pierwszy wyraz jest równy 0 a drugi jest to dystrybuanta na od \(-\infty \) do \( \infty \) więc wynosi ona 1.
\end{proof}
%</probabil-2025-12-19-wartosc-oczekiwana>
%</probabil-egzamin-22-wartosc-oczekiwana-wariancja>

%<*probabil-egzamin-22-uogolniony-rozklad-dwumianowy>
%<*probabil-2025-12-19-uogolniony-rozklad-dwumianowy>
\begin{lemma}
	Zmienna losowa ma rozkład normalny wtedy i tylko wtedy gdy jest transformacją liniową zmiennej losowej o standardowym rozkładzie normalnym
\end{lemma}

\begin{proof}
	Ponieważ zmienna losowa \( X\) z \(N(\mu, \sigma^2) \) ma ten sam rozkład co \( \sigma Z + \mu \) mamy że
	\[\expected{X} = \expected{\sigma Z + \mu} = \mu
	\]
	\[
		\variance{X} = \variance{\sigma Z + \mu} = \sigma^2
	\]
\end{proof}

czyli ten \sout{dzban} dzwon ma efektywnie przesunięcie o \( \mu \).

Stąd dostaniemy także dystrybuantę i gęstość:

\begin{align*}
	F_X(x) = \prob\pars{X \leq x} = \prob\pars{\frac{X-\mu}{\sigma} \leq \frac{x-\mu}{\sigma}} = \Phi\pars{\frac{x-\mu}{\sigma}}\\
	f_X(x) = \frac{1}{\sqrt{2\pi}\sigma}e^{-\frac{\pars{x-\sigma}^2}{2\sigma^2}}
\end{align*}

Możemy także wyznaczyć funkcję tworzącą momenty zmiennej o uogólnionym rozkładzie normalnym. Dokładne wyliczenie całki pozostawiamy dla czytelnika.

\begin{align*}
	\mathcal{M}_X(t) = \expected{e^{tX}} = \frac{1}{\sqrt{2\pi}\sigma}\int_{-\infty}^{\infty}e^{tX}e^{-\frac{\pars{X-\mu}^2}{2\sigma^2}}dx\\
	= \dots = e^{\frac{t^2\sigma^2}{2}+\mu t}
\end{align*}

\begin{theorem}
	Niech \(X \sim N(\mu_1, \sigma^2_1), Y \sim N(\mu_2, \sigma^2_2)\) to niezależne zmienne losowe. Wtedy \(X+Y \sim N(\mu_1+\mu_2, \sigma_1^2+\sigma_2^2)\).
\end{theorem}
\begin{proof}
	\begin{align*}
		\mathcal{M}_{X+Y}(t) = \pars{\mathcal{M}_X(t)}\pars{\mathcal{M}_Y(t)} = \pars{e^{\frac{t^2\sigma_1^2}{2}+\mu_1 t}}\pars{e^{\frac{t^2\sigma_2^2}{2}+\mu_2 t}} =\\
		= e^{\frac{t^2(\sigma_1^2+\sigma_2^2)}{2} + t(\mu_1 + \mu_2)}
	\end{align*}
\end{proof}

Podobnie możemy pokazać, że dla niezależnych \(X_1 \sim N(0, \sigma_1^2), X_2 \sim N(0, \sigma_2^2)\) dostajemy \(X_1+X_2 \sim N(0, \sigma_1^2+\sigma_2^2)\) oraz \(X_1-X_2 \sim N(0, \sigma_1^2+\sigma_2^2)\).
%</probabil-2025-12-19-uogolniony-rozklad-dwumianowy>
%</probabil-egzamin-22-uogolniony-rozklad-dwumianowy>
