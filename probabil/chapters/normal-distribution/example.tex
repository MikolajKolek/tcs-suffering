Przykład z książki i wykładu rozkładu normalnego: Detekcja Sygnału (p. 245)

\begin{exercise}
\end{exercise}
Przyjmijmy że mamy transmiter, który wysyła bit zakodowany za pomocą
\( S \in {-1, +1}\). Ale nie żyjemy w idealnym świecie i dodawany jest szum \( Y\), który jest zmienną losową o normalnym rozkładzie ze średnią 0 i odchyleniem standardowym \( \sigma \). \\
Sygnał ten jest odbierany i dekodowany w zależności jaki znak danego bita odbierzemy. \[ R = \sgn(S + Y)\]
Chcemy znaleźć prawdopodobieństwo, że otrzymany bit będzie inny niż ten, który nadamy (szum zmieni znak) \(P(R \neq S)\).\\

Prawdopodobieństwo, że błąd wyskoczy dla \( S = 1\) jest równe prawdopodobieństwu, że \( Y \leq -1 \)
\[
	P(Y \leq -1) = P \left( \frac{Y - \mu}{\sigma} \leq \frac{-1 - \mu}{\sigma} \right) = \Phi\left(-\frac{1}{\sigma}\right)
\]
Dla \( S = -1 \) mamy podobnie (to jest symetryczne btw)
\[
	P(Y \geq 1) = 1 - P \left( \frac{Y - \mu}{\sigma} \leq \frac{1 - \mu}{\sigma} \right) = 1 - \Phi\left(\frac{1}{\sigma}\right)
\]
Ponieważ symetria btw \( \Phi(-\frac{1}{\sigma}) = 1 - \Phi(\frac{1}{\sigma})\) \\ \\
Wynik wynosi zatem \( 2(1 - \Phi(\frac{1}{\sigma})\)

Dalej musimy odczytać z tabeli (BOOOOOOOOOORING)