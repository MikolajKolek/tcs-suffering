\begin{definition}
	\textbf{Algebra} to takie \(\Sigma \subset 2^{\Omega}\) dla danego uniwersum \(\Omega\), że spełnione są następujące warunki
	\begin{itemize}
		\item \(\emptyset, \Omega \in \Sigma\)
		\item \(e \in \Sigma \implies \Omega \setminus e \in \Sigma\)
		\item \(E_1, \dots, E_n \in \Sigma \implies \bigcup_{i=1}^n E_i \in \Sigma\)
	\end{itemize}
\end{definition}

\begin{definition}
	\textbf{\(\sigma\)-algebra} to algebra spełniająca dodatkowy warunek
	\[
		E_1, \dots \in \Sigma \implies \bigcup_{i=1}^{\infty} E_i \in \Sigma
	\]
\end{definition}

\begin{example}
	Proste przykłady algebr i \(\sigma-algebr\)
	\begin{itemize}
		\item \(\Sigma = 2^{\Omega}\) - \(\sigma\)-algebra
		\item \(\Sigma = \set{\emptyset, \Omega}\) - \(\sigma\)-algebra
		\item \(\Omega = \real, \Sigma = \set{\bigcup_{i=1}^n (a_i, b_i] \mid a_i, b_i \in \real \cup \set{-\infty, +\infty}, a_i < b_i}\) - algebra, ale nie \(\sigma\)-algebra, bo np. \(\bigcup_{i=1}^\infty (0, 1 - \frac{1}{n}] = (0, 1) \notin \Sigma\)
	\end{itemize}
\end{example}

\begin{lemma}
	Niech \(\Omega\) to uniwersum, a \((\Sigma_\alpha)_{\alpha \in I}\) to \(\sigma\)-algebry na \(\Omega\). Wtedy
	\[
		\bigcap_{\alpha \in I} \Sigma_\alpha \text{ jest \(\sigma\)-algebrą}
	\]
\end{lemma}
% TODO: dowód (nie był podany na wykładzie)

\begin{definition}
	\textbf{\(\sigma\)-algebra generowana przez \(F \subset 2^\Omega\)} to
	\[
		\sigma(F) = \bigcap_{\substack{\Sigma \text{ to } \sigma\text{-algebra}\\F \subset \Sigma}} \Sigma
	\]
	
	Intuicyjnie, jest to najmniejsza możliwa \(\sigma\)-algebra zawierająca w sobie \(F\)
\end{definition}

\begin{example}
	Przykład \(\sigma\)-algebry generowanej
	\[
		\sigma(\set{[a, b] \mid a, b \in \real, a \leq b}) := B(\real)
	\]
	\[
		\sigma(d\text{-wymiarowe kostki}) := B(\real^d)
	\]
	\(B\) to oznaczenie na zbiory borelowskie, które są przykładem zbiorów mierzalnych, nie mając problemów takich jak Banach-Tarski.
	
	\textbf{Uwaga. } Od teraz, gdy będziemy mówić o \(\real, \mathbb{C}\) w kontekscie miary, mamy na myśli przestrzenie mierzalne \((\real, B(\real)), (\mathbb{C}, B(\mathbb{C}))\), o ile nie jest powiedziane inaczej.
\end{example}
