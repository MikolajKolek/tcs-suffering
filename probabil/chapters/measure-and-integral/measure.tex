\begin{definition}
	\textbf{Przestrzeń mierzalna} to para \((\Omega, \Sigma)\) dla uniwersum \(\Omega\) i \(\Sigma\) będącego \(\sigma\)-algebrą na \(\Omega\).
\end{definition}

\begin{definition}
	\textbf{Miara} dla danej przestrzeni mierzalnej \((\Omega, \Sigma)\) to funkcja \(\mu: \Sigma \to [0, +\infty]\) spełniająca
	\begin{itemize}
		\item \(\mu(\emptyset) = 0\)
		\item \(E_1, E_2, \dots \in \Sigma \text{ parami rozłaczne} \implies \mu(\bigcup_{i=1}^\infty E_i}) = \sum_{i=1}^{\infty} \mu(E_i)\)
	\end{itemize}
\end{definition}

\begin{definition}
	\textbf{Przestrzeń z miarą} to trójka \((\Omega, \Sigma, \mu)\) gdzie \((\Omega, \Sigma)\) tworzą przestrzeń mierzalną, a \(\mu\) jest miarą na tej przestrzeni mierzalnej.
\end{definition}

\begin{definition}
	Funkcja \(f: \Omega_1 \to \Omega_2\) jest \textbf{mierzalna} dla przestrzeni probabilistycznych \((\Omega_1, \Sigma_1), (\Omega_2, \Sigma_2)\) gdy
	\[
		\forall_{B \in \Sigma_2} f^{-1}(B) \in \Sigma_1
	\]
\end{definition}

\begin{definition}
	\textbf{Miara probabilistyczna} to miara dla której \(\mu(\Omega) = 1\)
\end{definition}

\begin{lemma}
	Niech \(A_1, A_2, \dots, \in \Sigma, \forall_{i \in \natural_+} A_i \subset A_{i+1}\). Wtedy
	\[
		\mu(\bigcup_{i=1}^\infty A_i) = \lim_{x \to \infty} \mu(A_i)
	\]
\end{lemma}
% TODO: dowód (nie był podany na wykładzie)

\begin{lemma}
	Niech \(A_1, A_2, \dots, \in \Sigma, \forall_{i \in \natural_+} A_i \supset A_{i+1}\) oraz \(\mu(A_1) < \infty\). Wtedy
	\[
		\mu(\bigcap_{i=1}^\infty A_i) = \lim_{x \to \infty} \mu(A_i)
	\]
\end{lemma}
% TODO: dowód (nie był podany na wykładzie)

\begin{example}
	Miara licząca
	
	Miara licząca dla \(\Omega < \infty\) to \(\mu(A) = |A|\)
\end{example}

\begin{example} 
	Miara Lebesgue'a
	
	Miara Lebesgue'a to uogólnienie pojęcia długości, pola powierzchni i objętości na większą liczbę wymiarów. Przykładowo, w jednym wymiarze:
	\[
		\mu([a, b]) = b - a
	\]
	a więc
	\[
		\mu(\set{a}) = \mu(\bigcap_{n=1}^\infty [a, a + \frac{1}{n}]) = \lim_{n \to \infty} \frac{1}{n} = 0
	\]
	\[
		\mu(\rational) = \mu(\set{q_1, q_2, \dots}) = \mu(\bigcup_{n=1}^\infty \set{q_1, \dots, q_n}) = \lim_{n \to \infty} \mu(\set{q_1, \dots, q_n}) = 0
	\]
\end{example}
