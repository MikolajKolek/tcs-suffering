\subsection{Algebra i \texorpdfstring{\(\sigma\)}{sigma}-algebra}

\begin{definition}
	\textbf{Algebra} to takie \(\Sigma \subset 2^{\Omega}\) dla danego uniwersum \(\Omega\), że spełnione są następujące warunki
	\begin{itemize}
		\item \(\emptyset, \Omega \in \Sigma\)
		\item \(e \in \Sigma \implies \Omega \setminus e \in \Sigma\)
		\item \(E_1, \dots, E_n \in \Sigma \implies \bigcup_{i=1}^n E_i \in \Sigma\)
	\end{itemize}
\end{definition}

\begin{definition}
	\textbf{\(\sigma\)-algebra} to algebra spełniająca dodatkowy warunek
	\[
		E_1, \dots \in \Sigma \implies \bigcup_{i=1}^{\infty} E_i \in \Sigma
	\]
\end{definition}

\begin{example}
	Proste przykłady algebr i \(\sigma-algebr\)
	\begin{itemize}
		\item \(\Sigma = 2^{\Omega}\) - \(\sigma\)-algebra
		\item \(\Sigma = \set{\emptyset, \Omega}\) - \(\sigma\)-algebra
		\item \(\Omega = \real, \Sigma = \set{\bigcup_{i=1}^n (a_i, b_i] \mid a_i, b_i \in \real \cup \set{-\infty, \infty}, a_i < b_i}\) - algebra, ale nie \(\sigma\)-algebra, bo np. \(\bigcup_{i=1}^\infty (0, 1 - \frac{1}{n}] = (0, 1) \notin \Sigma\)
	\end{itemize}
\end{example}

\begin{lemma}
	Niech \(\Omega\) to uniwersum, a \((\Sigma_\alpha)_{\alpha \in I}\) to \(\sigma\)-algebry na \(\Omega\). Wtedy
	\[
		\bigcap_{\alpha \in I} \Sigma_\alpha \text{ jest \(\sigma\)-algebrą}
	\]
\end{lemma}
% TODO: dowód (nie był podany na wykładzie)

\begin{definition}
	\textbf{\(\sigma\)-algebra generowana przez \(F \subset 2^\Omega\)} to
	\[
		\sigma(F) = \bigcap_{\substack{\Sigma \text{ - } \sigma\text{-algebra}\\F \subset \Sigma}} \Sigma
	\]
	
	Intuicyjnie, jest to najmniejsza możliwa \(\sigma\)-algebra zawierająca w sobie \(F\)
\end{definition}

\begin{example}
	Przykład \(\sigma\)-algebry generowanej
	\[
		\sigma(\set{[a, b] \mid a, b \in \real, a \leq b}) := B(\real) \text{ (zbiory borelowskie)}
	\]
	Zbiory borelowskie są przykładem zbiorów mierzalnych, nie mając problemów takich jak Banach-Tarski.
\end{example}


\subsection{Definicja miary}

\begin{definition}
	\textbf{Przestrzeń mierzalna} to para \((\Omega, \Sigma)\) dla uniwersum \(\Omega\) i \(\Sigma\) będącego \(\sigma\)-algebrą na \(\Omega\).
\end{definition}

\begin{definition}
	\textbf{Miara} dla danej przestrzeni mierzalnej \((\Omega, \Sigma)\) to funkcja \(\mu: \Sigma \to [0, \infty]\) spełniająca
	\begin{itemize}
		\item \(\mu(\emptyset) = 0\)
		\item \(E_1, E_2, \dots \in \Sigma \text{ parami rozłaczne} \implies \mu(\bigcup_{i=1]^\infty E_i}) = \sum_{i=1}^{\infty} \mu(E_i)\)
	\end{itemize}
\end{definition}

\begin{definition}
	\textbf{Miara probabilistyczna} to miara dla której \(\mu(\Omega) = 1\)
\end{definition}

\begin{lemma}
	Niech \(A_1, A_2, \dots, \in \Sigma, \forall_{i \in \natural_+} A_i \subset A_{i+1}\). Wtedy
	\[
		\mu(\bigcup_{i=1}^\infty A_i) = \lim_{x \to \infty} \mu(A_i)
	\]
\end{lemma}
% TODO: dowód (nie był podany na wykładzie)

\begin{lemma}
	Niech \(A_1, A_2, \dots, \in \Sigma, \forall_{i \in \natural_+} A_i \supset A_{i+1}\) oraz \(\mu(A_1) < \infty\). Wtedy
	\[
		\mu(\bigcap_{i=1}^\infty A_i) = \lim_{x \to \infty} \mu(A_i)
	\]
\end{lemma}
% TODO: dowód (nie był podany na wykładzie)

\begin{example}
	Miara licząca
	
	Miara licząca dla \(\Omega < \infty\) to
	
	\[
		\mu(A) = |A|
	\]
\end{example}
\begin{example} 
	Miara Lebesgue'a
	
	Miara Lebesgue'a to uogólnienie pojęcia długości, pola powierzchni i objętości na większą liczbę wymiarów. Przykładowo, w jednym wymiarze:
	\[
		\mu([a, b]) = b - a
	\]
	a więc
	\[
		\mu(\set{a}) = \mu(\bigcap_{n=1}^\infty [a, a + \frac{1}{n}]) = \lim_{n \to \infty} \frac{1}{n} = 0
	\]
	\[
		\mu(\rational) = \mu(\set{q_1, q_2, \dots}) = \mu(\bigcup_{n=1}^\infty \set{q_1, \dots, q_n}) = \lim_{n \to \infty} \mu(\set{q_1, \dots, q_n}) = 0
	\]
\end{example}
