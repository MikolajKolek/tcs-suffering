Na początek skupmy się na prostym problemie: mamy \(\Omega\) - uniwersum, oraz \(A \subset \Omega\). Losujemy jednostajnie element \(x \in \Omega\) i chcemy poznać \(\prob(x \in A)\).

Naturalnie nasuwa się definicja
\[
	\(\prob(x \in A)\) = \frac{\text{wielkość}(A)}{\text{wielkość}(\Omega)}
\]
Jak możemy jednak zdefiniować wielkość? Dla \(|\Omega| < \infty\) jest to proste: wielkość\((X) = |X|\), lecz dla \(\Omega\) nieskończonego nie ma już oczywistej odpowiedzi. Na ratunek przychodzi pojęcie miary.

\begin{example} Przykładowe miary
	\begin{itemize}
		\item Długość odcinka
		\item Pole powierzchni na płaszczyźnie
		\item Objętość w przestrzeni trójwymiarowej
	\end{itemize}
\end{example}

Okazuje się jednak, że nie wszystkie zbiory można mierzyć.
\begin{example}
	Paradoks Banacha-Tarskiego
	
	Niech \(S\) to będzie sfera w przestrzeni trójwymiarowej. Możemy podzielić \(S\) na rozłączne części w następujący sposób
	\[
		S = P_1 \cup P_2 \cup P_3 \cup P_4 \cup P_5
	\]
	a następnie za pomocą ustalonych przesunięć \(\tau_1, \tau_2, \tau_3, \tau_4, \tau_5\) dojść do
	\[
		S = \tau_1(P_1) \cup \tau_2(P_2)
	\]
	\[
		S = \tau_3(P_3) \cup \tau_4(P_4) \cup \tau_5(P_5)
	\]
	
	Gdybyśmy chcieli mierzyć objętość naszej sfery i pozwalalibyśmy na podzielenie jej na części i wykonanie przesunięć doszlibyśmy więc do oczywiście sprzecznego wniosku
	\[
		\text{Objętość}(S) = 2 \cdot \text{Objętość}(S) 
	\]
\end{example}

Widzimy, że nie we wszystkich przypadkach intuicyjne pojęcie miary jest poprawne i nie wszystkie zbiory są mierzalne. W następnej sekcji skupimy się na znalezieniu zbiorów, które możemy mierzyć i dokładnie zdefiniujemy miarę.