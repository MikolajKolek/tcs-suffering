\subsection{Definicja}

W różnych sytuacjach, zwłaszcza w informatyce, mamy do czynienia z kolejkami do których przychodzą zdarzenia w pewnym sensie losowo, i które są potem obsługiwane również względnie losowo.
Kolejka Markowa służy właśnie do modelowania takich procesów.

Na opisanie modelu kolejki będziemy stosować notację \( Y / Z / n \), gdzie:
\begin{enumerate}
	\item \( Y \) opisuje rozkład z jakim przychodzą zdarzenia (klienci)
	\item \( Z \) opisuje rozkład czasu jaki zajmuje obsłużenie pojedynczego klienta
	\item \( n \) mówi ile mamy wątków, które jednocześnie będą obsługiwać zapytania z kolejki
\end{enumerate}

\subsection{Kolejka \( M / M / 1 \) }
\( M \) w naszej definicji oznacza \textit{Memoryless}.
Innymi słowy, nowi klienci przychodzą zgodnie z procesem Poissona z parametrem \( \lambda \)
a czas obsługi (przez jedyny wątek) ma rozkład wykładniczy z parametrem \( \mu \).

Liczbę klientów oczekujących w kolejce możemy modelować procesem Markowa
\( \set{ M_t \mid t \geq 0 } \)

Niech \( P_k(t) = P(M_t = k) \) oznacza prawdopodobieństwo na to, że obecnie w kolejce mamy \( k \) klientów.

\begin{theorem}[strona 231 P\&C]
	Niech \( \bar \pi \) będzie rozkładem stacjonarnym naszego procesu. Wtedy
	\[ \pi_k = \pars{1 - \frac{\lambda}{\mu}}\pars{\frac{\lambda}{\mu}}^k \]
\end{theorem}

Oczekiwana liczba klientów w kolejce wynosi
\begin{align*}
	C
	 & = \sum_{k=0}^{\infty} k \pi_k                                                                                  \\
	 & = \frac{\lambda}{\mu} \sum_{k=1}^\infty k \cdot \pars{1 - \frac{\lambda}{\mu}}\pars{\frac{\lambda}{\mu}}^{k-1} \\
	 & = \frac{\lambda}{\mu} \cdot \frac{1}{1 - \lambda / \mu}                                                        \\
	 & = \frac{\lambda}{\mu - \lambda}
\end{align*}
W drugim przejściu zauważamy, że wygląda to jak rozkład geometryczny z parametrem \( p = 1 - \frac{\lambda}{\mu} \),
którego wartość oczekiwana wynosi \( \frac{1}{p} \)

\subsection{Kolejka M/M/1/K}
Ograniczamy rozmiar naszej kolejki przez \( K \) -- każdy klient ponad nadmiar jest odsyłany z kwitkiem.
Jaki teraz mamy rozkład stacjonarny?

\begin{theorem}[strona 233 P\&C]
	Rozkład stacjonarny kolejki ograniczonej przez \( K \) ma postać:
	\[
		\pi_0 = \frac{1}{\sum_{k=0}^K \pars{\lambda/\mu}^k}
	\]
	\[
		\pi_k = \begin{cases}
			\pi_0 \cdot \pars{\frac{\lambda}{\mu}}^k & \text{ gdy } k \leq K \\
			0                                        & \text{ gdy } k > K
		\end{cases}
	\]

\end{theorem}

\subsection{Ćwiczenia}
\begin{exercise}
	Do małego kantoru przychodzą klienci zgodnie z procesem Poissona w tempie 0.2 klienta na minutę.
	Średni czas obsługi klienta zajmuje 2 minuty.
	Ponieważ pomieszczenie jest niewielkie to kolejka mieści maksymalnie dwóch klientów (nie licząc obecnie obsługiwanego), jeśli kolejka jest pełna to
	nowy klient udaje się gdzie indziej.

	\begin{enumerate}
		\item Zamodeluj ten proces jako proces Markowa -- wyznacz graf stanów i macierz przejścia.
		\item Wyznacz rozkład stacjonarny procesu oraz jego łańcucha
		\item Asymptotycznie jaką część czasu kolejka jest pełna?
		\item Asymptotycznie jaka część klientów uda się gdzie indziej z uwagi na pełną kolejkę?
		\item W tym momencie w kolejce czeka dokładnie jeden klient. Jakie jest prawdopodobieństwo, że zostanie obsłużony zanim przyjdzie kolejna osoba?
	\end{enumerate}
\end{exercise}
\begin{proof} \( \)
	\begin{enumerate}
		\item
		      Na początek warto przetłumaczyć opis słowny na model matematyczny.
		      Mamy dany proces Poissona z parametrem \( \lambda = 0.2 \)

		      Zakładamy, że czas przetwarzania ma rozkład wykładniczy (bo w sumie nie przerabialiśmy żadnej innej opcji)
		      i skoro oczekiwany czas wynosi 2 to musi mieć on parametr \( \mu = \frac{1}{2} \)


		      Będziemy mieli trzy możliwe stany \( 0, 1, 2, 3 \) na możliwe liczby osób w kantorze (wliczając w to osobę obsługiwaną) między którymi przejścia
		      opisuje macierz
		      \[
			      \mathbf{P} = \begin{bmatrix}
				      0 & 1 & 0     & 0     \\
				      p & 0 & 1 - p & 0     \\
				      0 & p & 0     & 1 - p \\
				      0 & 0 & 1     & 0
			      \end{bmatrix}
		      \]

		      Wyjaśnijmy co tu się dzieje:
		      \begin{itemize}
			      \item Jeśli w kantorze jest 0 klientów to jedyne co się
			            może zdarzyć to przyjść nowy, i w ten sposób mamy jednego klienta z prawdopodobieństwem 1

			      \item Kiedy mamy jednego lub dwóch klientów to są dwie opcje -- albo przychodzi nowy klient i kolejka rośnie o 1 albo obsłużony zostaje obecny klient i kolejka maleje o 1.
			            \( p \) jest tutaj prawdopodobieństwem, że kolejka maleje, czyli że zanim przyjdzie kolejna osoba to kończymy obsługiwać obecną.

			      \item Jeśli kantor jest pełny (czyli są w nim trzy osoby) to jedyne co może się stać, to może zostać obsłużony klient i zwolnić jedno miejsce.
		      \end{itemize}

		      Pasowałoby policzyć \( p \) - czyli prawdopodobieństwo, że obsłużymy klienta zanim przyjdzie kolejny.
		      Mamy dwie zmienne wykładnicze \( X \) z parametrem \(\lambda\) oraz \( Y \) z parametrem \( \mu \). \( X \) liczy czas do przyjścia kolejnego klienta a \( Y \) czas do końca obsługi obecnego.

		      \( p = P(\min(X, Y) = Y) = \frac{\mu}{\mu + \lambda} = \frac{0.5}{0.7} = \frac{5}{7} \)

		      Potrzebujemy jeszcze wyznaczyć parametry \( \lambda_0, \lambda_1, \lambda_2, \lambda_3 \) dla czasów spędzonych w każdym ze stanów.

		      Łatwo pokazać, że \[
			      \begin{cases}
				      \lambda_0 = \lambda = 0.2       \\
				      \lambda_1 = \lambda + \mu = 0.7 \\
				      \lambda_2 = \lambda + \mu = 0.7 \\
				      \lambda_3 = \mu = 0.5           \\
			      \end{cases}
		      \]

		\item
		      Dla procesu rozwiązujemy układ równań
		      \[
			      \begin{cases}
				      \pi_0 \lambda_0 = \pi_1 \lambda_1 \cdot p                                 \\
				      \pi_1 \lambda_1 = \pi_0 \lambda_0 \cdot 1 + \pi_2 \lambda_2 \cdot p       \\
				      \pi_2 \lambda_2 = \pi_1 \lambda_1 \cdot (1 - p) + \pi_3 \lambda_3 \cdot 1 \\
				      \pi_3 \lambda_3 = \pi_2 \lambda_2 \cdot (1 - p)                           \\
				      \pi_0 + \pi_1 + \pi_2 + \pi_3 = 1
			      \end{cases}
		      \]

		      Dla łańcucha jest podobnie, tylko bez lambd
		      \[
			      \begin{cases}
				      \pi'_0 = \pi'_1 \cdot p                        \\
				      \pi'_1 = \pi'_0 \cdot 1 + \pi'_2 \cdot p       \\
				      \pi'_2 = \pi'_1 \cdot (1 - p) + \pi'_3 \cdot 1 \\
				      \pi'_3 = \pi'_2 \cdot (1 - p)                  \\
				      \pi'_0 + \pi'_1 + \pi'_2 + \pi'_3 = 1
			      \end{cases}
		      \]

		\item Sprawdzamy \( \pi_3 \) wyliczone w poprzednim punkcie

		\item Klienci widzą stan kolejki zgodny z rozkładem stacjonarnym, zatem asymptotycznie \( \pi_3 \) klientów zobaczy pełną kolejkę.

		\item jest to dokładnie \( p \)

	\end{enumerate}
\end{proof}