Na początku powiedzmy sobie jasno: \textit{kill 0} ubija wszystkie procesy w grupie procesów. To oznacza, że w outpucie litera \texttt{C} pojawi się maksymalnie 2 razy (zanim dzieci obu równoległych procesów ubiją całą grupę).

Może jeszcze być taka heca, że \texttt{fork} się nie udał i zwrócił nam -1, a \texttt{kill -1} strzela do \textbf{wszystkich} w systemie. Na szczęście w naszym scenariuszu wszyscy to te same procesy, które ubija \texttt{kill 0}.

Oczywiście na początku pojawi się tylko jedna litera \texttt{A}. Pozostaje pytanie o to, co dzieje się z literami \texttt{B} -- musi pojawić się co najmniej jedna, ale, podobnie, mogą być też dwie.

Prowadzi nas to do następujących możliwości (hopefully żadnej nie pominęliśmy): \begin{enumerate}
	\item \texttt{AB}
	\item \texttt{ABB}
	\item \texttt{ABC}
	\item \texttt{ABBC}
	\item \texttt{ABBCC}
	\item \texttt{ABCB}
	\item \texttt{ABCBC}

\end{enumerate}

W pewnym momencie dziecko zabije wszystkich, którzy jeszcze żyją. Pytanie tylko kiedy.