Bo jest to kod biblioteki dynamicznie ładowanej (\texttt{so} oznacza Shared Object) więc może być tak, że wiele procesów naraz
używa tego kodu.
Pisanie po współdzielonej pamięci nie jest natomiast zbyt dobrym pomysłem.

Moglibyśmy dać każdemu procesowi osobną kopię, ale to nam zjada pamięć, bo te biblioteki nie są współdzielone bez powodu (są duże).