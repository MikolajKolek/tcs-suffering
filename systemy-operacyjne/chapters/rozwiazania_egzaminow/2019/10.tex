\textit{page fault frequency} mówi nam jak często programy odwołują się do ,,nie swojej' pamięci.
To znaczy -- obiecaliśmy im, że mają pamieć (pomijając przypadki że programista sam strzelił sobie w stopę, i wyleciał poza stronę i zarobił segfaultem na twarz), ale w celach optymalizacyjnych nie przydzieliśmy im fizycznie stron w pamięci. Oczywiście procesor nakrzyczy na program, że co on odwala i wtedy system operacyjny powinien grzecznie przeprosić i oddać programowi jego pamięć.

Ponadto, gdy przydzielimy już fizycznie jakąś pamięć programowi, ale ten od dłuższego czasu jej nie używał (a nam zaczyna brakować RAMu) możemy zrobić drobny fikołek i zapisać stan tej strony na dysk twardy, a pamięć fizyczną udostępnić innemu procesowi (który jej potrzebuje na teraz). Gdy proces który ordynarnie okradliśmy będzie chciał danych które sobie tam zapisał to się odwoła do tej pamięci w swojej przestrzeni adresów; wówczas memory management unit z CPU się oburzy i wywali page fault systemowi, a system spojrzy na ten wyjątek i powie ,,no tak, śmieszna sprawa, już oddaję tę pamięć'', weźmie to co miał na dysku i odda stronę okradzionemu procesowi.

Podczas gdy ten mechanizm umożliwia nam pracę nawet wtedy, gdy programy wymagają sumarycznie więcej RAMu niż my go mamy, jeśli wszyscy aktywnie korzystają ze swojej pamięci to takie page faulty będą latać bardzo często i w sumie to nasze sztuczki będą powodować straszne mulenie. Stąd też fajnie by system obserwował ten wskaźnik; jeśli zobaczy, że page fault frequency jest zbyt duże, po prostu wywali jakiś proces do swapa (no bo inaczej się już dać nie będzie).