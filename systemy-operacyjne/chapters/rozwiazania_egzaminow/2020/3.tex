Generalnie jak przeczytamy ten kod, to zauważymy, że pytanie jest o to, czy i kiedy \texttt{alarm} nas ubije (dziecka nie ubije, załączamy dokumentację forka).

Cytując dokumentację:
\begin{displayquote}
	alarm() arranges for a SIGALRM signal to be delivered to the
	calling process in seconds seconds.

	If seconds is zero, any pending alarm is canceled.

	In any event any previously set alarm() is canceled.
\end{displayquote}
oraz
\begin{displayquote}
	Alarms created by alarm() are preserved across execve(2) and are not inherited by children created via fork(2).
\end{displayquote}

W takim razie nieważne ile każemy spać dziecku, wypisze ono dwie kropki -- jedną wewnątrz ifa, a drugą zaraz przed returnem.

Jeśli ustawimy alarm na wystarczająco późno (tak, aby oba sleepy się wykonały a dziecko zakończyło), to rodzic wypisze swoją kropkę i łącznie zaobserwujemy trzy. W przeciwnym razie na wyjściu będą tylko dwie kropki dziecka.