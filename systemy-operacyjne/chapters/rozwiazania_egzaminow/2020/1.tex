Zdefiniujmy sobie radośnie oznaczenia:
\begin{enumerate}
	\item $p$ -- PID programu
	\item $q$ -- PID dziecka tego programu
	\item $r$ -- PID reapera
	\item $x$ -- PID rodzica naszego programu
\end{enumerate}

Pierwsza linijka jest zawsze taka sama: \(p, x\).

Znaczy, HAHA ŻARTUJEMY, bufor z printfa może się skopiować do dziecka jeśli się nie wypisało i będziemy mieć straszny syf. Nie rozpatrujemy tu tego przypadku.

Co do pozostałych dwóch, będą one wyglądać różnie zależnie od tego czy rodzic zdążył już spaść z rowerka przed dzieckiem i kto pierwszy co wypisał. Należy zauważyć, że teoretycznie reaper może być również rodzicem naszego programu. Wtedy $r$ i $x$ to to samo. Super.

Możemy zatem zaobserwować:

\texttt{p x \\ 0 q p \\ q p x}

albo

\texttt{p x \\ q p x \\ 0 q p}

albo

\texttt{p x \\ q p x \\ 0 q r}