\section{Wstęp}
Mamy na celu stworzyć serwer na Minixie, który implementuje mechanizm barier.

Bariera to taki cosiek, na którym procesy mogą czekać w oczekiwaniu na inne procesy.
Konkretniej -- bariera ma ustaloną szerokość \( w \). Jeśli na barierze czeka co najmniej \( w \) procesów
to wszystkie zostają wznowione. W przeciwym razie procesy nadal czekają.

\section{Podstawa dzialania}

Sama implementacja syscalli jest generalnie prosta i bazuje na jednej, fundamentalnej i trywialnej obserwacji: proces czekający na odpowiedź na wysłaną przez siebie wiadomość za pomocą mechanizmu \textit{sendrec} jest zablokowany do momentu otrzymania odpowiedzi. Jest to bardzo przydatne, bo to oznacza że nas serwer z barierami dostając syscalla mówiącego ,,ej zawieś mnie na takiej barierze'' po prostu nie musi na nią odpowiadać żeby proces sobie wisiał. W momencie gdy uznamy, że czas ,,puścić'' wszystkie procesy z danej bariery, po prostu posyłamy do każdego z nich jakąś wiadomość.

Można się zapytać, jak powinniśmy trzymać informacje o tym, które procesy są w jakiej barierze. Nasuwa się tu zwykła lista linkowana, ale po stronie systemu staramy się unikać wywołań funkcji \textit{malloc}. Wynika to z faktu, że to generuje potencjał na deadlock, bo wymaga zadzwonienia do jakiegoś serwera i zablokowania się (a jednocześnie sami jesteśmy serwerem i szybko może stać się głupia sytuacja). Korzystamy więc z istnienia makra \texttt{NR\_PROCS}, które jest ustalone na jakąś małą liczbę (presumably, 256). To oznacza, że możemy zaimplementować listę kursorowo, dowalając sobie ordynarnie tablicą na \texttt{NR\_PROCS} elementów. Fajnie.

Do dokladniejszej implementacji przejdziemy w kolejnych sekcjach.

\section{Strona użytkownika}

Od strony użytkownika implementacja wygląda tak, jak to robiliśmy w przypadku poprzednich zadań.
\begin{itemize}
	\item Dodajemy nagłówek \texttt{/usr/src/include/barriers.h}, w którym są definicje zadanych funkcji.
	\item Aktualizujemy \texttt{/usr/src/include/Makefile} o tenże nagłówek
	\item W pliku \texttt{/usr/src/lib/libc/sys-minix/barriers.c} piszemy implementację tych funkcji.

	      Ponieważ nie znamy adresu naszego serwera to będziemy potrzebowali wywołać \texttt{minix\_rs\_lookup} z nagłówka \texttt{<minix/rs.h>}.

	      Wykonujemy \texttt{\_syscall}\footnote{Dla fanów sportów ekstremalnych informujemy, że można po prostu użyć również gołego \textit{sendrec}-a. Nie powiemy kto tak zrobił, ale jego imię zaczyna się na ,,D'', a kończy na ,,ominik''. } i mapujemy errno z \texttt{EDEADSRCDST} na \texttt{EDEADEPT} żeby zgadzało się z opisem zadania.
	\item Aktualizujemy \texttt{/usr/src/lib/libc/sys-minix/Makefile.inc}
	\item opcjonalnie dodajemy sobie odpowiednie makra w \texttt{/usr/src/include/minix/com.h} -- nie trzeba wtedy pamiętać liczb syscalli ani pól wiadomości, w których znajdują się informacje.
\end{itemize}

\section{Serwer barier}
Warto zacząć od poczytania implementacji serwera \texttt{ipc} -- jest ona dość podobna do tego co my potrzebujemy zrobić. Generalnie to
\begin{enumerate}
	\item Podpinamy się do SEF-a -- \texttt{env\_setargs(argc, argv); sef\_startup();}
	\item W pętli odbieramy wiadomość -- \texttt{message m; sef\_receive(ANY, \&m);}
	\item Sprawdzamy \texttt{m.call\_type} i na jego podstawie wywołujemy odpowiednią funkcję lub stwierdzamy, że komuś się coś pomieszało.
	\item Implementujemy podstawowe mechanizmy
	      \begin{itemize}
		      \item \texttt{do\_barrier\_init} po prostu inicjalizuje odpowiednią barierę i od razu odpowiada czy się udało czy nie
		      \item \texttt{do\_barrier\_destroy} podobnie po prostu deinicjalizuje barierę (o ile nikt na niej nie czeka) i od razu odpowiada
		      \item \texttt{do\_barrier\_wait} dodaje proces do kolejki bariery, ale nie odpowiada mu.
		            Zamiast tego jeśli kolejka jest pełna to budzimy wszystkie procesy, które się w niej znajdują wysyłając do nich pustą wiadomość, której \texttt{m\_type} wynosi \texttt{OK}.
	      \end{itemize}

\end{enumerate}

Warto jeszcze może doprecyzować jak trzymać wszystkie te informacje sensownie.
Ponieważ będziemy musieli potem rozmawiać z procfs-em to polecamy taką implementację:

W pliku \texttt{/usr/src/servers/barriers/structures.h} definiujemy \texttt{\#BARRIERS 256}
oraz dwie struktury :

Strukturę \texttt{barrier}, która trzyma:
\begin{itemize}
	\item szerokość
	\item liczbę oczekujących procesów
	\item pierwszy element kolejki
\end{itemize}

Oraz strukturę \texttt{item}, która trzyma:
\begin{itemize}
	\item endpoint procesu
	\item id bariery na której on czeka
	\item następny proces
	\item poprzedni proces
\end{itemize}

Dane będziemy przechowywać w dwóch tablicach, których elementami będą powyższe struktury.

Tak jak powiedzieliśmy wcześniej -- procesy trzymane są jako listy kursorowe tj. zamiast wskaźników trzymamy indeks w tablicy. Mając endpoint procesu, jego indeksem będzie slot w tabeli procesów, który możemy wyłuskać makrem \texttt{\_ENDPOINT\_P}.

Aby uprościć sobie smutną implementację listy wiązanej warto zaimplementować kolejkę jako listę cykliczną z wartownikiem.

\section {Jak dogadać się z PMem}
Wszystko fajnie, ale w wymaganiach jest napisane, że serwer barier w przypadku gdy ktoś kto czekał w barierze i oberwał sygnałem ma przestać na niej czekać (analogicznie do tego, jak oberwanie sygnałem przerywa reada). Jak to realizujemy? Istnieje sobie serwer \textit{PM}, który zajmuje się właśnie sygnałami i takimi pięknymi rzeczami. W związku z powyższym, wystarczyłoby się tutaj ,,wstrzelić'' gdzieś PMowi do odpowiedniej funkcji i kazać mu informować nasz serwer, że jakiś proces oberwał właśnie sygnałem. Pojawia się tu spoiler: odpowiednie miejsce do takich akcji znajduje się w pliku \texttt{signal.c} (kto by pomyślał?), a funkcja która się tym zajmuje to \textit{unpause}. Można sobie spojrzeć na jej implementację -- jest tam już mechanizm dzwonienia do \textit{VFS}a w bardzo analogicznej sprawie.

Możemy wręcz skopiować funkcję \texttt{tell\_vfs}, która wysyła wiadomość, tyle że do naszego serwera.
Odbierając wiadomość potrzebujemy sprawdzić czy jest ona od PMa i wywalić przerwany/zabity proces oraz wysłać mu wiadomość z errno \texttt{EINTR}.

Jeśli proces spadł z rowerka również chcemy poinformować o tym barriersa -- funkcja która się wywołuje, gdy proces oberwał sygnałem i umarł to \textit{sig\_proc\_exit}.
Można też skipnąć zbędne gadanie i pójść bezpośrednio do \texttt{proc\_exit}, na jedno wyjdzie. Chyba.

Oczywiście pozostaje pytanie jak zrobić, żeby \textit{PM} wiedział, na jaki endpoint zadzwonić jeśli chodzi o serwer barier. Są różne podejścia:

\begin{enumerate}
	\item funkcja \textit{minix\_rs\_lookup}, która dzwoni do \textit{RS}a i prosi go o endpoint danego serwera: zły plan, prosto doprowadzić do deadlocku
	\item dodanie do \textit{barriers}a kodu, który każe mu dzwonić do \textit{PM}a gdy \textit{barriers} wstaje, aby podać mu swój endpoint (i żeby \textit{PM} wiedział do kogo dzwonić). Wtedy też musimy jakoś informować \textit{PM}a o tym, że \textit{barriers} jest wyłączany, żeby nie próbował dzwonić do serwera który ktoś włączył, a potem wyłączył
	\item wyszukiwanie \textit{barriersa} na pałę w tablicy procesów \textit{PM}a; wbrew pozorom nie jest to aż taki zły pomysł. While we're at it, przy iterowaniu przez tabelę procesów \textbf{zawsze sprawdzajcie flagę IN\_USE}. Inaczej zaczniecie łapać dane o procesach które zostały ubite; w szczególności możecie po resecie barriersa dowiedzieć się, że \textit{PM} źle adresuje wiadomości. Fun.
\end{enumerate}

\section{Procfs}
Kiedy wywołujemy \texttt{cat /proc/barriers} to naszym oczom powinno okazać się eleganckie podsumowanie stanu aktywnych barier.

Idziemy zatem do \texttt{/usr/src/servers/procfs/root.c} i tam się bawimy:
\begin{enumerate}
	\item Do tablicy \texttt{root\_files} dopisujemy zgodnie ze wzorem odpowiednią linijkę
	\item kopiujemy definicję dwóch tablic ze strukturami barier i procesów\
	\item W funkcji \texttt{root\_barriers}
	      \begin{enumerate}
		      \item Szukamy serwera barier w tabeli \texttt{mproc}
		      \item Wywołujemy fajną funkcję \texttt{getsysinfo} dwa razy -- raz na tablicę barier, drugi na tablicę procesów
		      \item Jeśli wszystko dobrze poszło to możemy wypisać te dwie struktury używając funkcji \texttt{buf\_printf}
	      \end{enumerate}
\end{enumerate}

Dobra, ale \texttt{getsysinfo} samo nie zadziała (w końcu to tylko syscall, który wysyła wiadomość)\footnote{Ponownie, możemy tu też zamiast bawić się w jakieś owijki zrobić własnego sendreca i być hepi}, musimy jeszcze przekonać serwer barier, żeby na nie odpowiadał.

Wracamy więc do serwera barier i sprawdzamy przypadkiem czy nie dostaliśmy wiadomości o typie \texttt{COMMON\_GETSYSINFO}.
Jeśli tak to odpowiadamy na nie w dość prosty sposób.
Aby wysłać jakąś tablicę \texttt{T} do gościa, który nas pyta wywołujemy po prostu:

\texttt{sys\_datacopy(SELF, (vir\_bytes)T, m.m\_source, (vir\_bytes)m.SI\_WHERE, m.SI\_SIZE)}

Sprawdzamy więc o którą tablicę jesteśmy pytani i odsyłamy mu jej zawartość.
Tak jak wcześniej wpisywaliśmy makra w \texttt{com.h} tak tutaj warto w pliku\\
\texttt{/usr/src/include/minix/sysinfo.h} dopisać sobie makra na informacje wyciągane od naszego serwera.
