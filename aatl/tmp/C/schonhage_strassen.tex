\subsection{Teoretyczna Transformata Numeryczna (NTT)}
Transformata NTT (\textit{Number Theoretic Transform}) jest wersją FFT, która nie używa liczb rzeczywistych -- zamiast nad \( \real \), obliczenia są wykonywane nad \( \integer_p \).
Wzory rekurencyjne i własność wzajemnej odwrotności DFT i IDFT przenoszą się na pierścień \( \integer_p \). Jedynym warunkiem jest, żeby \( \omega \in \integer_p \) spełniało \( \omega^n = 1 \) oraz  \( \omega^{\frac{n}{2}} = -1 \), czyli żeby \( \omega \) było rzędu \( n \).
Z twierdzenia Lagrange'a oraz faktu, że \( \integer_p \) jest cykliczna wiemy, że taki element istnieje, jeśli \( n \mid (p - 1) \). Algorytm FFT modulo \( p \) (czyli właśnie NTT) działa dla \( p \) postaci \( q \cdot n + 1 \).

\begin{greyframe}
    Splatanie ciągów \( A = (a_0, a_1, \ldots, a_n) \) i \( B = (b_0, b_1, \ldots, b_n) \):
    \begin{enumerate}
        \item Znajdź \( p \) postaci \( q \cdot n + 1 \), gdzie \( q \) jest niedużą liczbą pierwszą -- można wylosować i sprawdzić pierwszość lub szukać po kolei.
        \item Znajdź generator \( g \) grupy \( \integer^{*}_p \) -- generatorów jest \( \varphi(q \cdot n) = (q - 1) \cdot \frac{n}{2} \), więc losowanie zadziała.
        Generator musi spełniać \( g^{\frac{p-1}{2}} \neq 1 \pmod{p} \) oraz \( g^{\frac{p-1}{q}} \neq 1 \pmod{p} \).
        \item Ustal \( \omega = g^q \) rzędu \( n \) i oblicz splot za pomocą NTT.
    \end{enumerate}
\end{greyframe}
Można wybrać na tyle duże \( p \), żeby modulo nie miało wpływu na wynik. Przy mnożeniu liczb binarnych wystarczy \( p > n \).

\subsection{Algorytm Sch{\"o}nhage-Strassena}
Algorytm służy do mnożenia dużych liczb w zapisie binarnym. Ponieważ daje on zysk praktyczny dopiero w przypadku \emph{naprawdę} dużych liczb, musi radzić sobie z problemami wynikającymi z długości ich zapisu.

Pierwszym z wyzwań jest wybór pierścienia do obliczeń -- jeśli liczba \( p \) jest duże, to operacje w \( \integer_p \) nie są jednostkowe.

Liczba \( p \) nie musi być pierwsza, wystarczy nam względna pierwszość z \( n \), (potrzebne jest dzielenie przez \( n \)) i musi istnieć element \( \omega \) rzędu \( n \) modulo \( p \).
Te warunki spełniają \( p = 2^m + 1 \), \\ \( \omega = 2^{\alpha} \) takie, żeby \( \omega^n = 1 \pmod{p} \). Sprytny dobór liczb upraszcza obliczenia:
\begin{itemize}
    \item branie reszty modulo \( p \) -- zauważamy, że \( 2^m = p - 1 = -1 \pmod{p} \), więc możemy podzielić liczbę na \( m \)-cyfrowe fragmenty i dodać je z naprzemiennymi znakami:
    \[
        101001110_2 \pmod{2^3 + 1} = 101_2 - 001_2 + 110_2 \pmod{2^3 + 1},
    \]
    \item mnożenie przez \( \omega = 2^{\alpha} \) -- dopisujemy zera na końcu liczby binarnej i wyciągamy resztę modulo \( p \),
    \item dzielenie przez \( \omega = 2^{\alpha} \) -- mnożymy przez \( \omega^{-1} = -2^{m-\alpha} \), czyli też potęgę 2.
    \item mnożenie dwóch liczb w \( \integer_p \) -- nadal trudne, ale wykonujemy je niewiele razy i można sobie z nim poradzić rekurencją na liczbach o długości \( m \),
\end{itemize}

Algorytm Sch{\"o}nhage-Strassena służy do obliczania iloczynu \( N \)-bitowych liczb \( A \), \( B \) modulo \( p = 2^N + 1 \).
Jeśli chce się otrzymać wynik pełny (nie modulo), wystarczy przyjąć \( N \) większe od sumy długości liczb na wejściu. Zakładamy, że \( N \) jest potęgą 2. Algorym działa rekurencyjnie, sprowadzając iloczyn \( A \) i \( B \) do splotu wielomianów w mniejszym pierścieniu.

\begin{greyframe}
    Algorytm fastmul(\( A, B, N \)):
    \begin{enumerate}
        \item Rozbijamy liczby \( A \) i \( B \) na fragmenty o długości około \( \sqrt{N} \), czyli na \( t \) binarnych liczb \( b \)-cyfrowych. Liczby \( t, \ b \) powinny być potęgami dwójki wielkości mniej więcej \( \sqrt{N} \).
        \[
            A = u_0 + u_1 \cdot 2^b + u_2 \cdot 2^{2b} + \ldots + u_{t-1} \cdot  2^{(t-1)b}
        \]
        \[
            B = v_0 + v_1 \cdot 2^b + v_2 \cdot 2^{2b} + \ldots + v_{t-1} \cdot 2^{(t-1)b}
        \]
        Zadanie sprowadza się do pomnożenia dwóch liczb \( t \)-cyfrowych w systemie o podstawie \( 2^b \).

        \item Wykonujemy algorytm NTT w pierścieniu \( \integer_p \) dla \( p = 2^{b'} + 1 \). Wymaga to obliczenia \( 2t \) iloczynów modulo \( 2^{b'} + 1 \) rekurencyjnie.
    \end{enumerate}
\end{greyframe}

W punkcie 2 splatamy liczby w pierścieniu \( \integer_p \) dla \( p = 2^{b'} + 1 \), gdzie \( b' \approx 2b + \log t \) oraz \( t \mid b' \). Bierze się to stąd, że wynik splotu nie przekracza \( p \), czyli \( p > t \cdot \pars{2^b}^2 \), więc \( b' \geq 2b + \log t \).
Podzielność \( t \mid b' \) jest ważna, żeby rząd \( \omega = 2^{\frac{2b'}{t}} \) był równy \( t \), czyli \( \omega^{\frac{t}{2}} = 2^{b'} = -1 \pmod{p} \).

Algorytm można opisać ogólnym równaniem
\[
    T(n) = 2 \sqrt{n} \cdot T(\alpha \sqrt{n}) + \Theta(n \log n)
\]
dla pewnego całkowitego \( \alpha \).

Sprowadzamy mnożenie dwóch liczb \( N \)-bitowych do około \( 2\sqrt{N} \) mnożeń liczb o długości \( 4\sqrt{N} \), co daje złożoność \( \bigO(N \log^2 N) \).
Poprawienie jej do \( \bigO(N \log N \log \log N) \) wymaga zmniejszenia stałych \( (2, 4) \rightarrow (1, 2) \) technicznymi sztuczkami.
        
W praktyce rekursja jest płytka, ponieważ dla \( N < 10^3 \) lepiej użyć prostszych algorytmów, na przykład Tooma-Cooka albo nawet kwadratowego mnożenia.
