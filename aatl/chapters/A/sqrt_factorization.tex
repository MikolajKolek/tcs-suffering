Pomysł opiera się na tym, że możemy sprowadzić faktoryzację \( n \) do szukania nietrywialnych rozwiązań równania \( x^2 = y^2 \mod n \).

\begin{lemma}
	Niech \( n \) będzie nieparzystą liczbą złożoną, nie potęgą liczby pierwszej. Dla dowolnego \( s \) równanie:
	\[
		x^2 = s \pmod{n},
	\]
	jeśli ma jakiekolwiek rozwiązanie, to ma co najmniej 4 różne.
\end{lemma}
\begin{proof}
	Przedstawmy liczbę \( n \) jako \( n = p \cdot q \), gdzie \( p, \ q \) są względnie pierwsze. Jeśli \( x \) jest rozwiazaniem równania postaci \( x^2 = s \pmod{p} \), to \( -x \) również. Zapiszmy kongruencje:
	\[
		x^2 = s \pmod{p}
	\]
	\[
		x^2 = s \pmod{q}
	\]
	Z Chińskiego Twierdzenia o Resztach wynika, że rozwiązań oryginalnego równania jest co najmniej tyle, ile wynosi iloczyn liczby rozwiązań poszczególnych równań, czyli \( 2 \cdot 2 = 4 \).
\end{proof}
To wystarcza, żeby skonstruować algorytm.
\begin{greyframe}
	Algorytm na faktoryzację:
	\begin{enumerate}
		\item Wylosuj \( x < n \) i oblicz \( s = x^2 \pmod{n} \).
		\item Oblicz \( y = \textcolor{Blue}{\text{sqrt}(s, n)} \).
		\item Jeśli \( x \neq \pm y \), to \( \gcd(x + y, n) \) lub \( \gcd(x - y, n) \) jest nietrywialnym dzielnikiem \( n \).
		\item Jeśli \( x = y \) lub \( x = -y \), powtórz losowanie.
	\end{enumerate}
\end{greyframe}
{\small (Funkcja \( \textcolor{Blue}{\text{sqrt}(x, n)} \) to czarna skrzynka, która oblicza lub zgaduje w czasie wielomianowym pierwiastek z liczby \( x \) modulo \( n \).)}

Algorytm działa poprawnie, bo skoro \( x^2 = y^2 \pmod{n} \), to \( n \mid (x + y)(x - y) \). W każdej iteracji szansa na znalezienie nietrywialnego dzielnika jest równa \( \frac{1}{2} \), ponieważ tyle wynosi prawdopodobieństwo, że trafiliśmy na \( x \neq \pm y \), przy losowym wyborze \( x \), czyli każde z 4 rozwiązań równania \( y^2 = s \pmod{n} \) jest równie prawdopodobne.

Gdyby algorytm Tonellego-Shanksa działał na grupie \( \integer_n \) dla dowolnej liczby całkowitej \( n \), to dałoby się w czasie wielomianowym rozłożyć \( n \) na czynniki pierwsze.