\section{Zestaw 1}
\subsection{Zadanie 1}
\subsubsection{Wzór na $\phi(n)$}
Zauważmy, że liczb mniejszych równych od n które sa kandydatami na bycie wzglednie pierwszymi z n jest n. Więc mamy:
$$n$$
Wśród tych n liczb $n/p_1$ dzieli sie przez $p_1$, wiec nie sa one wzglednie pierwsze z n. Zatem musimy pozbyć sie ich z naszego n:
$$n - \frac{n}{p_1}$$
Teraz wypadalłby pozbyć sie liczb podzielnych przez $p_2$. Jest ich dokładnie $n/p_2$. I wszystko byłoby good, gdyby nie to że w naszym n sa też liczby które dziela sie przez $p_1 p_2$ i zostały one już odjete wcześniej (gdy odejmowaliśmy $p_1$). Zatem musimy je dodać z powrotem, by nie odejmować ich podwójnie. Wiec mamy:
$$n - \frac{n}{p_1} - \frac{n}{p_2} + \frac{n}{p_1 p_2}$$
Dla $p_3$ chcemy odjać $n/p_3$ ale musimy dodać zarówno $n/(p_3 p_2)$, jak i $n/(p_3 p_1)$ (bo wcześniej je odjeliśmy, a teraz chcemy je dodać z $p_3$). Trzeba zwrócić uwage też na liczbe $p_1 p_2 p_3$, która została odjeta od nas raz ($n/p_3$) a dodana dwa razy ($n/(p_1 p_2)$ i $n/(p_1 p_2)$). Wiec trzeba ja dodać by wyjść na zero (chcemy wyjść na zero bo ta liczba została odjeta w pierwszym kroku przy pomocy $p_1$):
$$n (1- \frac{1}{p_1} - \frac{1}{p_2} - \frac{1}{p_3} + \frac{1}{p_1 p_2} + \frac{1}{p_1 p_3} + \frac{1}{p_2 p_3} - \frac{1}{p_1 p_2 p_3})$$
No i łatwo zauważyć patern. Idac kolejnymi liczbami odpowiednio je dodajemy i usuwamy. Stosujemy zasade właczen i wyłaczen i otrzymujemy wzór (można to zapisać sumami, ale wtedy brzydko to wyglada):
$$\phi(n) = n (1- \frac{1}{p_1} - ... - \frac{1}{p_s} + \frac{1}{p_1 p_2} + ... + \frac{1}{p_s p_{s-1}} + ... + (-1)^s \frac{1}{p_1 ... p_s})$$
Co jak sie przyjrzymy teleskopuje sie do:
$$\phi(n) = n (1- \frac{1}{p_1})(1 - \frac{1}{p_2})...(1 - \frac{1}{p_s})$$
Co było do pokazania. $\blacksquare$

\subsubsection{Multiplikatywnosc}
Niech $n=ab$, gdzie $a, b$ sa wzglednie pierwsze. Przyjmijmy poniższe oznaczenia:
$$a=p_1^{\alpha_1}p_2^{\alpha_2}...p_s^{\alpha_s}$$
$$b=q_1^{\beta_1}q_2^{\beta_2}...q_r^{\beta_r}$$
Jako ze $n=ab$ to:
$$n=p_1^{\alpha_1}...p_s^{\alpha_s}q_1^{\beta_1}...q_r^{\beta_r}$$
Z pierwszego popunktu wiemy też że:
$$\phi(a) = a (1- \frac{1}{p_1})(1 - \frac{1}{p_2})...(1 - \frac{1}{p_s})$$
$$\phi(b) = b (1- \frac{1}{q_1})(1 - \frac{1}{q_2})...(1 - \frac{1}{q_r})$$
Jako że a i b sa wzglednie pierwsze, to $p_i \neq q_j$ dla każdego $i, j$.
$$\phi(n) = n (1- \frac{1}{p_1})...(1 - \frac{1}{p_s})(1 - \frac{1}{q_1})...(1 - \frac{1}{q_r})$$
Ale $\phi(n)$ można zapisać również jako:
$$\phi(n) = a(1- \frac{1}{p_1})...(1 - \frac{1}{p_s}) b(1 - \frac{1}{q_1})...(1 - \frac{1}{q_r})$$
Po czym zauważamy że lewa strona iloczynu to $\phi(a)$, a prawa to $\phi(b)$.
Zatem otrzymujemy:
$$\phi(n) = \phi(a)\phi(b) $$
Co było do pokazania. $\blacksquare$

\subsection{Zadanie 2}
\subsubsection{Suma dzielników n}
Niech $n=ab$, gdzie $a, b$ sa wzglednie pierwsze. Przyjmijmy poniższe oznaczenia:
$$a=p_1^{\alpha_1}p_2^{\alpha_2}...p_s^{\alpha_s}$$
$$b=q_1^{\beta_1}q_2^{\beta_2}...q_r^{\beta_r}$$
Jako ze $n=ab$ to:
$$n=p_1^{\alpha_1}...p_s^{\alpha_s}q_1^{\beta_1}...q_r^{\beta_r}$$
Patrzac na rozkład liczby a zauważamy że jej dzielniki sa postaci: $p_1^{i_1} \cdot p_2^{i_2} \cdot ... \cdot p_s^{i_s}$, gdzie $ 0\leq i_j \leq\alpha_j$ dla $j\leq s$. Zatem suma dzielnikow liczby a to suma wszystkich mozliwych kombinacji $i_1 ... i_s$. Zatem wzór na sume dzielnikow a ma postac:
$$\sigma (a) = ( p_1^0+p_1^1 + ...+p_1^{\alpha_1}) ( p_2^0+p_1^1 + ...+p_2^{\alpha_2})... ( p_s^0+p_s^1 + ...+p_s^{\alpha_s})$$
Analogicznie dla $b$ i $n$ otrzymujemy:
$$\sigma (b) = ( q_1^0+q_1^1 + ...+q_1^{\beta_1}) ( q_2^0+q_1^1 + ...+q_2^{\beta_2})... ( q_r^0+q_r^1 + ...+q_r^{\beta_r})$$
$$\sigma (n) = ( p_1^0+p_1^1 + ...+p_1^{\alpha_1})... ( p_s^0+p_s^1 + ...+p_s^{\alpha_s})( q_1^0+q_1^1 + ...+q_1^{\beta_1}) ... ( q_r^0+q_r^1 + ...+q_r^{\beta_r})$$
Zauważany że prawa strona wyrażenia to $\sigma(b)$, a lewa to $\sigma(a)$, zatem otrzymujemy:
$$\sigma(n) = \sigma(a)\sigma(b) $$
Co było do pokazania. $\blacksquare$
\subsubsection{Liczba dzielników n}
Dowód na liczbe dzielników przeprowadzamy jak wyżej, zauważajac, że:
$$d(a) = (1+\alpha_1) ( 1+\alpha_2)... ( 1+\alpha_s)$$
Czemu? Patrzymy na $\sigma (a)$. Czynnik zawierajacy $p_i$ wyprodukuje wszystkie potegi $p_i$, ktorych jest $\alpha_i + 1$
Zatem liczby dzielników liczb $b, n$ sa dane wzorem:
$$d(b) = (1+\beta_1) ( 1+\beta_2)... ( 1+\beta_r)$$
$$d(n) = (1+\alpha_1)... ( 1+\alpha_s)(1+\beta_1)... ( 1+\beta_r)$$
Zatem otrzymujemy:
$$d(n) = d(a)d(b) $$
Co było do pokazania. $\blacksquare$

\subsection{Zadanie 3}
Gdyby nie była właśnie godzina 2 to bym zrobił to zadanie...

\subsection{Zadanie 4}
BSO $b\leq a$. Niech $F_n$ to bedzie pierwsza liczba Fibonacciego wieksza lub równa $a$. Spójrzmy jak działa algorytm Euklidesa w zależnoci od tego jakiej wielkości jest $b$:
\begin{enumerate}
	\item $b\leq F_{n-2}$\\
	      Ponieważ $a\%b < b \leq F_{n-2}$ to w nastepnym wywołaniu funkcji otrzymujemy że wiekszy element z $a,b$ jest mniejszy lub równy $F_{n-2}$.
	\item $F_{n-2}< b\leq F_{n-1}$\\
	      Ponieważ $a\%b < b \leq F_{n-1}$ to w nastepnym wywołaniu funkcji otrzymujemy że wiekszy element z $a,b$ jest mniejszy lub równy $F_{n-1}$.
	\item  $F_{n-1}< b \leq F_{n}$\\
	      Jako że zachodzi:
	      $$F_{n-1}< b$$
	      $$2F_{n-1}< 2b$$
	      $$F_n=F_{n-1}+F_{n-2}\leq 2F_{n-1}< 2b$$
	      $$b \leq a \leq F_n < 2b$$
	      To mamy $a\%b = a-b$. Z kolejnych nierównoci;
	      $$a\leq F_{n-1}+F_{n-2} =  F_n$$
	      $$F_{n-2} \leq  F_{n-1}$$
	      $$F_{n-1}< b$$
	      Otrzymujemy:
	      $$a\%b=a-b\leq F_{n-2}$$
	      Zatem w nastepnym wykonaniu funkcji dostaniemy przypadek pierwszy
\end{enumerate}
Ponieważ w każdym kroku schodzimy o 1 lub o 2 w dół (a jak nie, to w nastepnym nadrabiamy 2 krokami) to funkcja wywoła sie co najwyżej $n$ razy.
Zauwazmy, że n-ta liczba Fibonacciego jest przedstawiana wzorem (Binet's formula):
$$F_n = \frac{\phi^n}{\sqrt{5}} -\frac{(- \phi)^{-n}}{\sqrt{5}}$$
Przy czym zauważmy że element który odejmujemy jest $\leq1$. Wiec możemy sobie przybliżyć (z góry):
$$F_n = \phi^n$$
$$log_{\phi}(F_n) = n$$
Zatem funkcja wywoła sie$ log_{\phi}(F_n) + O(1)$ razy. Przy czym $F_n = a$.

\subsection{Zadanie 5}
\subsubsection{Teza}
Używamy rozszerzonego algorytmu Euklidesa by otrzymać liczby $r$ i $s$ takie że $$g = gcd(m_1, m_2) = rm_1+sm_2$$ Wtedy rozwiazaniem równania bedzie:
$$x = \frac{a_1sm_2 +a_2rm_1}{g}$$
\subsubsection{Dowód}
$$x=\frac{a_1sm_2 +a_2rm_1}{g} =\frac{a_1(g-rm_1)+a_2rm_1}{g}= \frac{a_1g+(-a_1+a_2)rm_1}{g}=$$
$$= a_1+ \frac{(-a_1+a_2)rm_1}{g}$$
$$a_1+ \frac{(-a_1+a_2)rm_1}{g}\equiv a_1 mod (m_1)$$
Analogicznie
$$a_2+ \frac{(-a_2+a_1)sm_2}{g}\equiv a_2 mod (m_2)$$
Zaprezentowawne rozwiazanie działa. $\blacksquare$