\section{Zestaw 4 -  Pierścienie, wielomiany, ciała skończone}
\subsection{Zadanie 1}

$\textbf{Treść:}$  \\
\\
Pokaż, że jeśli pierścień jest euklidesowy, to jest pierścieniem ideałów głównych.
Podaj przykład pierścienia, który nie jest pierścieniem ideałów głównych. \\ \\
$\textbf{Rozwiązanie:}$ \\
\\
1. $I = (0)$ trywialne.

2. weźmy sobie dowolne $I \neq (0)$ \\ \\
Weźmy sobie takie $a \neq 0$ oraz $N(a)$ jest minimalne względem wszystkich $a \in S$.\\ \\
Pokażemy że $I \subseteq (a)$. \\ \\
Weźmy sobie dowolny $x \in S$. Z własności, że $S$ jest euklidesowy wiemy, że $x = aq + r$ oraz $r = 0$ lub $v(r) < v(a)$. Wybraliśmy takie $a$, że $v(a)$ jest minimalne, więc $v(r) \geq v(a)$ co oznacza, że $r = 0$, co oznacza, że $x = aq$. Czyli $I \subseteq (a)$.

Przykład: \\  \\
Weźmy sobie pierścień nad ciałem $\mathbb{F}_2[x] $ ( pierścień wielomianów stopnia co najwyżej $2$ ). Weźmy sobie ideał generowany przez elementy $(x,2)$. Teraz aby generował ten ideał jakiś jeden element (nazwijmy do $z$) to $x|z$ oraz $2|z$. Co oznacza że $z$ musi być w postaci $2k\cdot x$, a więc generuje nie generuje wielomianów stałych.

\subsection{Zadanie 2}

$\textbf{Treść:}$  \\
\\
Pokazać, że pierścień $\mathbb{Z}[i] = \{a + bi : a, b \in \mathbb{Z}\}$ (pierścień zespolonych liczb
całkowitych, zwany też pierścieniem Gaussa) jest euklidesowy. \\ \\
$\textbf{Rozwiązanie:}$ \\
\\
Weźmy sobie $v((a,b)) = a^2 + b^2$.\\
\\
Teraz musimy pokazać zachowanie reszty.\\ \\
Weźmy sobie $a = a_1 + a_2i, b = b_1 + b_2i$. Możemy teraz z własności liczb zespolonych obliczyć:
\\
$$\frac{a}{b} = \frac{a_1 + a_2i}{b_1 + b_2i} = \frac{(a_1 + a_2i)(b_1 - b_2i)}{b_{1}^{2} - b_{2}^{2}} = \frac{(a_1b_1 - a_2b_2) + (a_2b_1 - b_2a_1)i}{v(b)}$$ \\
Z faktu dzielenia liczb całkowitych wiemy że znajdziemy takie $q_1,q_2,r_1,r_2$ że: \\
$$a_1b_1 - a_2b_2 = v(b)q_1 + r_1$$
$$a_2b_1 - b_2a_1 = v(b)q_2 + r_2$$
możemy dobrać $q_1,q_2$ tak aby $r_1,r_2$ spełniały nierówność $$-\frac{1}{2}v(b) \leq r_1,r_2 \leq \frac{1}{2}v(b)$$ ( jak nie spełnia to odpowiednio zwiększamy $q_1,q_2$ lub zmniejszamy w zależności od potrzeby.)
\\
\\ Oznaczmy sobie teraz $q = q_1 + q_2i$ oraz $r = r_1 + r_2i$. Następnie zaobserwujmy że możemy zapisać: $$\frac{a}{b} = \frac{v(b)q + r}{v(b)}$$ co po uproszczeniu i skorzystaniu z definicji sprzężenia prezentuje się w formie: $$a = bq + \frac{r}{\overline{b}}$$ Wiemy że $a \in \mathbb{Z}$ oraz $bq \in \mathbb{Z}$ co implikuje, że $\frac{r}{\overline{b}} \in \mathbb{Z}$. Następnie korzystająć z właściwości naszej funkcji $v$ wiemy, że $$v(\frac{r}{\overline{b}}) = v(\frac{r}{b}) = v(b)^{-1}v(r)$$ \\ A teraz możemy zapisać $$r = r_1 + r_2i => v(r) = r_{1}^{2} = r_{2}^{2}$$ a z poprzeniej nierównośći: $$-\frac{1}{2}v(b) \leq r_1,r_2 \leq \frac{1}{2}v(b)$$ otrzymujemy, że $$v(r) \leq 2(\frac{1}{2}v(b))^2 = \frac{1}{2}v(b)^2$$ Podkłając to do równania $$v(\frac{r}{\overline{b}}) = v(\frac{r}{b}) = v(b)^{-1}v(r) \leq v(b)^{-1}\frac{1}{2}v(b)^2 = \frac{1}{2}v(b)$$
Wiec mamy już wszystko gdyż nasza reszte z dzielenia $\frac{a}{b} = bq + \frac{r}{\overline{b}}$ wynosi $\frac{a}{\overline{b}}$ co spełnia definicję.


\subsection{Zadanie 3}

$\textbf{Treść:}$  \\
\\
Charakterystyką ciała $F$ nazywamy taką liczbę k, że $1 + 1 + . . . + 1$ ($k$ razy) $ = 0$ \\ \\
(o ile taka istnieje, w przeciwnym wypadku charakterystyka wynosi $0$). Pokazać, że
charakterystyka ciała skończonego zawsze jest dodatnia i jest liczbą pierwszą.
\\ \\
$\textbf{Rozwiązanie:}$ \\ \\
1. Nie zerowość. \\ \\
Nie wprost zakładamy, że nie otrzymujemy 0 a wykonaliśmy więcej niz ilość elementów w ciele dodań 1. Co oznacza, że jakiś element musiał się powtórzyć. Jeżeli jakiś element się powtórzył to z definicji suma między powtórzeniami wynosi 0, gdyż jest ono elementem neutralnym dodawania. Czyli 0 musiało wystąpić sprzeczność.
\\
\\
2. Charakterystaka jest liczbą pierwszą.\\ \\
Załóżmy nie wprost, że charakterystyka jest liczbą złożoną równą $n$. Możemy z tego faktu rozbić $n = pq$ gdzie $p,q > 1$. W takim razie $(1+1+...+1) (n$ razy$) = (1+1+...+1) (q$ razy$) \cdot (1+1+...+1) (p$ razy$)$ z czego wynika, że $p$ lub $q$ jest 0. Co oznacza że $n$ nie jest najmniejsza taką liczbą która po dodaniu $n$ razy 1 otrzymamy 0. SPRZECZNOŚĆ z definicji charakterystyki.

\subsection{Zadanie 4}

$\textbf{Treść:}$  \\
\\
Dany jest wielomian o współczynnikach całkowitych $W(X) = a_0 + a_1X +
	... + a_nX^n$ oraz liczba całkowita $s$. Podaj algorytm, który w czasie $O(n)$ rozstrzygnie,
czy $W(s)$ jest liczbą dodatnią, ujemną, czy zerem.
Zakładamy, że liczby $s$ oraz $a_0, ..., a_n$ mieszczą się w słowie maszynowym i można
wykonywać na nich operacje w $O(1)$ (można o tym myśleć tak, że w $C++$ wszystkie dane
mieściłyby się w typie $int$). Operacje na większych liczbach nie są już stałe – liczba rzędu
$s^k$
będzie potrzebowała $\Omega(k)$ pamięci.


\textbf{Rozwiązanie:}
\newline
\newline
1. Jak $s = 0$ to zwracamy $a_0$. \\ \\
2. Jak $s < 0$ to zmieniamy współczynniki wielomianów przy nieparzystych potęgach na przeciwne i odpalamy się z $s = |s|$ dalej. \\ \\
3. Jeżeli $s = 1$ to sumujemy współczynniki. ( to będzie miało $O(n log n)$ jeżeli jest $n$ jest nie ograniczony ale to w treści jest blef i $n$ mieści się w słowie maszynyowym ) \\ \\
4. Zauważmy teraz że owy wielomian możemy czytać jako liczbe w systemie o podstawie $s$. Wystarczy nam teraz tylko usunąć przepełnienia ( zrobić tak aby współczynniki przy każdej potędze wielomianu z wyjątkiem najwyższej potęgi były w przedziale $[0,s-1]$ co jest prostą operacją na liczbach i nie wyjdziemy po za słowo maszynowe do jakiejś stałej ). Co kończy zadanie bo wystarczy nam sprawdzić czy wszystkie nowe współczynniki są zerami albo znak najbadziej znaczącego współczynnika.