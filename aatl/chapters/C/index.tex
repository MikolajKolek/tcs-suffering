Rachunek indeksów to obecnie najlepszy znany algorytm na logarytm dyskretny w \( \integer_p \).

Działając w grupie \( \integer_p \) o znanym generatorze \( g \), dla \( g^x = b \) chcemy znaleźć \( x = \log b \). Taki logarytm zachowuje własności zwykłego logarytmu.

\textbf{Faza I:} \\
Ustalamy zbiór małych liczby pierwszych \( Q = \{q_1, \dots, q_r\} \) -- bazę czynników.
Wybieramy lub losujemy \( k \), tak żeby uzyskać \( r \) różnych liczb, dla których \( g^k \pmod{p} \) rozkłada się na czynniki z bazy \( Q \). W ten sposób otrzymujemy \( r \) równań postaci:
\[
	g^k = q_1^{\beta_1} \cdot \ldots \cdot q_r^{\beta_r} \pmod{p},
\]
\[
	\text{czyli } k = \beta_1 \cdot \log q_1 + \ldots + \beta_r \cdot \log q_r \pmod{p - 1}
\]

\textbf{Faza II:} \\
Znając wszystkie \( \beta_i \) oraz \( k \), chcemy rozwiązać układ równań:
\[
	k^{(1)} = \beta_1^{(1)} \cdot \log q_1 + \ldots + \beta_r^{(1)} \cdot \log q_r
\]
\[
	\cdots
\]
\[
	k^{(r)} = \beta_1^{(r)} \cdot \log q_1 + \ldots + \beta_r^{(r)} \cdot \log q_r
\]
z niewiadomymi \( \log q_i \). Możemy wykorzystać do tego metody algebry liniowej z dokładnością do jednej trudności -- operacje są modulo \( p-1 \), które nie jest pierwsze, więc niekoniecznie da się policzyć odwrotność względem mnożenia.
Jako wynik otrzymujemy logarytmy dyskretny liczb \( q_1, \dots, q_r \).

\textbf{Faza III:} \\
Żeby znaleźć \( \log b \), próbujemy wygenerować jakąkolwiek liczbę postaci \( b^u \cdot g^v \), która jest gładka modulo \( p \). Wystarczy przyjąć \( u = 1 \) i losować \( v \) do skutku. Z równości
\[ b^u \cdot g^v = q_1^{\gamma_1} \cdot \ldots \cdot q_r^{\gamma_r} \]
ostatecznie wynika:
\[ x = \log b = u^{-1} \cdot (\gamma_1 \cdot \log q_1 + \ldots + \gamma_r \cdot \log q_r - v) \]

Oczekiwana złożoność algorytmu to \( e^{(\sqrt{2}+\bigO(1))\sqrt{\ln n \ln\ln n}} \), czyli podwykładnicza. Metoda rachunku indeksów daje się uogólnić na ciała skończone \( \integer_p[X]/(w) \), gdzie \( w \) jest pewnym wielomianem. Wtedy baza czynników, zamiast liczb pierwszych, zawiera nierozkładalne wielomiany małych stopni.

\subsection{Atak Logjam}
Rachunek indeksów, jeśli stosowany w sposób leniwy, ma słaby punkt -- fazy I i II są niezależne od wejściowego elementu \( b \), co gorsza można je łatwo zrównoleglić.
Żeby rozszyfrować ciąg informacji, dla których zawsze używana była ta sama liczba pierwsza \( p \), wystarczy wykonać fazy I i~II tylko raz, a faza III jest już bardzo szybka.
