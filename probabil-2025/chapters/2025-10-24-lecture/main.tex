{
\makeatletter
\def\input@path{{../probabil}}
\makeatother
\graphicspath{{../probabil}}

\section{Zastosowania nierówności na zmiennych losowych}
Pokażmy przykład zastosowania nierówności na zmiennych losowych poprzez wypełnienie tabelki pokazującej rezultaty jakie dają nam poszczególne nierówności.

Dla wprowadzenia, gdy mówimy o problemie kolekcjonera kuponów odwołujemy się do \ref{coupon-collectors-problem}. Za to gdy mówimy o rzutach monetą, to \(X_i\) to zmienna losowa dla której 
\[
	\prob(X_i = 0) = \prob(X_i = 1) = \frac{1}{2}
\]
oraz
\[
	X = \sum_{i=1}^n X_i, \ev{X} = \frac{n}{2}
\]

\begin{tabular}{p{6cm} p{5.5cm} p{5.5cm}}
\toprule
 & Rzuty monetą \newline Ograniczenie \(\prob(X \geq \frac{3}{4}n)\)  & Kolekcjoner kuponów \newline Ograniczenie \(\prob(X \geq 2 n H_n)\)\\
\midrule
Nierówność Markowa (\ref{markov-inequality}) & \(\frac{2}{3}\) (\ref{markov-coin-tosses}) & \(\frac{1}{2}\) (\ref{markov-coupon-collector}) \\

Nierówność Czebyszewa (\ref{chebyshev-inequality}) & \(\frac{4}{n}\) (\ref{chebyshev-coin-tosses}) & \(O(\frac{1}{ln^2 n})\) (\ref{chebyshev-coupon-collector})\\

Nierówność Chernoffa (\ref{chernoff-inequality}) & \(e^{-\frac{n}{20}}\) (\ref{chernoff-coin-tosses})& \(e^{-n}\) \\

Union bound (\ref{union-bound}) & \(\) & \(\frac{1}{n}\) (\ref{union-bound-coupon-collector}) \\
\bottomrule
\end{tabular}

W następnych sekcjach pokażdemy odpowiednie definicje tych nierówności oraz dowody wartości podanych w tabelce.

\begin{example} Union bound dla kolekcjonera kuponów\\
	\label{union-bound-coupon-collector}
	Niech \(A_i\) to zdarzenie polegające na nieuzyskaniu \(i\)-tego kuponu po \(n ln(n) + cn\) krokach
	\[
		\prob\left(A_i\right) = \left(1 - \frac{1}{n}\right)^{n ln(n) + cn} = \left(1 + \frac{1}{-n}\right)^{-n(-ln(n) -c)} \leq e^{-(ln(n) + c)} = \frac{1}{e^c n}
	\]
	Z \ref{union-bound} wiemy, że
	\[
		\prob\left(\bigcup_{i=1}^n A_i\right) = \sum_{i=1}^n \prob\left(A_i\right) = \frac{1}{e^c}
	\]
	Niech \(c = ln(n)\), a więc \(A_i\) to zdarzenie polegające na nieuzyskaniu \(i\)-tego kuponu po \(2n ln(n)\) krokach, a \(2n ln(n) \leq 2n H_n\), a więc
	\[
		\prob(X \geq 2n H_n) \leq \prob\left(X \geq 2n ln(n)\right) = \prob\left(\bigcup_{i=1}^n A_i\right) \leq \frac{1}{e^{ln(n)}} = \frac{1}{n}
	\]
\end{example}
{
	\begin{example} Nierówność Markowa dla kolekcjonera kuponów\\
		\ExecuteMetaData[chapters/inequalities/markov]{probabil-2025-10-24-markov-kolekcjoner-kuponow}
	\end{example}
}

\newpage
\section{Nierówność Czebyszewa}
{
	\ExecuteMetaData[chapters/inequalities/chebyshev]{probabil-2025-10-24-chebyshev-1}\\
	Potrzebne nam do tego będzie następujące twierdzenie
	\ExecuteMetaData[chapters/discrete-probability/geometric]{probabil-2025-10-24-wariancja-rozkladu-geometrycznego}
	Wracamy do zastosowania nierówności Czebyszewa dla kolekcjonera kuponów.
	\ExecuteMetaData[chapters/inequalities/chebyshev]{probabil-2025-10-24-chebyshev-2}
}

\newpage
\section{Funkcje tworzące momentów}
\begin{definition}
	\textbf{Funkcję tworzącą momenty} zmiennej losowej \( X \) definiujemy jako
	\[
		M_X(t) = \expected{e^{tX}}
	\]
\end{definition}

\begin{theorem}[Twierdzenie 4.1  P\&C]
	Jeśli \( M_X(t) \) tworzy momenty zmiennej \( X \) to
	\[
		\expected{X^n} = M_X^{(n)}(0)
	\]
\end{theorem}
\begin{proof}
	Zakładamy tutaj, że możemy zamieniać kolejnością operatory różniczkowania i wartości oczekiwanej.
	To założenie działa jeśli tworząca istnieje blisko zera i okazuje się, że zachodzi dla rozkładów, którymi się będziemy zajmować.

	\[
		M_X^{(n)}(t) = \expected{e^{tX}}^{(n)}
		= \expected{X^n e^{tX}}
	\]
	\[
		M_X^{(n)}(0) = \expected{X^n}
	\]
\end{proof}

%<*probabil-egzamin-generating-function-equality>
\begin{theorem}
	\label{generating-function-equality}
    (Bez dowodu)

    Niech \( X \), \( Y \) - zmienne losowe.

    Jeżeli zachodzi: 
    \[ 
        \forall_{t \in ( -\delta, \delta )} M_X(t) = M_Y(t) 
    \]
    gdzie \( \delta > 0 \) oraz \( M_X(t) \) i \( M_Y(t) \) istnieją w przedziale \( ( -\delta, \delta ) \) to \( X \) i \( Y \) mają ten sam rozkład.
\end{theorem}
%</probabil-egzamin-generating-function-equality>

\begin{theorem}[Twierdzenie 4.3 P\&C]
	Dla niezależnych zmiennych losowych \( X \) oraz \( Y \) zachodzi

	\[
		M_{X + Y}(t) = M_X(t) \cdot M_Y(t)
	\]
\end{theorem}
\begin{proof}
	\[
		M_{X + Y}(t) = \expected{e^{t(X + Y)}} = \expected{e^{tX}\cdot e^{tY}} = \expected{e^{tX}}\expected{e^{tX}} = M_X(t) \cdot M_Y(t)
	\]
\end{proof}

{
    \ExecuteMetaData[chapters/discrete-probability/geometric]{probabil-2025-10-24-funkcja-tworzaca-dla-rozkladu-geometrycznego}
}

\newpage
\section{Nierówność Chernoffa}
{
    \ExecuteMetaData[chapters/inequalities/chernoff]{probabil-2025-10-24-nierownosc-chernoffa-1}
}
{
    \ExecuteMetaData[chapters/inequalities/chernoff]{probabil-2025-10-24-nierownosc-chernoffa-2}
}
}
