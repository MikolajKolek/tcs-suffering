{
\makeatletter
\def\input@path{{../probabil}}
\makeatother
\graphicspath{{../probabil}}

\section{Przykładowe problemy z wartości oczekiwanej}
\ExecuteMetaData[chapters/discrete-probability/expected-value-problems]{probabil-2025-10-10-aproksymacja-szeregu-harmonicznego}
\ExecuteMetaData[chapters/discrete-probability/expected-value-problems]{probabil-2025-10-10-problem-kolekcjonera-kuponow}
\ExecuteMetaData[chapters/discrete-probability/expected-value-problems]{probabil-2025-10-10-oczekiwany-czas-quicksorta}

\newpage
\section{Nierówność Markowa}
\ExecuteMetaData[chapters/inequalities/markov]{probabil-2025-10-10-markov}

\newpage
\section{Wariancja i momenty zmiennej losowej}
\ExecuteMetaData[chapters/discrete-probability/variance]{probabil-2025-10-10-wariancja-1}
\ExecuteMetaData[chapters/discrete-probability/variance]{probabil-2025-10-10-wariancja-2}
\ExecuteMetaData[chapters/discrete-probability/variance]{probabil-2025-10-10-wariancja-3}
\ExecuteMetaData[chapters/discrete-probability/variance]{probabil-2025-10-10-wariancja-4}
\ExecuteMetaData[chapters/discrete-probability/expected-value]{probabil-2025-10-10-wariancja-5}
\ExecuteMetaData[chapters/discrete-probability/variance]{probabil-2025-10-10-wariancja-6}
\ExecuteMetaData[chapters/discrete-probability/binomial]{probabil-2025-10-10-wariancja-7}
\ExecuteMetaData[chapters/discrete-probability/variance]{probabil-2025-10-10-wariancja-8}
}
