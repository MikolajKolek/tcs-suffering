Twierdzenie Hall'a jest fajne, ale niestety działa tylko w~grafach dwudzielnych. Przydałby się nam jakiś warunek, który mówi o~istnieniu dopasowania doskonałego w~\emph{dowolnym} grafie.

Powiedzmy, że mamy graf \(G\)~o \(n\)~wierzchołkach. Dla uproszczenia będziemy te wierzchołki traktować jako liczby ze zbioru \([n]\), bo wyjdzie na to samo. Dla każdej krawędzi \(\{i, j\}\), gdzie \(i < j\), zróbmy sobie pewną zmienną, którą będziemy oznaczać \(x_{ij}\). Za chwilę wyjaśni się, co to dokładnie znaczy. Następnie zdefiniujemy sobie tzw. \emph{macierz symboliczną Tutte'a}. Jest to macierz \(n \times n\), a~wypełniamy ją zgodnie z~poniższą regułą:
\begin{equation*}
	M_G[i, j] = \begin{cases}
		0       & \iff \{i, j\} \not\in E\pars{G}         \\
		x_{ij}  & \iff \{i, j\} \in E\pars{G} \land i < j \\
		-x_{ij} & \iff \{i, j\} \in E\pars{G} \land i > j
	\end{cases}
\end{equation*}

Przykładowo, dla takiego grafu:
\begin{figure}[H]
	\centering
	\includegraphics[scale=0.5]{images/tutte/tutte_matrix_example_graph.png}
\end{figure}

będzie
\begin{equation*}
	M_G = \begin{bmatrix}
		0       & x_{12}  & x_{13}  & 0      \\
		-x_{12} & 0       & x_{23}  & x_{24} \\
		-x_{13} & -x_{23} & 0       & x_{34} \\
		0       & -x_{24} & -x_{34} & 0
	\end{bmatrix}
\end{equation*}
Zwróćmy uwagę, że jest to macierz skośnie symetryczna.

Prawdziwe okazuje się:
\begin{theorem}
	Graf \(G\)~ma dopasowanie doskonałe wtedy i~tylko wtedy, gdy wyznacznik symboliczny macierzy \(M_G\) jest niezerowy.
\end{theorem}
Spokojnie, powoli\dots Wyjaśnijmy sobie najpierw, czym jest ten wyznacznik symboliczny. Wyznacznik macierzy będzie (z~definicji) czymś takim:
\begin{equation*}
	\det\pars{M_G} = \sum_{\pi \in S_n}\sgn\pars{\pi} \cdot \pars{\prod_{i = 1}^{n}M_G[i, \pi\pars{i}]}
\end{equation*}
Gdyby w~naszej macierzy \(M_G\)~były konkretne liczby, to wyznacznik byłby po prostu wartością tego wyrażenia. Ale w~naszej macierzy są bliżej nieokreślone (na razie) zmienne, więc będziemy to wyrażenie traktować jako wielomian tych wszystkich zmiennych \(x_{ij}\) odpowiadających krawędziom i~nazwać \emph{wyznacznikiem symbolicznym}. Przez jego ,,niezerowość'' rozumiemy, że ten wielomian nie jest tożsamościowo równy zero. Skoro wyjaśniliśmy już, o~czym mówi to (na pierwszy rzut oka losowe) twierdzenie, to możemy przystąpić do jego dowodzenia.
\begin{proof}
	Najpierw zaobserwujmy, o~czym mówi nam wyznacznik symboliczny. Iterujemy się tam po wszystkich permutacjach \(\pi \in S_n\). Wiadomo, że każda permutacja rozkłada się na rozłączne cykle. Okazuje się, że każda permutacja, dla której wyrażenie pod sumą się nie zeruje, odpowiada pewnemu skierowanemu pokryciu cyklowemu \(G\). Co to jest pokrycie cyklowe? Najprościej mówiąc: bierzemy sobie kilka rozłącznych wierzchołkowo cykli w~\(G\) w~taki sposób, żeby każdy wierzchołek \(G\) należał do jakiegoś wybranego cyklu. Ponieważ mówimy o~\emph{skierowanym} pokryciu cyklowym, to jedna krawędź nieskierowana może się liczyć jako cykl długości \(2\), z~krawędziami skierowanymi w~obydwie strony. Poniżej przykładowy graf i~pewne jego pokrycie cyklowe:

	\begin{figure}[H]
		\centering
		\includegraphics[scale=0.75]{images/tutte/decomposed_graph.png}
		\caption{Przykładowe pokrycie cyklowe}
	\end{figure}

	Mając tę intuicję, faktycznie widzimy, że permutacja \(\pi\), dla której wyrażenie pod sumą się nie zeruje, odpowiada pokryciu cyklowemu. Gdyby bowiem nie było to pokrycie cyklowe, to nie istniałaby w~\(G\) któraś krawędź postaci \(\pars{i, \pi\pars{i}}\). Ale wtedy, z~definicji \(M_G\), byłoby \(M_G[i, \pi\pars{i}] = 0\), czyli jednak składnik odpowiadający \(\pi\) zerowałby się. Zatem tak naprawdę:
	\begin{equation*}
		\det\pars{M_G}
		= \sum_{\pi \in S_n}\sgn\pars{\pi} \cdot \pars{\prod_{i = 1}^{n}M_G[i, \pi\pars{i}]}
		= \sum_{\substack{\pi \in S_n\\\text{\tiny $\pi$ --- pokr. cykl. $G$}}}\sgn\pars{\pi} \cdot \pars{\prod_{i = 1}^{n}M_G[i, \pi\pars{i}]}
	\end{equation*}

	Z~tego od razu możemy wyprowadzić dowód w~jedną stronę. Załóżmy, że istnieje skojarzenie doskonałe w~\(G\) i, bez straty ogólności, składa się ono z~krawędzi \(\{1, 2\}, \{3, 4\}, \ldots, \{2k - 1, 2k\}\), gdzie \(2k = n\). Skoro to skojarzenie doskonałe, to te krawędzie nie dotykają się wierzchołkami, zatem możemy zrobić pokrycie wierzchołkowe transpozycjami \(\pars{1, 2}, \pars{3, 4}, \ldots, \pars{2k - 1, 2k}\), jak na poniższym rysunku:

	\begin{figure}[H]
		\centering
		\includegraphics[scale=0.75]{images/tutte/perfect_matching_cover.png}
		\caption{Pokrycie cyklowe zrealizowane za pomocą transpozycji}
	\end{figure}

	W~wyznaczniku symbolicznym składnik odpowiadający permutacji opisującej to pokrycie cyklowe będzie wyglądał następująco:

	\begin{equation*}
		x_{12} \cdot \pars{-x_{12}} \cdot x_{34} \cdot \pars{-x_{34}} \cdot \ldots \cdot x_{2k - 1\ 2k} \cdot \pars{-x_{2k - 1\ 2k}}
	\end{equation*}

	i~po prostu widać, że nic się z~tym nie zredukuje, bo są tu tylko transpozycje, więc jeśli nawet je odwrócimy to nic się nie zmieni --- w~szczególności znak tego wyrażenia. Zatem faktycznie wyznacznik symboliczny nie będzie tożsamościowo równy zero.

	Dowód w~drugą stronę, tzn.: jeśli wyznacznik symboliczny nie jest tożsamościowo równy zero, to mamy skojarzenie  doskonałe, jest nieco bardziej skomplikowany, ale opiera się na naturalnej obserwacji:
	\begin{equation*}
		\begin{split}
			\det\pars{M_G}
			 & = \sum_{\pi \in S_n}\sgn\pars{\pi} \cdot \pars{\prod_{i = 1}^{n}M_G[i, \pi\pars{i}]}
			= \sum_{\substack{\pi \in S_n                                                           \\\text{\tiny $\pi$ --- pokr. cykl. $G$}}}\sgn\pars{\pi} \cdot \pars{\prod_{i = 1}^{n}M_G[i, \pi\pars{i}]}\\
			 & = \pars{\sum_{\substack{\pi \in S_n                                                  \\\text{\tiny $\pi$ -- pokr. cykl. $G$}\\\text{\tiny $\pi$ bez niep. cykli}}}\sgn\pars{\pi}\pars{\prod_{i = 1}^{n}M_G[i, \pi\pars{i}]}} + \pars{\sum_{\substack{\pi \in S_n\\\text{\tiny $\pi$ -- pokr. cykl. \(G\)}\\\text{\tiny $\pi$ ma niep. cykle}}}\sgn\pars{\pi}\pars{\prod_{i = 1}^{n}M_G[i, \pi\pars{i}]}}
		\end{split}
	\end{equation*}
	Jak nam to pomaga? Zauważmy, że druga suma się zeruje. Dlaczego? Otóż dla każdej permutacji, w~której istnieje nieparzysty cykl, możemy wziąć nieparzysty cykl zawierający najmniejszą liczbą i~go odwrócić. Np. z~poniższej permutacji \(\pi\):

	\begin{figure}[H]
		\centering
		\includegraphics[scale=0.5]{images/tutte/pre_switching.png}
		\caption{Permutacja \(\pi = (1, 2)(3, 4, 5)(6, 7, 8)\)}
	\end{figure}

	powstanie permutacja \(\pi'\):

	\begin{figure}[H]
		\centering
		\includegraphics[scale=0.5]{images/tutte/post_switching.png}
		\caption{Permutacja \(\pi' = (1, 2)(5, 4, 3)(6, 7, 8)\)}
	\end{figure}

	To przekształcenie jest oczywiście bijekcją (i~dodatkowo inwolucją). Zauważmy też, że składnik sumy odpowiadający \(\pi'\) różni się tylko znakiem od składnika sumy odpowiadającego \(\pi\) --- końce krawędzi przecież się nie zmieniły, jedynie dla każdej zmiennej odpowiadającej krawędzi zmienił się znak, bo krawędź zmieniła kierunek. A~cykl był nieparzystej długości, co oznacza, że zmieniło się nieparzyście wiele znaków. Zatem składnik dla \(\pi\) zredukuje się ze składnikiem dla \(\pi'\) dla dowolnej permutacji \(\pi\), w~której istnieje nieparzysty cykl, czyli rzeczywiście cała druga suma się wyzeruje. Ale założyliśmy, że wyznacznik symboliczny jest niezerowy, czyli pierwsza suma jest niezerowa. Oznacza to, że istnieje pokrycie cyklowe \(G\), w~którym nie ma nieparzystych cykli --- innymi słowy, są same parzyste. Zauważmy teraz, że ponieważ są to rozłączne cykle parzystej długości pokrywające \(G\), to, biorąc co drugą krawędź z~każdego takiego cyklu, otrzymamy dopasowanie doskonałe:

	\begin{figure}[H]
		\centering
		\includegraphics[scale=0.75]{images/tutte/matching_from_cover.png}
		\caption{Pokrycie cyklowe oraz skojarzenie doskonałe; linia ciągła oznacza krawędź cyklu, którą bierzemy do skojarzenia }
	\end{figure}

	A~to właśnie chcieliśmy pokazać!
\end{proof}
