\epigraph{Dlaczego wysoki odsetek pracowników służby drogowej ma rodziny? Bo dużo Hall'ują.}{\textit{Niezwykle suchy żart pewnego studenta}}
\begin{theorem}[Halla]
	Graf dwudzielny \(G = (X,Y,E)\), gdzie \(|X|\) = \(|Y|\) ma dopasowanie doskonałe wtedy i tylko wtedy, gdy dla dowolnego \(A \subset X\) zachodzi: \begin{equation}
		|A| \leq |N(A)|
	\end{equation}
\end{theorem}

\begin{proof}
	Ponieważ twierdzenie mówi \textit{wtedy i tylko wtedy}, musimy udowadniać w dwie strony. Zacznijmy od tej prostszej strony, czyli pokażmy że gdy graf dwudzielny w którym \(|X| = |Y|\) ma dopasowanie doskonałe to \(|A| \leq |N(A)|\). Zasadniczo od razu to widać, bo skoro dopasowanie doskonałe istnieje to wystarczy sobie je wziąć. Każdy wierzchołek z \(|X|\) ma wówczas jakąś krawędź do wierzchołka z \(|Y|\). Zauważamy, że siłą rzeczy w samym dopasowaniu (czyli jakimś podgrafie oryginalnego grafu dwudzielnego, z którego być może ,,wywalono'' jakieś krawędzie) jest tak, że \(|A| = |N(A)|\), z czego w szczególności wynika teza. To chyba widać.

	W drugą stronę jest ciekawiej, bo po pierwsze musimy sobie wprowadzić instytucję \textit{ścieżki powiększającej}. Nie należy tego mylić ze ścieżką powiększającą z przepływów, bo one mówią o innych rzeczach (ale idea jest ta sama). Generalnie to załóżmy sobie, że mam jakieś dopasowanie \(M\) które nie jest doskonałe. Oznacza to, że w \(X\) jest jakiś wierzchołek \(x_0\) poza dopasowaniem. Jeśli \(x_0\) łączy się z jakimś wierzchołkiem \(y_0 \in Y\) i \(y_0 \in M\). \(y\) łączy się z jakimś \(x_1 \in M\) (bo są razem w dopasowaniu). Teraz jeśli \(x_1\) łączy się z jakimś \(y_1 \in Y\) takim, że \(y_1 \not \in M\) to ja to dopasowanie mogę przerobić: ,,połączyć'' \(x\) z \(y\) i \(x_1\) z \(y_1\), dorzucając dodatkowy wierzchołek do dopasowania. To jest przykład bardzo krótkiej ścieżki powiększającej, ale generalna idea to jest taki ,,zygzak'' którego można przerobić, żeby dopasowanie powiększyło się o jeden wierzchołek. At this point wszyscy już chyba wiedzą, że zamiłowania do formalizmu to ja nie mam.

	\begin{figure}[H]
		\centering
		\includegraphics[scale=0.3]{images/hall/augmenting_path_before.png}
		\includegraphics[scale=0.3]{images/hall/augmenting_path_after.png}
		\caption{Ścieżka powiększająca przed i po zamianie krawędzi}
	\end{figure}
	\pagebreak

	No dobra, ale co ma ścieżka powiększająca do twierdzenia Halla? Okazuje się że jest ona bardzo wygodnym narzędziem.

	Załóżmy sobie nie wprost, że mamy jakiś graf dwudzielny \(G = (X,Y,E)\), w którym zachodzi warunek Halla ale nie ma dopasowania doskonałego. W takim razie weźmy dopasowanie maksymalne \(M\). Istnieje jakiś \(x\), który nie należy do tego dopasowania (bo nie jest doskonałe). Z warunku Halla (\(|A| \leq |N(A)|\) dla dowolnego \(A \subset X\)) mam, że \(x\) musi mieć jakiegoś sąsiada w \(Y\). Zbiór wszystkich wierzchołków, z którymi połączony jest \(x\) (być może jest ich więcej, być może tylko jeden) oznaczam jako \(B_0\). Każdy wierzchołek z \(B_0\) musi należeć do dopasowania \(M\) (bo inaczej mógłbym je rozszerzyć, biorąc krawędź między tym wierzchołkiem a \(x\)). Wszystkie wierzchołki z \(X\) które są razem w dopasowaniu z wierzchołkami z \(B_0\) oznaczam jako \(A_1\). Oczywiście \(|A_1| = |B_0|\). Zauważmy, że \(|A_1 \cup \{x\}| \geq |B_0|\), a zatem musi istnieć jakiś zbiór wierzchołków \(B_1\) który ma krawędzie do wierzchołków zbioru \(A_1\). Ponownie, wszystkie krawędzie z \(B_1\) muszą być w dopasowaniu, bo inaczej mielibyśmy ścieżkę powiększającą (aha!) od \(x\) do jakiegoś wierzchołka z \(B_1\). W takim razie mamy jakiś zbiór \(A_2\) wierzchołków które są w dopasowaniu z wierzchołkami z \(B_1\), ponownie \(|A_2| = |B_1|\). \(|A_2| + |A_1| + 1 \geq |B_0| + |B_1|\), skąd wierzchołki z \(A_2\) łączą się jeszcze z jakimiś innymi wierzchołkami z \(Y\), ich zbiór nazwiemy \(B_2\). I ponownie, wierzchołki z \(B_2\) muszą być w dopasowaniu, bo inaczej mielibyśmy ścieżkę powiększającą. Korzystamy teraz z faktu który zawsze zakładamy, tj. faktu że grafy są skończone.

	\epigraph{I tak dalej, aż do wyczerpania zasobów}{\textit{Stefan ,,Siara'' Siarzewski do senatora Ferdynanda Lipskiego, ,,Kilerów-ów 2-óch''}}

	Nietrudno bowiem zauważyć, że w końcu wierzchołki się skończą i albo dostaniemy sprzeczność z założeniem że warunek Halla zachodzi, albo wyjdzie nam ścieżka powiększająca (a dopasowanie miało być maksymalne). Tym samym kończymy dowód.
\end{proof}
