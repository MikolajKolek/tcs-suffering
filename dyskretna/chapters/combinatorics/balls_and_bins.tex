\begin{table}[h]
	\centering
	\begin{tabular}{@{}cc|ccc@{}}
		\toprule
		Rozróżnialność obiektów & Rozróżnialność szuflad & Iniektywnie                                             & Surjektywnie                 & Dowolnie                         \\ \midrule
		Tak                     & Tak                    & \(k^{\underline{n}}\)                                   & \(\stirling{n}{k} \cdot k!\) & \(k^n\)                          \\
		Tak                     & Nie                    & \(\tiny \begin{cases} 0, n>k \\ 1 \leq k\end{cases}\)   & \(\stirling{n}{k}\)          & \(\sum_{i=1}^k \stirling{n}{i}\) \\
		Nie                     & Tak                    & \(\binom{k}{n} \)                                       & \(\binom{n-1}{k-1}\)         & \(\binom{n+k-1}{k-1}\)           \\
		Nie                     & Nie                    & \(\tiny\begin{cases}0, n>k \\ 1, n \leq k \end{cases}\) & \(p(n,k)\)                   & \(\sum_{i=1}^k p(n,i)\)          \\ \bottomrule
	\end{tabular}
\end{table}

Jako \(p(n,k)\) definiujemy liczbę podziałów liczby \(n\) na \(k\) niezerowych składników.
