\begin{definition}[Liczby pierwsze]
	Mówimy, że liczba \(p \in \natural\) jest \textit{pierwsza},
	jeżeli \(\card{\set{k \in \natural : k \mid p}} = 2\).
	Równoważnie, liczba \(p\) jest pierwsza jeżeli \(p \neq 1\) oraz
	\(\set{k \in \natural : k \mid p} = \set{1, p}\).
	Zbiór wszystkich liczb pierwszych oznaczamy \(\mathbb P\).
\end{definition}

\begin{theorem}[Fundamentalne twierdzenie arytmetyki]
	\label{nt:fundamentalne}
	Niech \(n \in \natural_1\). Istnieje dokładnie jeden multizbiór
	liczb pierwszych \(\mathcal S \subset \mathbb P\) spełniający
	\(\prod \mathcal S = n\). Ten multizbiór nazywamy \textit{rozkładem} \(n\) na czynniki pierwsze.
\end{theorem}

\begin{proof}[Dowód istnienia]
	Najpierw udowodnimy istnienie, przez indukcję po \(n\).
	Dla \(n = 1\) możemy wziąć \(\mathcal S = \emptyset\), więc bazę indukcji mamy załatwioną.
	Jeżeli \(n \in \mathbb P\) możemy wziąć \(\mathcal S = \set{n}\) i analogicznie dostać odpowiedź.
	Niech \(n \notin \mathbb P\). Z definicji liczb pierwszych istnieją \(a, k\) spełniające
	\(a \cdot k = n\) i \(a \notin \set{1, p}\). Ale to oznacza, że \(a, k < p\),
	czyli z indukcji istnieją multizbiory \(A, K\) spełniające \(a = \prod A, k = \prod K
	\implies n = \prod (A \cup K)\), co kończy dowód.
\end{proof}
Aby wykazać unikalność takiego rozkładu udowodnimy najpierw pomocniczy lemat:
\begin{lemma}[Lemat Euklidesa]
	\label{nt:lemateuklidesa}
	Niech \(p \in \mathbb P\) i \(a, b \in \natural_1\), oraz \(p \mid ab\).
	Wtedy \(p \mid a\) lub \(p \mid b\).
\end{lemma}
\begin{proof}[Dowód lematu]
	Jeżeli \(p \mid a\) to oczywiście mamy spełnioną tezę.
	Załóżmy więc, że \(p \nmid a\).
	Ale to oznacza, że \(\gcd(p, a) = 1\) (jest to jedyny inny dzielnik \(p\)).
	Z tożsamości Bezouta istnieją \(x, y\) spełniające \(ax + py = 1\).
	Mnożąc stronami przez \(b\) otrzymujemy \(axb + pyb = b\).
	Ale \(p \mid ab \implies p \mid axb \implies p \mid axb+pyb\), co dowodzi
	\(p \mid b\) i kończy dowód.
\end{proof}
\begin{proof}[Dowód unikalności]
	Mając ten lemat możemy przystąpić do dowodu nie wprost.
	Załóżmy, że istnieje liczba o dwóch różnych rozkładach na czynniki pierwsze.
	Niech \(n\) będzie najmniejszą taką liczbą, a \(\mathcal S, \mathcal T\) będą jej rozkładami.
	Oczywiście \(n \neq 1 \implies \mathcal S, \mathcal T \neq \emptyset\).
	Niech \(p \in \mathcal S\) będzie dowolną liczbą pierwszą z \(\mathcal S\). Z lematu Euklidesa i
	równości \(\prod \mathcal S = \prod \mathcal T\) wiemy, że istnieje \(q \in \mathcal T\) takie, że \(p \mid q\).
	Ale ponieważ \(p, q\in \mathbb P\) musi zachodzić \(p = q\).
	Oznacza to, że zachodzi \(\prod (\mathcal S \setminus{p}) = \prod (\mathcal T \setminus{p})\), co wraz z faktem, że \(p \geq 2\)
	pokazuje istnienie mniejszego \(n\) o nieunikalnym rozkładzie, co jest sprzeczne z założeniem
	o minimalności i kończy dowód.
\end{proof}
