\begin{theorem}[Nierówność LYM (Lubella, Yamamoto, Meshalkina)]
	Niech \(\mathcal{D}\) będzie antyłańcuchem kraty \(\mathbb B_n\).
	Wtedy zachodzi:
	\begin{equation}
		\sum_{X \in \mathcal{D}} \frac{1}{\binom{n}{\card{X}}} \leq 1.
	\end{equation}
	Alternatywnie, jeżeli \(f_n\) to liczba zbiorów o mocy \(n\) w \(\mathcal{D}\),
	to twierdzenie można zapisać jako:
	\begin{equation}
		\sum_{i=0}^{n} \frac{f_n}{\binom{n}{k}} \leq 1.
	\end{equation}
\end{theorem}

\begin{proof}
	Dowód opierać się będzie na pewnej ,,dziwnej'' funkcji \(\nu: \mathbb B_n \to \mathcal{P}(S_n)\),
	gdzie \(S_n\) to zbiór wszystkich permutacji \(n\)-elementowych.\footnote{Dla fanów algebry jest to zbiór podkładowy grupy symetrycznej na zbiorze \([n]\).}
	Naszą funkcję definiujemy w następujący sposób:
	\[\nu(X) = \set{\pi \in S_n: \set{\pi(1), \pi(2), \pi(\card{X})} = X},\]
	czyli innymi słowy jest to zbiór wszystkich permutacji, których pierwsze \(\card{X}\)
	elementów należy do \(X\). Zauważmy teraz dwa ciekawe fakty:
	\begin{enumerate}
		\item \(\card{\nu(X)} = \card{X} \cdot (n - \card{X})\) \\
		      Fakt ten wynika z prostego zliczania -- łatwo zauważyć, że wszystkie permutacje
		      w \(\nu(X)\) są postaci \[\pi = \left(\sigma(1), \sigma(2), \ldots, \sigma_{\card{X}},
			      \rho(1), \rho(2), \ldots, \rho(n - \card{X})\right),\]
		      gdzie \(\sigma\) jest ,,permutacją'' \(X\) (tj. bijekcją z \([\card{X}]\) na \(X\)),
		      a \(\rho\) jest analogicznym ustawieniem elementów z \([n] \setminus X\). Z reguły mnożenia
		      otrzymujemy więc moc zbioru \(\nu(X)\).
		\item \(X \neq Y,\, \nu(X) \cap \nu(Y) \neq \emptyset \implies X, Y - \text{porównywalne}\) \\
		      Załóżmy BSO, że \(\card{X} \leq \card{Y}\), i niech \(\pi\) będzie elementem świadczącym
		      niepustości przecięcia, tj. \(\pi \in \nu(X)\) oraz \(\pi \in \nu(Y)\). Z definicji \(\nu\)
		      wiemy, że \(X = \set{\pi(1), \pi(2), \ldots, \pi(\card{X})}\) oraz
		      \(Y = \set{\pi(1), \pi(2), \ldots, \pi(\card{Y})}\) -- ale z tego oczywiście
		      wynika, że \(X \subset Y\). Z tej obserwacji możemy wywnioskować, że dla dowolnego
		      antyłańcucha \(\mathcal{D} \subset \mathbb B_n\) i \(X, Y \in \mathcal{D},\, X \neq Y\)
		      zachodzi \(\nu(X) \cap \nu(Y) = \emptyset\).
	\end{enumerate}
	Czyli z drugiej obserwacji wynika, że \(\bigsqcup_{X \in \mathcal{D}} \nu(X) \subset S_n\), bo każdej permutacji
	odpowiada co najwyżej jeden zbiór z antyłańcucha \(\mathcal{D}\) -- wykonamy więc kilka transformacji:
	\begin{align*}
		\bigsqcup_{X \in \mathcal{D}} \nu(X)                              & \subset S_n     & \implies \\
		\sum_{X \in \mathcal{D}} \card{\nu(X)}                            & \leq \card{S_n} & \iff     \\
		\sum_{X \in \mathcal{D}} \card{X} \cdot (n - \card{X})            & \leq n!         & \iff     \\
		\sum_{X \in \mathcal{D}} \frac{\card{X} \cdot (n - \card{X})}{n!} & \leq 1          & \iff     \\
		\sum_{X \in \mathcal{D}} \frac{1}{\binom{n}{\card{X}}}            & \leq 1
	\end{align*}
\end{proof}
