\epigraph{
	Zgodnie z twierdzeniem Vizinga każdy graf można pokolorować przy użyciu co najwyżej \(\Delta(G)+1\) kolorów. Twierdzenie Brooksa określa dla jakich grafów to ograniczenie jest osiągane.
}{\textit{
		Użytkownik ,,Esculapa'' na polskojęzycznej Wikipedii w artykule ,,Twierdzenie Brooksa''}}

Oczywiście przytoczona wypowiedź jest nonsensem gdyż, jak dobrze wiemy, twierdzenie Vizinga mówi o kolorowaniu \textbf{krawędziowym}.
Wypowiedzmy zatem poprawną formę twierdzenia Brooksa.

\begin{theorem}[Brooks]
	Jeśli spójny graf \(G\) jest kliką lub cyklem nieparzystym to \(\chi(G) = \Delta(G) + 1\).
	W przeciwnym razie \(\chi(G) \leq \Delta(G)\)
\end{theorem}

\begin{proof}
	Widzimy, że cykl nieparzysty wymaga użycia \(3 = \Delta(G) + 1\) kolorów,
	a w przypadku kliki sąsiedzi każdego wierzchołka używają \(\Delta(G)\) kolorów, a jeszcze jeden potrzebujemy na ten wierzchołek. Przyjmijmy zatem, że nasz graf \(G\) nie jest
	ani cyklem nieparzystym ani kliką.

	Jeśli \(\Delta(G) \leq 2\) to \(G\) jest ścieżką lub cyklem parzystym i widzimy, że \(G\) jest dwudzielny.

	Niech \(\Delta(G) \geq 3\).

	Idea dowodu jest taka, że będziemy chcieli jakoś skonstruować kolorowanie używające co najwyżej \(\Delta(G)\) kolorów. W związku z tym będziemy inkrementalnie odfiltrowywać grafy, dla których takie kolorowanie stworzymy.
	Przedstawiam zatem kolejne własności grafu, dla którego będzie się trzeba trochę bardziej namęczyć.

	\begin{enumerate}
		\item \(G\) jest \(\Delta\)-regularny
		      Pokażemy, że w przeciwnym razie \(col(G) \leq \Delta\)
		      Jeśli \(G\) nie jest \(\Delta\)-regularny to w \(G\) istnieje wierzchołek \(v\) o stopniu mniejszym niż \(\Delta\).
		      Postawmy ten wierzchołek na końcu permutacji i spójrzmy na graf \(G - v\).
		      \(v\) miał jakichś sąsiadów, których stopień wynosił co najwyżej \(\Delta\).
		      W takim razie po usunięciu \(v\) jego byli sąsiedzi na pewno mają teraz stopień mniejszy niż \(\Delta\)
		      i możemy powtórzyć całe to rozumowanie aż skończą nam się wierzchołki i wygenerujemy całą permutację.
		      Zauważamy, że dzięki konstrukcji każdy wierzchołek ma na lewo mniej niż \(\Delta\) sąsiadów i dostajemy \(col(G) \leq \Delta\)

		      Wybierzmy dowolny wierzchołek \(v\) i oznaczmy \(H = G - v\).
		      Sąsiadom \(v\) zmniejszyliśmy stopień, zatem z powyższego wywodu wynika, że \(H\) jest \(\Delta\)-kolorowalny.
		      Pokolorujmy zatem \(H\) i przejdźmy do kolejnej własności.

		\item Sąsiedzi \(v\) dostają parami różne kolory w kolorowaniu grafu \(H\)
		      W przeciwnym razie któryś z \(\Delta\) kolorów jest wolny w \(v\) i możemy go użyć kończąc kolorowanie.

		      Nazwijmy sąsiadów \(v\) przez \(v_1, ..., v_\Delta\) i niech będą pokolorowani kolorami \(1, ..., \Delta\).

		      Oznaczmy \(C_{ij} = H[\{v \in V \mid c(v) \in \{i, j\}]\) - podgraf indukowany
		      grafu \(H\), który zawiera wszystkie wierzchołki w kolorach \(i, j\)

		\item Sąsiedzi \(v_i, v_j\) leżą w tym samym komponencie grafu \(C_{ij}\)
		      W przeciwnym razie możemy wziąć komponent do którego należy \(v_i\)
		      i przekolorować go tak, że wierzchołki o kolorze \(i\) dostają kolor \(j\),
		      a te o kolorze \(j\) dostają kolor \(i\). Tym samym wierzchołki \(v_i, v_j\)
		      otrzymują oba kolor \(j\) sprowadzając problem do poprzedniego podpunktu.

		\item Każdy \(C_{ij}\) jest ścieżką.
		      Założmy że tak nie jest i weźmy pierwszy licząc od \(v_i\) wierzchołek, który ma co najmniej trzech sąsiadów w \(C_{ij}\) i nazwijmy go \(x\).
		      Bez straty ogólności powiedzmy, że \(x\) ma kolor \(i\).
		      Rozważmy kolory jakie mają sąsiedzi \(x\).
		      Co najmniej trzech sąsiadów ma kolor \(j\), a pozostałych jest \(\Delta - 3\)
		      W takim razie sąsiedzi \(x\) używają co najwyżej \(\Delta - 2\) różnych kolorów. Oczywiście jeden z wolnych kolorów to \(i\), ale możemy teraz przekolorować \(x\) na ten drugi.
		      Ponieważ \(x\) był pierwszym rozgałęzieniem między \(v_i\) a \(v_j\)
		      to po przekolorowaniu \(v_i\) oraz \(v_j\) muszą leżeć w różnych komponentach nowego \(C_{ij}\)

		      \begin{figure}[ht]
			      \centering
			      \includegraphics[scale=0.5]{images/brooks/branching_path.png}
			      \caption{Wierzchołek \(x\) ma trzech sąsiadów w kolorze \(j\)}
		      \end{figure}

		\item Każde dwa \(C_{ij}, C_{ik}, k \neq j\) przecinają się tylko w \(v_i\)
		      W przeciwnym razie istnieje wierzchołek \(x\) w kolorze \(i\),
		      który ma dwóch sąsiadów w kolorze \(j\) i dwóch sąsiadów w kolorze \(k\).
		      Jak się dobrze policzy to tak jak poprzednio wyjdzie nam, że jakiś kolor jest wolny i możemy zrobić ten sam myk z przekolorowaniem.

		      \begin{figure}[ht]
			      \centering
			      \includegraphics[scale=0.5]{images/brooks/branching_path_three_colors.png}
			      \caption{Wierzchołek \(x\) ma dwóch sąsiadów w kolorze \(j\) i dwóch w kolorze \(k\)}
		      \end{figure}

		\item Istnieje para \(v_i, v_j\), która nie jest połączoną krawędzią.
		      W przeciwnym razie wierzchołki \(v, v_1, ..., v_\Delta\) tworzą klikę, co jest sprzeczne z założeniem.

		      Bez straty ogólności powiedzmy, że \(v_1\) i \(v_2\) nie są połączone krawędzią. Jednak z własności (5) musi istnieć ścieżka między nimi. Niech więc \(u\) to będzie pierwszy wierzchołek w kolorze \(1\) na ścieżce od \(v_2\) do \(v_1\).

		      Teraz dzieje się magia.
		      W \(C_{23}\) zamieniamy kolory \(2\) i \(3\) i takie kolorowanie przepuszczamy przez warunki \((2) - (5)\).
		      Jeśli w którymś miejscu udało nam się stworzyć dobre kolorowanie to super, a jeśli nie to ups.
		      Na szczęście zauważamy teraz fajną rzecz. Otóż wierzchołek \(u\) nadal jest połączony ścieżką w kolorach \(1\) i \(2\) z wierzchołkiem \(v_1\), zatem należy do komponentu \(C_{12}\), ale z drugiej strony jest połączony krawędzią z wierzchołkiem \(v_2\), który ma teraz kolor \(3\) zatem należy też do komponentu \(C_{13}\).
		      W takim razie nowe kolorowanie narusza warunek \((5)\) co już umiemy rozwiązać.

		      \begin{figure}[H]
			      \centering
			      \includegraphics[scale=0.4]{images/brooks/disconnected_before_swap.png}
			      \includegraphics[scale=0.4]{images/brooks/disconnected_after_swap.png}
			      \caption{Wierzchołki \(v_1\) i \(v_2\) przed i po przekolorowaniu komponentu \(C_{23}\)}
		      \end{figure}


	\end{enumerate}

	To tyle, nie ma więcej warunków, które musimy rozważać. Fajnie.

\end{proof}
