Jest to dowód, który najłatwiej zrozumieć samemu rozrysowując sobie proces na kartce. Niemniej, z pomocą rysunków spróbuję go Wam przybliżyć.

\begin{theorem}[Vizing]
	\[\Delta(G) \leq \chi'(G) \leq \Delta(G) + 1\]
\end{theorem}

\begin{proof}
	Ograniczenie dolne widzimy od razu -- \(\Delta\) krawędzi spotykających się w jednym wierzchołku musi dostać różne kolory.

	Ograniczenie górne pokazujemy indukcją po liczbie pokolorowanych krawędzi. Jedną krawędź umiemy pokolorować bez problemu.
	Weźmy zatem częściowe kolorowanie i powiedzmy, że chcemy pokolorować krawędź \((x, y)\).

	Skoro mamy do dyspozycji \(\Delta + 1\) kolorów to znaczy, że każdy wierzchołek ma jakiś kolor wolny (nie wychodzi z niego krawędź w tym kolorze).

	Obserwacja, z której będziemy dużo korzystać: jeśli dowolne wierzchołki \(x\), \(y\) połączone krawędzią mają wolny ten sam kolor \(\beta\)
	to krawędź między nimi możemy pokolorować na tenże kolor. Kolory wolne dla danego wierzchołka będziemy oznaczać linią przerywaną. dotykającą tego wierzchołka.

	\begin{figure}[ht]
		\centering
		\includegraphics[scale=0.6]{images/vizing/trivial_case.png}
		\caption{prosty przypadek; \(x\) i \(y\) mają wolny ten sam kolor \(\beta\)}
	\end{figure}

	Załóżmy więc, że mamy pecha i wierzchołki \(x, y\) nie mają wspólnych wolnych kolorów tj. jeśli \(x\) ma wolny kolor \(\beta\)
	to \(y\) ma ten kolor zajęty i vice versa. Dodatkowo niech \(\alpha_0\) będzie wolnym kolorem wierzchołka \(y\).

	Od tego momentu będziemy tak kombinować, żeby \(x\) zwolnić \(\alpha_0\), być może zajmując przy tym \(\beta\).

	\begin{figure}[H]
		\centering
		\includegraphics[scale=0.6]{images/vizing/pre_step_one.png}
		\caption{\(x\) i \(y\) nie mają wspólnych wolnych kolorów}
	\end{figure}


	Niech \(x_0\) będzie taki, że krawędź \((x, x_0)\) ma kolor \(\alpha_0\). Jeśli \(x_0\) ma wolny kolor \(\beta\)
	to krawędź \((x, x_0)\) możemy przekolorować na kolor \(\beta\),
	tym samym sprawiając, że \(x\) ma wolne \(\alpha_0\).
	Ale w takiej sytuacji możemy pokolorować \((x, y)\) na kolor \(\alpha_0\).

	Zatem sytuacja ma się teraz tak:

	\begin{figure}[ht]
		\centering
		\includegraphics[scale=0.6]{images/vizing/step_one.png}
		\caption{\(x\) i \(y\) mają wolne różne kolory, \(x_0\) ma zajętą \(\beta\) i wolne \(\alpha_0\)}
	\end{figure}

	Jeżeli teraz \(x\) ma wolny kolor \(\alpha_1\) to krawędź \((x, x_0)\)
	możemy przekolorować na \(\alpha_1\), a krawędź \((x, y)\) na \(\alpha_0\). Niech więc \((x, x_1)\) będzie w kolorze \(\alpha_1\).

	Jeśli \(x_1\) miałby wolny kolor \(\beta\) to możemy przekolorować \((x, x_1)\) na \(\beta\), wtedy \(x\) zwalnia się \(\alpha_1\) a z tym wiemy co robić.
	No to niech w \(x_1\) \(\beta\) będzie zajęta, a wolny będzie kolor \(\alpha_2\). Poniżej ilustracja:

	\begin{figure}[ht]
		\centering
		\includegraphics[scale=0.6]{images/vizing/step_two.png}
		\caption{\(x_1\) ma zajętą \(\beta\) a wolne \(\alpha_2\)}
	\end{figure}

	Podobnie jak wcześniej stwierdzamy, że z \(x\) wychodzi krawędź w kolorze \(\alpha_2\), bo inaczej przekolorujemy \((x, x_1)\) na \(\alpha_2\). Kontynuujemy to rozumowanie aż napotkamy wierzchołek \(x_k\) o wolnym kolorze \(\alpha_j\), który już znajduje się wśród kolorów \(\alpha_0, ..., \alpha_{k-1}\)


	\begin{figure}[ht]
		\centering
		\includegraphics[scale=0.6]{images/vizing/step_three.png}
		\caption{\(x_k\) ma wolny kolor \(\alpha_j\), który już widzieliśmy.}
	\end{figure}

	Niestety nie możemy wykonać tej samej sztuczki z przekolorowaniem co wcześniej, ale to nic nie szkodzi bo zrobimy co innego.
	Otóż wyjdźmy z wierzchołka \(x_k\) i pójdźmy ścieżką w kolorach na przemian \(\beta\) i \(\alpha_j\).
	Oczywiście kiedyś skończą nam się krawędzie i wylądujemy w jakimś wierzchołku \(v\).

	Rozważmy sobie teraz przypadki czym ten wierzchołek \(v\) jest.

	\begin{enumerate}
		\item \(v \notin \{x, x_0, x_1, ..., x_k\}\)
		      Najfajniejszy przypadek - ścieżka kończy się w niezbyt istotnym miejscu. Zamieniamy kolory na ścieżce miejscami. Możemy tak zrobić, bo wewnętrznym wierzchołkom się nic nie zmienia, a na końcach odpowiedni kolor jest wolny.

		      Teraz, począwszy od krawędzi \((x, x_k)\) przekolorujemy wachlarz.
		      Dzięki przekolorowaniu, \(x_k\) ma teraz wolną \(\beta\) tak jak \(x\), zatem \((x, x_k)\) możemy dać kolor \(\beta\)
		      W takim razie \(x\) ma teraz wolny kolor \(\alpha_k\) tak jak \(x_{k - 1}\),
		      zatem \((x, x_{k-1})\) dostanie kolor \(\alpha_k\).
		      Podobnie \((x, x_{k-2})\) dostanie kolor \(\alpha_{k-1}\).

		      W końcu dojdziemy do \((x, x_0)\), które dostanie kolor \(\alpha_1\). Sprawiliśmy, że \(x\) ma wolne \(\alpha_0\),
		      więc z czystym sumieniem kolorujemy \((x, y)\) na \(\alpha_0\).

		      \begin{figure}[H]
			      \centering
			      \includegraphics[scale=0.45]{images/vizing/fan_case_one_before.png}
			      \includegraphics[scale=0.45]{images/vizing/fan_case_one_after.png}
			      \caption{Przekolorowanie wachlarza gdy ścieżka z \(x_k\) kończy się w poza wierzchołkami \(x, x_0, ..., x_k\)}
		      \end{figure}

		\item \(v = x_{j - 1}\)
		      Taka sytuacja niestety może zajść, bo \(x_{j-1}\) ma wolny kolor \(x_j\) i zajętą \(\beta\).
		      Zauważmy, że nie możemy zrobić tego co w przypadku pierwszym, bo przekolorowanie ścieżki sprawia, że \(x_{j-1}\) ma kolor \(\alpha_j\) zajęty, a taki kolor by otrzymał przy poprawianiu wachlarza. Zrobimy zatem co innego.

		      Tak jak wcześniej przekolorujemy ścieżkę,
		      ale zamiast przekolorowywać krawędź \((x, x_k)\) na \(\beta\)
		      przekolorujemy \((x, x_{j-1})\) na \(\beta\).
		      Dalej możemy kontynuować tak jak poprzednio: \((x, x_{j-1})\) dostanie kolor \(\alpha_{j-2}\) itd. (Musieliśmy przyjść do \(v\) krawędzią o kolorze \(\beta\) bo inaczej nie byłby to koniec ścieżki)


		      \begin{figure}[H]
			      \centering
			      \includegraphics[scale=0.45]{images/vizing/fan_case_three_before.png}
			      \includegraphics[scale=0.45]{images/vizing/fan_case_three_after.png}
			      \caption{Przekolorowanie gdy ścieżka z \(x_k\) kończy się w \(x_{j-1}\)}
		      \end{figure}

	\end{enumerate}



\end{proof}
