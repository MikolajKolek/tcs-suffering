\subsection{Rozkład na ułamki proste}
To nie jest formalny dowód ani formalna własność ani nic, bardziej schemat postępowania przy rozkładzie na ułamki proste. Sam dowód tego, że rozkład na ułamki proste istnieje, to \textit{sprowadź do wspólnego mianownika i zobacz co Ci wyszło}.
Jeżeli \(deg(P(x)) < deg(Q(x))\) i \(Q(x) = (x-a)^n \cdot (x-b)^k\) to:
\begin{equation*}
	\frac{P(x)}{Q(x)} = \frac{P(x)}{(x-a)^n \cdot (x-b)^k} = \frac{A_1}{x-a} + \frac{A_2}{(x-a)^2} + \dots + \frac{A_n}{(x-a)^n} + \frac{B_1}{x-b} + \frac{B_2}{(x-b)^2} + \dots + \frac{B_k}{(x-b)^k}
\end{equation*}

Oczywiście ten schemat można rozszerzać na więcej śmiesznych rzeczy w mianowniku, ale chyba widać o co chodzi.
