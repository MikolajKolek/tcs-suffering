\section{Postać normalna Chomsky'ego}

\begin{definition}
	Mówimy, że symbol \(A \in N \cup \Sigma \) jest \textbf{generujący}, jeśli:
	\[
		\exists_{w \in \Sigma^*}A \rightarrow_{G}^* w
	\]

\end{definition}

\begin{definition}
	Mówimy, że symbol \(A \in N \cup \Sigma\) jest \textbf{osiągalny}, jeśli:

	\[
		\exists_{\alpha, \beta \in (N \cup \Sigma)^*} S \rightarrow \alpha A \beta
	\]
\end{definition}

\begin{definition}
	Mówimy, że symbol \(A \in N \cup \Sigma \) jest \textbf{użyteczny}, jeśli jest osiągalny i generujący.
\end{definition}

\begin{definition}
	Mówimy, że gramatyka G jest w \textbf{postaci normalnej Chomsky'ego} jeśli każda produkcja jest w jednej z następujących postaci:

	\[
		A \rightarrow BC
	\]

	gdzie \(B, C \in N \), oraz

	\[
		A \rightarrow a
	\]

	gdzie \(A \in N\) oraz \(a \in \Sigma\).

	Ponadto dopuszczamy, by dla stanu startowego \(S\) istniała produkcja \(S \rightarrow \eps\) (jeśli gramatyka akceptuje słowo puste), ale wówczas \(S\) nie może znaleźć się po prawej stronie żadnej produkcji.

\end{definition}

\begin{theorem}
	Dla każdej gramatyki bezkontekstowej \(G\) istnieje gramatyka bezkontekstowa \(G'\) w postaci normalnej Chomsky'ego taka, że \(L(G) = L(G')\).
\end{theorem}

\begin{proof}
	Dowód ten jest prosty, ale uciążliwy w realizacji (bo ma wiele kroków).

	Po pierwsze musimy pozbyć się nieterminali, które nie są użyteczne:

	\begin{enumerate}
		\item Symbole które nie są osiągalne znajdujemy najzwyklejszym DFSem puszczonym po produkcjach od symbolu startowego.
		\item Symbole które nie są generujące ,,diagnozujemy'' ustawiając wszystkie terminale i epsilon jako ,,generujące''; następnie przechodzimy po wszystkich produkcjach i ustawiamy jako generujące wszystkie nieterminale \(N\) takie, że \( N \rightarrow \alpha \), gdzie \( \alpha \in N \cup \Sigma \) i każdy symbol w \(\alpha\) jest generujący. Robimy tak dopóki coś się zmienia; widać, że działa to poprawnie.
	\end{enumerate}

	Poprawne działanie obu powyższych algorytmów formalnie należałoby dowieść indukcją, ale no, widać o co chodzi.

	Teraz zostały nam produkcje które są jakkolwiek sensownej postaci; musimy zatem wziąć na celownik te, które nadal nam mieszają szybki.

	Po pierwsze, chcemy wywalić wszystkie produkcje \textit{wymazujące}, to znaczy produkcje postaci \( A \rightarrow \eps \). W tym celu robimy bardzo zabawny fikołek i jeśli mamy jakąś produkcję z nieterminala \(B\) postaci \(B \rightarrow \alpha A \beta\) (gdzie \(  \alpha, \beta \in N \)) to dodajemy jeszcze produkcję \(B \rightarrow \alpha \beta\), by móc zamodelować sytuację gdy \(A\) przechodzi na epsilon bez konieczności trzymania produkcji \( A \rightarrow \eps\).

	Oczywiście, jeśli \(A\) występuje po prawej stronie jakiejś produkcji \(k\) razy, to dodajemy jakoś \(2^k-1\) produkcji tego typu (bo musimy dodać produkcję ,,symulującą zniknięcie'' \(A\) w dowolnym miejscu).

	Gdy już pozbyliśmy się produkcji wymazujących, to musimy rozprawić się z produkcjami \textit{jednostkowymi}, czyli produkcjami typu \(A \rightarrow B\) (gdzie \(A, B \in N\)). Intuicyjnie chcemy móc zrobić ,,skok'' z \(A\) do produkcji na którze przechodzi \(B\). Sformalizowanie tego jest trochę zabawniejsze (tak by działało).

	Definiujemy sobie zbiór wszystkich par nieterminali takich, że są w produkcjach jednostkowych:

	\[
		J = \set{(A, B) \in N^2 : A \rightarrow B}
	\]

	Teraz robimy kurczę fikołka roku: \textbf{domykamy to przechodnio}. Teraz dla każdego elementu \(J\) postaci \(A, X\), jeśli \(X \rightarrow \gamma\) gdzie \(\gamma\) jest taką formą zdaniową, że nie składa się wyłącznie z jednego nieterminala, dodajemy produkcję \(A \rightarrow \gamma\). Dzięki porobieniu takich ,,skrótów'' do wszystkich możliwych miejsc możemy wyrąbać produkcje jednostkowe i wszystko będzie super.

	Teraz musimy się uporać z dziwnymi produkcjami, gdzie po prawej stronie występują jednocześnie nieterminale i terminale. Mowa o produkcjach typu \(A \rightarrow \alpha a \beta\), gdzie \(a\in\Sigma\), \(\alpha, \beta \in (N \cup \Sigma)^*\).  Co robimy w takiej sytuacji? W sumie to nic ciekawego, dodajemy jakiś nieterminal \(A_a\) i produkcję \( A_a \rightarrow a\), dzięki czemu produkcję \(A \rightarrow \alpha a \beta\) możemy transformować w \(A \rightarrow \alpha A_a \beta\).

	Po tych wszystkich transformacjach wszystko jest już postaci \(A \rightarrow \alpha \) gdzie \( \alpha \in N^*\) lub \(A \rightarrow a\), gdzie \(a \in \Sigma\). Zmierzamy w bardzo dobrym kierunku, a jedyne co nam zostało to zamiana produkcji, które po prawej stronie mają ostro więcej niż 2 nieterminale, czyli:

	\[
		A \rightarrow A_1 A_2 A_3\dots A_k
	\]

	dla \(A,A_1, A_2, A_3, \dots, A_k \in N\).

	Aby sobie z tym poradzić, wprowadzamy sobie nieterminale \(X_1, X_2, \dots, X_{k-2}\) :

	\[
		X_1 \rightarrow A_2 X_2
	\]

	\[
		X_2 \rightarrow A_3 X_3
	\]

	\[
		\dots
	\]

	\[
		X_{k-3} \rightarrow A_{k-2} X_{k_2}
	\]

	\[
		X_{k-2} \rightarrow A_{k-1} A_{k}
	\]

	Po wprowadzeniu takich nieterminali i takich produkcji, elegancko przerabiamy naszą problematyczną produkcję do postaci \( A \rightarrow A_1 X_1 \). Jeśli procedurę tę wykonamy dla wszystkich produkcji, to teraz wszystko, elegancko, po prawej stronie będzie mieć 2 nieterminale (jeśli będziemy je mieć).

	Jeśli mamy literkę po prawej, to mamy jedną, więc jest super.

	Teraz jeszcze technikalium: jeśli nasz język akceptuje puste słowo, to nieterminal startowy \(S\) musi przechodzić w epsilon; wtedy też nie chcemy by stan startowy pojawiał się po prawej stronie jakiejkolwiek produkcji. Wobec tego robimy nowy stan startowy \(S'\) i dajemy mu produkcje \(S' \rightarrow S\) oraz \(S' \rightarrow \eps\). To kończy naszą konwersję.

\end{proof}