\section{Automaty ze stosem}

\subsection{Nieformalna definicja}
Tak jak w przypadku DFA i NFA mieliśmy automaty z jakimiś stanami, które przechodziły (albo nie) po kolejnych literach, tak automaty ze stosem mają jeszcze dodatkowo stos, na podstawie którego możemy podejmować decyzję co zrobić.

\subsection{Formalnie}
\begin{definition}
	\textbf{Automat ze stosem} (pushdown automaton PDA) definiujemy jako
	\[
		P = (Q, \Sigma, \Gamma, \delta, q_0, Z_0, F)
	\]
	gdzie
	\begin{itemize}
		\item \( Q \) -- zbiór stanów
		\item \( \Sigma \) -- skończony alfabet słów
		\item \( \Gamma \) -- skończony alfabet stosu
		\item \( \delta : Q \times (\Sigma \cup \set{\eps}) \times \Gamma \rightarrow \powerset\pars{Q \times \Gamma^*} \) -- funkcja przejścia
		\item \( q_0 \) -- stan startowy
		\item \( Z_0\) -- stosowy symbol startowy
		\item \( F \subseteq Q \) -- zbiór stanów akceptujących
	\end{itemize}
\end{definition}
Intuicyjnie \( \delta \) dla każdego stanu \( q \), litery \( a \), symbolu na szczycie stosu \( z \) oddaje zbiór nowych możliwych stanów wraz z symbolami które mają być dodane na stos.
Symbol \( z \) jest usuwany ze szczytu stosu w momencie przejścia, wiele nowych symboli może zostać dodanych.
Jeśli zdarzy się, że opróżnimy stos to mamy tak zwany przypał (automat się zacina), ale się tym nie przejmujemy.

Warto zauważyć, że \( \delta(q, a, z) \) jest \textbf{zbiorem}, a więc PDA jest niedeterministyczny (automat może wybrać dowolne z przejść).
Istnieje deterministyczna wersja PDA o której powiemy więcej w sekcji \nameref{cfl-subclassses}.


\begin{definition}
	\textbf{Konfiguracja} PDA to trójka
	\[
		(q, w, \gamma)
	\]
	gdzie
	\begin{itemize}
		\item \( q \) -- aktualny stan
		\item \( w \) -- część słowa pozostała do przeczytania
		\item \( \gamma \) -- (wszystkie) symbole na stosie
	\end{itemize}
\end{definition}

\begin{definition}
	Dla konfiguracji PDA \( P \) definiujemy relację \( \vdash_P \)
	\[
		(q, aw, X\beta) \vdash_P (p, w, \alpha \beta)
		\iff
		(p, \alpha) \in \delta(q, a, X)
	\]
\end{definition}

\begin{definition}
	Definiujemy \( \vdash_P^* \) zwrotne i przechodnie domknięcie \( \vdash_P \)
\end{definition}

Intuicja stojąca za \( \vdash_P \) i \( \vdash_P^* \) jest taka sama jak za \( \rightarrow_G \) i \( \rightarrow_G^* \) tj. chcemy opisać które konfiguracje są osiągalne z których.

\begin{lemma}
	\label{pda-stack-extending-lemma}
	Dla PDA \( P \) jeśli
	\[
		(q, x, \alpha) \vdash_P^* (p, y, \beta)
	\]
	to
	\[
		(q, xw, \alpha\gamma) \vdash_P^* (p, yw, \beta\gamma)
	\]
\end{lemma}
\begin{proof}
	Skoro
	\[
		(q, x, \alpha) \vdash_P^* (p, y, \beta)
	\]
	to znaczy że \( x = a_0a_1\dots a_{n-1}y \) oraz istnieje ciąg \( n \) przejść
	\[
		(q, a_0a_1\dots a_ny, \alpha) \vdash_P (q_1, a_1\dots a_ny, \alpha_1) \vdash_P \dots
		\vdash_P (p, y, \beta)
	\]
	przy czym niektóre z \( a_i \) mogą być \( \varepsilon \) (bo PDA robi \(\varepsilon\)-przejścia).

	Zauważamy, że nigdy nie możemy wyjść poza \( x \) bo nie możemy dodawać liter do słowa, a także nie możemy opróżnić stosu bo to zatrzymuje automat.

	W takim razie możemy zastosować te same przejścia z \( \delta \) uzyskując ciąg
	\[
		(q, a_0a_1\dots a_nyw, \alpha\gamma) \vdash_P (q_1, a_1\dots a_nyw, \alpha_1\gamma) \vdash_P \dots
		\vdash_P (p, yw, \beta\gamma)
	\]
\end{proof}

\subsection{Akceptacja}
\subsubsection{Akceptacja stanem akceptującym}
\[
	L(P) = \set{ w \mid (q_0, w, Z_0) \vdash_P^* (q_F, \eps, \gamma) \land q_F \in F}
\]

\subsubsection{Akceptacja pustym stosem}
\[
	N(P) = \set{ w \mid (q_0, w, Z_0) \vdash_P^* (q, \eps, \eps)}
\]

\subsubsection{Równoważność akceptacji}
Mamy dwie różne definicje tego co znaczy, że jakieś słowo jest akceptowane -- pytanie czy są one równoważne tj. czy jeśli mamy słowo \( w \in L(P) \) to umiemy skonstruować automat \( Q \) taki że \( w \in N(Q)  \) i na odwrót.

Okazuje się że tak, te dwie definicje są równoważne.
\begin{theorem}
	\label{pda-type-equivalence}
	Dla języka \( L \) następujące stwierdzenia są równoważne:
	\begin{enumerate}
		\item Istnieje \( P \) taki że \( L(P) = L \)
		\item Istnieje \( Q \) taki że \( N(Q) = L \)
	\end{enumerate}
\end{theorem}
\begin{proof} \( \)

	\begin{description}
		\item \( (1) \implies (2) \)

		      Dodajemy do \( \Gamma \) specjalny symbol \( \perp \), który będzie oznaczał dno stosu.

		      Na początku \( \varepsilon \)-prześciami modyfikujemy stos tak, aby zawierał on \( Z_0\perp \).


		      Następnie tworzymy stan opróżniający stos \( q_{pop} \) oraz przejścia
		      \[
			      \delta(q_{pop}, \varepsilon, z) = \set{(q_{pop}, \varepsilon)}
		      \]
		      oraz dla każdego stanu akceptującego \( q_F \in F \)
		      \[
			      \delta(q_F, \varepsilon, z) = \set{(q_{pop}, \varepsilon)}
		      \]

		      gdzie \( z \in \Gamma\cup \set{\perp} \)

		      Pokażemy teraz, że \( L(P) = N(Q) \)
		      \begin{description}
			      \item \( L(P) \subseteq N(Q) \)

			            Niech \( w \in L(P) \). Z definicji \( L(P) \) mamy dla pewnego \( q_F \in F \)
			            \[
				            (q_0, w, Z_0) \vdash_P^* (q_F, \varepsilon, \gamma)
			            \]
			            Dzięki nowym przejściom mamy
			            \[
				            (q_F, \varepsilon, z\gamma) \vdash_Q (q_{pop}, \gamma)
			            \]
			            oraz
			            \[
				            (q_{pop}, \varepsilon, \gamma) \vdash_Q^* (q_{pop}, \varepsilon, \varepsilon)
			            \]

			            Łącząc powyższe trzy otrzymujemy
			            \[
				            (q_0, w, Z_0) \vdash_Q^* (q_{pop}, \varepsilon, \varepsilon)
			            \]
			            zatem \( w \in N(Q) \)

			      \item \( N(Q) \subseteq L(P) \)

			            Niech \( w \in N(Q) \) czyli z definicji, dla pewnego \( q \) mamy
			            \[
				            (q_0, w, Z_0) \vdash_Q^* (q, \varepsilon, \varepsilon)
			            \]

			            Zauważmy jednak, że zaczynamy z symbolem \( \perp \) na dnie stosu, którego musieliśmy się w którymś momencie pozbyć.

			            Jedyne przejścia które na to pozwalają to dodane przez nas
			            \[
				            \delta(q_F, \varepsilon, \perp) = \set{(q_{pop}, \varepsilon)}
			            \]
			            oraz
			            \[
				            \delta(q_{pop}, \varepsilon, \perp) = \set{(q_{pop}, \varepsilon)}
			            \]
			            Ponado jeśli dla \( q \neq q_{pop} \)
			            \[
				            \delta(q, a, z) = {(q_{pop}, \gamma)}
			            \]
			            to musi być tak, że \( q \in F \).

			            W takim razie, aby z konfiguracji startowej zaakceptować \( w \) przez pusty stos to musieliśmy przejść z \( q_F \) do \( q_{pop} \).

			            Ponieważ od wyjścia z \( q_F \) wolno nam wykonywać jedynie \(\varepsilon\)-przejścia to skoro na końcu mamy konfigurację \( (q_{pop}, \varepsilon, \varepsilon) \) to w \( q_F \) musieliśmy mieć konfigurację \( (q_F, \varepsilon, \gamma) \) co jest równoważne faktowi, że \( w \in L(P) \).


		      \end{description}

		\item \( (2) \implies (1) \)

		      Zaczynamy tak samo jak poprzednio, dodając symbol dna stosu do \( \Gamma \) i włożeniem go pod \( Z_0 \).

		      Tworzymy jeden stan akceptujący \( q_{acc} \) oraz dla każdego stanu \( q \) dodajemy przejście
		      \[
			      \delta(q, \varepsilon, \perp) = \set{(q_{acc}, \varepsilon)}
		      \]

		      Pokażemy teraz, że \( N(Q) = L(P) \)

		      \begin{description}
			      \item \( N(Q) \subseteq L(P) \)

			            Niech \( w \in N(Q) \). Mamy zatem
			            \[
				            (q_0, w, Z_0) \vdash_Q^* (q, \varepsilon, \varepsilon)
			            \]
			            Z lematu \ref{pda-stack-extending-lemma} mamy
			            \[
				            (q_0, w, Z_0\perp) \vdash_P^* (q, \varepsilon, \perp)
			            \]

			            Dzięki nowym przejściom mamy
			            \[
				            (q, \varepsilon, \perp) \vdash_P (q_{acc}, \varepsilon, \varepsilon)
			            \]
			            czyli
			            \[
				            (q_0, w, Z_0\perp) \vdash_P^* (q_{acc}, \varepsilon, \varepsilon)
			            \]
			            i w efekcie \( w \in L(P) \)


			      \item \( L(P) \subseteq N(Q) \)

			            Niech \( w \in L(P) \) czyli
			            \[
				            (q_0, w, Z_0) \vdash_Q^* (q_{acc}, \varepsilon, \gamma)
			            \]

			            Ponieważ (poza inicjalizacją) nigdy nie dodajemy \( \perp \) na stos to jedyne przejścia które prowadzą z \( q \) do \( q_{acc} \) to
			            \[
				            (q, v, \perp) \vdash_P (q_{acc}, v, \varepsilon)
			            \]
			            Zauważmy, że skoro \( w \) zostało zaakceptowane to \( v = \varepsilon \).

			            Ponadto, poza ostatnim, wszystkie przejścia które wykonuje \( P \) są legalne w \( Q \) z tym, że nie mają \( \perp \) na dnie stosu.

			            W takim razie dla pewnego \( q \) mamy
			            \[
				            (q_0, w, Z_0) \vdash_Q^* (q, \varepsilon, \varepsilon)
			            \]
			            czyli \( w \in N(Q) \).

		      \end{description}

	\end{description}
\end{proof}

\subsection{Równoważność PDA i CFG}

\begin{theorem}
	Niech \( L \) będzie językiem. Następujące stwierdzenia są równoważne:
	\begin{enumerate}
		\item Istnieje gramatyka bezkontekstowa \( G \) taka że \( L(G) = L \)
		\item Istnieje PDA \( P \) taki że \( L(P) = L \)
		\item Istnieje PDA \( Q \) taki że \( N(Q) = L \)
	\end{enumerate}
\end{theorem}
\begin{proof} \( \)
	Oczywiście mamy \( (2) \iff (3) \) z twierdzenia \ref{pda-type-equivalence}.
	Pokażemy, że \( (3) \iff (1) \)


	\begin{description}


		\item \( (3) \implies (1) \)

		      Mając PDA \( P \) akceptujący pustym stosem chcemy stworzyć gramatykę \( G \), która w pewnym sensie symuluje ten automat. Będziemy chcieli żeby produkcje w gramatyce odpowiadały jakimś przejściom tego automatu.

		      Definiujemy nieterminale jako
		      \[ N = \set{S} \cup\set{(q, z, p) : q, p \in Q, z \in \Gamma} \]

		      Intuicyjnie chcemy, żeby taki nieterminal \( (q, z, p) \) reprezentował nam takie słowa \( w \), że
		      \[
			      (q, w, z\gamma) \vdash_P^* (p, \varepsilon, \gamma)
		      \]
		      Dla każdego \( p \in Q \) dodajemy produkcję:
		      \[
			      S \rightarrow (q_0, Z_0, p)
		      \]

		      Ponadto jeśli dla \( a \in \Sigma \) lub \( a = \varepsilon \) mamy
		      \[
			      (r, z_1 \dots z_n) \in \delta(q, a, z)
		      \]
		      to dodajemy dla wszystkich \( (r_1, \dots, r_{n-1}) \in Q^n \) produkcję:
		      \[
			      (q, z, p) \rightarrow a(r, z_1, r_1)(r_1, z_2, r_2)\dots(r_{n-1}, z_n, p)
		      \]

		      Innymi słowy -- jeśli zdecydujemy, że chcemy przejść ze stanu \( q \) do stanu \( p \) ściągając przy tym \( z \) ze stosu i użyjemy przy tym
		      przejścia automatu \( (q, z, p, r_1, z_1\dots z_n) \in \delta \)
		      to musimy teraz przejść ze stanu \( r \) do stanu \( p \) i ściągnąć wszystkie dodane symbole \( z_1, \dots, z_n \) -- nie wiemy jak to zrobimy, dlatego przechodzimy po wszystkich możliwych kombinacjach stanów pośrednich.

		      Po drodze możemy przechodzić przez dowolne stany automatu -- stąd dodajemy taką produkcję dla każdych możliwych \( r_1, \dots, r_{n-1} \).


		      Pozostaje pokazać, że \( N(P) = L(G) \) -- to się robi jakąś indukcją.


		\item \( (1) \implies (3) \)

		      Idea jest analogiczna jak poprzednio tylko tym razem z gramatyki \( G \) robimy automat \( P \). Na szczęście jest on niedeterministyczny więc będziemy mogli ,,zgadnąć'' produkcję a następnie za jej pomocą sparsować słowo z wejścia.


		      Zaczynamy od zmodyfikowania naszej gramatyki tak aby każda produkcja była postaci:
		      \[
			      A \rightarrow a
		      \]
		      gdzie \( w \in \Sigma \cup \set{\varepsilon} \)

		      lub
		      \[
			      A \rightarrow B_1 \dots B_n
		      \]
		      gdzie \( B_1, \dots, B_n \in N \)

		      Robimy to w dwóch krokach:
		      \begin{enumerate}
			      \item Dla każdej litery \( a \in \Sigma \) dodajemy nieterminal \( A \) oraz produkcję \( A \rightarrow a \)
			      \item zastępujemy każde (poprzednio istniejące) wystąpienie litery \( a \) w produkcjach na nieterminal \( A \)
		      \end{enumerate}

		      Zmiana jest niewielka, łatwo pokazać że dostajemy gramatykę produkującą ten sam język.



		      Teraz tworzymy PDA akceptujący przez pusty stos:
		      \begin{itemize}
			      \item \( Q = \set{q} \) -- mamy jeden stan bo będziemy korzystać wyłącznie ze stosu
			      \item \( Z_0 = S \) -- stosowy symbol startowy jest początkowym nieterminalem gramatyki
			      \item alfabetem automatu jest \( \Sigma \)
			      \item \( \Gamma = N \) -- na stos będziemy dodawać jedynie nieterminale
		      \end{itemize}

		      Ponadto prześcia definiujemy:
		      \begin{itemize}
			      \item Jeśli \( A \rightarrow a \) to
			            \[
				            (q, \varepsilon) \in \delta(q, a, A)
			            \]

			      \item Jeśli \( A \rightarrow B_1\dots B_n \) to
			            \[
				            (q, B_1\dots B_n) \in \delta(q, \varepsilon, A)
			            \]
		      \end{itemize}

		      Innymi słowy -- na stosie będziemy trzymać nieterminale, które pozostały nam do przeparsowania.
		      Jeśli widzimy nieterminal stworzony dla litery \( a \) to zjadamy \( a \) z czytanego słowa i parsujemy dalej.

		      Teraz pokazujemy, że faktycznie \( L(G) = N(P) \)

		      Chcemy pokazać indukcją, że dla każdego \( n \in \natural \setminus \set{0} \) oraz każdych \( w, A, \beta \) zachodzi
		      \[
			      A \rightarrow_G^n w\beta \iff (q, w, A) \vdash_P^n (q, \varepsilon, \beta)
		      \]
		      przy czym w gramatyce robimy wywód lewostronny.

		      \begin{itemize}
			      \item Baza indukcji \( n = 1 \)

			            Ponieważ wykonujemy jedną produkcję/jedno przejście to są dwie sytuacje -- albo \( k = 0, w \in \Sigma \cup \set{\varepsilon} \) albo \( w = \varepsilon, k > 0 \).

			            W pierwszym przypadku mamy
			            \[
				            A \rightarrow_G w \iff (q, \varepsilon) \in \delta(q, w, A)
			            \]
			            natomiast w drugim
			            \[
				            A \rightarrow_G B_1 \dots B_k \iff (q, B_1\dots B_k) \in \delta(q, \varepsilon, A)
			            \]
			            obie równoważności wynikają bezpośrednio z konstrukcji.

			      \item Krok indukcyjny



		      \end{itemize}


	\end{description}
\end{proof}