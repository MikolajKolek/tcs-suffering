\section{Drzewa wywodów}

\begin{definition}
    \textbf{Derywacja} albo \textbf{wywód} to ciąg form zdaniowych \( \alpha_0, \dots, \alpha_n \)
    takich, że \( \alpha_i \rightarrow_G \alpha_{i+1}, \alpha_0 = S, \alpha_n = w \in \Sigma^*\)
\end{definition}
\begin{definition}
    \textbf{Wywód lewostronny} to taki wywód \( \alpha_0, \dots, \alpha_1 \)
    w którym jeśli \( \alpha_i = xA\beta_i \)
    to \( \alpha_{i+1} = x\gamma\beta_i \)
    
    Innymi słowy - rozwijamy zawsze skrajnie lewy nieterminal.
\end{definition}

\begin{definition}
    \textbf{Drzewo wywodu} (parse tree) to ukorzenione drzewo z porządkiem na dzieciach w którym:
    \begin{itemize}
        \item każdy wierzchołek ma etykietę z \( \Sigma \cup N \cup \set{\eps} \)
        \item etykieta korzenia to \( S \)
        \item Jeśli wierzchołek ma etykietę \( A \in N \) a jego dzieci \( X_1, \dots, X_n \) to \( (A, X_1\dots X_2) \in P \)
    \end{itemize}
\end{definition}
Wywód lewostronny otrzymujemy przechodząc DFSem odwiedzając dzieci od lewej do prawej.

\begin{definition}
    \textbf{Gramatyka} \( G \) jest \textbf{niejednoznaczna} (ambiguous) jeśli istnieje słowo \( w \in L(G) \) dla którego istnieje więcej\
    niż jedno drzewo wywodu (więcej niż jeden wywód lewostronny).
\end{definition}

Analogicznie definiujemy gramatykę jednoznaczną.

\begin{definition}
    \textbf{Język bezkontekstowy} \(L\) jest \textbf{wewnętrznie niejednoznaczny} (inherently ambiguous) jeśli każda gramatyka która go definiuje\
    jest niejednoznaczna (czyli nie istnieje gramatyka jednoznaczna, która go definiuje).
\end{definition}