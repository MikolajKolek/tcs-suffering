\section{Podstawowe problemy decyzyjne dotyczące języków regularnych}

W poniższych sekcjach będziemy zakładać, że mamy do czynienia z \( \eps\)-NFA lub wyrażeniami regularnymi i będziemy poszukiwać algorytmów działających w czasie wielomianowym (tak jak na ćwiczeniach). 

\subsection{Należenie określonego słowa do języka}

Mając dane \(\eps\)-NFA oraz słowo \(w = a_1a_2a_3\dots a_n\) będziemy (trochę podobnie jak przy konwersji z \(\eps\)-NFA na DFA) trzymać wszystkie stany, w których możemy się znaleźć po ,,przejściu'' prefiksu danego słowa.

Z domknięcia przechodniego stanu startowego idziemy sobie literką \(a_0\) do wszystkich stanów, które są ,,osiągalne'' i ten zbiór również epsilon-domykamy (zauważmy, że każdy krok tego typu można robić liniowo od liczby stanów). 

Formalniej, możemy chcieć sobie zdefiniować ciąg zbiorów stanów \(B_{0}, B_{1}, \dots, B_{n} \):

\[ 
B_{0} = \set{s}^{A, \eps}
\]
\[
B_1 = \Delta(B_0, a_0) 
\]
\[
B_2 = \Delta(B_1, a_1)
\]
\[ 
\dots 
\]

Pamiętacie te delty, co nie? Były w definicji. Hehe. 

W każdym razie, po wyliczeniu tego (widać że wielomianowo) chcecie sprawdzić, czy \( \exists_{q \in B_n} q \in F \).

\subsection{Niepustość języka}

Mając dane \(\eps\)-NFA możemy sprawdzić, czy akceptuje jakieś słowo puszając DFS od stanu startowego, szukając stanów akceptujących. Istnienie jakiejkolwiek ścieżki tego typu jest równoważne istnieniu słowa, które zostanie zaakceptowane; problem osiągalności w grafie jest, jak wiemy, dosyć mocno wielomianowy. 

\subsection{Skończoność i nieskończoność języka regularnego}

Dostajemy \(\eps\)-NFA i chcielibyśmy sprawdzić, czy język który jest przez nie akceptowany jest skończony. Pierwsze co robimy, to wywalamy stany które nie są osiągalne (wielomianowo). 

Jeśli wyjdzie nam że język jest pusty (czyli wywaliliśmy wszystkie stany akceptujące) to znaczy, że język jest skończony (bo jest pusty).

Ponadto wywalamy wszystkie stany, z których nie da się osiągnąć żadnego stanu akceptującego. 

Co dalej? Zauważmy, że jeśli język jest nieskończony to teraz w \(\eps\)-NFA musi istnieć cykl \textbf{nieepsilonowy}. W takim razie zaczynamy ważyć wszystkie krawędzie; epsilonowym dajemy \(0\), wszystkim innym \(-1\). Pytanie nam się degeneruje do pytania, czy istnieje cykl ujemny -- na to pytanie z kolei jesteśmy w stanie wielomianowo odpowiedź algorytmem Bellmana-Forda. 

Puszczamy algorytm Bellmana-Forda; jeśli znajdzie cykl ujemny to oznacza że istnieje cykl nieepsilonowy, a biorąc pod uwagę że stan końcowy jest osiągalny z tego cyklu, to wiemy że można po nim chodzić tyle razy ile chcemy (a więc język jest nieskończony czy coś). 

Nieskończoność języka robimy analogicznie.

\subsection{Minimalność liczby stanów DFA}

Puszczamy algorytm D i sprawdzamy czy coś zmienił (takie zadanie serio było na liście). 

\subsection{Równoważność DFA}

Sprowadzamy to do ww. problemów oraz operacji na DFA. Wystarczy zawuażyć, że 
\[
\( L(A_1) = L(A_2) \iff (L(A_1) \cap \complement{L(A_2)}) \cup (\complement{L(A_1)} \cap L(A_2)) = \varnothing \).
\]