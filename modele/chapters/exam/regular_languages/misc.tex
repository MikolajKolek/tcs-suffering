\section{Słowo resetujące}
\begin{definition}
	Mówimy, że słowo \(w\) jest \textbf{resetujące}, jeśli:
	\[
		\exists_{q \in Q} \forall_{q' \in Q} \hat\delta(q',w) = q
	\]
\end{definition}

Innymi słowy: jest to takie słowo, że niezależnie z którego stanu byśmy je puścili, wszystko koniec końców ,,znajdzie się'' w tym samym stanie.

Jeśli dla DFA istnieje słowo resetujące, to ma ono długość co najwyżej \(\frac{n^3 -n}{6}\). Nie wiemy natomiast, czy istnieje lepsze ograniczenie. Istnieje \textit{Hipoteza Černego}\footnote{Ján Černý był Czechem}, obniżająca to ograniczenie górne do \((n-1)^2\), ale nikt jej jeszcze nie dowiódł.

\begin{lemma}
	Dla DFA o \( n \) stanach oraz stanów \( q_i, q_j \) jeśli istnieje \( w \) takie, że \( \widehat \delta(q_i, w) = \widehat \delta(q_j, w) \) to istnieje też \( v \), \( \card{v} \leq n^2 \) dla którego \( \widehat \delta(q_i, v) = \widehat \delta(q_j, v) \)
\end{lemma}
\begin{proof}
	Tworzymy multigraf skierowany par stanów \( G = (Q^2, E) \).
	Krawędź skierowaną z etykietą \( a \) między wierzchołkami \( (q, p), (q', p') \) istnieje wtedy i tylko wtedy gdy \( \delta(q, a) = q' \land \delta(p, a) = p' \)

	Słowo synchronizujące dla dwóch stanów odpowiada ścieżce w naszym grafie z wierzchołka \( (q_i, q_j) \) do pewnego \( (q, q) \).

	Z założenia mamy, że istnieje pewna taka ścieżka, być może dłuższa niż \( n^2 \). Zauważmy jednak że mamy \( n^2 \) wierzchołków, a wierzchołków na ścieżce nie ma sensu powtarzać, zatem istnieje ścieżka o co najwyżej \( n^2 \) krawędziach.

	Słowo \( v \) z tezy odzyskujemy czytając etykiety na krawędziach tejże ścieżki.
\end{proof}


\begin{theorem}
	Jeśli DFA ma słowo resetujące to ma ono długość co najwyżej \( \card{Q}^3 \)
\end{theorem}
\begin{proof}
	Niech \( Q = \set{q_1, \dots, q_n} \).
	Skoro istnieje słowo resetujące dla całego automatu to w szczególności jest ono resetujące dla każdej pary \( (q_i, q_j) \). Z poprzedniego lematu wiemy, że każdą parę umiemy zsynchronizować słowem długości co najwyżej \( n^2 \).

	Prowadzi nas to do następującego algorytmu:
	\begin{enumerate}
		\item Niech \( q_{1, 1} = q_1, \dots, q_{n, 1} = q_n \)
		\item Niech \( w_1 = \varepsilon \)
		\item Dla \( i = 1, \dots, n - 1 \):
		      \begin{enumerate}
			      \item Znajdź słowo resetujące \( v \) dla stanów \( q_{i, i}, q_{i + 1, i} \)
			      \item \( w_{i + 1} = w_iv \)
			      \item Ustaw \( q_{j, i + 1} = \widehat \delta(q_{j, i}, v) \)
		      \end{enumerate}
		\item \( w_n \) jest słowem resetującym całego automatu.
	\end{enumerate}

	Skonkatenowaliśmy \( n - 1 \) słów długości co najwyżej \( n^2 \) -- zatem \( \card{w_n} \leq n^3 \).

	Indukcyjnie pokazujemy, że \( w_i \) synchronizuje pierwsze \( i \) stanów, a zatem \( w_n \) jest resetujące dla całego automatu.

\end{proof}