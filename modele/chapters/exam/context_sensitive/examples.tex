\section{Przykład konstrukcji CSG i LBA dla prostego języka kontekstowego}

Na ćwiczeniach robiliśmy \( a^nb^nc^n\) który w jednej grupie ćwiczeniowej zajął około 45 minut.\( kmwtw\)

...w innej grupie jednak udało się uporać z tym istotnie szybciej, i przytoczymy to tutaj.

\subsection{LBA}

Automat jest w istocie dość prosty. W każdym kroku głowica wraca na początek, a następnie szuka podciągu \(abc\) (niespójnego!)\
(czyli po prostu przechodzi sprawdzając literki kolejno) jednocześnie oznaczając wykorzystane literki. Jak nie znajdzie, to\
sprawdza czy wszystkie literki są oznaczone (wykorzystane). Jeśli tak to słowo akceptuje, jeśli nie - odrzuca.

\subsection{CSG}

Tu jest istotnie trudniej. Zaczynamy od wygenerowania słowa \(A^nB^n\), a następnie chcemy przypychać żetony.

\begin{enumerate}
    \item Nieterminal \(A\) może wygenerować żeton, jeśli on nie był wykorzystany, a po lewej nic nie ma, lub jest nieterminal wykorzystany \(A'\).
    \item Generowanie żetonu to produkcja \(A \rightarrow [A', Z_1]\).
    \item Żeton możemy przepychać: jeśli lewy sąsiad ma żeton, to ja mogę go wziąc do siebie (\(A \rightarrow [A, Z_1]\)\). Jeśli prawy sąsiad ma żeton,\
    to znaczy że go oddałem, więc mogę się go u siebie pozbyć: \([A, Z_1] \rightarrow A\).
    \item Jeśli żeton \(Z_1\) dotrze do \(B\), to \(B\) zostaje wykorzystane, a żeton zamienia się w \(Z_2\).
    \item Jeśli żeton \(Z_2\) dotrze do ostatniego \(B\), to możemy wygenerować literę \(c\), a żeton znika.
    \item Litery \(a, b\) powstają tylko z \(A', B'\) - czyli wykorzystanych do produkcji C. Oznacza to że nieterminale mogą zniknąć tylko jeśli\
    wyprodukowaliśmy \(n\) razy \(c\) na końcu (żeton nie może zniknąć inaczej); jednocześnie, nie da się wyprodukować więcej żetonów (więc także literek \(c\)).
\end{enumerate}