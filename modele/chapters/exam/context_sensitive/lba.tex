\section{Automat ograniczony liniowo (LBA)}


\begin{definition}

 \textbf{Linear-Bounded Automaton} (LBA, automat ograniczony liniowo) to ograniczona MT będąca największą tuplą na tym przedmiocie 
    \[
        LBA = (Q, \Sigma, \Gamma, \delta, q_0, \blank \footnote{Symbol ten nie jest w zasadzie potrzebny, ale pojawił się na wykładzie - dla kompatybilności z maszyną Turinga.}, F, \vdash, \dashv)
    \]
    gdzie
    \begin{itemize}
        \item \( Q \) -- skończony zbiór stanów
        \item \( \Sigma \subseteq \Gamma \) -- skończony alfabet wejściowy
        \item \( \Gamma \) -- skończony alfabet taśmowy
        \item \( \delta : Q \times \Gamma \rightarrow 
        Q \times \Gamma \times \set{-1, +1}
        \) -- stan, litera \( \rightarrow \) nowy stan, zmiana litery, ruch głowicą (lewo, prawo)
    \end{itemize}
    Przy czym jeśli \( ((q, \vdash), (q', X, d)) \in \delta \) to \( X = \vdash, d = 1 \) analogicznie po drugiej stronie taśmy
    
    \begin{itemize}
        \item \( q_0 \) -- stan startowy
        \item \( \blank \in \Gamma \setminus \Sigma \) -- pusty symbol / zero (domyślny symbol taśmy)
        \item \( F \) -- zbiór stanów akceptujących
    \end{itemize}
    
    Ponadto zakładamy, że \(Q \cap \Gamma = \varnothing\). 
\end{definition}

\subsection{Równoważność CSG i LBA}

\begin{theorem}
    Dla każdej CSG G istnieje LBA A takie że \( L(G) = L(A) \)
\end{theorem}

\begin{proof} 
Automat zaczyna ze słowem i zgaduje jaką produkcję wykonać (do tyłu). Następnie ją podstawia, skraca słowo jeśli trzeba, i sprawdza czy mamy przypadkiem sam symbol startowy. Jeśli tak, to akceptuje słowo.
Jeśli gramatyka ma produkcję \(S \rightarrow \epsilon\), to sprawdzamy rownież puste słowo.
\end{proof}

\begin{theorem}
    Dla każdego LBA A istnieje CSG G takie że \( L(A) = L(G) \)
\end{theorem}

\begin{proof}
Gramatyka będzie operować na klockach domino. Na górze będą literki, na dole symulacja pracy automatu: odpowiedni symbol oraz być może również głowica i / lub symbol końca taśmy. Zaczynamy od wylosowania takich klocków domino, które na górze mają słowo, a na dole mają stan początkowy automatu.
Następnie korzystając z kontekstowości gramatyki, symulujemy przechodzenie głowicy oraz podmienianie przez nią symboli. Jeśli głowica automatu przejdzie do stanu akceptującego, "zwijamy interes", czyli zamieniamy domino z tym stanem w literkę z góry, a następnie każde domino które sąsiaduje
z literką również zwijamy w literkę z góry tego domina. Należy jeszcze zauważyć, że ponieważ automat jest ograniczony liniowo, to nie zabraknie nam miejsca do jego symulacji.

Formalnie niestety należy opisać wszystkie możwlie produkcje, których jest dużo. Nie jest to jednak trudne.
\end{proof}