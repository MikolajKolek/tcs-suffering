\section{Własności języków kontekstowych}

\begin{lemma}
	Języki kontekstowe są zamknięte na przecięcie.
\end{lemma}

\begin{proof}
	Dla dwóch \( LBA A_1, A_2\) tworzymy \(LBA A\), który zapamiętuje wejście na osobnej taśmie, i symuluje automaty kolejno.
	Jeśli jeden z automatów nie skończy pracy to nie ma problemu, bo słowo i tak nie należy do przecięcia.
\end{proof}

\begin{lemma}
	Języki kontekstowe są zamknięte na konkatencaję.
\end{lemma}

\begin{proof}
	Dla dwóch \(CSG G_1, G_2\) tworzymy \(CSG G\) z produkcją \(S \rightarrow S_1S_2\).
\end{proof}

\begin{lemma}
	Języki kontekstowe są zamknięte na dopełnienie.
\end{lemma}

\begin{proof}
	Idzie to z twierdzenia 6.6 (Immerman-Szelepcsény).
\end{proof}

\begin{lemma}
	Języki kontekstowe są zamknięte na sumę.
\end{lemma}

\begin{proof}
	Dla dwóch CSG \(G_1\), \(G_2\) tworzymy CSG \(G\) z produkcją \(S \rightarrow S_1|S_2\).
\end{proof}

\begin{lemma}
	Języki kontekstowe są zamknięte na gwiazdkę Kleene'go.
\end{lemma}

\begin{proof}
	Dla dwóch CSG \(G_1\), \(G_2\) tworzymy CSG \(G\) z produkcją \(S \rightarrow S_1S'\), \(S' \rightarrow S_1S'|\epsilon\) (należy uważać, aby nie zaakceptować pustego słowa).
\end{proof}