\section{Problem TQBF}

Problem TQBF to niejako rozszerzenie SATa: mamy formułę boolowską, ale każda zmienna jest dodatkowo skwantyfikowana \(\forall\) lub \(\exists\).\
Na przykład: \(\forall x \exists y \exists z ((x \vee z) \wedge y)\). Mamy oczywiście odpowiedzieć, czy ta formuła jest prawdziwa. Analogicznie do SATa\
problem ten jest zupełny dla pewnej klasy problemów - w tym przypadku PSPACE.

\begin{theorem}
    Problem TQBF jest PSPACE-zupełny.
\end{theorem}

\begin{proof}[Zarys dowodu.]
    Problem ten jest oczywiście w PSPACE, bo możemy sprawdzić wszystkie konfiguracje. Wystarczy nam do tego wielomianowo wiele miejsca.

    Istotnie trudniej jest z pokazaniem zupełności. Będziemy chcieli zredukować dowolny inny problem z PSPACE, do TQBF. Dla tekiego innego problemu\
    musi istnieć maszyna Turinga, która go rozwiązuje, i zużywa wielomianowo pamięci. Chcemy więc aby nasz TQBF odpowiadał na pytanie, czy istnieje\
    ścieżka obliczająca na tej maszynie która zaczyna w \(q_begin\), kończy w \(q_acc\), i wykonuje nie więcej niż jakąś liczbę kroków \(T\) - skoro miejsce\
    jest ograniczone, to wszystkie obliczenia dłuższe muszą się zapętlić.

    Chcemy więc zdefiniować \(\phi_{q_{begin},q_{acc},T}\). Zrobimy to rekurencyjnie. Niech 
    \[
        \phi_{c_1,c_2,t} = \exists m_1 (\phi_{c_1, m_1, \ceil{t/2}} \wedge \phi_{m_1, c_1, \ceil{t/2}})
    \]
    Oczywiście jeśli w formule \(t=1\), to sprawdzamy czy dane przejście istnieje (przypadek bazowy rekurencji). 

    Taka formuła jednak byłaby zbyt długa (wykładniczo długa). Musimy więc ją uprościć i istotnie to teraz robimy. Zauważmy, że powyższa formuła jest\
    równoważna następującej (gdzie \((c_3, c_4)\) to jakaś para stanów):
    \[
        \exists m_1 (\phi_{c_1, m_1, \ceil{t/2}} \wedge \phi_{m_1, c_1, \ceil{t/2}}) = \exists m_1 \forall(c_3, c_4) \in \{(c_1,m_1),(m_1,c_2)\}\
        (\phi_{c_3,c_4,\ceil{t/2}})
    \]
    Ta formuła już jest odpowiedniej długości, oraz obliczalna wielomianowo. Dowolny problem z PSPACE możemy zatem zredukować do TQBF, oraz on sam jest\
    w PSPACE, więc jest on PSPACE-zupełny.
\end{proof}