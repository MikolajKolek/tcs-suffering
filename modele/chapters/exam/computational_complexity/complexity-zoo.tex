\section{Klasy złożoności}

\subsection{Podstawowe klasy}

\begin{definition}
    Maszyna M (deterministyczna lub nie) \textbf{działa w czasie} \( T : \natural \rightarrow \natural \) jeśli dla każdej konfiguracji startowej \( q_0 w \) każde obliczenie jest akceptujące lub odrzucające i ma długość co najwyżej \( T(\abs{w}) \)
\end{definition}

\begin{definition}
    \( \textsc{DTIME}(f(n)) \) (również po prostu \( \textsc{TIME}(f(n)) \)) to zbiór problemów rozwiązywalnych przez deterministyczną maszynę Turinga działającą w czasie \(f(n)\).
\end{definition}

Analogicznie definiujemy \( \textsc{NTIME}(f(n)) \) dla maszyn niedeterministycznych.

\begin{definition}
    \( \textbf{P} = \textsc{PTIME} = \bigcup_{k=0}^\infty \textsc{DTIME}(n^k) \)
\end{definition}

\begin{definition}
    \( \textbf{NP} = \textsc{NPTIME} = \bigcup_{k=0}^\infty \textsc{NTIME}(n^k) \)
\end{definition}

\begin{definition}
    \( \textbf{\textsc{EXPTIME}} = \bigcup_{k=0}^\infty \textsc{DTIME}(2^{n^k}) \)
\end{definition}

\begin{definition}
    \( \textsc{SPACE}(f(n)) \) to zbiór problemów \textbf{rozwiązywalnych} przez deterministyczną maszynę Turinga, która nigdy nie zużyje więcej niż \(f(n)\) pamięci na raz.
\end{definition}

Analogicznie definiujemy \( \textsc{NSPACE}(f(n)) \) dla maszyn niedeterministycznych.

\begin{definition}
    \( \textbf{\textsc{PSPACE}} = \bigcup_{k=0}^\infty \textsc{SPACE}(n^k) \)
\end{definition}

\begin{definition}
    \( \textbf{\textsc{NPSPACE}} = \bigcup_{k=0}^\infty \textsc{NSPACE}(n^k) \)
\end{definition}

\begin{definition}
    Problem jest w klasie \textsc{coC} jeśli jego dopełnienie jest w klasie \textsc{C}.
\end{definition}

\begin{corollary}
Dla klas deterministycznych \( \textsc{C} = \textsc{coC} \).
\end{corollary}

\subsection{Inne klasy złożoności}

Nie należy się nimi szczególnie przejmować, chociaż warto je znać.

\begin{definition}
    Maszyna Turinga \(M\) ma wyrocznię dla języka \(L\), jeśli ,,ma ona dostęp'' do zbioru \(L\) (w czasie stałym), tj. wpisuje jakieś słowo na specjalną taśmę, przechodzi do odpowiedniego stanu \(q_?\) po czym w jednym kroku automatycznie przechodzi do stanu \(q_{TAK}\) albo \(q_{NIE}\), zależnie od tego czy dane słowo należy do \(L\) czy nie.    
\end{definition}

\begin{definition}
    Język L jest w klasie \(P^{NP}\), jeśli istnieje taki wielomian \(p\) i taka deterministyczna Maszyna Turinga \(M\) z wyrocznią dla jakiegoś problemu z klasy \(\np\) taka, że \(L(M) = L\) i \(M\) działa w czasie \(p\). 
\end{definition}

\begin{definition}
    Język L jest w klasie \(NP^{NP}\), jeśli istnieje taki wielomian \(p\) i taka niedeterministyczna Maszyna Turinga \(M\) z wyrocznią dla jakiegoś problemu z klasy \(\np\) taka, że \(L(M) = L\) i \(M\) działa w czasie \(p\). 
\end{definition}

\begin{definition}
        \(  L \in  \textsc{coNP}^\textsc{NP} \iff \complement{L} \in \textsc{NP}^\textsc{NP} \).
\end{definition}

