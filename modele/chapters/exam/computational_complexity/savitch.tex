\begin{definition}
	Definiujemy problem \textsc{REACHABILITY} jako:
	\begin{itemize}
		\item wejście: Nieskierowany graf \( G = (V, E) \), wierzchołki \(v_1\) i \(v_2\).
		\item pytanie: Czy istnieje ścieżka z \(v_1\) do \(v_2\)?
	\end{itemize}
\end{definition}

\begin{lemma}
	Problem \textsc{REACHABILITY} można rozwiązać w \textsc{SPACE(\(\log^2(n)\))} (nie licząc pamięci na opis grafu).
\end{lemma}

\begin{proof}

	Dodajemy dodatkowy parametr do problemu \(n\) oznaczający ilość maksymalną ilość kroków na trasie (oczywiście na początku równy jest ilości wierzchołków w grafie). Dzięki temu możemy skonstruować algorytm typu dziel i zwyciężaj:

	function reachable\((v_1, v_2, n) \rightarrow TAK/NIE\):

	\begin{itemize}
		\item \( n = 1 \): TAK jeśli \(v_1, v_2\) są sąsiadami w grafie; NIE, jeśli nie są
		\item \( n > 1 \): TAK jeśli istnieje \(v_i\) t.że reachable\((v_1, v_i, \floor{{n+1}/2})\) i reachable\((v_i, v_2, \floor{{n+1}/2}) \) zwracają TAK. Iterujemy się po wszystkich \(v_i\).
	\end{itemize}

	Złożoność pamięciowa wychodzi ładnie, bo głębokość wywołań rekurencyjnych jest logarytmiczna, i w każdym wywołaniu trzymamy liczniki logarytmicznej długości. Stąd też mamy \(\log^2(n)\).

\end{proof}

\begin{theorem}[Savitch]
	\( \textsc{NSPACE}(f(n)) \subset \textsc{PSPACE}(f(n)^2)\)
\end{theorem}

\begin{proof}

	Rozważmy drzewo możliwych konfiguracji osiąganych przez niedeterministyczną Maszynę Turinga \(M\). Jeśli zużywa ona wielomianowo wiele pamięci, to
	znaczy że istnieje taki wielomian \(p\), że \(M\) zużywa maksymalnie \(p(n)\) pamięci roboczej, gdzie \(n\) to rozmiar wejścia.

	Liczbę możliwych konfiguracji maszyny \(M\) możemy przeszacować przez \(c^{p(n)}\), gdzie \(c\) jest jakąś stałą
	(np. \(c = |\Gamma| + |Q|\)). Zauważamy, że to co nas interesuje to to, czy z konfiguracji startowej możemy uzyskać konfigurację końcową.
	Jest to problem \textsc{REACHABILITY}, tyle że sąsiadów naszego grafu skierowanego stanów maszyny - i zamiast sprawdzać czy istnieje krawędź,
	sprawdzamy czy z jednego stanu można przejść w jednym kroku do drugiego.

	Co zatem możemy zrobić, to ordynarnie generować po kolei wszystkie możliwe konfiguracje takie, że na taśmie znajduje się stan akceptujący i pytać,
	czy taka konfiguracja jest osiągalna z tej startowej. Zauważmy, że możemy reuse'ować pamięć, więc taka generacja kolejno wszystkich konfiguracji
	końcowych nas nie boli.

	Jako, że \textsc{REACHABILITY} zużywa deterministycznie \(\log^2(n)\) pamięci, mamy że nasz algorytm będzie zużywać celem sprawdzenia
	osiągalności \(\log^2(c^{p(n)}) = p(n)^2 \cdot \log^2(c) = p(n)^2 \), (\(c\) jest stałą). Miejsce zużyte na trzymanie na jakichś taśmach
	konfiguracji startowej i końcowej jest natomiast oczywiście liniowe od maksymalnej długości maszyny.
\end{proof}

\begin{corollary}
	\( \pspace = \npspace \).
\end{corollary}