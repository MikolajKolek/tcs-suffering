\section{Zawierania klas złożoności}

\subsection{Podstawowe relacje}

Podsumujemy tu (prawie) wszystkie znane nam klasy złożoności.

\begin{corollary}
	\( \textsc{REG} \subset \textsc{CFL}\footnote{Nie zapominajmy o podklasach CFL} \subset \textsc{CSL} \subset \textsc{RE} \)
\end{corollary}

Co więcej, tutaj wiemy, że wszystkie zawierania są ostre.

\begin{corollary}
	\( \textsc{P} \subset \textsc{NP} \subset \textsc{PSPACE} \subset \textsc{EXPTIME} (\subset \textsc{R}) \)
\end{corollary}

Hierarchią tą można również wydłużać w obie strony - np. po lewej mamy \textsc{LOGSPACE}, a po prawej \textsc{EXPSPACE}.

Dla tej hierarchi (jak pewnie wszyscy wiedzą) nie wiemy czy \(\textsc{P} = \textsc{NP}\). Okazuje się również, że nie wiemy czy pozostałe zawierania są
ostre (poza zawieraniem w R) - poza jednym, \( \textsc{P} \neq \textsc{EXPTIME} \).

\begin{lemma}
	\( \textsc{P} \neq \textsc{EXPTIME} \)
\end{lemma}

\begin{proof}
	Skonstruujmy język, który należy do \(\textsc{TIME}(f(n^3))\), ale nie należy do \(\textsc{TIME}(f(n))\). Niech ten język będzie składał się z par (maszyna Turniga, słowo)\
	takich, że maszyna akceptuje to słowo po co najwyżej liniowej liczbie kroków. Ten język jest w \(\textsc{TIME}(f(n^3))\) (damy radę zasymulować), ale nie jest w\
	\(\textsc{TIME}(f(n))\). Pokażmy to ostatnie nie wprost: niech istnieje maszyna \(K\), która odpowie w czasie liniowym, czy dana maszyna zaakceptuje dane\
	wejście w czasie wykładniczym. Skonstruujmy teraz maszynę \(N\), która bierzę na wejściu inną maszynę \(M_w\), odpala \(K\) z pytaniem (\(M_w,M_w\)),\
	i odwraca jej odpowiedź. Wejście do wewnętrznej \(K\) jest ok. 2 razy większe niż do naszej \(N\), więc tu problemu nie ma. Co się jednak stanie, jeśli\
	zapytamy \(N\) o samą siebie? Ona zapyta \(K\) czy \(N\) powinno zaakceptować \(N\) - czyli dokładnie interesujące nas pytanie - po czym odwróci\
	odpowiedź! Oczywiście prowadzi to do sprzeczności, czyli maszyna \(K\) nie może istnieć.

	Wiemy zatem że \(\textsc{TIME}(f(n)) \subsetneq \textsc{TIME}(f(n^3))\). Z tego wynika już to co chcemy:
	\[
		\p \subset \textsc{TIME}(2^n) \subsetneq \textsc{TIME}(2^{3n}) \subset \textsc{EXPTIME}
	\]
\end{proof}

Jest to szczególny przypadek Twierdzenia o hierarchi czasu i de facto udowodniliśmy po drodzę jedną z jego wersji. Pewnie są prostsze sposoby, ale ten działa.

\subsection{Tw. Savitch'a}

Pozostaje jeszcze zająć się relacją między \( \pspace\) a \(\npspace \).

\begin{definition}
	Definiujemy problem \textsc{REACHABILITY} jako:
	\begin{itemize}
		\item wejście: Nieskierowany graf \( G = (V, E) \), wierzchołki \(v_1\) i \(v_2\).
		\item pytanie: Czy istnieje ścieżka z \(v_1\) do \(v_2\)?
	\end{itemize}
\end{definition}

\begin{lemma}
	Problem \textsc{REACHABILITY} można rozwiązać w \textsc{SPACE(\(\log^2(n)\))} (nie licząc pamięci na opis grafu).
\end{lemma}

\begin{proof}

	Dodajemy dodatkowy parametr do problemu \(n\) oznaczający ilość maksymalną ilość kroków na trasie (oczywiście na początku równy jest ilości wierzchołków w grafie). Dzięki temu możemy skonstruować algorytm typu dziel i zwyciężaj:

	function reachable\((v_1, v_2, n) \rightarrow TAK/NIE\):

	\begin{itemize}
		\item \( n = 1 \): TAK jeśli \(v_1, v_2\) są sąsiadami w grafie; NIE, jeśli nie są
		\item \( n > 1 \): TAK jeśli istnieje \(v_i\) t.że reachable\((v_1, v_i, \floor{{n+1}/2})\) i reachable\((v_i, v_2, \floor{{n+1}/2}) \) zwracają TAK. Iterujemy się po wszystkich \(v_i\).
	\end{itemize}

	Złożoność pamięciowa wychodzi ładnie, bo głębokość wywołań rekurencyjnych jest logarytmiczna, i w każdym wywołaniu trzymamy liczniki logarytmicznej długości. Stąd też mamy \(\log^2(n)\).

\end{proof}

\begin{theorem}[Savitch]
	\( \textsc{NSPACE}(f(n)) \subset \textsc{PSPACE}(f(n)^2)\)
\end{theorem}

\begin{proof}

	Rozważmy drzewo możliwych konfiguracji osiąganych przez niedeterministyczną Maszynę Turinga \(M\). Jeśli zużywa ona wielomianowo wiele pamięci, to
	znaczy że istnieje taki wielomian \(p\), że \(M\) zużywa maksymalnie \(p(n)\) pamięci roboczej, gdzie \(n\) to rozmiar wejścia.

	Liczbę możliwych konfiguracji maszyny \(M\) możemy przeszacować przez \(c^{p(n)}\), gdzie \(c\) jest jakąś stałą
	(np. \(c = |\Gamma| + |Q|\)). Zauważamy, że to co nas interesuje to to, czy z konfiguracji startowej możemy uzyskać konfigurację końcową.
	Jest to problem \textsc{REACHABILITY}, tyle że sąsiadów naszego grafu skierowanego stanów maszyny - i zamiast sprawdzać czy istnieje krawędź,
	sprawdzamy czy z jednego stanu można przejść w jednym kroku do drugiego.

	Co zatem możemy zrobić, to ordynarnie generować po kolei wszystkie możliwe konfiguracje takie, że na taśmie znajduje się stan akceptujący i pytać,
	czy taka konfiguracja jest osiągalna z tej startowej. Zauważmy, że możemy reuse'ować pamięć, więc taka generacja kolejno wszystkich konfiguracji
	końcowych nas nie boli.

	Jako, że \textsc{REACHABILITY} zużywa deterministycznie \(\log^2(n)\) pamięci, mamy że nasz algorytm będzie zużywać celem sprawdzenia
	osiągalności \(\log^2(c^{p(n)}) = p(n)^2 \cdot \log^2(c) = p(n)^2 \), (\(c\) jest stałą). Miejsce zużyte na trzymanie na jakichś taśmach
	konfiguracji startowej i końcowej jest natomiast oczywiście liniowe od maksymalnej długości maszyny.
\end{proof}

\begin{corollary}
	\( \pspace = \npspace \).
\end{corollary}

Możemy to podsumować w dość ciekawą hierarchię:

\begin{corollary}
	\( \textsc{REG} \subset \textsc{CFL} \subset \textsc{P} \subset \textsc{NP} \subset (\textsc{PSPACE} = \textsc{NPSPACE}) \subset \textsc{EXPTIME} \subset \textsc{R}
	\subset \textsc{RE} \)
\end{corollary}

\subsection{CSL, a P}

Czy nie brakuje nam w poprzedniej hierarchi CSL? Okazuje się, że nie.

Przypomnijmy, LBA ma rozwiązywalny problem stopu. Aby zapisać licznik ilości kroków w tym problemie, dzięki systemom liczbowym, wystarczy nam liniowo miejsca.

Inplikuje to CSL \(\subset\) NPSPACE, zatem z tw. Savitch'a

\begin{corollary}
	CSL \(\subset\) PSPACE
\end{corollary}

Podamy również bez dowodu następujące fakty:

\begin{lemma}
	CSL \( \not \subset \) NP
\end{lemma}

Jest to intuicyjne - nie ma powodu, by nie miał istnieć problem o liniowej pamięci ale wykładniczym czasie.

\begin{lemma}
	NP \( \not \subset \) CSL
\end{lemma}

Również jest to intuicyjne - NP dopuszcza wielomianową pamięć, LBA jedynie liniową.

Pozwala nam to umiejscowić CSL na naszej hierarchi - jest on równolegle do P, NP:

\begin{corollary}
	\( \textsc{CFL} \subset \textsc{CSL} \subset \textsc{PSPACE} \subset \textsc{R} \)
\end{corollary}

Tutaj można jednak zauważyć, że przecież LBA może rozwiązać 3-SATa, który jest NP-zupełny. Dlaczego więc NP \( \not \subset \) CSL? Jest to prawda,
jednak 3-SAT jest NP-zupełny względem redukcji wielomianowej, nie liniowej. Prowadzi nas to jednak do ciekawej obserwacji: możemy na maszynie Turinga
zredukować dowolny problem z NP do 3-SAT, który możemy już rozwiązać LBA. Ostatecznie uzyskujemy ciekawą obserwację

\begin{corollary}
	Jeżeli istniała by wyrocznia LBA w P, to P = NP.
\end{corollary}

Nie ma to wielkiego praktycznego znaczenia, jednak warto to rozumieć. Podobne zadania pojawiały się również na egzaminie.

Zakończymy tę sekcję oczywistym faktem

\begin{lemma}
	CFL \(\subset\) CSL
\end{lemma}

Jest to poniekąd ciekawe -- przecież automat ze stosem może mieć na tym stosie nieskończenie wiele symboli; LBA - tylko liniowo. Okazuje się jednak że
nic to PDA nie daje, ponieważ ma dostęp tylko do jednego na raz.