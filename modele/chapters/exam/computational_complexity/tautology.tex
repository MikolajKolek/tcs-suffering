\section{Problem TAUTOLOGY}

\begin{definition}
	Problem \textsc{TAUTOLOGY} definiujemy jako problem w którym:
	\begin{itemize}
		\item wejście: formuła rachunku zdań \( \varphi \)
		\item wyjście: Czy \( \varphi \) jest tautologią tj. czy każde wartościowanie wartościuje \( \varphi \) na 1 ?
	\end{itemize}
\end{definition}

\begin{theorem}
	\( \textsc{TAUTOLOGY} \) jest \( \textsc{coNP} \)-zupełny
\end{theorem}
\begin{proof}
	Pokażemy że \( \overline{\textsc{TAUTOLOGY}} \) jest \( \textsc{NP} \)-zupełny.

	Mamy zatem formułę \( \varphi \) i chcemy sprawdzić czy tautologią nie jest, czyli czy istnieje \( v \) takie, że \( v(\varphi) = 0 \).

	Okazuje się, że \( v(\lnot \varphi) = 0 \iff v(\varphi) = 1 \). Oznacza to, że problem \( \overline{\textsc{TAUTOLOGY}} \) i \textsc{SAT} są sobie równoważne (wystarczy zanegować formułę na wejściu) -- czyli \( \overline{\textsc{TAUTOLOGY}} \) jest \textsc{NP}-zupełny czyli \textsc{TAUTOLOGY} jest \textsc{coNP}-zupełny.

\end{proof}