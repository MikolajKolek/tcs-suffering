\section{Przykłady konstrukcji Maszyn Turinga dla prostych języków}

\subsection{Maszyna Turinga rozpoznająca język będący zbiorem liczb pierwszych zapisanych w systemie jedynkowym}

Maszyny Turinga ,,lubią się'' z systemem jedynkowym, a najwygodniej się je konstruuje na wielu taśmach.

\begin{enumerate}
    \item Na pierwszej taśmie mamy więc wejście. Możemy zacząć od sprawdzenia, czy to wejście jest w interesującym nas formacie, oraz czy jest równa przynajmniej 2.
    \item Na drugiej taśmie, mamy obecnie sprawdzaną liczbę. Jak jest ona równa wejściu, to akceptujemy. Zaczynamy od 2.
    \item Na trzeciej taśmie, sprawdzamy czy liczba z pierwszej jest podzielna przez libczę z drugiej. Ale dzielenie jest trudne (kto był na numerkach ten wie),\
    więc po prostu dodajemy liczbę sprawdzaną do obecnej na trzeciej taśmie (zaczynamy od 0), i rozpatrujemy przypadki: mniejsza od liczby z wejścia - znowu dodajemy,\
    równa - odpowiadamy nie, większa - zerujemy taśmę i zwiększamy o 1 liczbę z drugiej taśmy.
\end{enumerate}

\subsection{Maszyna Turinga rozpoznająca język będący zbiorem maszyn, które wyznaczają niepusty język}

Trzymamy kolejkę obecnych stanów maszyny dla różnych wejść, na których ją jednocześnie symulujemy. W jednym kroku, symulujemy jeden krok na każdym\
wejściu, z kolejki, oraz dodajemy nowe (kolejne leksykograficznie) wejście wraz z początkowym stanem maszyny. Widać, że jeśli maszyna symulowana\
kiedyś coś zaakceptuje, to dojdziemy do tego w skończonym czasie - nawet jeżeli maszyna symulowana nigdy się nie zatrzyma na niektórych wejściach!