\section{Własności języków rekurencyjnych i rekurencyjnie\texorpdfstring{\\}{}przeliczalnych}

\subsection{Zamkniętość na przecięcie}

\begin{theorem}
	Języki rekurencyjne są zamknięte na przecięcie.
\end{theorem}

\begin{proof}
	Mając maszynę Turinga z własnością stopu \(A\) i maszynę Turinga z własnością stopu \(B\) możemy stworzyć maszynę Turinga z własnością stopu \(C\), która najpierw symuluje maszynę \(A\) na swoim wejściu, a potem maszynę \(B\). Akceptuje wtedy i tylko wtedy, gdy obie maszyny akceptują. Zauważamy, żę ma ona własność stopu, bo jedyne ciekawe co robi to symuluje MT z własnością stopu. Tym samym przecięcie dwóch problemów z R musi również być w R.
\end{proof}

\begin{theorem}
	Języki rekurencyjnie przeliczalne są zamknięte na przecięcie.
\end{theorem}
\begin{proof}
	Konstrukcja przebiega identycznie jak w powyższym dowodzie, przy czym któraś z maszyn może się nam zaciąć na wejściu -- niespecjalnie nas to obchodzi, bo w takim razie dane słowo i tak nie należałoby do przecięcia dwóch języków. Tym samym mamy maszynę Turinga (niekoniecznie z własnością stopu) która akceptuje wtedy i tylko wtedy, gdy słowo należy do przecięcia dwóch języków rekurencyjnie przeliczalnych; tym samym ta klasa również jest zamknięta na przecięcie.
\end{proof}

\subsection{Zamkniętość na konkatenację}

\begin{theorem}
	Języki rekurencyjne są zamknięte na konkatenację.
\end{theorem}

\begin{proof}
	Konstruujemy niedeterministyczną maszynę Turinga która zgaduje miejsce konkatenacji i symuluje obie częsci na odpowiednich maszynach wyznaczających oba języki.
\end{proof}

\begin{theorem}
	Języki rekurencyjnie przeliczalne są zamknięte na konkatenację.
\end{theorem}

\begin{proof}
	Analogicznie.
\end{proof}

\subsection{Zamkniętość na dopełnienie w językach rekurencyjnych}

\begin{theorem}
	\label{recursive-languages-closed-under-complement}
	Języki rekurencyjne są zamknięte na dopełnienie.
\end{theorem}
\begin{proof}
	Jeśli język jest rekurencyjny, to znaczy, że istnieje Maszyna Turinga \(M\) taka, że zwraca \textsc{TAK} lub \textsc{NIE} w zależności od tego, czy określone słowo do niego należy czy nie.

	W takim razie, otrzymując jakieś słowo \(w\) możemy otrzymać dopełnienie języka \(L(M)\) negując odpowiedź \(M\). Powstała maszyna ma nadal własność stopu, bo \(M\) ma własność stopu.
\end{proof}

\subsection{Rozpacz przy dopełnianiu języków rekurencyjnie przeliczalnych}

\begin{theorem}
	Jeśli \( L \in RE \setminus R \), to \( \complement{L} \not \in RE\)
\end{theorem}
\begin{proof}
	Załóżmy, że \( L \in RE \setminus R \) i \( \complement{L} \in RE\).
	Mamy więc maszynę \( M \) rozpoznającą \( L \) oraz \( N \) rozpoznającą \( \complement{L} \)

	Konstruujemy DMT \( M' \) która:
	\begin{enumerate}
		\item Wczytuje wejście \( w \)
		\item Aż do akceptacji:
		      \begin{enumerate}
			      \item wykonaj jeden krok symulacji \( M \) na \( w \)
			      \item wykonaj jeden krok symulacji \( N \) na \( w \)
		      \end{enumerate}
		\item Jeśli \( M \) zaakceptowało to wypisz ,,TAK''
		\item Jeśli \( N \) zaakceptowało to wypisz ,,NIE''
	\end{enumerate}

	Oczywiście \( w \in L \lor w \in \complement{L} \) więc albo \( M \) zaakceptuje \( w \) albo zrobi to \( N \) -- któraś w końcu musi.
	Stanie się to po skończonej liczbie kroków niezależnie od \( w \) zatem \( L(M') = M \) oraz \( M' \) ma własność stopu.

	Trochę przypał bo w takim razie \( L \in (RE \setminus R) \cap R \) czyli nie istnieje.

\end{proof}

\subsection{Zamkniętość na sumę}

\begin{theorem}
	Języki rekurencyjne są zamknięte na sumę.
\end{theorem}

\begin{proof}
	Symulujemy obie maszyny równolegle.
\end{proof}

\begin{theorem}
	Języki rekurencyjnie przeliczalne są zamknięte na sumę.
\end{theorem}

\begin{proof}
	Analogicznie.
\end{proof}

\subsection{Zamkniętość na gwiazdkę Kleenego}

\begin{theorem}
	Języki rekurencyjne są zamknięte na gwiazdkę Kleenego.
\end{theorem}

\begin{proof}
	Analogicznie jak konkatenacja.
\end{proof}

\begin{theorem}
	Języki rekurencyjnie przeliczalne są zamknięte na gwiazdkę Kleenego.
\end{theorem}

\begin{proof}
	Również analogicznie.
\end{proof}