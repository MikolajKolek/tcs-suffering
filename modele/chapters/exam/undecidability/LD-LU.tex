\section{Język \texorpdfstring{\( L_d \)}{Ld} oraz \texorpdfstring{\(L_u\)}{Lu}}

\subsection{Język \texorpdfstring{\( L_d \)}{Ld}}

\begin{definition}
	Definiujemy język \(L_d\) następująco:

	\[
		L_d = \set{w_i : w_i \not \in L(M_i)}
	\]
\end{definition}

\begin{theorem}
	\( L_d \not \in \re \)
\end{theorem}
\begin{proof}[Dowód przekątniowy.]
	Załóżmy, że istnieje Maszyna Turinga M taka, że \(L(M) = L_d\). Wówczas ma ona jakiś indeks \(k\) (Maszyn Turinga jest przeliczalnie wiele). Rozważmy co się dzieje ze słowem \(w_k\):

	\begin{itemize}
		\item Jeśli \(w_k \in L_d\), to znaczy, że \(w_k \in L(M_k)\), ale to znaczy, że \(w_k \not \in L_d\), co prowadzi nas do sprzeczności.

		\item Jeśli \(w_k \not\in L_d\), to znaczy że \(w_k \not\in L(M_k)\), ale to znaczy, że \(w_k \in L_d\), co również prowadzi do sprzeczności.
	\end{itemize}

	W obu przypadkach otrzymujemy sprzeczność, co kończy dowód.
\end{proof}

\subsection{Język \texorpdfstring{\(L_u\)}{Lu}}

\begin{definition}
	Definiujemy język \(L_u\) następująco:

	\[
		L_u = \set{ (M, w) : w \in L(M)}
	\]

\end{definition}

Język \(L_u\) nazywamy \textit{językiem uniwersalnym}, bo zawiera wszystkie pary (maszyna, słowo) takie, że dana maszyna akceptuje określone słowo.

\begin{theorem}
	\( L_u \in \re \setminus \r \)
\end{theorem}
\begin{proof}
	Oczywiście \( L_u \in \re \) bo możemy po prostu zasymulować maszynę \( M \) na słowie \( w \) na Uniwersalnej Maszynie Turinga.

	Aby pokazać, że \( L_u \notin \r \) załóżmy nie wprost, że mamy maszynę \( M \) z własnością stopu taką, że \( L(M) = L_u \).

	Konstruujemy maszynę M', która:
	\begin{enumerate}
		\item wczytuje wejście \( x \)
		\item symuluje \( M \) na \( (x, x) \)
		\item neguje wyjście \( M \) -- możemy to zrobić bo \( M \) ma własność stopu
	\end{enumerate}

	Dajemy teraz maszynie \( M' \) na wejściu \( w \).

	Mamy teraz dwie sytuacje:
	\begin{itemize}
		\item \( M' \) zaakceptowała \( w \)
		      \[
			      w \in L(M') \Rightarrow (w, w) \notin L_u \Rightarrow w \notin L(w) \Rightarrow w \in L_d
		      \]

		\item \( M' \) odrzuciła \( w \)
		      \[
			      w \notin L(M') \Rightarrow (w, w) \in L_u \Rightarrow w \in L(w)
			      \Rightarrow w \notin L_d
		      \]
	\end{itemize}
	No i mamy przypał bo \( L(M') = L_d \notin \re \) -- w takim razie nie może istnieć maszyna \( M \) z własnością stopu czyli \( L_u \notin \r \).

	Warto zauważyć, że jeśli \( M \) nie ma własności stopu to krok (3) maszyny \( M' \) nie ma sensu -- chcemy, żeby \( M' \) zatrzymywała się na wszystkich TAK-instancjach (z definicji akceptacji), ale NIE-instancje maszyny \( M \) mogą działać w nieskończoność. Wiemy zatem raczej co do \( L(M') \) nie należy niż należy.

\end{proof}