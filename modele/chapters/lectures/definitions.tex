\section{Alfabet}
\begin{description}
	\item \( \Sigma = \set{0, 1} \) -- dla ustalenia uwagi, w~ogólności może to być dowolny niepusty skończony zbiór
	\item \( \Sigma^i \) -- słowa długości \emph{dokładnie} \( i \)
	\item \( \Sigma^0 = \set{\eps} \) -- język jednostkowy (przez \(\eps\) oznaczamy słowo puste)
	\item \( \Sigma^* = \bigcup_{i=0}^\infty \Sigma^i \) -- wszystkie słowa
	\item \( \Sigma^+ = \Sigma^* \setminus \set{\eps} \) -- wszystkie niepuste słowa
\end{description}

\section{Słowa}
Nie wiemy co to są słowa.

\section{Język}
\begin{definition}[Język]
	\textbf{Językiem} nad alfabetem \( \Sigma \) nazywamy dowolny podzbiór \( \Sigma^* \)
\end{definition}

Pytanie teoretyczno-abstrakcyjne -- czy każdy język ma skończony opis?
Otóż nie bardzo, bo programów (opisów) mamy przeliczalnie wiele (\(\aleph_0\), bo są to \emph{elementy} \(\Sigma^*\)), a możliwych języków mamy nieprzeliczalnie wiele (\(\continuum\), bo są to \emph{podzbiory} \(\Sigma^*\)).

Ok, to teraz interesuje nas jakie języki są opisywalne -- tj. dla jakich języków istnieje program \( \Pi \) który dla ustalonego języka \( L \) oraz słowa \( z \) sprawdza, czy \( z \in L \).

\section{Problemy obliczeniowe}
Ogólnie mówiąc \textbf{problem decyzyjny} to problem pytający się czy jakiś obiekt spełnia określony warunek, tj. zwraca odpowiedź TAK/NIE.

Jak się to ma do języków?
Otóż nasze obiekty musimy umieć zakodować binarnie -- np. liczby możemy zapisać ich reprezentacją binarną, grafy macierzą incydencji (zapisujemy jej kolejne wiersze), itp.

W takim razie języki to problemy obliczeniowe a pytanie ,,Czy dany obiekt spełnia warunek?'' zamieniamy na ,,Czy słowo opisujące ten obiekt należy do języka, w którym słowa to kody wszystkich poprawnych instancji?''.

\section{Opis języka}
Języki możemy scharakteryzować pod względem złożoności.

Dostajemy coś, co się nazywa \textbf{hierarchią Chomsky'ego}
\begin{enumerate}
	\item języki regularne
	      \begin{enumerate}
		      \item automat skończony
		      \item gramatyka liniowa
		      \item wyrażenie regularne
	      \end{enumerate}

	\item języki bezkontekstowe
	      \begin{enumerate}
		      \item automat ze stosem
		      \item gramatyka bezkontekstowa
	      \end{enumerate}

	\item języki kontekstowe
	      \begin{enumerate}
		      \item gramatyka kontekstowa
	      \end{enumerate}

	\item języki rekurencyjnie przeliczalne
	      \begin{enumerate}
		      \item maszyna Turinga
	      \end{enumerate}
\end{enumerate}

\section{Operacje}
\subsection{Na słowach}
Słowa możemy traktować jako funkcję \(n \mapsto \Sigma\) (gdzie \( n \in \natural \) -- tzn. możemy zapytać taką funkcję, jaka literka jest pod danym indeksem.

\begin{definition}
	\textbf{Konkatenacją} nazywamy funkcję
	\begin{equation*}
		{\cdot}\colon \Sigma^* \times \Sigma^* \mapsto \Sigma^*
	\end{equation*}
	zdefiniowaną następująco:
	\begin{equation*}
		\pars{v \cdot w}\pars{i} = \begin{cases}
			v\pars{i}           & \text{ gdy } i \leq \len{v}                     \\
			w\pars{i - \len{v}} & \text{ gdy } \len{v} < i \leq \len{v} + \len{w}
		\end{cases}
	\end{equation*}
	Mówiąc po ludzku, jest to po operacja sklejenia dwóch słów w~jedno.
	Zazwyczaj będziemy pomijać kropkę i~zapisywać po prostu \(vw\).
\end{definition}
Własności konkatenacji:
\begin{itemize}
	\item łączność -- \( (ab)c = a(bc) \)
	\item element neutralny -- \( a\eps = \eps a = a \)
	\item \(\len{ab} = \len{a} + \len{b}\)
\end{itemize}
Zatem \( (\Sigma ^*, \cdot , \eps) \) to \emph{monoid}.
\begin{definition}[Potęga słowa]
	\[
		w^n = \begin{cases}
			0        & \text{ gdy } n = 0 \\
			w^{n-1}w & \text{ gdy } n > 0
		\end{cases}
	\]
\end{definition}

\subsection{Na językach:}
\begin{itemize}
	\item suma \( L_1 \cup L_2 \)
	\item przecięcie \( L_1 \cap L_2 \)
	\item katenacja
	      \begin{equation*}
		      L_1\cdot L_2 = \set{w_1w_2 : w_1 \in L_1 \land w_2 \in L_2}
	      \end{equation*}
	      Widzimy, że dla dowolnego \(L \subseteq \Sigma^*\):
	      \begin{gather*}
		      L\emptyset = \emptyset L = \emptyset\\
		      L\set{\eps} = \set{\eps}L = L
	      \end{gather*}
	\item potęga
	      \begin{equation*}
		      L^n = \begin{cases}
			      \set{\varepsilon} & \text{ gdy } n = 0 \\
			      L^{n-1} \cdot L   & \text{ gdy } n > 0
		      \end{cases}
	      \end{equation*}
	\item gwiazdka Kleenego
	      \begin{equation*}
		      L^* = \bigcup_{n=0}^\infty L^n
	      \end{equation*}
\end{itemize}

Dla rozróżnienia sklejanie słów będziemy nazywać \textit{konkatenacją} a sklejanie języków \textit{katenacją}.



\section{Algebra Kleenego}
Napis \(a \leq b\) jest skrótem dla \(a + b = b\).

\begin{definition}[Algebra Kleenego]
	\textbf{Algebrą Kleenego} nazywamy tuplę \( \mathbf{A} = (A, +, \cdot, ^*, 0, 1) \), która spełnia następujące warunki:
	\begin{enumerate}
		\item \((a + b) + c = a + (b + c)\)
		\item \(a + b = b + a\)
		\item \(a + a = a\)
		\item \(a + 0 = a\)
		\item \( (a \cdot b) \cdot c = a \cdot (b \cdot c) \)
		\item \( a \cdot 1 = 1 \cdot a = a \)
		\item \( a \cdot 0 = 0 \cdot a = 0 \)
		\item \( a\cdot(b + c) = a\cdot b + a \cdot c \)
		\item \( (a + b)c = ac + bc \)
		\item \( 1 + aa^* \leq a^* \)
		\item \( 1 + a^*a \leq a^* \)
		\item \( ax + b \leq x \implies a^*b \leq x \)
		\item \( xa + b \leq x \implies ba^* \leq x \)
	\end{enumerate}
\end{definition}
Tak się składa, że
\begin{equation*}
	\pars{\powerset\pars{\Sigma^*}, {\cup}, {\cdot}, {}^*, \emptyset, \set{\eps}}
\end{equation*}
jest algebrą Kleenego.

\begin{lemma}[Lemat Ardena]
	Niech \(A, B, X \subseteq \Sigma^*\) i załóżmy, że
	\begin{equation*}
		A\cdot X \cup B = X
	\end{equation*}
	Wtedy język \(X = A^*B\) jest rozwiązaniem tego równania, przy czym
	\begin{enumerate}[label={(\roman*)}]
		\item jest rozwiązaniem najmniejszym w sensie~inkluzji
		\item jeśli \(\eps \not\in A\), to jest jedynym rozwiązaniem
	\end{enumerate}
\end{lemma}
\begin{proof}
	Pokażmy najpierw, że \(A^*B\) spełnia równanie.
	\begin{equation*}
		A\pars{A^*B} \cup B
		= \pars{AA^*}B \cup B
		= \pars{AA^*\cup \set{\eps}}B
		= A^*B
	\end{equation*}
	Przyjrzyjmy się teraz dodatkowym wnioskom.
	\begin{enumerate}[label={(\roman*)}]
		\item Weźmy dowolne rozwiązanie \(X\).

		      Najpierw pokażemy indukcyjnie że dla każdego \( k \) mamy \( A^kB \subseteq X \).
		      \begin{enumerate}[label={{\arabic*}.}]
			      \item Baza indukcji -- przyjmujemy \(k = 0\):
			            \begin{equation*}
				            A^0B = \set{\eps}B = B \subseteq X
			            \end{equation*}
			            Ostatnie zawieranie wynika bezpośrednio z~równania: \(X = AX \cup B\).
			      \item Krok indukcyjny -- zakładamy, że \(A^kB \subseteq X\) i~dowodzimy, że również \(A^{k + 1}B \subseteq X\). Wiemy, że \(A^{k + 1}B = AA^kB\). Ponieważ \(A^kB \subseteq X\), to \(AA^kB \subseteq AX \subseteq X\), gdzie ostatnią inkluzję ponownie bierzemy bezpośrednio z~równania.
		      \end{enumerate}
		      Co nam to daje? To, że
		      \begin{equation*}
			      \bigcup_{k \in \natural} A^kB \subseteq X
		      \end{equation*}
		      bo skoro każdy z~sumowanych zbiorów zawiera się w~\(X\), to ich suma też. Zwróćmy teraz uwagę, że
		      \begin{equation*}
			      \bigcup_{k \in \natural} A^kB = A^*B
		      \end{equation*}
		      i~mamy \(A^*B \subseteq X\), tak, jak chcieliśmy.
		\item Załóżmy, że \( \eps \notin A\) i pokażmy, że jeśli \(X\) jest rozwiązaniem to \(X \subseteq A^*B\). Z~tego otrzymamy już równość, bo inkluzję w~drugą stronę udowodniliśmy w~poprzednim podpunkcie.

		      Prowadzimy dowód nie wprost: załóżmy, że istnieją słowa należące do \(X\), ale nie do \(A^*B\). Niech \(w\) będzie najkrótszym takim słowem (w~przypadku remisu dowolnym z~najkrótszych).

		      Zastanówmy się najpierw: czy może być \(w \in B\)? Otóż nie, ponieważ \(\eps \in A^*\), więc gdyby \(w \in B\), to \(w = \eps w \in A^*B\), a~przecież przed chwilą założyliśmy, że tak nie jest.

		      Zatem \(w \not\in B\). Przypomnijmy, że \(w \in X = AX \cup B\), bo \(X\) jest rozwiązaniem naszego równania. Zatem skoro \(w \not\in B\), to siłą rzeczy \(w \in AX\), bo ma należeć do sumy. Oznacza to, że \(w = uv\) dla pewnych \(u \in A\), \(v \in X\).

		      Niewątpliwie \(u \neq \eps\), bo \(\eps \not\in A\), na mocy założeń twierdzenia. Zatem długość \(v\) jest \emph{silnie mniejsza} od długości \(w\).

		      Na koniec zastanówmy się, czy może zachodzić \(v \in A^*B\). Okazuje się, że nie może. Dlaczego? Bo gdyby tak było, to
		      \begin{equation*}
			      w = uv \in AA^*B
		      \end{equation*}
		      ale
		      \begin{equation*}
			      AA^*B
			      = A\pars{\bigcup_{k = 0}^{\infty}A^k}B
			      = \pars{\bigcup_{k = 1}^{\infty}A^k}B
			      = \bigcup_{k = 1}^{\infty}\pars{A^kB}
			      \subseteq \bigcup_{k = 0}^{\infty}\pars{A^kB}
			      = \pars{\bigcup_{k = 0}^{\infty}A^k}B
			      = A^*B
		      \end{equation*}
		      Czyli mielibyśmy \(w \in A^*B\), a~przecież założenie, że tak nie jest, już śni nam się po nocach.

		      Znaleźliśmy zatem śmieszne słowo \(v\), które należy do \(X\), \emph{nie} należy do \(A^*B\), i~jest krótsze od \(w\). Ale chwila, przecież \(w\) miało być najkrótszym należącym do \(X\) i~nienależącym do \(A^*B\)! Uzyskana sprzeczność dowodzi tezy.
	\end{enumerate}
\end{proof}

\begin{lemma}
	\(\leq\) jest porządkiem częściowym
\end{lemma}
\begin{proof} Porządek częściowy musi spełniać trzy warunki:
	\begin{enumerate}
		\item zwrotność, czyli \( a \leq a \) -- z~warunku (3) dla algebry Kleenego mamy dokładnie \(a + a = a\).
		\item przechodniość, czyli jeśli \(a \leq b\) oraz \(b \leq c\) to \(c \leq a\)
		      \begin{equation*}
			      a + c = a + \pars{b + c} = \pars{a + b} + c = b + c = c
		      \end{equation*}
		\item antysymetryczność, czyli jeśli \(a \leq b\) oraz \(b \leq a\) to \(a = b \)
		      \begin{equation*}
			      b = a + b = b + a = a
		      \end{equation*}
	\end{enumerate}
\end{proof}