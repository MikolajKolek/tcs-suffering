\section{Minimalizacja automatu}

\begin{definition}
	Stany \( q_1, q_2 \) zadanego DFA są \textbf{A-równoważne} jeśli
	\[
		\forall_{w \in \Sigma^*} \hat \delta(q_1, w) \in F \iff \hat \delta(q_2, w) \in F
	\]
	i \textbf{A-rozróżnialne} jeśli
	\[
		\exists_{w \in \Sigma^*} \hat \delta(q_1, w) \in F \iff \hat \delta(q_2, w) \notin F
	\]
\end{definition}

\subsection{Algorytm D}
\begin{itemize}
	\item Wejście: DFA \( A \)
	\item Wyjście: (trójkątna) macierz Boolowska \( M \), w której \( M_{i, j} = 1 \iff a_i, a_j \) są rozróżnialne
\end{itemize}

Na początku wypełniamy \( M \) zerami.
Jeśli \( a_i \in F, a_j \notin F \) to \( M_{i, j} = 1 \)
Dopóki macierz \( M \) się zmienia to dla każdych dwóch stanów \( q_1, q_2 \) oraz litery \( a \) jeśli \( \hat \delta(q_1, a) = q_k, \hat \delta(q_2, a) = q_l \) oraz \( M_{k, l} = 1 \) to \( M_{i, j} := 1 \)

\begin{lemma}
	A-równoważnosć jest relacją równoważności
\end{lemma}
\begin{proof}
	Wszystkie własności wynikają wprost z faktu, że
	w definicji mamy logiczną równoważność pod dużym kwantyfikatorem.
\end{proof}

\begin{lemma}
	Relacja \(\sim_A\) A-równoważności jest kongruencją tzn.
	\[
		\forall_{T \in [Q]_{\sim_A}}
		\forall_{q_1, q_2 \in T}
		\forall_{a \in \Sigma}
		\exists_{T' \in [Q]_{\sim_A}}
		\delta(q_1, a) \in T' \land \delta(q_2, a) \in T'
	\]
\end{lemma}
\begin{proof}

\end{proof}

Wszystko super, umiemy minimalizować automaty, ale pojawia się ważne pytanie -- Czy zminimalizowany automat jest \textit{najmniejszy} tj. czy ma najmniej stanów ze wszystkich automatów, które rozpoznają dany język?

Otóż okazuje się, że tak -- w przeciwnym razie biorąc sumę dwóch równoważnych automatów, z których jeden ma mniej stanów, jakieś dwa stany większego automatu muszą być równoważne jednemu stanowi mniejszego, a co za tym idzie, ten większy nie był minimalny.

\begin{theorem}[Myhill--Nerode]
	Język \( L \) jest regularny wtedy i tylko wtedy gdy
	istnieje relacja równoważności \( \equiv \subseteq \Sigma^* \times \Sigma^*\) o skończenie wielu klasach równoważności, która jest \(L\)-kongruentna tzn.
	\begin{enumerate}
		\item \(\forall_{v \in \Sigma^*} x \equiv y \implies xv \equiv yv \)
		\item \( x \equiv y \implies (x \in L \iff y \in L)\)
	\end{enumerate}
\end{theorem}
\begin{proof} \( \)
	\begin{description}
		\item \( \implies \)

		      Skoro \( L \) jest regularny to istnieje DFA \( A \), na podstawie którego definiujemy
		      \[
			      x \equiv y \iff \hat \delta(s, x) = \hat \delta(s, y)
		      \]
		      Łatwo sprawdzić że to działa.

		\item \( \impliedby \)

		      Tworzymy automat, którego stanami są klasy równoważności słów wyznaczone przez relację \( \equiv \)
	\end{description}
\end{proof}