\section{Deterministyczne Automaty Skończone}
\subsection{Definicja}
\begin{definition}
	\textbf{Deterministyczny Automat Skończony} (DFA od deterministic finite automaton) to tupla:
	\[
		A = (Q, \Sigma, \delta, s, F)
	\]
	gdzie
	\begin{itemize}
		\item \( Q \) jest skończonym zbiorem stanów
		\item \( \Sigma \) jest skończonym alfabetem
		\item \( \delta: Q \times \Sigma \rightarrow Q \) jest funkcją przejścia
		\item \( s \in Q \) jest stanem startowym
		\item \( F \subseteq Q \) jest zbiorem stanów akceptujących (końcowych)
	\end{itemize}
\end{definition}

To jest bardzo abstrakcyjna i formalna definicja, w praktyce automat będziemy reprezentować jako graf skierowany, albo jako tabelkę przejść.

O automacie możemy myśleć jako o maszynie, która rozpoznaje czy zadane słowo należy do jakiegoś języka (związanego z tymże automatem). Zaczynamy w stanie startowym i przechodzimy do kolejnego stanu zjadając przy tym kolejne litery ze słowa.

Aby nieco sformalizować powyższe zdanie wprowadzamy funkcję \( \hat \delta \), która definiuje w jakim stanie kończymy jeśli zaczynamy w stanie \( q \) z danym słowem \( w \)
\begin{definition}
	\( \hat \delta : Q \times \Sigma^* \rightarrow Q \)
	\[
		\hat \delta(q, w) = \begin{cases}
			q                            & \text{ jeśli } w = \eps             \\
			\delta(\hat \delta(q, x), a) & \text{ jeśli } w = xa, a \in \Sigma \\
		\end{cases}
	\]
\end{definition}

\begin{definition}
	Językiem akceptowanym przez automat \( A = (Q, \Sigma, \delta, s, F) \) nazywamy
	\[
		L(A) = \set{w \in \Sigma^* \mid \hat \delta(s, w) \in F}
	\]

	Podobnie, słowo \( w \) jest akceptowane przez automat \( A \) jeśli \( w \in L(A) \)
\end{definition}


\subsection{Przykład}
Rozważmy automat zadany tabelą:
\begin{center}
	\begin{tabular}{c|c c}
		        & \(a\)   & \(b\)   \\
		\hline
		\(q_0\) & \(q_1\) & \(q_0\) \\
		\(q_1\) & \(q_0\) & \(q_1\) \\
	\end{tabular}
\end{center}
\( s = q_0, F = \set{q_0} \)

\begin{figure}[H]
	\centering
	\includegraphics[scale=0.75]{img/even-a-automaton.png}
	\caption{Zadany automat w formie grafu}
\end{figure}


Niech \( L \) -- wszystkie słowa nad \( \set{a, b}^* \) w których występuje parzyście wiele \( a \).
Twierdzimy, że \( L(A) = L \).

\begin{proof} \( \)

	\texttt{@PuchatyPompon}

	\begin{description}
		\item ,,\(\subseteq\)''

		\item ,,\(\supseteq\)''
	\end{description}
\end{proof}