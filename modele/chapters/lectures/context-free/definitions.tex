\section{Definicje}
\begin{definition}
	\textbf{Gramatyka bezkontekstowa} (Context-Free Grammar) to \( G = (N, \Sigma, P, S \) gdzie
	\begin{itemize}
		\item \( N \) to skończony zbiór zmiennych (nieterminale)
		\item \( \Sigma \) - alfabet (terminale)
		\item \( P \) - produkcje \( P \subseteq N \times (N \cup \Sigma)^* \)
		\item \( S \in N \) - symbol startowy
	\end{itemize}
\end{definition}

\begin{definition}
	\textbf{Forma zdaniowa} to dowolne słowo nad \( (N \cup \Sigma)^* \)
\end{definition}

\begin{definition}
	Dla gramatyki \( G \) definiujemy relację \( \rightarrow_G \) na formach zdaniowych.
	\[
		\alpha \rightarrow_G \beta \iff
		\exists \alpha_1, \alpha_2, \gamma \in (N \cup \Sigma)^* \exists A \in N : (A, \gamma) \in P \land
		\alpha = \alpha_1  A \alpha_2 \land \beta = \alpha_1 \gamma \alpha_2
	\]
\end{definition}
\begin{definition}
	\( \rightarrow_G^* \) to domknięcie i przechodnie domknięcie \( \rightarrow_G \)
\end{definition}
\begin{definition}
	\textbf{Język  generowany} przez gramatykę G to
	\[
		L(G) = \set{w \in \Sigma^* \mid S \rightarrow_G^* w}
	\]
	podobnie dla \( A \in N \)
	\[
		L(G) = \set{w \in \Sigma^* \mid A \rightarrow_G^* w}
	\]
\end{definition}
\begin{definition}
	\textbf{Derywacja} albo \textbf{wywód} to ciąg form zdaniowych \( \alpha_0, \dots, \alpha_n \)
	takich, że \( \alpha_i \rightarrow_G \alpha_{i+1}, \alpha_0 = S, \alpha_n = w \in \Sigma^*\)
\end{definition}
\begin{definition}
	Gramatyka \( G \) jest \textbf{jednoznaczna} (unambiguous) jeśli dla każdego słowa \( w \in L(G) \) istnieje dokładnie jedno drzewo wywodu
	(dokładnie jeden wywód lewostronny).
\end{definition}
\begin{definition}
	\textbf{Wywód lewostronny} to taki wywód \( \alpha_0, \dots, \alpha_1 \)
	w którym jeśli \( \alpha_i = xA\beta_i \)
	to \( \alpha_{i+1} = x\gamma\beta_i \)

	Innymi słowy - rozwijamy zawsze skrajnie lewy nieterminal.
\end{definition}

\begin{definition}
	\textbf{Drzewo wywodu} (parse tree) to ukorzenione drzewo z porządkiem na dzieciach w którym:
	\begin{itemize}
		\item każdy wierzchołek ma etykietę z \( \Sigma \cup N \cup \set{\eps} \)
		\item etykieta korzenia to \( S \)
		\item Jeśli wierzchołek ma etykietę \( A \in N \) a jego dzieci \( X_1, \dots, X_n \) to \( (A, X_1\dots X_2) \in P \)
	\end{itemize}
\end{definition}
Wywód lewostronny otrzymujemy przechodząc DFSem odwiedzając dzieci od lewej do prawej.

\begin{definition}
	Język jest bezkontekstowy (Context-Free Language) jeśli jest generowany przez jakąś gramatykę bezkontekstową.
\end{definition}
\begin{definition}
	Język jest niejednoznaczny (inherently ambiguous) jeśli nie istnieje jednoznaczna gramatyka, która go generuje.
\end{definition}
\begin{definition}
	Gramatyka $G$ jest w postaci normalnej Chomsky'ego (CNF) jeżeli dla każdej produkcji jest tak, że jest ona postaci:

	\begin{enumerate}
		\item \(A \rightarrow BC\) lub
		\item \(A \rightarrow a\) lub
		\item \(S \rightarrow \eps\)
	\end{enumerate}

	gdzie \(A,B,C \in N\) (możliwe że są sobie równe), \(S\) to symbol startowy, a \(a\) to jakiś symbol terminalny. W przypadku, w którym istnieje produkcja \(S \rightarrow \eps\), \(S\) nie może znajdywać się po prawej stronie żadnej produkcji.
\end{definition}