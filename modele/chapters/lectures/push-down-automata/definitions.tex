\section{Nieformalna definicja}
Tak jak w przypadku DFA i NFA mieliśmy automaty z jakimiś stanami, które przechodziły (albo nie) po kolejnych literach, tak automaty ze stosem mają jeszcze dodatkowo stos, na podstawie którego możemy podejmować decyzję co zrobić.

\section{Formalnie}
\begin{definition}
	\textbf{Automat ze stosem} (pushdown automaton PDA) definiujemy jako
	\[
		P = (Q, \Sigma, \Gamma, \delta, q_0, Z_0, F)
	\]
	gdzie
	\begin{itemize}
		\item \( Q \) -- zbiór stanów
		\item \( \Sigma \) -- skończony alfabet słów
		\item \( \Gamma \) -- skończony alfabet stosu
		\item \( \delta : Q \times (\Sigma \cup \set{\eps}) \times \Gamma \rightarrow \powerset\pars{Q \times \Gamma^*} \) -- funkcja przejścia
		\item \( q_0 \) -- stan startowy
		\item \( Z_0\) -- stosowy symbol startowy
		\item \( F \subseteq Q \) -- zbiór stanów akceptujących
	\end{itemize}
\end{definition}
Intuicyjnie \( \delta \) dla każdego stanu \( q \), litery \( a \), symbolu na szczycie stosu \( z \) oddaje zbiór nowych stanów wraz z symbolami które mają być dodane na stos.
Symbol \( z \) jest usuwany ze szczytu stosu w momencie przejścia, wiele nowych symboli może zostać dodanych.
Jeśli zdarzy się, że opróżnimy stos to mamy tak zwany przypał, ale się tym nie przejmujemy.

\begin{definition}
	\textbf{Konfiguracja} PDA to trójka
	\[
		(q, w, \gamma)
	\]
	gdzie
	\begin{itemize}
		\item \( q \) -- aktualny stan
		\item \( w \) -- część słowa pozostała do przeczytania
		\item \( \gamma \) -- (wszystkie) symbole na stosie
	\end{itemize}
\end{definition}

\begin{definition}
	Dla konfiguracji PDA \( P \) definiujemy relację \( \vdash_P \)
	\[
		(q, aw, X\beta) \vdash_P (p, w, \alpha \beta)
		\iff
		(p, \alpha) \in \delta(q, a, X)
	\]
\end{definition}

\begin{definition}
	Definiujemy \( \vdash_P^* \) jako zwrotne i przechodnie domknięcie \( \vdash_P \)
\end{definition}

\begin{lemma}
	Dla PDA \( P \) jeśli
	\[
		(q, x, \alpha) \vdash_P^* (p, y, \beta)
	\]
	to
	\[
		(q, xw, \alpha\gamma) \vdash_P^* (p, yw, \beta\gamma)
	\]
\end{lemma}

\section{Akceptacja}
\subsection{Akceptacja stanem akceptującym}
\[
	L(P) = \set{ w \mid (q_0, w, Z_0) \vdash_P^* (q_F, \eps, \gamma) \land q_F \in F}
\]

\subsection{Akceptacja pustym stosem}
\[
	N(P) = \set{ w \mid (q_0, w, Z_0) \vdash_P^* (q, \eps, \eps)}
\]

\subsection{Równoważność}
Mamy dwie różne definicje tego co znaczy, że jakieś słowo jest akceptowane -- pytanie czy są one równoważne tj. czy jeśli mamy słowo \( w \in L(P) \) to umiemy skonstruować automat \( Q \) taki że \( w \in N(Q)  \) i na odwrót.

Z \( P, L(P) \) łatwo jesteśmy w stanie skonstruować \( Q, N(Q) \) -- jeśli stan \( q_F \) był akceptujący to przechodzimy do stanu, który epsilon przejściami opróżniamy stos.

Z \( Q, N(Q) \)



