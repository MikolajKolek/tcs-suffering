\section{Maszyna Turinga}
\begin{definition}
	\textbf{Maszyna Turinga} to tupla
	\[
		MT = (Q, \Sigma, \Gamma, \delta, q_0, B, F)
	\]
	gdzie
	\begin{itemize}
		\item \( Q \) -- skończony zbiór stanów
		\item \( \Sigma \subseteq \Gamma \) -- skończony alfabet wejściowy
		\item \( \Gamma \) -- skończony alfabet taśmowy
		\item \( \delta : Q \times \Gamma \rightarrow
		      Q \times \Gamma \times \set{-1, +1}
		      \) -- stan, litera \( \rightarrow \) nowy stan, zmiana litery, ruch głowicą
		\item \( q_0 \) -- stan startowy
		\item \( B \in \Gamma \setminus \Sigma \) -- pusty symbol / zero (domyślny symbol taśmy)
		\item \( F \) -- zbiór stanów akceptujących
	\end{itemize}
\end{definition}

\begin{definition}
	\textbf{Opis chwilowy} (konfiguracja) Maszyny Turinga to
	ciąg
	\[
		X_1, \dots, X_{k-1}, q, X_k, \dots, X_m
	\]
	gdzie \( X_i \in \Gamma, q \in Q\) oraz \( X_1, X_m \neq B \) przy czym możemy mieć sytuację \( X_1 = B \) jeśli mamy \( q, X_1 \) oraz \( X_m = B \) gdy mamy \( q, X_m \)

	Maszyna Turinga zaczyna w stanie \( q_0, w \)
\end{definition}

\begin{definition}
	Definiujemy relację \( \vdash_M \) na konfiguracjach MT.

	Przejście w lewo ,,wewnątrz'':
	\[
		X_1, \dots, X_{k-1}, q, X_k, \dots, X_m
		\vdash_M
		X_1, \dots, X_{k-2}, q, X_{k-1}, Y, \dots X_m
	\]
	Przejście w prawo ,,wewnątrz'':
	\[
		X_1, \dots, X_{k-1}, q, X_k, \dots, X_m
		\vdash_M
		X_1, \dots, Y, q, X_{k+1}, \dots X_m
	\]

	Podobnie definiujemy skrajne przejścia.

	Przejście w lewo ,,na zewnątrz''
	\[
		q X_1, \dots, X_m \vdash_M
	\]


\end{definition}


\begin{definition}
	\textbf{Język akceptowany przez Maszynę Turinga} definiujemy jako
	\[
		L(M) = \set{w \in \Sigma^* \mid q_0, w\vdash_M^* \alpha q, \beta, q \in F}
	\]
\end{definition}

\begin{definition}
	Język \( L \subseteq \Sigma^* \) jest \textbf{rekurencyjnie przeliczalny} (recursively enumerable RE) jeśli istnieje Maszyna Turinga \( M \)
	\[
		L(M) = L
	\]
\end{definition}

\begin{definition}
	Mówimy że Maszyna Turinga  \( M \) ma \textbf{własność stopu} jeśli istnieją dwa stany \( q_{acc} \neq q_{rej} \in Q \)
	i jednocześnie \[
		\forall_{w \in \Sigma^*} : q_0, w \vdash_M^* \alpha, q_f, \beta
	\]
	gdzie \( q_f \in \set{q_{acc}, q_{rej}} \).
	Przy czym \( F = \set{q_{acc}} \) oraz nie ma przejsć ze stanów \( q_{acc}, q_{rej} \)
\end{definition}

\begin{definition}
	Mówmy że język \( L \) jest \textbf{rekurencyjny} (rozstrzygalny) jeśli istnieje Maszyna Turinga z własnością stopu taka, że
	\( L(M) = L \)
\end{definition}

\begin{lemma}
	Każdy język regularny jest rekurencyjny.
\end{lemma}
\begin{lemma}
	\[
		HALT = \set{
			(M, w) \mid
			M \text{ zatrzymuje się na } w
		} \in RE \setminus R
	\]
\end{lemma}
Innymi słowy, problem stopu jest nierozstrzygalny.

\begin{definition}
	\(k\)-taśmowa Maszyna Turinga to
	\[
		M_k = (Q, \Sigma, \Gamma, \delta, q_0, B, F)
	\]
	gdzie
	\( \delta : Q \times \Gamma^k \rightarrow Q \times \pars{\Gamma \times \set{-1, +1}}^k\)
\end{definition}

\begin{theorem}
	Dla każdego \( k \) każdy język akceptowany przez \(k\)-taśmową jest rekurencyjnie przeliczalny.
\end{theorem}
\begin{proof}
	Konstruujemy maszynę jednotaśmową, która symuluje \(k\) taśm.
\end{proof}

Zastanawiamy się teraz, czy jak mamy jakieś słowo \( w \) i zadaną MT \( M \) to umiemy sprawdzić czy \( w \in L(M) \)

Maszynę Turinga możemy zakodować używając tylko zer i jedynek:
\begin{itemize}
	\item Stany \( Q = \set{q_1, \dots, q_k} \) kodujemy unarnie -- stan \( q_i \) zamieniamy na \( i \) zer.

	\item Podobnie \( \Gamma = \set{x_1, \dots x_l} \) -- każdy \( x_i \) jest reprezentowany jako \( i \) zer.

	\item Przejście w lewo kodujemy jako \(0\) a w prawo jako \(00\)

	\item funkcję przejścia \( \delta \) kodujemy jako ciąg zakodowanych tupli \( (q_i, x_j, q_i', x_j', a) \)
\end{itemize}

Czynimy obserwacje:
\begin{itemize}
	\item Każdą MT umiemy zakodować
	\item Być może mamy więcej niż jedno kodowanie na każdą MT
	\item Niektórym kodowaniom nie odpowiada żadna maszyna.
\end{itemize}

Ponieważ nie wszystkie kodowania są poprawne to tym niepoprawnym przypisujemy jakąś maszynę tak, żeby wszystkie kodowania dawały nam maszyny.

\begin{lemma}
	Istnieją języki, które nie są rekurencyjnie przeliczalne
\end{lemma}

\begin{proof}
	Maszyn jest przeliczalnie wiele (bo każde kodowanie jest skończone) możemy więc je ponumerować \( M_1, M_2, \dots \)

	Możemy zrobić nieskończoną tabelę maszyny vs słowa, w której \( A_{i, j} = 1 \iff w_j \in L(M_i) \)

	Robimy teraz język \( L_d = \set{w_i \mid w_i \notin L(M_i)} \) tj. bierzemy te słowa, które \textbf{nie} są akceptowane przez maszynę, którą opisują.

	Załóżmy nie wprost, że istnieje maszyna \( M_k \), która akceptuje język \( L_d \)

	Zastanawiamy się teraz czy jakieś słowo \( w_k \in L_d \)
	Jeśli \( w_k \in L_d \) to \( w \notin L(M_k) \) czyli \( w_k \notin L_d \)
	Tak samo jeśli \( w_k \notin L_d \) to \( w \in L(M_k) \) czyli \( w_k \in L_d \)

	W takim razie \( w_k \in L_d \iff w_k \notin L_d \) -- sprzeczność, zatem \( L_d \) nie jest rekurencyjnie przeliczalny.
\end{proof}



A co z dopełnieniem \( L_d \) ? Czy \( \Sigma^* \setminus L_d \in R \) ?

Zadajmy ogólniejsze pytanie -- Czy jeśli \( L \in R \) to \( \Sigma^* \setminus L \in R \) ?

Okazuje się że tak, bo jeśli mamy MT z własnością  stopu to możemy po prostu zamienić stan akceptujący z odrzucającym i dostać MT, która akceptuje dokładnie dopełnienie i nadal ma własność stopu.

W takim razie \( \Sigma^* \setminus L_d \notin R \) bo inaczej \( L_d \) jest rekurencyjny.

\begin{lemma}
	\( \Sigma^* \setminus L_d \) jest rekurencyjnie przeliczalny.
\end{lemma}
\begin{proof}
	Robimy uniwersalną maszynę \( U \), która potrafi zasymulować dowolną zakodowaną maszynę \( M \) na słowie startowym \( w \).
\end{proof}

\begin{lemma}
	\[
		L_U = \set{(M, w) : w \in L(M)} \in RE \setminus R
	\]
\end{lemma}

\begin{definition}
	\textbf{Redukcja} z języka \( L_1 \) do języka \( L_2 \) to funkcja obliczalna przez maszynę Turinga, taka że \( x \in L_1 \iff L(f(x)) \in L_2 \)

	Jeśli istnieje redukcja z \( L_1 \) do \( L_2 \) to zapisujemy to faktem \( L_1 \leq L_2 \)
\end{definition}

\begin{lemma}
	Jeśli \( L_1 \leq L_2 \) i \( L_1 \) nie jest R (RE) to \( L_2 \) też nie jest R (RE).
\end{lemma}
\begin{proof}
	Nie wprost załóżmy, że \( L_2 \) jest R o czym świadczy maszyna \( M_2 \)

	Skonstruujemy maszynę \( M_1 \), która akceptuje \( L_1 \).

	Przepuszcza ona wejście \( x \) przez funkcję redukującą \( L_1 \) do \( L_2\) a następnie uruchamia \( M_2 \) na słowie \( f(x) \).
\end{proof}

\begin{lemma}
	\[
		L_{NE} = \set{ M : L(M) \neq \varnothing } \in RE
	\]
\end{lemma}
\begin{proof}
	Możemy generować kolejne słowa, dodawać je do kolejki, i wykonywać po jednym kroku symulacji dla każdego słowa z kolejki.
\end{proof}

\begin{lemma}
	\[ L_{NE} \notin R \]
\end{lemma}
\begin{proof}
	Przeprowadzimy redukcję z \( L_u\) do \( L_{NE} \).

	Dla wejścia \( (M, w) \) konstruujemy maszynę \( M' \) taką, że
	\( w \in L(M) \iff M' \neq \varnothing \)
\end{proof}

\begin{definition}
	Własność \( P \) języków RE to podklasa RE (tj. jakiś zbiór języków)

	Własność \( P \) jest nietrywialna jeśli \( P \neq \varnothing \) oraz \( P \neq RE \).

	Własności reprezentujemy jako \( L_p = \set{M \mid L(M) \in P} \)
\end{definition}

\begin{theorem}[Rice]
	Każda nietrywialna własność RE jest nierozstrzygalna.
\end{theorem}
\begin{proof}
	Weźmy dowolną własność \( P \)

	Zredukujemy \( L_U \) do \( L_P \)


\end{proof}

\begin{lemma}
	Dla dwóch gramatyk bezkontekstowych \( G_1, G_2 \) problem
	\[
		L(G_1) \cap G(L_2) \neq \varnothing ?
	\]
	jest nierozstrzygalny
\end{lemma}
\begin{proof}
	Zredukujemy \( L_{NE} \) do tego problemu.
\end{proof}