{
\makeatletter
\def\input@path{{../probabil}}
\makeatother
\graphicspath{{../probabil}}

\begin{questiontext}
    Rozkład dwumianowy, rozkład geometryczny i ich własności. Własność bez pamięci, wartość oczekiwana, wariancja, wyższe momenty. Funkcje tworzące momentów.
\end{questiontext}

\section{Rozkład dwumianowy}
{
	\ExecuteMetaData[chapters/discrete-probability/binomial]{probabil-egzamin-rozklad-dwumianowy-1}
	\ExecuteMetaData[chapters/discrete-probability/binomial]{probabil-egzamin-rozklad-dwumianowy-2}
	\ExecuteMetaData[chapters/discrete-probability/binomial]{probabil-egzamin-rozklad-dwumianowy-3}
}

\section{Rozkład geometryczny}
{
	\ExecuteMetaData[chapters/discrete-probability/geometric]{probabil-egzamin-rozklad-geometryczny-1}
	\ExecuteMetaData[chapters/discrete-probability/geometric]{probabil-egzamin-rozklad-geometryczny-2}
	W oparciu na funkcję tworzącą momenty, możemy obliczyć wartość oczekiwaną, wariancję, oraz dowolne wyższe momenty
	\ExecuteMetaData[chapters/discrete-probability/geometric]{probabil-egzamin-rozklad-geometryczny-3}
}

}