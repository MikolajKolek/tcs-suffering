{
\makeatletter
\def\input@path{{../probabil}}
\makeatother
\graphicspath{{../probabil}}

\begin{questiontext}
    Łańcuch Markowa. Nieprzywiedlność, okres stanu i okres łańcucha. Prawdopodobieństwa przejść pomiędzy stanami w nieprzywiedlnym i nieokresowym łańcuchu Markowa. Stany powracające (dodatnie i zerowe) i chwilowe. W każdym nieprzywiedlnym, skończonym łańcuchu Markowa oczekiwany czas przejścia pomiędzy dwoma stanami jest skończony.
\end{questiontext}

\section{Definicja łańcucha Markowa}
{
	\ExecuteMetaData[chapters/markov-chains/definitions]{probabil-egzamin-9-definicja-lancucha-markowa}
}

\section{Nieprzywiedlność i okres stanu}
{
	\ExecuteMetaData[chapters/markov-chains/definitions]{probabil-egzamin-9-okres-stanu}
	\ExecuteMetaData[chapters/markov-chains/state-classification]{probabil-egzamin-9-nieprzywiedlnosc-1}
	\ExecuteMetaData[chapters/markov-chains/state-classification]{probabil-egzamin-9-nieprzywiedlnosc-2}
}

\section{Okres nieprzywiedlnego łańcucha Markowa}
{
	\ExecuteMetaData[chapters/markov-chains/state-classification]{probabil-egzamin-9-okres-nieprzywiedlnego}
}

\section{Prawdopodobieństwa przejść pomiędzy stanami w nieprzywiedlnym i nieokresowym łańcuchu Markowa}
{
	\ExecuteMetaData[chapters/markov-chains/state-classification]{probabil-egzamin-9-wlasnosc-nieprzywiedlnego-nieokresowego}
}

\section{Stany powracające i chwilowe}
{
	\ExecuteMetaData[chapters/markov-chains/state-classification]{probabil-egzamin-9-stany-powracajace-chwilowe}
}

\section{Własność nieprzywiedlnego, skończonego łańcucha Markowa}
{
	\ExecuteMetaData[chapters/markov-chains/state-classification]{probabil-egzamin-9-wlasnosc-nieprzywiedlnego-skonczonego}
}
}