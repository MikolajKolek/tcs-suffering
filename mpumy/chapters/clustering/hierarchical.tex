\section{Grupowanie hierarchiczne}

Idea jest bardzo prosta -- mamy \( m \) punktów, chcemy je pogrupować.
No to grupujemy je poprzez scalanie dwóch grupek w kolejnych iteracjach.
To które dwie grupki będziemy łączyć to zależy od metody jaką wybierzemy.

Zaczynamy od wyznaczenia odległości między każdymi dwoma punktami \( d(a, b) \) -- może to być odległość w dowolnej metryce w zależności od problemu.

Następnie wybieramy klastry które są najpodobniejsze do siebie.
Możliwe kryteria:
\begin{itemize}
	\item łączenie pojedyncze
	      \[
		      D(A, B) = \min\braces{d(a, b) : a \in A, b \in B}
	      \]

	      Radzi sobie z niekulistymi kształtami ale jest wrażliwe na obserwacje odstające

	\item łączenie pełne
	      \[
		      D(A, B) = \max\braces{d(a, b) : a \in A, b \in B}
	      \]

	      Jest mniej czułe na obserwacje odstające ale rozbija duże klasy.

	\item łączenie średnie
	      \[
		      D(A, B) = \frac{1}{\card{A}\card{B}}\sum_{a\in A}\sum_{b \in B}d(a, b)
	      \]

	      Ma tendencje kuliste i jest mniej podatne na obserwacje odstające.

	\item łączenie centroidalne
	      \[
		      D(A, B) = d(c_A, c_B)
	      \]

	      Dobrze się sprawdza ale jest kosztowne ze względu na liczenie centroidów

	\item łączenie Warda
	      \[
		      D(A, B) = \sum_{a \in A \cup B} d(a, c_{a \cup B})^2
	      \]

	      Działa trochę jak średnie.

\end{itemize}
gdzie centroid to punkt minimalizujący sumę kwadratów odległości.

Aby nieco usprawnić liczenie odległości istnieje algorytm Lance'a-Williamsa -- zaczynamy od zainicjowania macierzy \( D \) odległościami między punktami, a następnie gdy łączymy dwa klastry to kasujemy odpowiadające im wiersze/kolumny i dodajemy nowy wiersz i nową kolumnę w których wartości obliczamy na podstawie właśnie usuniętych danych.


\subsection{Wady i zalety}
Zalety:
\begin{itemize}
	\item Nie wymaga znajomości liczby klastrów
	\item Algorytm nie jest randomizowany -- na tych samych danych działa tak samo
	\item Intuicyjny
\end{itemize}
Wady:
\begin{itemize}
	\item Wolny
	\item Nie działa gdy brakują wartości
	\item Nie działa dobrze jeśli cechy są różnych typów
\end{itemize}