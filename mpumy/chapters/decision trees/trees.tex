\section{Drzewa decyzyjne}

Niech \( x = (x_1, \dots, x_k) \in X \) będzie elementem z przestrzeni wejść.

Będziemy konstruować drzewa binarne w których węzły są jednej z dwóch postaci:
\begin{itemize}
	\item \( x_i \leq t \), \(t \in \real\) gdy \(i\)-ta cecha jest ciągła
	\item \( x_i \in A_0 \) gdzie \( A_0 \) jest jakimś nietrywialnym podzbiorem wartości jakie przyjmuje ta cecha.
\end{itemize}

W ten sposób otrzymujemy podział \( X \) na parami rozłączne zbiory \( R_1, \dots, R_M \)

\subsection{Regresja}

Każdemu \( R_j \) przypisujemy jakąś wartość, którą przewidujemy dla wszystkich jego elementów.
Innymi słowy funkcja decyzyjna jest postaci
\[
	h(x) = \sum_{j = 1}^M \widehat c_j \cdot \indicator{x \in R_j}
\]
Jeśli \( l(y, \widehat y) = (y - \widehat y)^2 \)
to \( \widehat c_j \) jest średnią arytmetyczną wartości w danym obszarze.
\[
	\widehat c_j = \frac{1}{\card{R_j}} \sum_{i : x^{(i)} \in R_j} y^{(i)}
\]

\subsection{Złożoność drzewa}
Oczywiście minimalny błąd treningowy uzyskujemy przy najmniejszym rozdrobnieniu, gdy każdy obszar zawiera dokładnie jeden punkt -- zazwyczaj nie chcemy takiej sytuacji.

Dla uproszczenia załóżmy, że \( x \in R^k \)

Dla każdej cechy \( j \) oraz punktu \( s \) definiujemy dwa zbiory uzyskane w wyniku podziału:
\begin{itemize}
	\item \( R_1(j, s) = \set{x : x_j \leq s} \)
	\item \( R_2(j, s) = \set{x : x_j > s} \)
\end{itemize}
oraz odpowiadające im stałe
\begin{itemize}
	\item \( \widehat c_1 = \sum_{i : x^{(i)} \in R_1} \frac{y^{(i)}}{\card{R_1}} \)
	\item \( \widehat c_2 = \sum_{i : x^{(i)} \in R_2} \frac{y^{(i)}}{\card{R_2}} \)
\end{itemize}

Chcemy teraz zminimalizować błąd:
\[
	J(j, s) =
	\sum_{i : x^{(i)} \in R_1(j, s)}\pars{y^{(i)} - \widehat c_1(j, s)}^2
	+
	\sum_{i : x^{(i)} \in R_2(j, s)}\pars{y^{(i)} - \widehat c_2(j, s)}^2
\]

W tym celu dla każdej cechy \( j \) osobno sortujemy obserwacje rosnąco po \( x_j \) uzyskując ciąg \( x^{(j_1)}, \dots, x^{(j_m)} \).

Możliwych punktów podziału jest teraz \( m - 1 \) -- są to środki pomiędzy kolejnymi punktami.
Dla każdej cechy wybieramy ten punkt, który minimalizuje błąd, a następnie wybieramy tę cechę dla której ten błąd jest najmniejszy.
Dzielimy drzewo i kontynuujemy rekurencyjnie.

\subsection{Pruning}
Kiedyś musimy zakończyć budowę drzewa aby nie doszło do przeuczenia.

Idea pruningu polega na tym, że celowo konstruujemy duże (przeuczone) drzewo, z którego będziemy stopniowo kasować węzły aby otrzymać coś sensownego.

\begin{definition}
	Niech \( T_0 \) będzie drzewem decyzyjnym.
	Mówimy, że \( T \subset T_0 \) jest jego \textbf{poddrzewem} jeśli możemy je otrzymać poprzez usunięcie niektórych węzłów z \( T_0 \).
\end{definition}
\begin{definition}
	\textbf{Złożoność} drzewa \( T \), oznaczaną przez \( \card{T} \) będziemy oznaczać liczbę liści tego drzewa.
\end{definition}

Dla ustalonego \( \alpha \) definiujemy funkcję kosztu drzewa \( T \)
\[
	J_\alpha(T) = \widehat R(T) + \alpha \card{T}
\]

Intuicyjnie odpowiada to znalezieniu balansu pomiędzy ryzykiem empirycznym a rozmiarem drzewa -- w naturalny sposób większe drzewa mają mniejsze ryzyko empiryczne kosztem możliwego przeuczenia ze względu na rozmiar.

Parametr \( \alpha \) dobieramy eksperymentalnie za pomocą danych walidacyjnych.

Optymalnego drzewa szukamy zachłannie w kolejnych iteracjach -- \( T_1 \) powstaje przez usunięcie jednego z liści \( T_0 \) minimalizując przy tym \( \widehat R(T_1) - \widehat R(T_0) \).

\( T_2 \) konstruujemy kasując minimalny liść z \( T_1 \) i tak dalej, aż do wyczerpania zapasów (tj. aż zostanie nam jeden węzeł).

Dostajemy w ten sposób \( \card{T_0} \) drzew, wśród których znajduje się \( \argmin_{T \subseteq T_0}  J_\alpha(T) \) dla dowolnego \( \alpha \geq 0 \)


\subsection{Klasyfikacja}

Załóżmy że mamy \( K \) klas które jakoś chcemy przypisać danym.

Mając drzewo decyzyjne dzielące \( X \) na obszary \( R_1, \dots, R_M \)
obliczamy prawdopodobieństwa
\[
	\widehat p_{n, k} = \frac{1}{\card{R_n}} \sum_{i : x^{(i)} \in R_n} \indicator{y^{(i)} = k}
\]

Naturalnym wyborem funkcji decyzyjnej jest przypisanie każdemu liściowi najczęstszej klasy.

Aby wartości \( \widehat p_{n, k} \) miały jakiś sens wprowadzamy \textit{miarę nieczystości}.
Intuicyjnie liść jest czysty jeśli jakaś klasa wyraźnie w nim dominuje.

Przykładowe funkcje nieczystości to:
\begin{itemize}
	\item błąd klasyfikacji
	      \[
		      1 - \widehat p_{n, h(n)}
	      \]
	\item współczynnik Giniego
	      \[
		      \sum_{k=1}^K \widehat p_{n, k}(1 - \widehat p_{n, k})
	      \]
	\item entropia
	      \[
		      -\sum_{k=1}^K \widehat p_{n, k} \ln \widehat p_{n, k}
	      \]
\end{itemize}

Tak jak w przypadku regresji wybieraliśmy podział minimalizujący ryzyko empiryczne, tak tutaj będziemy chcieli minimalizować ważoną nieczystość -- dla miary nieczystości \( Q \) oraz potencjalnego podziału na \( R_1, R_2 \) minimalizujemy
\[
	\card{R_1} \cdot Q(R_1) + \card{R_2} \cdot Q(R_2)
\]

\subsection{Niebinarne cechy}

Jeśli jakaś cecha \( x_j \) może przyjąć \( q \) możliwych, nieporównywalnych wartości to potencjalnie mamy \( 2^{q - 1} - 1 \) podziałów na niepuste podzbiory -- to dość dużo.

Zamiast tego sortujemy cechy \( A_1, \dots, A_q \) rosnąco po \( p(y = 1 \given x_j = A_i) \).

Mamy dzięki temu \( q - 1 \) możliwych podziałów, co jest już dużo lepsze.

\subsection{Wady i zalety}

Zalety:
\begin{itemize}
	\item łatwo zrozumieć co robią
	\item szybko klasyfikują jak już je skonstruujemy
	\item nie wymagają przetwarzania danych
\end{itemize}
Wady:
\begin{itemize}
	\item nie zawsze są optymalne
	\item wrażliwe na zmiany
	\item długi czas uczenia
	\item dla danych ciągłych podziały są bardzo sztywne
\end{itemize}