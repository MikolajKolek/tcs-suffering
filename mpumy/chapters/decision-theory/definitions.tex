\section{Definicje}

Definiujemy trzy przestrzenie, którymi będziemy się zajmować:
\begin{itemize}
	\item \( X \) -- wejścia
	\item \( A \) -- akcje, czyli nasze przewidywania
	\item \( Y \) -- oczekiwane przewidywania
\end{itemize}

\begin{definition}
	\textbf{Funkcja decyzyjna} to dowolna funkcja \( f : X \rightarrow A \)
\end{definition}

\begin{definition}
	\textbf{Funkcja straty} (kosztu) to dowolna funkcja \( \ell : A \times Y \rightarrow \real \)
\end{definition}

Będziemy ustalali funkcję straty \( \ell \), która opisuje jak dobry jest wynik i starali się dopasować funkcję \( f \), która podejmuje dobre decyzje.


\begin{definition}
	\textbf{Ryzyko} funkcji decyzyjnej \( f \) przy stracie \( \ell \) definiujemy jako
	\[
		R(f) = \expected{\ell(f(X), Y)}
	\]
	gdzie \( X, Y \) to zmienne losowe opisujące dane.
\end{definition}

\begin{definition}
	\textbf{Bayesowska funkcja decyzyjna} to
	\[
		f^* = \argmin_{f : X \rightarrow A} R(f)
	\]
\end{definition}

Ponieważ w praktyce rzadko znamy rozkłady \( X, Y \) to definiujemy też ryzyko empiryczne, które odnosi się bezpośrednio do jakiegoś zbioru danych.

\begin{definition}
	\textbf{Ryzyko empiryczne} funkcji decyzyjnej \( f \) na danych \( D = \set{ \pars{x^{(1)}, y^{(1)}}, \dots, \pars{x^{(m)}, y^{(m)}} }\) definiujemy jako
	\[
		\widehat R_m(f) = \frac{1}{m} \sum_{i=1}^m \ell\pars{f(x^{(i)}), y^{(i)}}
	\]
\end{definition}
Z silnego prawa wielkich liczb wiemy, że prawie na pewno mamy zbieżność \( \widehat R_m \rightarrow R_f \)

Aby znaleźć dobrą funkcję decyzyjną \( f \) mając jakieś dane \( D \) będziemy minimalizować ryzyko empiryczne, czyli średnią stratę \( f \) na danych.

Ponieważ potencjalnych funkcji decyzyjnych może być dużo i nie wszystkie są dla nas sensowne to wprowadzamy pojęcie przestrzeni hipotez.

\begin{definition}
	\textbf{Przestrzeń hipotez} to dowolny zbiór funkcji decyzyjnych.
\end{definition}

Dla zadanego zbioru hipotez \( H \) będziemy szukać
\[
	\widehat f_H = \argmin_{f \in H} \frac{1}{m} \sum_{i=1}^m \ell\pars{f(x^{(i)}), y^{(i)}}
\]