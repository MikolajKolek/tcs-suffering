\section{Metody oceny klasyfikatora}

Załóżmy że mamy klasyfikator binarny. Jak ocenić czy dobrze sobie radzi?

Dla danego wejścia mamy cztery możliwości jakości odpowiedzi:
\begin{itemize}
	\item True Positive -- poprawnie odpowiedzieliśmy ,,TAK''
	\item True Positive -- poprawnie odpowiedzieliśmy ,,NIE''
	\item False Positive -- błędnie odpowiedzieliśmy ,,TAK''
	\item False Negative -- błędnie odpowiedzieliśmy ,,NIE''
\end{itemize}

Mamy trzy miary jakości:
\begin{itemize}
	\item dokładność -- jak wiele danych jest poprawnie zaklasyfikowanych
	      \[
		      \frac{TP + TN}{TP + TN + FP + FN}
	      \]
	\item precyzja -- jak wiele danych, które uznaliśmy za pozytywne faktycznie jest pozytywnych. Wysoka precyzja to niski poziom fałszywych alarmów.
	      \[
		      \frac{TP}{TP + FP}
	      \]
	\item czułość -- jak dużo faktycznie pozytywnych danych uznaliśmy za pozytywne. Wysoka czułość to wysoki poziom detekcji pozytywnych przypadków.
	      \[
		      \frac{TP}{TP + FN}
	      \]
\end{itemize}

Chcielibyśmy osiągnąć zarówno wysoką precyzję jak i wysoką czułość -- np. jeśli wykrywamy obecność wirusa to z jednej strony chcemy przegapiać jak najmniej zarażeń (czułość), ale nie chcemy też wszystkich kierować na kwarantannę (precyzja).

\subsection{Miara F-Beta}

\begin{definition}
	Miarę \( F_\beta \) definiujemy jako
	\[
		F_\beta = (1 + \beta^2) \cdot \frac{\textnormal{precyzja} \cdot \textnormal{czułość}}{\beta^2 \cdot \textnormal{precyzja} + \textnormal{czułość}}
	\]
\end{definition}

Szczególnym przypadkiem jest miara \( F_1 \)
\[
	F_1 = 2 \cdot \frac{\text{precyzja} \cdot \text{czułość}}{\text{precyzja} + \text{czułość}}
\]

\( \beta > 1 \) premiuje czułość, natomiast \( \beta < 1 \) premiuje precyzję.

\subsection{Krzywa ROC}

Krzywa ROC to wykres, który dla różnych modeli ilustruje zależność czułości czyli poziomu wyników prawdziwie pozytywnych \( \frac{TP}{TP + FN} \) od poziomu wyników fałszywie pozytywnych \( \frac{FP}{FP + TN} \)

Oś \( y = x \) odpowiada losowej klasyfikacji, obszar nad nią oznacza dobry klasyfikator, obszar pod nią oznacza zły klasyfikator.

Idealny klasyfikator znajduje się w punkcie (0, 1) -- nie ma w ogóle wyników fałszywie pozytywnych a wszystkie prawdziwe dane są poprawnie zaklasyfikowane.
