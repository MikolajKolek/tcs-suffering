\chapter{Regularyzacja}

W regresji staramy się znaleźć hipotezę \( h \), która osiąga jak najmniejszy błąd na danych treningowych. Co jeśli dochodzi jednak do przeuczenia?
Jednym z powodów może być nadmierna złożoność hipotezy, która jest na tyle ,,elastyczna''/,,złożona'' że praktycznie zapamiętuje cały zbiór danych.
Będziemy próbowali jakoś odsiać takie hipotezy, które są zbyt złożone.

Zaczynamy w tym celu od wprowadzenia miary złożoności.
\begin{definition}
	\textbf{Miara złożoności} dla przestrzeni hipotez \( H \)
	to dowolna funkcja
	\[
		\Omega : H \rightarrow [0, +\infty)
	\]
\end{definition}
Miarą złożoności może być:
\begin{itemize}
	\item liczba rozważanych cech
	\item stopień dopasowanego wielomianu
	\item głębokość drzewa
	\item \( \ell_0(\theta) = \card{\set{i : \theta_i \neq 0}} \)
	\item \( \ell_1(\theta) = \sum_{i=1}^k \abs{\theta_i} \)
	\item \( \ell_2(\theta) = \sum_{i=1}^k \theta_i^2 \)
\end{itemize}

Mamy dwa rodzaje regularyzacji:
\begin{itemize}
	\item Regularyzacja Tichonowa -- dodajemy do funkcji straty koszt za złożoność hipotezy:
	      \[
		      l(h) = \ell(h) + \lambda \Omega(h)
	      \]

	\item Regularyzacja Iwanowa -- zawężamy się do tych hipotez, które są mało złożone
	      \[
		      H' = \set{h \in H : \Omega(h) \leq \omega}
	      \]
\end{itemize}

Okazuje się, że te dwie definicje są sobie równoważne tj. jeśli mamy optimum \( h^*_\lambda \) dla regularyzacji Tichonowa to jest ono również rozwiązaniem optymalnym dla regularyzacji Iwanowa dla jakiegoś \( \omega \).

Analogicznie \( h^*_\omega \) jest optymalne dla Tichonowa przy pewnym \( \lambda \).

Jeśli używamy funkcji \( \ell_1 \) to mamy do czynienia z \textbf{regresją lasso}, a jeśli używamy \( \ell_2 \) to z \textbf{regresją grzbietową}.