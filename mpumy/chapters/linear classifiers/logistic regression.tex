\section{Regresja logistyczna}

Regresja logistyczna korzysta z funkcji decyzyjnej zadanej wzorem
\[
	h_\theta(x) = \sigma(\theta^T x) = \frac{1}{1 + \exp(-\theta^Tx)}
\]

Traktujemy wyjście jako rozkład prawdopodobieństwa:
\[
	p(y = 1 \given x, \theta) = h_\theta(x)
\]
\[
	p(y = 0 \given x, \theta) = 1 - h_\theta(x)
\]
Możemy też zapisać sprytnie jako
\[
	p(y \given x, \theta) = \pars{h_\theta(x)}^y\pars{1 - h_\theta(x)}^{1-y}
\]

Chcemy zmaksymalizować log-wiarygodnosć
\begin{align*}
	\ell(\theta)
	 & = \log \prod_{i=1}^m p(y^{(i)} \given x^{(i)}, \theta)                                            \\
	 & = \sum_{i=1}^m \pars{y^{(i)} \log h_\theta(x^{(i)}) + (1 - y^{(i)}) \log (1 - h_\theta(x^{(i)}))}
\end{align*}