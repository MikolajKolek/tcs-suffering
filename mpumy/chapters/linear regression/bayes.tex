\section{Bayesowska regresja liniowa}

W normalnej regresji liniowej staraliśmy się pokonać szum i dopasować jedną funkcję.
Teraz będziemy chcieli znaleźć rozkład prawdopodobieństwa szukanych funkcji co nam daje dużo więcej informacji.

Mamy:
\[
	p(\theta) = \Normal(\theta \given \mu_0, S_0)
\]
oraz
\[
	p(Y \given X, \theta) = \prod_{i=1}^m \Normal(y^{(i)}, \theta^T\phi(x^{(i)}), \sigma^2)
\]

Rozkład a posteriori zadany jest wzorem:
\[
	p(\theta \given Y, X) = \frac{p(Y, X \given \theta)p(\theta)}{p(X, Y)}
\]
Mianownik jest stałą, więc \( p(\theta \given Y, X) \) jest iloczynem wielu rozkładów normalnych zatem
\[
	p(\theta \given Y, X) = \Normal(\theta \given \mu_m, S_m)
\]

\subsection{Rozkład predykcyjny}

Jeszcze lepiej będzie jak zamiast przewidywać sam parametr \( \theta \) będziemy umieli obliczać rozkład prawdopodobieństwa nowej danej \( x^{(m+1)}, y^{(m+1)} \).