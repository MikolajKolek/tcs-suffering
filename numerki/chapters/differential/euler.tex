\section{Metoda Eulera}
\textit{Intuicja}: Narysować strzałki wskazujące kierunek funkcji w punkcie, po czym pójść po strzałkach. Dla równania \( y' = F(y) \) wyliczamy kolejne wartości \( y \), zaczynając od \( y_0 = y(0) \).
\[
	y_{k+1} = y_k + h \cdot F(y_k)
\]
I tak rozwiązujemy równanie różniczkowe numerycznie. Dla bardzo małych \( h \) będzie stabilne, dla większych nie. Problemy, dla których potrzeba bardzo małych \( h \) do dobrych wyników są \textit{sztywne}.

\noindent
\textbf{Niejawna metoda Eulera} \\
Zaczynając od \( y_0 = y(0) \)
\[
	y_{k+1} = y_k + h \cdot F(y_{k+1})
\]
Daje lepszą stabilność niż podstawowy wariant.