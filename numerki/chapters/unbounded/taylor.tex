\section{Wzór Taylora}
Wzór Taylora dla jednej zmiennej:
\[
	f(x) \approx f(x_0) + f'(x_0)(x - a) + \frac{1}{2}f''(x_0)(x - x_0)^2
\]
Wtedy ekstremum \( f \) jest gdzieś w okolicy \( x_0 - \frac{f'(x_0)}{f''(x_0)} \). \\
Przybliżenie ze wzoru Taylora dla funkcji wielu zmiennych:
\[
	f(x) \approx f(x_0) + \nabla f(x_0) \cdot (x - x_0) + (x - x_0)^T \cdot H_f(x_0) \cdot (x - x_0)
\]
\[
	\nabla f = \left[ \frac{\partial f}{\partial x_1}, \cdots, \frac{\partial f}{\partial x_n} \right]
\]
W punkcie stacjonarnym \( x_0 \) ( spełniającym \( \nabla f = 0 \)):
\begin{itemize}
	\onehalfspacing
	\item jeśli \( H_f(x_0) \) jest dodatnio określona, to w \( x_0 \) jest minimum.
	\item jeśli \( H_f(x_0) \) jest ujemnie określona, to w \( x_0 \) jest maksimum.
	\item jeśli \( H_f(x_0) \) jest nieosobliwa, ale nie ma maksimum ani minimum, to w \( x_0 \) jest punkt siodłowy.
	\item jeśli \( H_f(x_0) \) jest osobliwa, mogą wystąpić sytuacje zdegenerowane (może też dowolna \linebreak z powyższych).
\end{itemize}