\section{Metoda gradient descent}
Będąc w punkcie \( x_k \), idziemy w kierunku wektora gradientu:
\[
	x_{k+1} = x_k - \nabla f(x_k) \cdot t,
\]
gdzie t jest pewną stałą. Nie potrzeba wielu obliczeń f.
\begin{warning}
	Można kluczyć lub przeskoczyć minimum.
\end{warning}

\subsection{Metoda gradientu prostego – wariant}
Będąc w punkcie \( x_k \), idziemy w kierunku wektora gradientu od razu do optimum na prostej:
\[
	h_k(t) = f(x_k - \nabla f(x_k) \cdot t)
\]
\[
	x_{k+1} = \min_{t}\; h_k(t)
\]
Szukamy minimum \( h_k \) za pomocą metod dla jednej zmiennej (Newton, wyszukiwanie ternarne). Daje to większą dokładność, ale też trzeba więcej razy obliczać f.