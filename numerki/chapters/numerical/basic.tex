\section{Podstawowe kwadratury}
Jeśli \( x_1, \dots, x_n \) są równo położone na przedziale \( [a, b] \), tę metodę nazywa się kwadraturą Newtona-Cotesa. \\
\textbf{Kwadratury Newtona-Cotesa}
\begin{itemize}
	\onehalfspacing
	\item zamknięte (\( a \) i \( b \) są pierwszym i ostatnim węzłem interpolacji)
	\item otwarte (\( a \) i \( b \) nie są węzłami interpolacji).
\end{itemize}
\vspace{1.2em}

\begin{itemize}
	\onehalfspacing
	\item Metoda prostokątów – kwadratura otwarta dla \( n = 1 \)
	      \[
		      \int_{a}^{b} f = (b-a) \cdot \frac{f(a+b)}{2}
	      \]
	      Błąd: \( O((b - a)^3) \)
	\item Metoda trapezów – kwadratura zamknięta dla \( n = 1 \)
	      \[
		      \int_{a}^{b} f = (b-a) \cdot \frac{f(a)+f(b)}{2}
	      \]
	      Błąd: \( O((b - a)^3) \)
	\item Metoda Simpsona – kwadratura zamknięta dla \( n = 2 \)
	      \[
		      \int_{a}^{b} f = \frac{b-a}{6} \cdot \left(f(a) + 4f\left(\frac{a+b}{2}\right) + f(b)\right)
	      \]
	      Błąd: \( O((b - a)^5) \)
	\item Metoda Simpsona 3/8 – kwadratura zamknięta dla \( n = 2 \)
	      \[
		      \int_{a}^{b} f = \frac{b-a}{8} \cdot \left(3f\left(\frac{2a+b}{3}\right) + 3f \left(\frac{a+2b}{3}\right) + f(b)\right)
	      \]
\end{itemize}
Dla dużych przedziałów błąd jest duży, więc lepiej zastosować kwadratury złożone, czyli dla podziału na mniejsze przedziały. Przykład dla prostokątów:
\[
	\sum \Delta x \cdot f\left(\frac{x_i + x_{i+1}}{2}\right)
\]
dla przedziałów \( [x_i,\:x_{i+1}] \) wielkości \( \Delta x \). \\
Dla kwadratur złożonych błędy wynoszą \( O(\Delta x^2) \) (prostokąty, trapezy) i \( O(\Delta x^4) \) (Simpson).