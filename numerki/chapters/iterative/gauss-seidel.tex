\section{Metody Gaussa-Seidela}
Przyjmujemy za \( Q = U \) dolny trójkąt macierzy \( A \). Jeśli rozłożymy \( A = L + U \), gdzie \( U \) ma zera na przekątnej:
\[
	x_{n+1} = L^{-1}(b - Ux_n)
\]
Metoda jest zbieżna, jeśli \( A \) ma dominującą przekątną albo jeśli \( A \) jest symetryczna i dodatnio określona.