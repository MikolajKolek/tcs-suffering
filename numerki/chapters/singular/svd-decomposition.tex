\textbf{Prawe wektory szczególne} \( A \) to \( v_1, \dots, v_n \), dla których \( A^TA \cdot v_i = \lambda_iv_i \), dla \( \lambda_i \in \mathbb{R} \). \\
\textbf{Wartości szczególne} \( A \) to \( \sqrt{\lambda_1}, \dots, \sqrt{\lambda_n} \). \\
\textbf{Lewe wektory szczególne} \( A \) to \( u_1, \dots, u_n \), gdzie \( u_i = \frac{1}{\sqrt{\lambda_i}}A v_i \). \\
Wektor szczególny to taki, który odpowiada najbardziej wydłużanemu kierunkowi.

\section{Rozkład SVD}
Przekształcamy macierz A liniowo:
\[
	A = U\Sigma V^T
\]
złożeniem odwzorowań, gdzie \( V \) ma wektory \( v_i \)  jako wiersze, \( U \) ma \( u_i \) jako kolumny, \( \Sigma \) jest diagonalna z \( \sqrt{\lambda_i} \).
\begin{enumerate}
	\singlespacing
	\item zmiana bazy (obrót przestrzeni) \( (v_i \rightarrow e_i) \)
	\item skalowanie \( (e_i \rightarrow \sqrt{\lambda_i} e_i) \)
	\item powrót (obrót przestrzeni) \( (e_i \rightarrow u_i) \)
\end{enumerate}
\textbf{Przypadek I: A jest osobliwa} \\
Wtedy niektóre \( \lambda_i \) są zerami. Niezerowych \( u_i \) jest mniej niż \( n \), więc uzupełniamy je dowolnymi wektorami do bazy ortonormalnej. \\
\textbf{Przypadek II: A nie jest kwadratowa} \\
Jeśli \( m < n \), to \( A^TA \) jest rozmiaru \( n \times n \), ale ma mniejszy rząd (\( m \)) i część wartości szczególnych jest zerami. Mamy \( n \) prawych i \( m \) lewych wektorów szczególnych. \\
Jeśli \( m > n \), to \( A^TA \) jest rozmiaru \( n \times n \). Mamy \( n \) prawych wektorów szczególnych – dopełniamy \( u_i \) wektorami \( Av_i \) do bazy. \\
Macierz \( V \) jest \( n \times n \), \( \Sigma \) – \( m \times n \), a \( U \) jest \( m \times m \).
