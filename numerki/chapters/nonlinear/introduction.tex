Celem jest znaleźć pierwiastek funkcji na przedziale \( [a, b] \). Metody korzystają z tego: \\
\textbf{Twierdzenie Darboux} \\
Jeśli funkcja \( f: [a, b] \rightarrow \mathbb{R} \) jest ciągła i \( f(a) \leq y \leq f(b) \) lub \( f(b) \leq y \leq f(a) \), to istnieje \( x \in [a, b] \), t.że \( f(x) = y \). \\
\textbf{Twierdzenie Lagrange’a} \\
Jeśli \( f: [a, b] \rightarrow \mathbb{R} \) jest różniczkowalna, to istnieje \( x \in (a, b) \) t.że \( f'(x) = \frac{f(b) - f(a)}{b - a} \). \\
\textbf{Wzór Taylora}
\[
	f(x) = f(a) + \frac{f'(a)}{1!}(x - a) + \frac{f''(a)}{2!}(x - a)^2 + \cdots + \frac{f^{(k)}(a)}{k!}(x - a)^k + R_{k+1}(x, a),
\]
gdzie \( R_{k+1} = \frac{f'(\xi)}{(k+1)!}(x - a)^{k+1} \).