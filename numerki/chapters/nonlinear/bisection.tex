\section{Metoda bisekcji}
\begin{multicols}{2}
	\noindent
	\( p_0 = a, q_0 = b \) \\
	\( x_n = \frac{p_n + q_n}{2} \) \\
	\columnbreak
	\begin{equation*}
		(p_n, q_n) =
		\begin{cases}
			(p_{n-1}, x_{n-1}) & f(p_{n-1}) \cdot f(x_{n-1}) < 0 \\
			(x_{n-1}, q_{n-1}) & \text{wpp.}
		\end{cases}
	\end{equation*}
\end{multicols}
\noindent
Zbieżność liniowa - \( k \) cyfr znaczących wymaga \( O(k) \) kroków
