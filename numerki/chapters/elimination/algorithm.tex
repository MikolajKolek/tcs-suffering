\section{Algorytm - rozkład LU}
\begin{lstlisting}[language=Cpp]
for i = 1, ..., n
    m = max_j(|A[j][i]|)
    swap(A[i], A[m]) // partial pivoting
    swap(L[i][:i], L[m][:i])
    a = A[i][i]
    for j = i+1, ..., n
        r = A[j][i] / a
        L[i][j] = r
        A[j] -= r * A[i]
\end{lstlisting}
\noindent
Złożoność: \( O(n^3) \), dokładnie \( \frac{2}{3}n^3 \) operacji na liczbach rzeczywistych

\begin{warning}
	Przy wyborze małego elementu głównego stabilność numeryczna pogarsza się przez błędy zaokrągleń - mogą pojawić się bardzo małe i bardzo duże wartości.
\end{warning}

Pivoting (częściowy) pomaga. Wybieramy wiersz z największym na wartość bezwzględną elementem na przekątnej i to daje większą stabilność numeryczną. Przy rozkładzie LU trzeba wtedy pamiętać, że wynikiem jest rozkład \( LU = PA \), gdzie \( P \) jest macierzą permutacji.
